\documentclass[12pt]{letter} % Copiata dal nostro stile

\usepackage{setspace}
\usepackage{geometry} % Per modificare margini e dimensioni varie
\usepackage{graphicx} % Per inserire immagini

\geometry{
	paper=a4paper, % Change to letterpaper for US letter
	top=3cm, % Top margin
	bottom=1.5cm, % Bottom margin
	left=2.2cm, % Left margin
	right=2.2cm, % Right margin
	%showframe, % Uncomment to show how the type block is set on the page
}

\usepackage[T1]{fontenc} % Output font encoding for international characters
\usepackage[utf8]{inputenc} % Required for inputting international characters
\usepackage{charter} % stix di default, copiato dagli answer credo
\usepackage{scrextend} % indentazione per il corpo della lettera
\usepackage{microtype} % Improve justification
\usepackage{eurosym} % wanna see the muney

\date{11/01/2019} % comment to show today date

\signature{
	Project manager 7DOS \\
	\includegraphics[width=6cm]{../PianoDiProgetto/Res/Firme/andrea.png}
} % firma a piè di pagina

%\address{123 Broadway \\ City, State 12345 \\ (000) 111-1111} % Your address and phone number











%\usepackage{calc} %introduce la notazione infissa per le op. aritmetiche interne a LaTeX

\usepackage[utf8]{inputenc}
\usepackage[T1]{fontenc}
\usepackage[italian]{babel} %il documento è in italiano
%\usepackage{textcomp} %The pack­age sup­ports the Text Com­pan­ion fonts, which pro­vide many text sym­bols
%(such as baht, bul­let, copy­right, mu­si­cal­note, onequar­ter, sec­tion, and yen), in the TS1 en­cod­ing.

\usepackage{graphicx}       %permette di inserire delle immagini
%\usepackage{caption}        %numerazione figure e loro descrizione testuale
\usepackage[labelformat=empty]{caption}
\usepackage{subcaption}     %sottofigure numerabili
\usepackage{float}  %permette di inserire un # qualsiasi di figure fluttuanti
\usepackage{xcolor}
\usepackage{rotating} %permette di ruotare le immagini
%\usepackage{changepage} %utile se c'è bisogno di aggiustare margini per centrare figure

%package utili per la math mode ( $ ... $ o \[ ... \] )
\usepackage{amsmath}
\usepackage{amssymb}
\usepackage{amsfonts}
%\usepackage{euler}    %font 'ams euler', lo stesso di 'Concrete Mathematics' di Knuth
\usepackage{amsthm}
\usepackage{mathtools}

% package utili per tabelle(\thead in particolare)
\usepackage{array, booktabs, caption}
\usepackage{makecell}
\renewcommand\theadfont{\bfseries}
\usepackage{boldline}

\usepackage{listings} %permette di inserire degli spezzoni di codice

\usepackage{tikz} %disegno di immagini vettoriali a schermo. Utile per grafi
\usetikzlibrary{arrows.meta}
\usetikzlibrary{graphs}
\usetikzlibrary{arrows}
%\usepackage{tikz-uml} %serve per disgnare l'UML, fantastica guida:
%https://perso.ensta-paristech.fr/~kielbasi/tikzuml/var/files/doc/tikzumlmanual.pdf
%download package: http://perso.ensta-paristech.fr/~kielbasi/tikzuml/

%package per le tabelle
\usepackage{booktabs} %permette di poter usare delle liste nelle tabelle
\usepackage{tabularx} 
\usepackage{longtable} %una tabella può continuare su più pagine
\usepackage{multirow} %utile per visualizzare una cella su più righe
%\usepackage{multicolumn} %cella su più colonne
%\usepackage[table]{xcolor} %rende disponibile l'utilizzo di un colore per lo sfondo
                        %delle celle di una tabella

%crea una cella per le tabelle in grado di andare a capo con \newline
%https://tex.stackexchange.com/questions/12703/how-to-create-fixed-width-table-columns-with-text-raggedright-centered-raggedlef
\usepackage{array}
\newcolumntype{L}[1]{>{\raggedright\let\newline\\\arraybackslash\hspace{0pt}}m{#1}}
\newcolumntype{C}[1]{>{\centering\let\newline\\\arraybackslash\hspace{0pt}}m{#1}}
\newcolumntype{R}[1]{>{\raggedleft\let\newline\\\arraybackslash\hspace{0pt}}m{#1}}


%indice con i puntini
\usepackage{tocloft}

%http://ctan.mirror.garr.it/mirrors/CTAN/macros/latex/contrib/appendix/appendix.pdf
\usepackage{appendix} %aggiunge dei comandi per l'appendice
\usepackage{parskip} %aiuta LaTeX a trovare il miglior stile per i page break
\setcounter{secnumdepth}{5} % numera i sottoparagrafi
\setcounter{tocdepth}{5} %aggiunge all'indice i sottoparagrafi
%\usepackage{titlesec} %\begin{paragraph} si può usare come subsubsubsection!


\usepackage{breakurl}%\url{...} può continare alla linea successiva. (si può andare a capo)

%Pacchetto per il simbolo degli euro
\usepackage{eurosym}

%Pacchetto per i colori delle tabelle
\usepackage{color, colortbl}

\definecolor{Maroon}{cmyk}{0, 0.87, 0.68, 0.32}
\usepackage[colorlinks=true]{hyperref}
\hypersetup{
    colorlinks=true,
    citecolor=black,
    filecolor=black,
    linkcolor=black, % colore dei link interni
    urlcolor=blue  % colore dei link interniesterni
}

