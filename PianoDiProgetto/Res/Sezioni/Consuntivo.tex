\newsection{Consuntivi di periodo e preventivo}
\section{Consuntivi di periodo}
%volendo possiamo mettere breve introduzione e spiegare la notazione usata nelle tabelle dopo
\subsection{Periodo di analisi dei requisiti}
\subsubsection{Consuntivo}
%tabella presa da analisi con numeri che rappresentano ore spese in più o in meno tra parentesi 
\subsubsection{Variazioni dalla pianificazione}
%indicare perchè ci sono state variazioni
\subsubsection{Considerazioni}
%brevi considerazioni, indicare perchè sforato o non
\section{Preventivo}
%tabella preventivo
%arrotondamento per eccesso preventivo con frase tipo: il budget in più ci aiuterà in caso si dovessero verificare situazioni di rischio
\pagebreak