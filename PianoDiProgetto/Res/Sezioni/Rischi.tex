\newsection{Analisi dei Rischi}
L'analisi dei rischi è strutturata nel seguente modo:
La fase di identificazione dei rischi ha lo scopo di individuare i rischi.
Per ogni rischio individuato viene poi specificata la probabilità che si verifichi,
la gravità delle conseguenze, il sistema di monitoraggio ed il piano di contingenza.
\subsection{Identificazione dei rischi}
3 categorie di rischi sono state individuate:
\begin{itemize}
\item Rischi umani:
	\begin{itemize}
	\item Conflitti tra i membri del team (es. disaccordi, tensioni);
	\item Problemi di natura personali dei membri del team (es. periodi di malattia, impegni personali).
	\end{itemize}
\item Rischi tecnologici:
	\begin{itemize}
	\item Guasti agli strumenti di lavoro personali (es. guasti a pc o connessione alla rete personali);
	\item Guasti ai servizi usati a supporto (es. GitHub).
	\end{itemize}
\item Rischi di progetto:
	\begin{itemize}
	\item Instabilità dei requisiti;
	\item Sottostima delle risorse necessarie;
	\item Errori di gestione del progetto.
	\end{itemize}
\end{itemize}
\subsection{Rischi umani}
\subsubsection{Conflitti tra i membri del team}
\begin{itemize}
\item \textbf{Probabilità di occorrenza:} bassa;
\item \textbf{Conseguenze:} lievi;
\item \textbf{Monitoraggio:} Spetta ai membri del gruppo segnalare al responsabile eventuali problemi interpersonali attinenti al progetto;
\item \textbf{Piano di contingenza:} Il responsabile interverrà per mitigare le tensioni e in caso estremo riassegnare le attività per evitare interamente i conflitti.
\end{itemize}
\subsubsection{Problemi di natura personali dei membri del team}
\begin{itemize}
\item \textbf{Probabilità di occorrenza:} media;
\item \textbf{Conseguenze:} gravi;
\item \textbf{Monitoraggio:} Spetta ai membri del gruppo segnalare al responsabile eventuali periodi di indisponibilità quanto prima possibile;
\item \textbf{Piano di contingenza:} La segnalazione tempestiva permette al responsabile di progetto di ripianificare le attività opportunamente.
\end{itemize}
\subsection{Rischi tecnologici}
\subsubsection{Guasti agli strumenti di lavoro personali}
\begin{itemize}
\item \textbf{Probabilità di occorrenza:} bassa;
\item \textbf{Conseguenze:} gravi;
\item \textbf{Monitoraggio:} Spetta ai membri del gruppo segnalare l'impossibilità di lavorare per via di eventuali guasti agli strumenti di lavoro personali;
\item \textbf{Piano di contingenza:} Se il periodo di inattività non può essere recuperato, il membro del gruppo che dovesse avere problemi sarà tenuto a procurarsi degli strumenti (computer, connessione ad internet) di fortuna oppure ad utilizzare quelli dei laboratori dell'università.
\end{itemize}
\subsubsection{Guasti ai servizi usati a supporto}
\begin{itemize}
\item \textbf{Probabilità di occorrenza:} bassa;
\item \textbf{Conseguenze:} gravi;
\item \textbf{Monitoraggio:} I membri del team che dovessero notare disservizi degli strumenti usati come GitHub o nTask sono tenuti a segnalarlo al responsabile;
\item \textbf{Piano di contingenza:} I membri del gruppo sono tenuti ad effettuare almeno un backup della repository remota al giorno. In caso di disservizi duraturi, saranno utilizzati servizi alternativi tra quelli considerati	durante la scelta degli strumenti da utilizzare.
\end{itemize}
\subsection{Rischi di progetto}
\subsubsection{Instabilità dei requisiti}
\begin{itemize}
\item \textbf{Probabilità di occorrenza:} media;
\item \textbf{Conseguenze:} gravi;
\item \textbf{Monitoraggio:} Il responsabile ha il compito di monitorare i requisiti;
\item \textbf{Piano di contingenza:} Dell'instabilità è aspettata e sintomo del raffinamento dei requisiti.	Se dovesse essere dovuta a requisiti volatili poiché non abbastanza specifici o perché in costante mutamento, il responsabile dovrà contattare il Proponente per accordare un incontro per risolvere il problema.
\end{itemize}
\subsubsection{Sottostima delle risorse necessarie}
\begin{itemize}
\item \textbf{Probabilità di occorrenza:} media;
\item \textbf{Conseguenze:} medie;
\item \textbf{Monitoraggio:} Il responsabile deve accertarsi che il team sia in-schedule. I membri del gruppo devono avvisare il responsabile se la probabilità di sforare con le tempistiche dovesse aumentare;
\item \textbf{Piano di contingenza:} Il responsabile è tenuto in tal caso a determinare se una diversa pianificazione può risolvere il problema o se dei requisiti (di bassa priorità) debbano rimanere insoddisfatti.
\end{itemize}
\subsubsection{Errori di gestione del progetto}
\begin{itemize}
\item \textbf{Probabilità di occorrenza:} media;
\item \textbf{Conseguenze:} medie;
\item \textbf{Monitoraggio:} Il responsabile è tenuto ad effettuare l'attività di monitoraggio;
\item \textbf{Piano di contingenza:} Il responsabile è tenuto a risolvere il problema assicurandosi che le politiche di gestione del progetto siano rispettate e eventualmente modificandole.
\end{itemize}
\pagebreak