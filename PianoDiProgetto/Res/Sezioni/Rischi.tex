\newsection{Analisi dei rischi}

\newlength\colB \setlength\colB{6em}
\newlength\colC \setlength\colC{4em}
\newlength\colD \setlength\colD{16em}
\newlength\combinedlength
\setlength\combinedlength{%
    \dimexpr\colB+\colC+\colD+4\tabcolsep+%
    2\arrayrulewidth\relax}

L'analisi dei rischi è strutturata nel seguente modo:
\begin{itemize}
	\item \textbf{Identificazione}: nel quale vengono individuati i potenziali \gl{rischi} che possono essere incontrati durante lo svolgimento del progetto;
	\item \textbf{Analisi}: quantificare la probabilità che si verifichi,
	la gravità delle conseguenze e dedurne le criticità;
	\item \textbf{Sistema di monitoraggio}: identificazione del sistema con il quale il team controllerà i rischi, per far si che si possano prevenire ed evitare;
	\item \textbf{Piano di contingenza}: istituire dei piani atti alla mitigazione degli effetti dannosi di un rischio nel caso questo dovesse verificarsi.
\end{itemize}
\subsection{Identificazione dei rischi}
	Tre categorie di rischi sono state individuate:
	\begin{itemize}
		\item Rischi umani;
		\item Rischi tecnologici;
		\item Rischi di progetto.
	\end{itemize}
\subsection{Analisi dei rischi}
	\subsubsection{Rischi umani}

\renewcommand{\arraystretch}{1.5}
\begin{longtable}{| m{12em} | m{6em} | m{4em} | m{16em} |}
\hline \rowcolor{title_row}
\centering\textbf{\color{title_text}Rischio} &
\centering\textbf{\color{title_text}Probabilità di occorrenza} &
\centering\textbf{\color{title_text}Impatto} &
\centering\textbf{\color{title_text}Monitoraggio}
\endhead

\hline
\textbf{Conflitti tra i membri del team} (es. disaccordi, tensioni) & Bassa & Lieve & Spetta ai membri del gruppo segnalare al responsabile eventuali problemi interpersonali attinenti al progetto \\

\hline \rowcolor{Gray}
\textbf{Piano di contingenza} &
\multicolumn{3}{ m{\combinedlength} |}{Il responsabile interverrà per mitigare le tensioni e in caso estremo riassegnare le attività per evitare interamente i conflitti}\\

\hline
\textbf{Problemi di natura personali dei membri del team} (es. periodi di malattia, impegni personali) & Media & Grave &
Spetta ai membri del gruppo segnalare al responsabile eventuali periodi di indisponibilità quanto prima possibile \\

\hline \rowcolor{Gray}
\textbf{Piano di contingenza} &
\multicolumn{3}{ m{\combinedlength} |}{La segnalazione tempestiva permette al responsabile di progetto di ripianificare le attività opportunamente}\\

\hline
\textbf{Inesperienza del gruppo} & Alta & Grave &
Spetta ai membri del gruppo segnalare lacune dal punto di vista professionale o l'imbattersi in un attività che non si riesca a svolgere \\

\hline \rowcolor{Gray}
\textbf{Piano di contingenza} &
\multicolumn{3}{ m{\combinedlength} |}{
La segnalazione tempestiva permette al responsabile di progetto di trovare una soluzione adeguata. In particolare il responsabile dovrà decidere se assegnare più ore/uomo per lo svolgimento dell'attività o se incaricare un membro del gruppo in funzione di supporto. Alla decisione del responsabile dovrà aver seguito un assessment delle risorse riassegnate e una proiezione delle conseguenze di tali cambiamenti sulla pianificazione} \\
\hline
\caption{Tabella 2.2.1: Rischi umani\label{}}
\end{longtable}

\pagebreak
\renewcommand{\arraystretch}{1}
\subsubsection{Rischi tecnologici}
\renewcommand{\arraystretch}{1.5}
\begin{longtable}{| m{12em} | m{6em} | m{4em} | m{16em} |}
\hline \rowcolor{title_row}
\centering\textbf{\color{title_text}Rischio} &
\centering\textbf{\color{title_text}Probabilità di occorrenza} &
\centering\textbf{\color{title_text}Impatto} &
\centering\textbf{\color{title_text}Monitoraggio}
\endhead

\hline
\textbf{Guasti agli strumenti di lavoro personali} (es. guasti al pc od alla connessione alla rete personale) & Bassa & Grave &
Spetta ai membri del gruppo segnalare l'impossibilità di lavorare per via di eventuali guasti agli strumenti di lavoro personali \\

\hline \rowcolor{Gray}
\textbf{Piano di contingenza} &
\multicolumn{3}{ m{\combinedlength} |}{
Se il periodo di inattività non può essere recuperato, il membro del gruppo che dovesse avere problemi sarà tenuto a procurarsi degli strumenti (computer, connessione ad internet) di fortuna oppure ad utilizzare quelli dei laboratori dell'università}\\

\hline
\textbf{Guasti ai servizi usati a supporto} (es. GitHub) & Bassa & Grave &
I membri del team che dovessero notare disservizi degli strumenti usati come GitHub o nTask sono tenuti a segnalarlo al responsabile \\

\hline \rowcolor{Gray}
\textbf{Piano di contingenza} &
\multicolumn{3}{ m{\combinedlength} |}{
I membri del gruppo sono tenuti ad effettuare almeno un backup della repository remota al giorno. In caso di disservizi duraturi, saranno utilizzati servizi alternativi tra quelli considerati	durante la scelta degli strumenti da utilizzare}\\
\hline
\caption{Tabella 2.2.1: Rischi tecnologici\label{}}
\end{longtable}
\pagebreak
\renewcommand{\arraystretch}{1}
	\subsubsection{Rischi di progetto}
\renewcommand{\arraystretch}{1.5}
\begin{longtable}{| m{12em} | m{6em} | m{4em} | m{16em} |}
\hline \rowcolor{title_row}
\centering\textbf{\color{title_text}Rischio} &
\centering\textbf{\color{title_text}Probabilità di occorrenza} &
\centering\textbf{\color{title_text}Impatto} &
\centering\textbf{\color{title_text}Monitoraggio}
\endhead

\hline
\textbf{Instabilità dei requisiti} & Media & Grave &
Il responsabile ha il \gl{compito} di monitorare i requisiti \\

\hline \rowcolor{Gray}
\textbf{Piano di contingenza} &
\multicolumn{3}{ m{\combinedlength} |}{
Dell'instabilità è aspettata e sintomo del raffinamento dei requisiti.	Se dovesse essere dovuta a requisiti volatili poiché non abbastanza specifici o perché in costante mutamento, il responsabile dovrà contattare il \gl{Proponente} per accordare un incontro per risolvere il problema} \\

\hline
\textbf{Sottostima delle risorse necessarie} & Media & Media &
Il responsabile deve accertarsi che il team sia \gl{in-schedule}. I membri del gruppo devono avvisare il responsabile se la probabilità di sforare con le tempistiche dovesse aumentare \\

\hline \rowcolor{Gray}
\textbf{Piano di contingenza} &
\multicolumn{3}{ m{\combinedlength} |}{
Il responsabile è tenuto in tal caso a determinare se una diversa pianificazione può risolvere il problema o se dei requisiti (di Bassa priorità) debbano rimanere insoddisfatti}\\

\hline
\textbf{Errori di gestione del progetto} & Media & Media &
Il responsabile è tenuto ad effettuare l'attività di monitoraggio \\

\hline \rowcolor{Gray}
\textbf{Piano di contingenza} &
\multicolumn{3}{ m{\combinedlength} |}{
Il responsabile è tenuto a risolvere il problema assicurandosi che le politiche di gestione del progetto siano rispettate ed eventualmente modificandole}\\
\hline
\caption{Tabella 2.2.1: Rischi di progetto\label{}}
\end{longtable}
\pagebreak
\renewcommand{\arraystretch}{1}
