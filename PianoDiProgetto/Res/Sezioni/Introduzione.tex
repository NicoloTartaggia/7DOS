\newsection{Introduzione}
\subsection{Scopo del documento}
Questo documento ha lo scopo di presentare la \gl{pianificazione} del gruppo 7DOS per lo sviluppo del \gl{progetto} "Grafana \& Bayes" presentato da Zucchetti. In questo documento sarà presente anche un'analisi dei rischi e dei costi dello sviluppo del \gl{capitolato} scelto.
In particolare, il documento conterrà:
\begin{itemize}
	\item Un'analisi dei rischi relativi al progetto;
	\item Una breve descrizione del modello di sviluppo scelto per il progetto;
	\item Una dettagliata pianificazione dei tempi delle \gl{attività} da svolgere;
	\item Una \gl{stima} preventiva dell'utilizzo delle risorse a disposizione.
\end{itemize}

\subsection{Scopo del prodotto}
Il prodotto da realizzare consiste in un \gl{plug-in} per il \gl{software} di monitoraggio \gl{Grafana}, da sviluppare in linguaggio \gl{JavaScript}. Il prodotto dovrà svolgere almeno le seguenti funzioni:
\begin{itemize}
	\item{Leggere la definizione di una \gl{rete Bayesiana}, memorizzata in formato \gl{JSON};}
	\item{Associare dei nodi della rete Bayesiana ad un flusso di dati presente nel sistema di Grafana;}
	\item{Ricalcolare i valori delle probabilità della rete secondo regole temporali prestabilite;}
	\item{Derivare nuovi dati dai nodi della rete non collegati al flusso di dati, e fornirli al sistema di Grafana;}
	\item{Visualizzare i dati mediante il sistema di creazione di grafici e \gl{dashboard} a disposizione.}
\end{itemize}

\subsection{Glossario}
Per rendere la lettura del documento più semplice, chiara e comprensibile viene allegato il \emph{Glossario v4.0.0} nel quale sono contenute le definizioni dei termini tecnici, dei vocaboli ambigui, degli acronimi e delle abbreviazioni. La presenza di un termine all'interno del Glossario è segnalata con una "g" posta come pedice (esempio: $Glossario_{g}$).
\subsection{Maturità del documento}
Il presente documento sarà soggetto ad incrementi futuri. Per questo motivo, non si pone l'obiettivo di risultare completo.
Tutto ciò che riguarda la pianificazione degli incrementi, può essere trovato nel \emph{\gl{Piano di Progetto v4.0.0}} in §4.  
\subsection{Riferimenti}

\subsubsection{Normativi}
\begin{itemize}
	\item \textbf{Verbali:} \emph{Verbale del 2018-12-04, del 2018-12-11, del 2019-01-02, del 2019-01-05, del 2019-03-18 e del 2019-04-04};
	\item \textbf{Norme di Progetto:} \emph{Norme di Progetto v4.0.0}.
\end{itemize}
\subsubsection{Informativi}
\begin{itemize}	
	\item \textbf{Capitolato C3:} G\&B: monitoraggio intelligente di processi \gl{DevOps}
 \\ \url{https://www.math.unipd.it/~tullio/IS-1/2018/Progetto/C3.pdf};
 	\item \textbf{ISO/IEC 12207:} \url{https://www.math.unipd.it/~tullio/IS-1/2009/Approfondimenti/ISO_12207-1995.pdf};
 	\item \textbf{ISO/IEC 25010:} \url{https://iso25000.com/index.php/en/iso-25000-standards/iso-25010};
 	\item \textbf{Slide del corso "Ingegneria del Software"} - \gl{Ciclo di vita del software}
 \\ \url{https://www.math.unipd.it/~tullio/IS-1/2018/Dispense/L05.pdf};
	\item \textbf{Slide del corso "Ingegneria del Software"} - Gestione di progetto
 \\ \url{https://www.math.unipd.it/~tullio/IS-1/2018/Dispense/L06.pdf};
 	\item \textbf{Slide del corso "Ingegneria del Software"} - Regole del progetto didattico
 \\ \url{https://www.math.unipd.it/~tullio/IS-1/2018/Dispense/P01.pdf}.
 
\end{itemize}

\subsection{Scadenze}
Il gruppo 7DOS ha deciso di rispettare le seguenti scadenze, su cui si basa la pianificazione per lo svolgimento del progetto:
\begin{itemize}
	\item \textbf{Revisione dei \gl{Requisiti}:} 21-01-2019;
	\item \textbf{Revisione di Progettazione:} 15-03-2019;
	\item \textbf{Revisione di Qualifica:} 19-04-2019;
	\item \textbf{Revisione di Accettazione:} 17-05-2019.
\end{itemize}

\pagebreak
