\newsection{Eventi di rischio}
In questa sezione viene riportata una tabella che riassume gli eventi di rischio che si sono verificati.
\vspace{1cm}
\renewcommand{\arraystretch}{1.5}
\begin{longtable}{| L{7em} | L{15em} | L{16em} |}
\hline
\rowcolor{title_row}
\textbf{\color{title_text}{Periodo}}  & \textbf{\color{title_text}{Evento}} & \textbf{\color{title_text}{Reazione}} \\
\endhead
\hline
Analisi dei Requisiti & Difficoltà iniziali nell'organizzare e nello scrivere i documenti in linguaggio \LaTeX. & Tutti i membri del team hanno provveduto ad auto-formarsi con risorse on-line riguardo il linguaggio \LaTeX. \\
\hline

Analisi dei Requisiti & Durante questi periodi due membri del gruppo hanno dovuto assentarsi a causa di impegni personali. Periodi:
\begin{itemize}
	\item dal 23/12 al 27/12/2018;
	\item dal 28/12/2018 al 03/01/2019.
\end{itemize} & Il \emph{Responsabile} ha deciso di ridistribuire il carico di lavoro in maniera omogenea, in modo tale che tutti gli altri membri del team riuscissero a far fronte a queste assenze. \\
\hline

Analisi dei Requisiti & Problema hardware al notebook di un membro del gruppo. &  Il \emph{Responsabile} ha permesso di lavorare da remoto restando però in comunicazione attraverso i canali Discord.\\
\hline

Progettazione architetturale & Durante questi periodi tre membri del gruppo hanno dovuto assentarsi per malattia. Periodi:\begin{itemize}
    \item dal 13/02 al 17/02/2019;
    \item dal 20/02 al 23/02/2019;
    \item dal 21/02 al 26/02/2019.
\end{itemize} & Abbiamo ridistribuito il carico di lavoro sui membri del gruppo operativi.
Il peso dell'assenza di tre membri del gruppo ha comportato un adattamento del PoC, che copre un numero minore di requisiti rispetto a quanto pianificato inizialmente. \\
\hline

Progettazione architetturale & Nessun membro del team aveva mai utilizzato le tecnologie richieste per lo sviluppo del prodotto, questo ha reso la loro integrazione non banale. & Abbiamo cercato il più possibile di formarci a vicenda, in modo che i più veloci a comprendere una tecnologia la spiegassero agli altri membri del team. \\
\hline

Progettazione architetturale & L'inesperienza del gruppo ha portato a sopravvalutare la documentazione riguardante i plug-in presente su Grafana. & Il \emph{Responsabile} ha deciso di assegnare due membri del gruppo alla ricerca di materiale utile e documentazione aggiuntiva riguardante la realizzazione del plug-in. \\
\hline

Progettazione di dettaglio e codifica & Durante questo periodo un membro del gruppo ha dovuto assentarsi a causa di impegni personali. Periodo:\begin{itemize}
	\item dal 24/03 al 28/03/2019.
\end{itemize} & Il \emph{Responsabile} ha deciso di ridistribuire il carico di lavoro in maniera omogenea, in modo tale che tutti gli altri membri del team riuscissero a far fronte a questa assenza. \\
\hline

Progettazione di dettaglio e codifica & Problema hardware al pc di uno dei membri del gruppo, che lo ha costretto a subire un rallentamento per due giorni. & Il \emph{Responsabile} ha deciso di designare due membri del team per supportare il suo lavoro, in modo da evitare un rallentamento generale. \\
\hline

Progettazione di dettaglio e codifica & Durante questo periodo di tempo la piattaforma \emph{TravisCI} non era funzionante. Periodo:\begin{itemize}
	\item dal 27/03 al 29/03/2019.
\end{itemize} & Il \emph{Responsabile} ha deciso di assegnare alcuni membri del team che stavano lavorando sul testing del codice ad altre attività, durante questo periodo di tempo. \\
\hline

Progettazione di dettaglio e codifica & Dal giorno 27/03/2019 le API di \emph{Gmail} necessarie per notificare l'invio e la ricezione di email attraverso il nostro canale Discord sono state disattivate. & Il \emph{Responsabile} ha deciso usare l'inoltro automatico delle email in arrivo sull'indirizzo del gruppo ad ogni membro. \\
\hline
\caption{Tabella 8.1: Eventi di rischio verificatesi\label{}}
\end{longtable}
\renewcommand{\arraystretch}{1}
\newpage
