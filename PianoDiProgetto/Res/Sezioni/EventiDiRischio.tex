\newsection{Eventi di rischio}
In questa sezione viene riportata una tabella che riassume gli eventi di rischio che si sono verificati.
\\
\renewcommand{\arraystretch}{1.5}
\begin{longtable}{| m{7em} | m{15em} | m{16em} |}
\hline
\rowcolor{title_row}
\textbf{\color{title_text}{Periodo}}  & \textbf{\color{title_text}{Evento}} & \textbf{\color{title_text}{Reazione}} \\
\endhead
\hline
Analisi dei Requisiti & Difficoltà iniziali nell'organizzare e nello scrivere i documenti in linguaggio \LaTeX. & Tutti i membri del team hanno provveduto ad auto-formarsi con risorse on-line riguardo il linguaggio \LaTeX. \\
\hline
Progettazione architetturale & Durante questo periodo 3 membri del gruppo si sono ammalati. Periodi:\begin{itemize}
    \item dal 13/02 al 17/02/2019;
    \item dal 20/02 al 23/02/2019;
    \item dal 21/02 al 26/02/2019.
\end{itemize} & Abbiamo ridistribuito il carico di lavoro sui membri del gruppo operativi.
Il peso dell'assenza, seppur momentanea, di 3 membri del gruppo ha comportato un adattamento del PoC che copre un numero minore di requisiti rispetto a quanto pianificato inizialmente. \\
\hline
Progettazione architetturale & Nessun membro del team aveva mai utilizzato le tecnologie richieste per lo sviluppo del prodotto, questo ha reso la loro integrazione non banale. & Abbiamo cercato il più possibile di formarci a vicenda, in modo che i più veloci a comprendere una tecnologia la spiegassero agli altri membri del team; \\
\hline
Progettazione architetturale & L'inesperienza del gruppo ha portato a sopravvalutare la documentazione riguardante i plugin presente su Grafana. & Il \emph{Responsabile} ha deciso di assegnare 2 membri del gruppo alla ricerca di materiale utile e documentazione aggiuntiva riguardante la realizzazione del plugin. \\
\hline
\caption{Tabella 8.1: Eventi di rischio verificatesi\label{}}
\end{longtable}
\renewcommand{\arraystretch}{1}
\newpage
