\newsection{Preventivo}
In questa sezione viene riportata una tabella che riassume i costi preventivati e viene proposto il costo totale del progetto.
\\
\renewcommand{\arraystretch}{1.5}
\begin{table}[H]
\begin{center}
\begin{tabular}{|c|c|c|}
\hline
\rowcolor{title_row}
\textbf{\color{title_text}{Periodo}}  & \textbf{\color{title_text}{Preventivo \euro}} & \textbf{\color{title_text}{Consuntivo \euro}} \\ \hline
Analisi dei requisiti   & 3.635 & 3.725 \\ \hline
Consolidamento dei requisiti   & 1.130 & - \\ \hline
Progettazione architetturale    & 4.007 & - \\ \hline
Progettazione di dettaglio e codifica    & 6.580 & - \\ \hline
Verifica e validazione    & 2.450 & - \\ \hline
Totale   & 3.635 & - \\ \hline
Rendicontato   & 13.037 & - \\ \hline
\end{tabular}
\caption{Tabella 8.1: Preventivo dei costi e costi effettivi\label{}}
\end{center}
\end{table}
\renewcommand{\arraystretch}{1}

Il totale rendicontato viene arrotondato a 13.500 \euro{}  per ottenere il costo totale preventivato.  
Il margine di differenza tra i due costi sarà utile, nel caso si verifichi un evento di rischio, a aumentare
il numero di ore di lavoro senza impattare sul costo proposto.

\pagebreak
