\subsection{Analisi}

Il periodo di analisi comincia il 15-11-2018 e si conclude il 14-01-2019. L'inizio coincide con la data di formazione del gruppo e l'avvio dei primi lavori. La conclusione coincide con la scadenza scelta per presentare la documentazione d'ingresso al progetto.

\subsubsection{Incrementi}
Durante il periodo sopra descritto vengono effettuati 6 incrementi, ognuno per un'attività principale.\\
Le attività principali sono:
\begin{itemize}

	\item \textbf{Studio di Fattibilità:} questa attività consiste nell'analisi dei vari capitolati proposti ed è importante per scegliere con attenzione il capitolato da svolgere. Viene redatto il documento di supporto \textit{Studio di fattibilità} contenente l'analisi effettuata per ogni capitolato. Questa attività va obbligatoriamente svolta prima dell'Analisi dei requisiti in quanto bisogna essere certi del capitolato che si intende svolgere;

	\item \textbf{Norme di progetto:} in questa prima attività l'Amministratore stabilisce tutte le norme che i membri del gruppo 7dos devono rispettare fino alla conclusione del progetto. Viene redatto il documento di supporto \textit{Norme di progetto} che contiene tutte le norme stabilite. Questa attività è di massima importanza considerando che le Norme di progetto stabiliscono anche le norme e gli strumenti utilizzati per la stesura dei documenti;

	\item \textbf{Analisi dei Requisiti:} partendo dalla bozza di analisi ad alto livello redatta durante lo \textit{Studio di Fattibilità} si genera un'analisi approfondita. Durante questa analisi si ricavano e analizzano tutti i requisiti del capitolato scelto e si riportano nel documento \textit{Analisi dei requisiti};

	\item \textbf{Piano di Qualifica:} in questa attività l'Analista insieme al Responsabile di Progetto individua i metodi per garantire la qualità di prodotto. Una volta individuati vengono redatti all'interno del documento \textit{Piano di qualifica};

	\item \textbf{Piano di Progetto:} il Responsabile di progetto, partendo dalle date ufficiali e dalle relative scadenze, redige il Piano di Progetto così da organizzare le attività del gruppo. Si analizzano anche i rischi nei quali il gruppo può incombere e le relative soluzioni. Si suddividono anche le risorse disponili per l'intera durata del progetto.

	\item \textbf{Glossario:} in questa attività si individuano tutti i termini considerati poco chiari o ambigui e li si aggiungono nel documento contenente il \textit{Glossario}.

\end{itemize}

\subsubsection{Analisi - Gantt delle attività}

\subsubsection{Analisi - Ripartizione ore}