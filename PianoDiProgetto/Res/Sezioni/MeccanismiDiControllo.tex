\newsection{Meccanismi di controllo e di rendicontazione}
\subsection{Meccanismi di controllo}
Il sistema di ticketing adottato, descritto nel documento \emph{Norme di Progetto v1.0.0} permette di suddividere e controllare le varie attività. In particolare, è possibile visualizzare:
\begin{itemize}
\item{\textbf{Calendario attività}}: la data di fine delle varie attività è indicata in un calendario e gestita automaticamente dal sistema di ticketing;  
\item{\textbf{Dettaglio attività}}: ogni attività include diverse informazioni, quali personale incaricato, stato dell'attività e data di fine prevista. 
\end{itemize}
\subsubsection{Controllo dei ritardi}
Ogni attività, assegnata a uno o più componenti del gruppo, deve essere svolta entro una data prestabilita. In questo modo è possibile pianificare in modo chiaro ed efficace l'insieme delle varie attività. Il mancato completamento entro la data prevista comporta un ritardo, il quale viene annotato (decidere dove?) e rendicontato.

\subsubsection{Controllo delle fasi di processo}
//va inserito il grafico delle attività basato sul PDCA

\subsubsection{Controllo delle metriche di progetto}

\subsection{Meccanismi di rendicontazione}
Il sistema di ticketing adottato, descritto nel documento \emph{Norme di Progetto v1.0.0}, permette ad ogni membro di rendicontare le ore spese per una specifica attività. In particolare, è possibile visualizzare:
\begin{itemize}
\item{Ore di lavoro complessive per un attività;}
\item{Ore di lavoro complessive per un ruolo.}
\end{itemize}
\pagebreak