\newsection{Informazioni generali}
\subsection{Informazioni incontro}
% - INSERT INFORMAZIONI INCONTRO
\begin{itemize}
	\item { \textbf{Luogo:} Torre Archimede;  }
	\item { \textbf{Data:} 11 Dicembre 2018; }
	\item { \textbf{Ora:} 14.30; }
	\item { \textbf{Partecipanti del gruppo:} gruppo al completo; }
	\item { \textbf{Partecipanti esterni:} nessuno. }
\end{itemize}


\subsection{Argomenti affrontati}
Riunione durante la quale sono avvenute le seguente attività:
\begin{itemize}
	\item{Resoconto sullo stato di avanzamento dei documenti Piano di Qualifica e Piano di Progetto;}
	\item{Approvazione documento Studio di Fattibilità;}
	\item{Suddivisione ruoli per il documento Analisi dei Requisiti;}
	\item{Discussione sull'utilizzo di un \emph{work flow} da seguire per il \gl{versionamento};}
	\item{Organizzazione incontro con il \gl{proponente} \emph{Zucchetti Software srl} per chiarire dubbi sul capitolato. }
\end{itemize}

\subsubsection{Piano di Qualifica e Piano di Progetto}
Il gruppo ha visionato lo sviluppo dei documenti Piano di Qualifica e Piano di Progetto. Sono state definite e concordate le modifiche da apportare e sono stati definiti i passi successivi per il completamento degli stessi. Inoltre, è stata fissata una \gl{milestone} per il giorno 18 Dicembre 2018 per la redazione del Piano di Progetto fino alla sezione Preventivo e del Piano di Qualifica fino alla sezione Metriche, con una prima bozza dei casi d'uso e \gl{requisiti} ad alto livello.

\subsubsection{Approvazione documento Studio di Fattibilità}
Il documento, dopo il termine della verifica, è stato approvato dal \emph{Responsabile}.

\newpage

\subsubsection{Organizzazione}
\textbf{Ruoli} per il documento Analisi dei Requisiti:
\begin{itemize}
\item{\textbf{Redattori}: Giacomo Barzon, Lorenzo Busin, Giovanni Sorice, Andrea Trevisin;}
\item{\textbf{Verificatori}: Marco Costantino, Nicolò Tartaggia;}
\item{\textbf{Responsabile}: Michele Roverato.}
\end{itemize}

\subsubsection{Work flow}
Il gruppo ha deciso che, per ora, non verrà utilizzato nessun tipo di \gl{workflow} per il versionamento. La motivazione deriva dal fatto che, per le attività da svolgere, è sufficiente l'utilizzo del \emph{branch master} come ramo in cui effettuare i commit. In una fase successiva, quando inizierà lo sviluppo del software richiesto, il gruppo discuterà nuovamente se utilizzare questa opzione.

\subsubsection{Incontro con \emph{Zucchetti Software srl}}
La data stabilita è il 12 Dicembre 2018.