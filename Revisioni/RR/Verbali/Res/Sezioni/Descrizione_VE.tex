\newsection{Informazioni generali}
\subsection{Informazioni incontro}
% - INSERT INFORMAZIONI INCONTRO
\begin{itemize}
	\item { \textbf{Luogo:} sede Zucchetti Software srl, via Cittadella 7, Padova};
	\item { \textbf{Data:} 12 Dicembre 2018; }
	\item { \textbf{Ora:} 9.30; }
	\item { \textbf{Partecipanti del gruppo}:}
	{	\begin{itemize}
		 	\item Nicolò Tartaggia; 
		 	\item Giovanni Sorice;
		 	\item Andrea Trevisin.   		  
	 	\end{itemize} 
 	}
	\item { \textbf{Partecipanti esterni}:
		 \begin{itemize}
		 	\item  dott. Gregorio Piccoli.
		 \end{itemize}
		 }
\end{itemize}


\subsection{Domande e risposte}
Vengono riportate di seguito le domande chieste durante questo incontro con le relative risposte: 
\begin{itemize}
\item\textbf{Quale standard è opportuno usare per il \gl{progetto}? Avete delle preferenze?}\\
\emph{Per quanto riguarda gli standard, non vi sono restrizioni.  \gl{Grafana} rende disponibili i suoi standard interni, i quali possono essere utilizzati come base d'appoggio per lo sviluppo del \gl{plug-in}. In generale, la scelta di utilizzare uno standard in particolare è a completa discrezione del team;}

\item\textbf{In termini di sicurezza del prodotto, lo sviluppo del plug-in deve trattare anche questa parte?}\\
\emph{La sicurezza è un argomento importante senza ombra di dubbio. Tuttavia il tema centrale del progetto è lo sviluppo del plug-in, come spiegato nelle specifiche del \gl{capitolato}, le quali non menzionano lo sviluppo di una sezione relativa alla sicurezza. Inoltre va considerato il fatto che ognuno di voi ha un monte ore da rispettare e occuparsi di sicurezza comporta una dispendio di risorse importante. Di conseguenza, dentro questi termini, non è necessario occuparsi di essa. Ciò non toglie che il team possa comunque rendere il prodotto sicuro;}

\item\textbf{Per quanto riguarda l'argomento casi d'uso relativi al prodotto da realizzare, ci può fornire un esempio?}\\
\emph{Questo plug-in richiede due step: uno step di specifica, che riguarda il produttore della \gl{rete Bayesiana}. Per semplificare la cosa si può utilizzare un file \gl{JSON}. Avete un link di esempio di file JSON che configura la rete sul capitolato d'appalto.\\
Dopo la produzione della rete, occorrerà inserirla all'interno del plug-in. Questo è il secondo step. Sarà compito vostro la realizzazione del plug-in che permette all'utente di interfacciarsi con esso. }
\item\textbf{Quali categorie di utenti possono utilizzare il plug-in?} \\
\emph{Il progetto prevede lo sviluppo di una rete Bayesiana applicata al plug-in per Grafana che riesca a rendere "più intelligente" il monitoraggio del flusso di dati. Una rete Bayesiana può e dovrebbe funzionare su diversi tipi di applicazioni, quindi non solamente con Grafana. Di conseguenza, le categorie di utenti sono potenzialmente molteplici;}

\item\textbf{Avete un server di prova dal quale prendere i dati per testare il plug-in?}\\
\emph{C'è questa possibilità. Considerate che per effettuare dei test potete monitorare i dati provenienti, per esempio, dalla CPU della vostra macchina mentre sono in esecuzione applicazioni differenti;}

\item\textbf{Come deve essere visualizzato il plug-in?}\\
\emph{La rete Bayesiana produce risultati a partire dall'input fornito. Ciò che interessa particolarmente è il funzionamento della rete. La modalità di visualizzazione è una scelta del gruppo che non prevede specifiche. Grafana mette a disposizione diversi strumenti per la realizzazione dell'interfaccia del plug-in, come grafici, indici ecc;}

\item\textbf{Cosa curare di più riguardo il progetto?}\\
\emph{Non sottovalutate ciò che non conoscete. Molte volte, gruppi di lavoro come voi, utilizzano diverse tecnologie senza documentarsi sufficientemente. È preferibile utilizzare pochi elementi e sfruttare in modo efficace le loro potenzialità. \\
Siete un gruppo e, come tale, dovete lavorare insieme. Le ore a vostra disposizione richiedono che ognuno dia il proprio contributo per lo sviluppo del prodotto.\\
Prestate attenzione alle librerie che trovate in rete. Esse possono esservi d'aiuto, tuttavia, prima di utilizzarle, verificate la loro solidità.}
\end{itemize}
