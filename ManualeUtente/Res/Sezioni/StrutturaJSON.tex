\pagebreak
\newsection{Struttura del JSON}
I file JSON che contengono la definizione di una rete Bayesiana devono essere strutturati strutturati nel seguente modo:
\begin{itemize}
	\item \textbf{nodes}: elenco dei nodi della rete;
	\begin{itemize}
		\item \textbf{name}: nome del nodo;
		\item \textbf{values}: elenco degli stati che il nodo può assumere:
		\begin{itemize}
			\item name: nome dello stato;
			\item type: tipo dello stato. Al momento ne sono state implementate 3 tipologie: \textit{range}, \textit{string} e \textit{boolean};
			\item value: valore dello stato;
			\item rangeMin/rangeMax: da utilizzare per al posto di \textit{value} nel caso in cui si utilizzino valori di tipo \textit{range}.
		\end{itemize}
		\item \textbf{parents}: elenco dei \gl{nodi padri};
		\item \textbf{cpt}:  tabella delle probabilità condizionali associata al nodo.
	\end{itemize}
\end{itemize}
Per maggiori informazioni sulle regole sintattiche dei file JSON, consultare la documentazione W3C al seguente link:
\\[0.2cm]
\hspace*{10mm}
\url{https://www.w3schools.com/js/js_json_syntax.asp}
\Spazio
Ecco un esempio di rete con due nodi in formato JSON:
\begin{lstlisting}
{
	"nodes":[
		{
		"name":"Node1",
		"values":[
			{
				"name":"Example of string value",
				"type":"string",
				"value":"Value1"
			},
			{
				"name":"Another example of string value",
				"type":"string",
				"value":"Value2"
			}
		],
		"parents":["Node2"],
		"cpt":[[0.2,0.8],[0.5,0.5],[0.4,0.6]]
		},
		{
		"name":"Node2",
		"values":[
			{
				"name":"Low Range",
				"type":"range",
				"rangeMin":0,
				"rangeMax":10
			},
			{
				"name":"Normal Range",
				"type":"range",
				"rangeMin":11,
				"rangeMax":80
			},
			{
				"name":"Alert Range",
				"type":"range",
				"rangeMin":81,
				"rangeMax":100
			}
		],
		"parents":[],
		"cpt":[[0.6,0.2,0.2]]
		}
	]
}
\end{lstlisting}


