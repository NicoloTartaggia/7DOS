\appendix
\addcontentsline{toc}{section}{Appendice}
%\section*{Appendice}
\newsection{Glossario}
\subsection*{D}

\subsubsection*{Datasource}
\label{sec:datasource} 
Una datasource è una sorgente di dati in Grafana. Solitamente si tratta di un database. Grafana supporta nativamente:
	\begin{itemize}
	\item CloudWatch;
	\item Elasticsearch;
	\item Graphite;
	\item InfluxDB;
	\item Microsoft SQL;
	\item MySql;
	\item OpenTSDB;
	\item PostgreSQL;
	\item Prometheus;
	\item Stackdriver.
	\end{itemize}

\subsection*{I}
\subsubsection*{Issue}
\label{sec:issue}
Termine inglese che significa "problema, questione". In una repository GitHub, una \textit{issue} viene utilizzata per segnalare bug, miglioramenti o fare richieste agli sviluppatori.

\subsection*{J}
	\subsubsection*{JavaScript}
	\label{sec:javascript}
	JavaScript è un linguaggio di programmazione, comunemente utilizzato in ambito web.
	
\subsection*{R}
\subsubsection*{Root folder}
\label{sec:radice}
Anche nota come "radice" (o "cartella radice"), è la cartella principale in cui è contenuto un prodotto documentale o software. 