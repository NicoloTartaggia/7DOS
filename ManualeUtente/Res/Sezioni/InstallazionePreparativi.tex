\newsection{Installazione e preparativi}
\subsection{Requisiti}
Per poter utilizzare il plug-in e tutto ciò che permette la sua esecuzione vanno soddisfatti i seguenti requisiti minimi:
\begin{itemize}
	\item{Installazione ed accesso a Grafana;}
	\item{Installazione del plug-in;}
	\item{Connessione di uno o più datasource a Grafana, su cui si abbiano permessi di scrittura.}
	\item{Altro?} 
\end{itemize}
\subsection{Browser compatibili}
Di seguito vengono riportate le versioni minime dei browser sui quali è garantito il funzionamento del nostro plug-in:
\begin{itemize}
	\item{Google Chrome v.73}
	\item{Mozilla Firefox v.66}
	\item{Safari v.}
	\item{Microsoft Internet Explorer v.11}
	\item{Microsoft Edge v.41} \\
\end{itemize}

Affinché tutte le funzionalità offerte dal plug-in vengano eseguite correttamente, è necessario che \gl{JavaScript} sia abilitato.
\subsection{Installazione}
Per installare il plug-in all'interno di Grafana è necessario scaricare l'intero progetto dal seguente link: \\[0.2cm]
\hspace*{10mm}\url{https://github.com/NicoloTartaggia/7DOS-plugin}\\[0.2cm]
ed inserirlo all'interno della cartella  del folder di Grafana (se non presente il folder precedentemente indicato, il progetto dovrà essere inserito all'interno del folder ~ \texttt{/data/plugins} ~).
Infine, per abilitare l'utilizzo dell'applicativo, servirà abilitarlo nell'apposita \textbf{Config} tab del plug-in.\\
Per ulteriori informazioni consultare il seguente link:\\[0.2cm]
\hspace*{10mm}
\url{https://grafana.com/docs/plugins/installation/}
\pagebreak
