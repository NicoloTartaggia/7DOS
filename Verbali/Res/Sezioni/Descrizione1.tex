\newsection{Informazioni generali}
\subsection{Informazioni incontro}
% - INSERT INFORMAZIONI INCONTRO
\begin{itemize}
	\item { \textbf{Luogo:} Torre Archimede};
	\item { \textbf{Data:} 18 Marzo 2019};
	\item { \textbf{Ora di inizio:} 10.30};
	\item { \textbf{Ora di fine:} 12.00};
	\item { \textbf{Partecipanti del gruppo:} gruppo al completo};
	\item { \textbf{Partecipanti esterni:} nessuno}.
\end{itemize}


\subsection{Agenda}
\begin{itemize}
	\item {Modifiche ai documenti.}
\end{itemize}

\subsection{Resoconto}
\begin{itemize}
	\item{ \textbf{Documenti:} abbiamo stabilito gli interventi da per le modifiche da apportare ai documenti:
		\begin{itemize}
			\item \textbf{Piano di Progetto}: aggiungere rischi non inseriti nel documento, modificare "suddivisione" con "assegnazione" e spostare i contenuti di §6 nelle Norme;
			\item \textbf{Analisi dei Requisiti}: rimuovere elenco puntato in §2 e renderla più descrittiva, inserire lo scenario principale in tutti gli UC, sistemare i sottocasi di UC1, usare l'ereditarietà per ottenere UC9 e UC10 da UC8 ed aggiungere Raintank come attore secondario;
			\item \textbf{Piano di Qualifica}: spostare nelle Norme i contenuti di §2 e §3, aggiungere in appendice la specifica dei test inserendo nei TU il codice in modo da facilitarne la stesura ed effettuare misurazioni più spesso per poi riportarne i risultati nei grafici per mostrarne l'andamento;
			\item \textbf{Norme di Progetto}: ampliare §2 e §3, spostare norme sul versionamento in Processi di supporto e correggere la definizione del PDCA;
			\item \textbf{Riferimenti}: rimuovere riferimenti ai documenti che causano riferimenti circolari e spostare gli standard nei riferimenti informativi.
		\end{itemize}}	

\end{itemize}

