\section{Informazioni generali}
\subsection{Informazioni incontro}
% - INSERT INFORMAZIONI INCONTRO
\begin{itemize}
	\item { \textbf{Luogo:} Torre Archimede  }
	\item { \textbf{Data:} 7 Dicembre 2017 }
	\item { \textbf{Ora:} 15.00 }
	\item { \textbf{Partecipanti del gruppo:} Gruppo al completo }
	\item { \textbf{Partecipanti esterni:} Nessuno }
\end{itemize}


\subsection{Argomenti affrontati}
Nel corso della riunione sono stati discusse questioni inerenti all'organizzazione del gruppo ed ai requisiti del progetto facendo anche riferimento alla riunione con IKS del 6 dicembre.

\subsubsection{Organizzazione}
\begin{itemize}
	\item { \textbf{Strumenti di comunicazione team:} Si è deciso che lo strumento che il team utilizzerà per le comunicazioni sarà \textit{Slack}. Lo strumento sarà suddiviso in diversi canali:	
		\begin{itemize}
			\item { \textbf{general:} Per le comunicazioni generiche;}
			\item { \textbf{incontri:} Per approvare e avere informazioni sugli incontri;}
			\item { \textbf{todo:} Canale dove verranno riportati i vari punti lasciati in sospeso dei documenti;}
		\end{itemize}
	}
	\item { \textbf{Dropbox:} Si è deciso di creare un account Dropbox così da poter aggiungere materiale di vario genere utile per i membri del team come per esempio guide, tutorial e file di grandi dimensioni;}
	\item { \textbf{Diario:} Nell'intestazione del documento sarà presente il diario, nel quale verranno segnate in modo formale le modifiche apportate al documento da ogni membro del gruppo.
		Per effettuare ogni modifica è necessario riportare autore, ruolo, data e versione del documento;}
	\item { \textbf{Diagrammi UML:} Nel corso della riunione è stato assegnato a Giuseppe Merlino il compito di cercare, per poi discutere nella prossima riunione un software per disegnare questo tipo di diagrammi;}
	\item { \textbf{Stesura Norme di Progetto:} Si è deciso di iniziare a lavorare su questo documento. La suddivisione delle sezioni da scrivere è così: }
	\begin{itemize}
		\item { \textbf{Processi Primari:} Alberto Gallinaro, Lisa Parma ed Elia Montecchio;}
		\item { \textbf{Processi di Supporto} Giuseppe Merlino, Paolo Eccher e Davide Zago;}
		\item { \textbf{Processi Organizzativi:} Francesco Parolini.}
	\end{itemize}
\end{itemize}

\subsubsection{Requisiti del progetto}
Abbiamo discusso ed approvato che il plugin riguardante la mappa topologica deve poter essere ridimensionabile, quindi deve essere possibile effettuare ingrandimento e restringimento. Deve anche essere trattato il caso in cui i dati non possano essere caricati, visualizzando un opportuno messagio di errore.

Per quanto riguarda il plugin per la visualizzazione delle stack trace: esso deve essere realizzato come un elenco numerato crescente di trace che parte dal numero 1. Deve anche qui essere trattato il caso in cui i dati non possano essere caricati, visualizzando un opportuno messagio di errore.
