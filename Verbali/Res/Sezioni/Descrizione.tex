\newsection{Informazioni generali}
\subsection{Informazioni incontro}
% - INSERT INFORMAZIONI INCONTRO
\begin{itemize}
	\item { \textbf{Luogo:} Torre Archimede};
	\item { \textbf{Data:} 27 Novembre 2018};
	\item { \textbf{Ora:} 14.30};
	\item { \textbf{Partecipanti del gruppo:} gruppo al completo};
	\item { \textbf{Partecipanti esterni:} nessuno}.
\end{itemize}


\subsection{Argomenti affrontati}
Riunione durante la quale ognuno ha conosciuto i vari membri del gruppo. Sono state discusse questioni riguardanti l'organizzazione del gruppo, gli strumenti per la comunicazione e i ruoli per l'analisi dei \gl{capitolati} e le Norme di Progetto. Inoltre sono stati decisi nome e logo del gruppo ed è stata presa una decisione comune sulla scelta del capitolato.

\subsubsection{Organizzazione}
\begin{itemize}
	\item { \textbf{Strumenti di comunicazione del team:} lo strumento che il gruppo utilizzerà per la comunicazione sarà Discord;}
	\item{ \textbf{Google:} è stato creato un account Google con indirizzo 7dos.swe@gmail.com a cui tutti gli utenti hanno accesso;}
	\item { \textbf{GitHub:} si è deciso di creare una \gl{repository} su GitHub per il \gl{versionamento};}
	\item { \textbf{Diario:} nell'intestazione del documento sarà presente il diario, nel quale verranno segnate in modo formale le modifiche apportate al documento da ogni membro del gruppo.
		Per effettuare ogni modifica è necessario riportare autore, ruolo, data e versione del documento;}
	\item { \textbf{Ruoli:}  }
	\begin{itemize}
		\item { \textbf{Analisti:} Giovanni Sorice, Marco Costantino, Michele Roverato, Lorenzo Busin;}
		\item { \textbf{Verificatori:} Giacomo Barzon, Andrea Trevisin;}
		\item { \textbf{Responsabile:} Nicolò Tartaggia.}
	\end{itemize}
	\item { \textbf{Nome:} \emph{7DOS}, acronimo per \emph{DevOps Seven}. Il nome deriva dallo scenario in cui il \gl{progetto} si posiziona, \gl{DevOps} appunto, e dal numero di componenti del gruppo.}
\end{itemize}

\subsubsection{Capitolato}
Ogni membro del gruppo prima della riunione ha analizzato ciascun capitolato, traendone aspetti positivi e negativi. Successivamente è stato fatto un confronto collettivo, il quale ha portato alla scelta del capitolato C3, \emph{G\&B: monitoraggio intelligente di processi DevOps}, proposto dall'azienda Zucchetti. Il gruppo, infatti, ha ritenuto interessante l'obiettivo del capitolato e le tecnologie previste.
