\newsection{Informazioni generali}
\subsection{Informazioni incontro}
% - INSERT INFORMAZIONI INCONTRO
\begin{itemize}
	\item { \textbf{Luogo:} Torre Archimede  }
	\item { \textbf{Data:} 11 Dicembre 2018 }
	\item { \textbf{Ora:} 14.30 }
	\item { \textbf{Partecipanti del gruppo:} Gruppo al completo }
	\item { \textbf{Partecipanti esterni:} Nessuno }
\end{itemize}


\subsection{Argomenti affrontati}
Riunione durante la quale sono avvenute le seguente attività:
\begin{itemize}
	\item{Resconto sullo stato di avanzamento dei documenti \emph{Piano di Qualifica} e \emph{Norme di Progetto}; }
	\item{Discussione sull'utilizzo di un \emph{work flow} da seguire per il versionamento;}
	\item{Organizzazione incontro con il proponente \emph{Zucchetti Software srl} per chiarire dubbi sul capitolato. }
\end{itemize}

\subsubsection{Piano di Qualfica e Norme di Progetto}
Il gruppo ha visionato lo sviluppo dei documenti \emph{Piano di Qualifica} e \emph{Norme di Progetto}. Sono state definite e concordate le modifiche da apportare e sono stati definiti i passi successivi per il completamento degli stessi. Inoltre, il 18-12-2018 è stata individuata come data di scadenza per la redazione dei documenti in modo da poter permettere ai verificatori e al responsabile di progetto di svolgere rispettivamente la verifica e l'approvazione prima delle vacanze natalizie.

\subsubsection{Work flow}
Il gruppo ha deciso che, per ora non, verrà utilizzato nessun tipo di \emph{work flow} per il versionamento. La motivazione deriva dal fatto che, per le attività da svolgere, è sufficiente l'utilizzo del \emph{branch master} come ramo in cui effettuare i commit. In una fase successiva, quando inizierà lo sviluppo del software richiesto, il gruppo discuterà nuovamente se utilizzare questa opzione.

\subsubsection{Incontro con \emph{Zucchetti Software srl}}
La data stabilita è mercoledì 12 dicembre.