\newsection{Informazioni generali}
\subsection{Informazioni incontro}
% - INSERT INFORMAZIONI INCONTRO
\begin{itemize}
	\item { \textbf{Luogo:} Torre Archimede};
	\item { \textbf{Data:} 4 Aprile 2019};
	\item { \textbf{Ora di inizio:} 9.30};
	\item { \textbf{Ora di fine:} 12.30};
	\item { \textbf{Partecipanti del gruppo:} gruppo al completo};
	\item { \textbf{Partecipanti esterni:} nessuno}.
\end{itemize}


\subsection{Agenda}
\begin{itemize}
	\item {Stato di avanzamento del progetto;}
	\item {Incontro con il proponente;}
	\item {Stesura consuntivo;}
	\item {Ripianificazione;}
	\item {Presentazione per PB.}
\end{itemize}

\subsection{Resoconto}
\begin{itemize}
	\item {\textbf{Stato di avanzamento:} abbiamo fatto il punto della situazione in merito all'avanzamento del progetto. Tutti i requisiti obbligatori sono stati soddisfatti, circa metà dei requisiti desiderabili anche, tranne per quanto riguarda quelli relativi all'editor grafico; anche la maggior parte dei requisiti opzionali sono stati implementati;}
	\item {\textbf{Incontro con proponente:} in seguito ad accordi presi con il proponente di incontrarci quando il progetto avesse raggiunto un buon livello di funzionalità implementate, dato che non vi era la necessità di ulteriori chiarimenti, abbiamo stabilito l'incontro con Zucchetti per il giorno 2019-04-09;}
	\item {\textbf{Consuntivo:} abbiamo steso i consuntivi riguardanti il periodo di "Progettazione di dettaglio e codifica";}
	\item { \textbf{Ripianificazione}: in conclusione alle varie ripianificazioni eseguite in questa fase di progetto, abbiamo scelto di riportare nel \emph{Piano di Progetto}, in una apposita sezione dedicata, la nostra ultima ripianificazione prima dell'ingresso in RQ;}
	\item { \textbf{Presentazione}: per la presentazione abbiamo deciso di descrivere il tipo di architettura scelto dal gruppo e i diagrammi delle classi, presentando i design pattern utilizzati, il tutto con un buon livello di dettaglio, ma senza essere troppo specifici o troppo generici.}
\end{itemize}

