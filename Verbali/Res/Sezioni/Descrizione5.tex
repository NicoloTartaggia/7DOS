\newsection{Informazioni generali}
\subsection{Informazioni incontro}
% - INSERT INFORMAZIONI INCONTRO
\begin{itemize}
	\item { \textbf{Luogo:} sede Zucchetti Software srl, via Cittadella 7, Padova;}
	\item { \textbf{Data:} 8 Maggio 2019};
	\item { \textbf{Ora di inizio:} 14.30};
	\item { \textbf{Ora di fine:} 15.30};
	\item { \textbf{Partecipanti del gruppo:}	\begin{itemize}
			\item Giovanni Sorice;
			\item Michele Roverato;
			\item Nicolò Tartaggia.
	\end{itemize}}
	\item { \textbf{Partecipanti esterni:} 
	\begin{itemize}
			\item dott. Gregorio Piccoli.
	\end{itemize}}
\end{itemize}


\subsection{Agenda}
\begin{itemize}
	\item {Visione funzionalità aggiunte;}
	\item {Metodologia di sviluppo;}
	\item {Risposte.}
\end{itemize}

\subsection{Resoconto}
\begin{itemize}
	\item \textbf{Visione funzionalità aggiunte:} abbiamo mostrato una demo del nostro prodotto al proponente che ne è rimasto piacevolmente soddisfatto, affermando che tutto ciò che si aspettava siamo riusciti a portarlo a termine. Abbiamo inoltre mostrato le funzionalità aggiunte quali:
	\begin{itemize}
		\item Miglioramento grafica;
		\item Aggiunta visualizzazione grafica della rete Bayesiana;
		\item Aggiunta funzionalità di import/export completa del panel;
		\item Aggiunta sistema di warning;
		\item Aggiunta funzionalità di disconnesione dei nodi.
	\end{itemize}
	Anche di queste nuove funzionalità ne è rimasto soddisfatto, notando l'effettivo ritorno visivo della grafica e della visualizzazione della rete Bayesiana;
	\item {\textbf{Metodologia di sviluppo}:
	il dott. Gregorio Piccoli ci ha fatto ragionare sulla metodologia di sviluppo da noi utilizzata e sui possibili miglioramenti apportabili, mostrandoci anche un esempio aziendale verificatosi all'interno dell'azienda proponente.}
\end{itemize}

