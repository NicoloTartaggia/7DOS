\newsection{Informazioni generali}
\subsection{Informazioni incontro}
% - INSERT INFORMAZIONI INCONTRO
\begin{itemize}
	\item { \textbf{Luogo:} sede Zucchetti Software srl, via Cittadella 7, Padova;}
	\item { \textbf{Data:} 9 Aprile 2019};
	\item { \textbf{Ora di inizio:} 14.30};
	\item { \textbf{Ora di fine:} 15.30};
	\item { \textbf{Partecipanti del gruppo:} gruppo al completo};
	\item { \textbf{Partecipanti esterni:} 
	\begin{itemize}
			\item dott. Gregorio Piccoli.
	\end{itemize}}
\end{itemize}


\subsection{Agenda}
\begin{itemize}
	\item {Demo del progetto;}
	\item {Domande;}
	\item {Risposte.}
\end{itemize}

\subsection{Resoconto}
\begin{itemize}
	\item \textbf{Demo:} abbiamo mostrato una demo del nostro prodotto al proponente che ne è rimasto piacevolmente soddisfatto, affermando che tutto ciò che si aspettava siamo riusciti a portarlo a termine. Abbiamo inoltre mostrato l'architettura del nostro plug-in e le scelte implementative da noi effettuate ed anche in questo caso egli ha espresso un parere positivo. Infine abbiamo discusso di quali altre funzionalità preferirebbe vedere implementate per la prossima revisione;
	\item {\textbf{Domande}:
		\begin{enumerate}
			\item Il nostro plug-in è stato sviluppato per funzionare con la versione di Grafana 5.4.x; Zucchetti ha l'intenzione futura di passare alla nuova versione di Grafana 6 e quindi dobbiamo adottare delle tecniche che garantiscano la retrocompatibilità?
			\item Ci sono altre feature che vorreste che implementassimo per la RA oltre a quelle da noi pianificate?
			\item Per quanto riguarda l'editor grafico che vorremmo implementare, preferireste che sia una funzionalità interna al plug-in oppure un piccolo programma stand-alone?
		\end{enumerate}}
	\item {\textbf{Risposte}:
		\begin{enumerate}
			\item Il proponente ha affermato che Zucchetti non ha intenzione di passare alla nuova versione di Grafana e che la nostra soluzione va bene così. Non vi è dunque la necessità di garantire la retrocompatibilità;
			\item Il proponente ci ha consigliato di implementare una funzionalità per mostrare in forma di grafo la rete Bayesiana; un'altra funzionalità di interesse, anche se di difficile realizzazione, consiste nell'analisi a posteriori dei risultati raccolti per verificare la coerenza delle probabilità di rete, con l'uso possibile di algoritmi di Bloom;
			\item Il proponente ha espresso un parere positivo in merito alla realizzazione di un applicativo indipendente dal plug-in; esso consiste in un editor grafico per la creazione/modifica di una rete Bayesiana che consente di esportarne la definizione in JSON.
		\end{enumerate}}
\end{itemize}

