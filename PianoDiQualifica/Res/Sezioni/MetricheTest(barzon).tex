\newsection{Metriche dei test}
Il processo di verifica, per essere informativo, deve essere quantificabile. Il team ha quindi stabilito delle metriche di natura numerica, in modo da misurare accuratamente le varie caratteristiche di qualità del prodotto. Per ogni metrica sono presenti  spiegazione, motivazione, formula usata per il calcolo e tre intervalli di valori. Questi ultimi sono distinti come segue:
\begin{itemize}
	\item {\textbf{Intervallo limite:} eventuali intervalli di valori in cui la metrica deve essere racchiusa per poter essere considerata come un valore legale che possiede una certa rilevanza;}
	\item {\textbf{Intervallo accettabile:} i valori che vengono considerati oltre la soglia di accettabilità per la metrica;}
	\item {\textbf{Intervallo obiettivo:} i valori che vengono considerati l'obiettivo ottimale da raggiungere.}
\end{itemize}
\pagebreak
