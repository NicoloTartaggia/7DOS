\newsection{Metriche dei test}
\subsection{Metriche per i processi}
Per la valutazione della qualità dei processi il team 7DOS ha deciso di basarsi sullo standard ISO \textbackslash IEC 15504 anche conosciuto come SPICE (Software Process improvement and Capability Determination). \\
Lo standard definise per ogni processo una scala di capability a cinque livelli, più un livello base, chiamato livello 0.
\begin{itemize}
	\item{\textbf{Optimizing process(Level 5)}}:Il processo persegue i principi del miglioramento continuo.
	\item{\textbf{Predictable process(Level 4)}}:Sono stati definiti dei limiti entro cui il processo può operare.
	\item{\textbf{Estabilished process(Level 3)}}:Il processo è basato su standard ben definiti.
	\item{\textbf{Managed process(Level 2)}}:L'esecuzione del processo è pianificata, monitorata ed eventualmente corretta se necessario.
	\item{\textbf{Performed process(Level 1)}}:Il processo è stato implementato e raggiunge gli obbiettivi prefissati.
	\item{\textbf{Incomplete process(Level 0)}}:Il processo non è stato ancora implementato o non raggiunge gli obbiettivi prefissati.
\end{itemize}
La capability di un processo è misurata tramite nove attributi di processo definiti a livello internazionale:
\begin{itemize}
	\item{\textbf{Livello 0}}:
	\item{\textbf{Livello 1}}:
	\begin{itemize}
		\item{\textbf{Process performance}}:Il processo è in grado di produrre in output un prodotto identificabile 	
	\end{itemize}
	\item{\textbf{Livello 2}}:
	\begin{itemize}
		\item{\textbf{Performance management}}:Il processo è in grado di produrre un prodotto coerente con gli obbiettivi fissati.
		\item{\textbf{Work product management}}:Il prodotto del processo possiede dei requisiti ben definiti ed una documentazione esaustiva. Inoltre vengono continuamente effettuate verifiche sul prodotto ed eventuali correzzioni se necessario.	
	\end{itemize}
	\item{\textbf{Livello 3}}:
	\begin{itemize}
		\item{\textbf{Process definition}}:L'esecuzione del processo è basata su standard ben definiti.
		\item{\textbf{Process deployment}}:Sono stati definiti ed assegnati dei ruoli a ciascun membro del team, ogni risorsa necessaria per l'esecuzione del processo è disponibile ed utilizzabile.
	\end{itemize}
	\item{\textbf{Livello 4}}:
	\begin{itemize}
		\item{\textbf{Process measurement}}:Vengono utilizzate delle metriche per garantire che il prodotto del processo rispetti gli standard qualitativi aziendali. 
		\item{\textbf{Process control}}:Il processo è facilmente gestibile, produce risultati facilmente prevedibili, ed opera all'interno di limiti ben definiti.	
	\end{itemize}
	\item{\textbf{Livello 5}}:
	\begin{itemize}
		\item{\textbf{Process innovation}}:Vengono definite delle proposte di miglioramento del processo sulla base dei dati raccolti.
		\item{\textbf{Process Optimization}}:L'impatto di tutti i miglioramenti proposti viene attentamente analizzato in modo tale da assicurarsi che essi apportino degli effettivi benefici al processo.
	\end{itemize}

\end{itemize}
Ciascuno degli attributi elencati qui sopra è valutato secondo una scala a quattro valori N-P-L-F:
\begin{itemize}
	\item{\textbf{Not achieved}}(0\% - 15\%)
	\item{\textbf{Partially achieved}}(>15\% - 50\%)
	\item{\textbf{Largely achieved}}(>50\% - 85\%)
	\item{\textbf{Fully achieved}}(85\% - 100\%)
\end{itemize}
\subsection{Metriche per i prodotti}
\subsubsection{Metriche relative al codice software}
\paragraph{Percentuale di requisiti soddisfatti}
\begin{flushleft}
Fondamentale per comprendere lo stato di avanzamento del progetto.\\
Calcolata attraverso la seguente formula:
$$
	(\frac{n^{\circ}\:requisiti\:soddisfatti}{n^{\circ}\:totale\:di\:requisiti})*100
$$
\textbf{Range Ottimale}:100\%-100\% \\
\textbf{Range Accettabile}:100\%-100\%
\end{flushleft}
\paragraph{Percentuale di test case passati}
\begin{flushleft}
Fondamentale per determinare l'effiacia del prodotto software.\\
Calcolata tramite la seguente formula:
$$
	(\frac{n^{\circ}\:test\:superati}{n^{\circ}\:test\:eseguiti})*100
$$
\textbf{Range Ottimale}:100\%-100\% \\
\textbf{Range Accettabile}:100\%-100\%
\end{flushleft}
\paragraph{Average Response Time}
\begin{flushleft}
Tempo medio richiesto per completare l'esecuzione di una funzionalità richiesta ed esporne il risultato.\\
Calcolato tramite la seguente formula:
$$\frac{\sum\limits_{i=1}^N TFE{\small{i}}}{NFE}$$
Dove
\begin{itemize}
\item{\textbf{TFE${\small{i}}$}}:Tempo trascorso dal momento in cui viene richiesta l'esecuzione dell'i-esima funzionalità fino al momento in cui la funzionalita' viene effettivamente completata ed il risultato viene esposto.
\item{\textbf{NFE}}:Numero di funzionalità eseguite durante una singola esecuzione del prodotto software.
\end{itemize}
\textbf{Range Ottimale}:0.3s-0.5s \\
\textbf{Range Accettabile}:0.5s-1.0s
\end{flushleft}
\paragraph{Average time spent per feature development}
\begin{flushleft}
Tempo medio speso dall'intero team di sviluppo per lo sviluppo di nuove funzionalità.\\
Calcolato attraverso la seguente formula:
	$$\frac{\sum\limits_{i=1}^N TF{\small{i}}}{NF}$$
Dove
\begin{itemize}
	\item{\textbf{TF${\small{i}}$}}:Tempo uomo speso dall'intero team di sviluppo per lo sviluppo dell'iesima funzionalita'.
	\item{\textbf{NF}}:Numero di funzionalita' sviluppate
\end{itemize}
\textbf{Range Ottimale}:3h-4h 30m \\
\textbf{Range Accettabile}:4h 30m-5h 30m
\end{flushleft}
\paragraph{Average time spent per error correction}
\begin{flushleft}
Tempo medio speso dall'intero team di sviluppo per la risoluzione di una singolo errore.\\
Calcolato tramite la seguente formula
	$$A\frac{\sum\limits_{i=1}^N TE{\small{i}}}{NE}$$
Dove
\begin{itemize}
	\item{\textbf{TE${\small{i}}$}}:Tempo uomo speso per la risoluzione dell'i-esimo errore rilevato
	\item{\textbf{NE}}:Numero complessivo di errori rilevati
\end{itemize}
\textbf{Range Ottimale}:45m-1h 30m \\
\textbf{Range Accettabile}:1h 30m-2h 30m
\end{flushleft}
\paragraph{Average time spent per unit testing}
\begin{flushleft}
Tempo medio speso dall'intero team di sviluppo per la scrittura di test corretti e completi per una singola unita'.
Calcolato attraverso la seguente formula
	$$\frac{\sum\limits_{i=1}^N TU{\small{i}}}{NU}$$
Dove
\begin{itemize}
	\item{\textbf{TU${\small{i}}$}}:Tempo uomo speso dall'intero team di sviluppo per la scrittura di test corretti e completi per l'i-esima unita'.
	\item{\textbf{NU}}:Numero unita' testate.
\end{itemize}
\textbf{Range Ottimale}:1h 30m-2h \\
\textbf{Range Accettabile}:2h-2h 30m
\end{flushleft}
\subsubsection{Metriche relative alla documentazione}
\paragraph{Numero di errori grammaticali}
\begin{flushleft}
Tutti i documenti verrano analizzati da un apposito strumento di analisi grammaticale. Per poter essere accettati non potranno avere un numero di errori grammaticali superiore a zero.
\end{flushleft}

\paragraph{Gunning fog index}
\begin{flushleft}
Utilizzato per misurare la facilita' di lettura e di comprensione di un testo. Il numero risultante è un indicatore del numero di anni di educazione formale della quale una persona necessita al fine di leggere il testo con facilità. \\
L'indice di Gunning fog è calcolabile tramite la seguente formula:
$$
	0.4*((\frac{n^{\circ}\:parole}{n^{\circ}\:frasi})+100*(\frac{n^{\circ}\:parole\:complesse}{n^{\circ}\:parole}))
$$
\textbf{Range Ottimale}:<=12 \\
\textbf{Range Accettabile}:>12-15
\end{flushleft}
\paragraph{Indice di Gulpease}
\begin{flushleft}
Utilizzato per misurare la leggibilita' di un testo in lingua italiana.\\
L'indice di Gulpease è calcolabile tramite la seguente formula:
$$
	89+\frac{(numero\:delle\:frasi)-10*(numero\:delle\:lettere)}{numero\:delle\:parole}
$$
I risultati sono compresi tra 0 e 100 dove il valore 0 indica la leggibilità più bassa e 100 indica la leggibilità più alta. In generale risulta che testi con indice:
	\begin{itemize}
		\item{\textbf{Inferiore	a 80}}: Sono difficili da leggere per chi ha la licenza elementare
		\item{\textbf{Inferiore	a 60}}: Sono difficili da leggere per chi ha la licenza media
		\item{\textbf{Inferiore	a 40}}: Sono difficili da leggere per chi ha la licenza superiore
	\end{itemize}
\textbf{Range Ottimale}:60-100 \\
\textbf{Range Accettabile}:40-100
\end{flushleft}
\paragraph{Indice di Flesh}
\begin{flushleft}
Utilizzato per misurare la leggibilita' di un testo in lingua inglese.\\
L'indice di Flesh è calcolabile tramite la seguente formula:
	$$
		206,835-(0,846*S)-(1,015*P)
	$$
Dove
	\begin{itemize}
		\item{\textbf{S}}:Numero di sillabe calcolato su un campione di 100 parole.
		\item{\textbf{P}}:Numero medio di parole per frase.
	\end{itemize}
La leggibilita è considerata come:
	\begin{itemize}
		\item{\textbf{Alta}}:se l'indice e' superiore a 60.
		\item{\textbf{Media}}:se l'indice e' compresa tra 50 e 60.
		\item{\textbf{Bassa}}:se l'indice e' inferiore a 50.
	\end{itemize}
\textbf{Range Ottimale}:60-100 \\
\textbf{Range Accettabile}:50-60
\end{flushleft}
\pagebreak
