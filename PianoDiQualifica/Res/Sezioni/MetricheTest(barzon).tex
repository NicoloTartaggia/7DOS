\newsection{Metriche dei test}
\subsection{Metriche per i processi}
Per la valutazione della qualità dei processi il team 7DOS ha deciso di basarsi sullo standard ISO \textbackslash IEC 15504 anche conosciuto come SPICE (Software Process improvement and Capability Determination). \\
Lo standard definise per ogni processo una scala di capability a cinque livelli, più un livello base, chiamato livello 0.
\begin{itemize}
	\item{\textbf{Optimizing process(Level 5)}}:Il processo persegue i principi del miglioramento continuo.
	\item{\textbf{Predictable process(Level 4)}}:Sono stati definiti dei limiti entro cui il processo può operare.
	\item{\textbf{Estabilished process(Level 3)}}:Il processo è basato su standard ben definiti.
	\item{\textbf{Managed process(Level 2)}}:L'esecuzione del processo è pianificata, monitorata ed eventualmente corretta se necessario.
	\item{\textbf{Performed process(Level 1)}}:Il processo è stato implementato e raggiunge gli obbiettivi prefissati.
	\item{\textbf{Incomplete process(Level 0)}}:Il processo non è stato ancora implementato o non raggiunge gli obbiettivi prefissati.
\end{itemize}
La capability di un processo è misurata tramite nove attributi di processo definiti a livello internazionale:
\begin{itemize}
	\item{\textbf{Livello 0}}:
	\item{\textbf{Livello 1}}:
	\begin{itemize}
		\item{\textbf{Process performance}}:Il processo è in grado di produrre in output un prodotto identificabile 	
	\end{itemize}
	\item{\textbf{Livello 2}}:
	\begin{itemize}
		\item{\textbf{Performance management}}:Il processo è in grado di produrre un prodotto coerente con gli obbiettivi fissati.
		\item{\textbf{Work product management}}:Il prodotto del processo possiede dei requisiti ben definiti ed una documentazione esaustiva. Inoltre vengono continuamente effettuate verifiche sul prodotto ed eventuali correzzioni se necessario.	
	\end{itemize}
	\item{\textbf{Livello 3}}:
	\begin{itemize}
		\item{\textbf{Process definition}}:L'esecuzione del processo è basata su standard ben definiti.
		\item{\textbf{Process deployment}}:Sono stati definiti ed assegnati dei ruoli a ciascun membro del team, ogni risorsa necessaria per l'esecuzione del processo è disponibile ed utilizzabile.
	\end{itemize}
	\item{\textbf{Livello 4}}:
	\begin{itemize}
		\item{\textbf{Process measurement}}:Vengono utilizzate delle metriche per garantire che il prodotto del processo rispetti gli standard qualitativi aziendali. 
		\item{\textbf{Process control}}:Il processo è facilmente gestibile, produce risultati facilmente prevedibili, ed opera all'interno di limiti ben definiti.	
	\end{itemize}
	\item{\textbf{Livello 5}}:
	\begin{itemize}
		\item{\textbf{Process innovation}}:Vengono definite delle proposte di miglioramento del processo sulla base dei dati raccolti.
		\item{\textbf{Process Optimization}}:L'impatto di tutti i miglioramenti proposti viene attentamente analizzato in modo tale da assicurarsi che essi apportino degli effettivi benefici al processo.
	\end{itemize}

\end{itemize}
Ciascuno degli attributi elencati qui sopra è valutato secondo una scala a quattro valori N-P-L-F:
\begin{itemize}
	\item{\textbf{Not achieved}}(0\% - 15\%)
	\item{\textbf{Partially achieved}}(>15\% - 50\%)
	\item{\textbf{Largely achieved}}(>50\% - 85\%)
	\item{\textbf{Fully achieved}}(85\% - 100\%)
\end{itemize}
\subsection{Metriche per i prodotti}
Tutte le metriche elencate di seguito saranno relative alla documentazione. Il team 7DOS ha deciso di rimandare la scelta di metriche relative a prodotti software in quanto poco interessanti per la prima revisione.
\subsubsection{Numero di errori ortografici}
\pagebreak