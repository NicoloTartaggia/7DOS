\newsection{Introduzione}
\subsection{Scopo del documento}
Il presente documento ha lo scopo di esporre dettagliatamente le norme, le metodologie e gli standard che il gruppo 7DOS intende adottare per assicurare che ogni \gl{prodotto}, di natura documentale o applicativa, aderisca ai vincoli di \gl{qualità} stabiliti dal proponente. Per garantire il rispetto di tali vincoli si prevede un continuo \gl{processo} di verifica delle attività svolte dal gruppo, al fine di individuare eventuali problematiche nel minor tempo possibile permettendo immediati interventi risolutivi.
\subsection{Scopo del prodotto}
Il prodotto da realizzare consiste in un \gl{plug-in} per il software di monitoraggio \gl{Grafana}, da sviluppare in linguaggio \gl{JavaScript}. Il prodotto dovrà svolgere almeno le seguenti funzioni:
\begin{itemize}
	\item{Leggere la definizione di una \gl{rete Bayesiana}, memorizzata in formato \gl{JSON};}
	\item{Associare dei nodi della rete Bayesiana ad un flusso di dati presente nel sistema di Grafana;}
	\item{Ricalcolare i valori delle probabilità della rete secondo regole temporali prestabilite;}
	\item{Derivare nuovi dati dai nodi della rete non collegati al flusso di dati, e fornirli al sistema di Grafana;}
	\item{Visualizzare i dati mediante il sistema di creazione di grafici e \gl{dashboard} a disposizione.}
\end{itemize}
Viene previsto un utilizzo del prodotto da parte dell'azienda proponente per il monitoraggio di sistemi gestionali in Cloud; tuttavia, dato l'obiettivo di rendere il prodotto open-source, esso dovrà essere utilizzabile indipendentemente dal particolare sistema che si desidera monitorare.
\subsection{Glossario}
Per rendere la lettura del documento più semplice, chiara e comprensibile viene allegato il \emph{Glossario v1.0.0} nel quale sono contenute le definizioni dei termini tecnici, dei vocaboli ambigui, degli acronimi e delle abbreviazioni. La presenza di un termine all'interno del Glossario è segnalata con una "g" posta come pedice (esempio: $Glossario_{g}$).
\subsection{Maturità del documento}
Il presente documento sarà soggetto ad incrementi. Futuri per questo motivo, non si pone l'obiettivo di risultare completo già in questa fase del progetto.
Tale decisione è dovuta al fatto che sono state trattate le esigenze di attività di progetto più impellenti e ricorrenti.
Tutto ciò che riguarda la pianificazione degli incrementi, può essere trovato nel \emph{Piano di Progetto V1.0.0} all'interno della quarta sezione.  
\subsection{Riferimenti}

\subsubsection{Normativi}
\begin{itemize}
	\item \textbf{Norme di Progetto:} \emph{Norme di Progetto v1.0.0.};
	\item \textbf{Capitolato d'appalto C3}: G\&B monitoraggio intelligente di processi DevOps \\
	\url{https://www.math.unipd.it/~tullio/IS-1/2018/Progetto/C3.pdf};
	\item \textbf{ISO/IEC 12207}:\\ \url{https://www.math.unipd.it/~tullio/IS-1/2009/Approfondimenti/ISO_12207-1995.pdf};
\end{itemize}
\subsubsection{Informativi}
\begin{itemize}
	\item \textbf{\gl{Grafana} Code Styleguide:} \\
	\url{http://docs.grafana.org/plugins/developing/code-styleguide/};
	\item \textbf{\gl{Angular TypeScript} Code Styleguide:} \\
	\url{https://angular.io/guide/styleguide};
	\item \textbf{Verbali:} \emph{Verbale del 2018-12-12};
	\item \textbf{Software Engineering - Ian Sommerville - 10th Edition}.
\end{itemize}
\pagebreak