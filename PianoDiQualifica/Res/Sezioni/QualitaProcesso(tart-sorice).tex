\newsection{Strategia generale per la verifica}
Il Piano di Qualifica qualità prevede che vengano delineati obiettivi da portare a termine seguendo strategia e metriche ben definite.
\subsection{Obiettivi}
La qualità del prodotto richiesta dal proponente è strettamente legata con la qualità dei processi che portano al suo sviluppo. Pertanto, attraverso questa sezione, si descrivono gli obiettivi per il perseguimento di entrambe le parti basandosi sugli standard riportati di seguito. 
\subsection{Qualità di processo}
	É impossibile produrre prodotti caratterizzati da un alto livello di qualità se il proprio way of working non è di qualità. Per poter raggiungere gli obiettivi prefissati è fondamentale che i processi attuati garantiscano un elevato livello di qualità. Il gruppo 7dos, per questo motivo, ha deciso di adottare la normativa ISO/IEC 15504 (chiamata anche SPICE) e di seguire il principio di miglioramento continuo (\gl{PDCA}). 
\subsubsection{ISO/IEC 15504(SPICE)}
Per ogni \gl{processo} lo standard ISO/IEC 15504 definisce una scala di maturità a cinque livelli (più il livello base, detto "livello 0"), riportati di seguito:
	\begin{itemize}
	\item \textbf {Livello 0 - Incomplete Process}: il processo riporta \gl{performance} e risultati incompleti, inoltre è gestito in modo caotico.
	\item \textbf {Livello 1 - Performed Process}:  il processo raggiunge i risultati attesi ma viene eseguito in modo non controllato. Gli attributi di tale processo sono:
		\begin{itemize}
		\item \textbf{1.1 - Process Performance}: capacità di un processo di raggiungere gli obiettivi definiti.
		\end{itemize}
	\item \textbf {Livello 2 - Managed Process}: il processo è pianificato e tracciato secondo standard prefissati, dunque il suo prodotto è controllato, manutenuto e soddisfa determinati criteri di qualità. Gli attributi di tale processo sono:
		\begin{itemize}
		\item \textbf{2.1 - Performance Management}: capacità di un processo di identificare gli obiettivi e di definire, monitorare e modificare le sue performance;
		\item \textbf{2.2 - Work Product Management}: capacità di un processo di identificare, elaborare, documentare e controllare i propri risultati.
		\end{itemize}
	\item \textbf {Livello 3 - Established Process}: il processo possiede specifici standard organizzativi che includono linee guida personalizzate, il tutto è consolidato tramite una politica di feedback del prodotto. Gli attributi di tale processo sono:
		\begin{itemize}
		\item \textbf{3.1 - Process Definition}: il processo specifico si basa su processi standard, individuando e incorporando caratteristiche fondamentali di questi;
		\item \textbf{3.2 - Process Deployment}: sono stati definiti ed assegnati dei ruoli a ciascun membro del team, ogni risorsa necessaria per l'esecuzione del processo è disponibile ed utilizzabile.
		\end{itemize}
	\item \textbf {Livello 4 - Predictable Process}: il processo è quantitativamente misurato e statisticamente analizzato per permettere di prendere decisioni oggettive e per assicurare che le prestazioni rimangano all'interno di limiti definiti. Gli obiettivi sono, di conseguenza, supportati in maniera consistente. Gli attributi di tale processo sono:
		\begin{itemize}
		\item \textbf{4.1 - Process Measurement}: i risultati di misurazione dei processi vengono utilizzati per garantire che le prestazioni del processo supportino il raggiungimento degli obiettivi determinati;
		\item \textbf{4.2 - Process Control}: il processo è gestito quantitativamente in modo da renderlo stabile, capace e prevedibile entro limiti definiti.
		\end{itemize}
	\item \textbf {Livello 5 - Optimizing Process}: il processo è in continuo miglioramento per raggiungere adeguatamente gli obiettivi prefissati. Gli attributi di tale processo sono:
		\begin{itemize}
		\item \textbf{5.1 - Process Innovation}: gli obiettivi di miglioramento del processo supportano gli obiettivi aziendali rilevanti;
		\item \textbf{5.2 - Process Optimization}: le modifiche alla definizione, gestione e le prestazioni del processo si traducono in un impatto efficace per il raggiungimento degli obiettivi.
		\end{itemize}
	\end{itemize}
	
	Lo standard SPICE offre una scala di valutazione per ogni processo, in modo da misurare il livello di raggiungimento degli stessi:
	\begin{itemize}
	\item \textbf{N - Not Achieved}: 0 - 15\%;
	\item \textbf{P - Partially Achieved}: >15\% - 50\%;
	\item \textbf{L - Largely Achieved}: >50\% - 85\%;
	\item \textbf{F - Fully Achieved}: >85\% - 100\%.
	\end{itemize}
\subsubsection{Ciclo di miglioramento continuo (PDCA)}
	Il ciclo di miglioramento continuo (PDCA, \emph{Plan-Do-Check-Act)} prevede quattro fasi iterative che permettono di controllare costantemente lo sviluppo di un processo, in modo da poter perseguire la miglior qualità di quest'ultimo:
	\begin{itemize}
	\item \textbf{Plan}: in questa fase vengono definiti elementi estremamente importanti che riguardano il ciclo di miglioramento continuo. In particolare vengono fissati obiettivi, processi da utilizzare, risultati da ottenere, personale incaricato per i vari processi e scadenze da rispettare;
	\item \textbf{Do}: in questa fase vengono avviate tutte le attività previste da completare entro la data stabilita;
	\item \textbf{Check}: in questa fase vengono confrontati i risultati ottenuti dalle varie attività con quelli ipotizzati durante la fase Plan;
	\item \textbf{Act}: in questa fase vengono individuate le possibili problematiche che hanno prodotto risultati differenti da quelli attesi. Di conseguenza, verranno determinate tutte le attività da revisionare per migliorare la qualità del processo. 
	\end{itemize}

\subsubsection{Gestione}
	Il gruppo 7dos ha deciso di seguire l'approccio a maturità di processo per i seguenti motivi:
	\begin{itemize}
	\item predisposizione alla cura della qualità del prodotto e dei processi;
	\item previsto nelle buone pratiche di \gl{management};
	\item adatto per i neofiti della programmazione progettuale.
	\end{itemize}
	
\subsection{Qualità di prodotto}
Per poter garantire che il prodotto realizzato sia di alta qualità, è necessario definire un modello per la valutazione di quest'ultima; il team 7dos, per questo motivo, ha scelto di adottare il modello di qualità delineato nello standard ISO/IEC 25010, anche noto come \gl{SQuaRE}.

Tale modello comprende 8 caratteristiche (ciascuna divisa in sotto-caratteristiche, per un totale di 31) che vanno prese in considerazione durante lo sviluppo del progetto per garantire un'elevata qualità complessiva del prodotto finale.

Per praticità e rilevanza ai fini del prodotto, sono state selezionate 5 caratteristiche da considerare e per ciascuna sono state individuate le sotto-caratteristiche più rilevanti al progetto, da perseguire come obiettivi prioritari. In particolare, sono state scartate: \gl{Compatibility}, in quanto andando a realizzare un plugin (di natura integrato in un sistema preesistente) è stata giudicata superflua; \gl{Security}, in quanto il plugin non dovrà gestire autenticazione o raccolta di dati; ed infine \gl{Portability} in quanto essendo il prodotto un plugin per un determinato sistema, non è rilevante la sua portabilità ad altri ambienti. 

Di seguito vengono riportate le caratteristiche e sotto-caratteristiche selezionate:
\subsubsection{Functional Suitability}
Questa caratteristica esprime il grado di soddisfacimento dei requisiti espliciti ed impliciti da parte di un prodotto o servizio, quando utilizzato sotto determinate condizioni.\\ \\
\textbf{Sotto-caratteristiche notevoli:}
\begin{itemize}
	\item{\textbf{Functional Completeness}}: esprime il grado con cui l'insieme di funzioni copre i compiti specificati e gli obiettivi dell'utente;
	\item{\textbf{Functional Correctness}}: esprime il grado con cui il prodotto restituisce risultati corretti, entro il livello di precisione desiderato.
\end{itemize}

\subsubsection{Performance Efficiency}
Questa caratteristica esprime le prestazioni relative al sistema come, quantità di risorse utilizzate per eseguire una determinata funzionalità del sistema sotto specifiche condizioni.  \\ \\
\textbf{Sotto-caratteristiche notevoli:}
\begin{itemize}
	\item{\textbf{Time Behaviour}}: esprime il grado con cui i tempi di risposta ed elaborazione e i volumi di produzione di un prodotto o sistema, durante l'esecuzione delle sue funzionalità, rispettano i requisiti;
	\item{\textbf{Resource Utilization}}: esprime il grado con cui il numero e tipo di risorse utilizzate da un prodotto o sistema, durante l'esecuzione delle sue funzionalità, rispetta i requisiti.
\end{itemize}

\subsubsection{Usability}
Questa caratteristica esprime il grado con cui un prodotto o sistema può essere usato da un determinato utente per raggiungere determinati scopi con efficacia, efficienza e soddisfazione in uno specifico contesto d'uso.\\ \\
\textbf{Sotto-caratteristiche notevoli:}
\begin{itemize}
	\item{\textbf{Learnability}}: esprime il grado con cui determinati utenti sono in grado di imparare ad utilizzare il prodotto o sistema con efficacemente, efficientemente, con sicurezza da rischi e soddisfazione in un dato contesto d'uso;
	\item{\textbf{Operability}}: esprime il grado con cui un prodotto o sistema ha attributi che lo rendono facile da operare e controllare;
	\item{\textbf{User Error Protection}}: esprime il grado di efficacia ed efficienza con cui un sistema protegge gli utenti dal commettere errori.
\end{itemize}

\subsubsection{Reliability}
Questa caratteristica esprime il grado con cui un sistema, prodotto o componente esegue determinate funzioni sotto specifiche condizioni per un dato periodo di tempo. \\ \\
\textbf{Sotto-caratteristiche notevoli:}
\begin{itemize}
	\item{\textbf{Maturity}}: esprime il grado con cui un sistema, prodotto o componente raggiunge i requisiti di affidabilità in normali condizioni operative;
	\item{\textbf{Fault Tolerance}}: esprime il grado con cui un sistema, prodotto o componente opera come previsto nonostante la presenza di malfunzionamenti hardware o software.
\end{itemize}

\subsubsection{Maintainability}
Questa caratteristica esprime il grado di efficacia ed efficienza con cui un prodotto, o componente può essere modificato per migliorarlo, correggerlo o adattarlo a dei cambiamenti all'ambiente. \\ \\
\textbf{Sotto-caratteristiche notevoli}
\begin{itemize}
	\item{\textbf{Modularity}}: esprime il grado di scomposizione del sistema in parti minimali tali che un cambiamento ad una specifica componente ha il minimo impatto su tutte le altre componenti;
	\item{\textbf{Analisability}}: esprime il grado di efficacia ed efficienza con cui è possibile analizzare l'impatto nel sistema di uno specifico cambiamento ad una o più delle sue parti, ai fini di rilevare eventuali casi di fallimento;
	\item{\textbf{Modificability}}: esprime il grado con cui un prodotto o sistema può essere modificato efficacemente ed efficientemente senza introdurre difetti che ne possano intaccare la qualità complessiva;
	\item{\textbf{Testability}}: esprime il grado di efficacia ed efficienza con cui è possibile stabilire ed eseguire dei test per valutare la qualità del sistema.
\end{itemize}


\pagebreak
\pagebreak