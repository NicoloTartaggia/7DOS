\newsection{Strategia generale per la verifica}
Il Piano di Qualifica prevede che vengano delineati obiettivi da portare a termine seguendo strategia e metriche ben definite.\\
La qualità di un qualsiasi prodotto o sistema è strettamente legata alla qualità dei processi che portano al suo sviluppo. Pertanto, nelle seguenti sezioni, verranno descritti gli obiettivi da perseguire nell'intento di garantire la massima qualità di processi e prodotti nella realizzazione del progetto.

\subsection{Qualità di processo}
	È impossibile creare prodotti di alta qualità se il proprio \gl{way of working} è scadente: risulta quindi fondamentale che i processi attuati, \emph{in primis}, garantiscano un elevato livello qualitativo. Il gruppo 7DOS ha deciso di adottare la normativa \gl{ISO/IEC 15504}, anche nota come \gl{SPICE}, e di seguire il principio di miglioramento continuo (\gl{PDCA}). 


\subsubsection{Procedure di qualità di processo}

\paragraph{Obiettivi}
Gli obiettivi da rispettare durante lo sviluppo del progetto sono i seguenti:

\begin{itemize}
	\item{\textbf{Pianificazione}: organizzazione oraria del lavoro, prestando attenzione nell'assegnazione dei compiti ai vari membri del gruppo;
	}
	\item{\textbf{Budget}: controllo e verifica della pianificazione monetaria cercando di attenersi ai costi preventivati;
	}
	\item{\textbf{Rischi}: quantificazione ed individuazione di rischi interni ed esterni, con particolare attenzione al calcolo dei rischi non previsti;
	}
	\item{\textbf{Build e versionamento}: quantificazione di build e versionamenti dei prodotti
	}
\end{itemize}

Le metriche utilizzate sono le seguenti:
\begin{itemize}
	\item{\textbf{Pianificazione}: 
		\begin{itemize}
			\item{\emph{Schedule Variance} (\gl{SV}).}
		\end{itemize}
	}
	\item{\textbf{Budget}: 
		\begin{itemize}
			\item{\emph{Budget Variance} (\gl{BV}).}
		\end{itemize}	
	}
	\item{\textbf{Rischi}:
		\begin{itemize}
			\item\emph{Numero rischi non previsti.}
			\item\emph{Indisponibilità servizi esterni.}	
		\end{itemize}	
	}
	\item{\textbf{Build e versionamento}: 
		\begin{itemize}
			\item\emph{Media commit a settimana.}
		\end{itemize}	
	}
\end{itemize}

\subsection{Tabella riassuntiva delle metriche relative ai processi}
\renewcommand{\arraystretch}{1.5}
\begin{table}[H]
	\centering
	\begin{tabular}{|C{5cm}|C{4cm}|C{4cm}|}
		\hline
		\rowcolor{title_row}
		\textbf{\color{title_text}{Nome Metrica}} &  \textbf{\color{title_text}{Range accettabile}} & \textbf{\color{title_text}{Range ottimale}} \\ \hline
		Schedule Variance & $\geq$-5 giorni  & $\geq$0 giorni \\ \hline
		Budget Variance&$\geq$-10\% &$\geq$0\% \\ \hline
		Numero rischi non previsti&$\geq$-3 &$\geq$0 \\ \hline
		Indisponibilità servizi esterni&$\geq$-5 &$\geq$0 \\ \hline
		Media commit a settimana&$\geq$20 &$\geq$30 \\ \hline

	\end{tabular}
	\caption{Riassunto delle metriche dei test sui processi}
	\label{tabella:riassunto metriche dei test sui processi}
\end{table}
\renewcommand{\arraystretch}{1}

\subsubsection{Gestione}
	Il gruppo 7DOS ha deciso di seguire l'approccio a maturità di processo per i seguenti motivi:
	\begin{itemize}
	\item Predisposizione alla cura della qualità del prodotto e dei processi;
	\item Previsto nelle buone pratiche di \gl{management};
	\item Adatto per i neofiti della programmazione progettuale.
	\end{itemize}
	
\subsection{Qualità di prodotto}
Per poter garantire che il prodotto realizzato sia di alta qualità, è necessario definire un modello per la valutazione di quest'ultima; il team 7DOS, per questo motivo, ha scelto di adottare il modello di qualità delineato nello standard \gl{ISO/IEC 25010}), anche noto come \gl{SQuaRE}.

Tale modello comprende 8 caratteristiche (ciascuna divisa in sotto-caratteristiche, per un totale di 31) che vanno prese in considerazione durante lo sviluppo del progetto per garantire un'elevata qualità complessiva del prodotto finale.

Per praticità e rilevanza ai fini del prodotto, sono state selezionate 5 caratteristiche da considerare e per ciascuna sono state individuate le sotto-caratteristiche più rilevanti al progetto, da perseguire come obiettivi prioritari. In particolare, sono state scartate: \gl{Compatibility}, in quanto andando a realizzare un plug-in (di natura integrato in un sistema preesistente) è stata giudicata superflua; \gl{Security}, in quanto il plug-in non dovrà gestire autenticazione o raccolta di dati; ed infine \gl{Portability} in quanto essendo il prodotto un plug-in per un determinato sistema, non è rilevante la sua portabilità ad altri ambienti. 

\subsection{Metriche relative ai documenti}

\begin{itemize}
	\item{\textbf{Grammatica}: controllo e verifica della correttezza grammaticale
		\begin{itemize}
			\item{\emph{Numero di errori grammaticali}.}
		\end{itemize}	
	}
	\item{\textbf{Lettura e comprensione}: facilità di lettura e comprensione di un testo
		\begin{itemize}
			\item{\emph{Gunning fog index}.}
			\item{\emph{Indice di Gulpease}.}
		\end{itemize}	
	}
\end{itemize}

\subsection{Metriche relative ai prodotti software}

\begin{itemize}
	\item{\textbf{Sviluppo e rispetto dei requisiti}:
		\begin{itemize}
			\item{\emph{Functional Implementation Completeness}.}
			\item{\emph{Average Functional Implementation Correctness}.}			
		\end{itemize}	
	}
	\item{\textbf{Interazioni con l'utente}: 
		\begin{itemize}
			\item{\emph{Average Learning Time}.}
		\end{itemize}	
	}
	\item{\textbf{Test e completezza}: 
	\begin{itemize}
		\item{\emph{Failure Density}.}
	\end{itemize}	
}
\end{itemize}

\subsection{Tabella riassuntiva delle metriche relative ai prodotti}
\renewcommand{\arraystretch}{1.5}
\begin{table}[H]
	\centering
	\begin{tabular}{|C{4cm}|C{3cm}|C{3cm}|C{3cm}|}
		\hline
		\rowcolor{title_row}
		\textbf{\color{title_text}{Nome Metrica}} & \textbf{\color{title_text}{Intervallo limite}} & \textbf{\color{title_text}{Range accettabile}} & \textbf{\color{title_text}{Range ottimale}} \\ \hline
		Functional Implementation Completeness & 0-100 & 75-100 & 100 \\ \hline
		Average Functional Implementation Correctnes&0-100&80-100&95-100 \\ \hline
		Numero di errori grammaticali & / & 0 & 0 \\ \hline
		Gunning fog index & / &12-15&0-12 \\ \hline
		Indice di Gulpease & 0-100& 40-100& 60-100 \\ \hline
		Indice di Flesh& 0-100 & 50-100& 60-100\\ \hline
		Average Learning Time& / &0-15 & 0-30 \\ \hline
		Failure Density&0-100&0-10&0\\ \hline
	\end{tabular}
	\caption{Riassunto delle metriche dei test sui prodotti}
	\label{tabella:riassunto metriche dei test sui prodotti}
\end{table}
\renewcommand{\arraystretch}{1}
\pagebreak