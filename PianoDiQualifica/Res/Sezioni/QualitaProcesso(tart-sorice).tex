\newsection{Strategia generale per la verifica}
Il Piano di Qualifica prevede che vengano delineati obiettivi da portare a termine seguendo strategia e metriche ben definite.\\
La qualità di un qualsiasi prodotto o sistema è strettamente legata alla qualità dei processi che portano al suo sviluppo. Pertanto, nelle seguenti sezioni, verranno descritti gli obiettivi da perseguire nell'intento di garantire la massima qualità di processi e prodotti nella realizzazione del progetto.

\subsection{Qualità di processo}
	È impossibile creare prodotti di alta qualità se il proprio \gl{way of working} è scadente: risulta quindi fondamentale che i processi attuati, \emph{in primis}, garantiscano un elevato livello qualitativo. Il gruppo 7DOS ha deciso di adottare la normativa \gl{ISO/IEC 15504}, anche nota come \gl{SPICE}, e di seguire il principio di miglioramento continuo (\gl{PDCA}). 


\subsubsection{Pianificazione e controllo} 
Questo processo ha come scopo la pianificazione dello sviluppo del progetto, andando a definire suddivisione delle attività da svolgere, organizzazione oraria del lavoro, pianificazione e controllo dei costi e standard di supporto e accertandosi che il team abbia padronanza delle tecnologie utilizzate.
\paragraph{Obiettivi} \Spazio
Gli obiettivi da rispettare durante lo sviluppo del progetto sono i seguenti:
	\begin{itemize}
		\item{\textbf{Pianificazione}: organizzazione oraria del lavoro, prestando attenzione nell'assegnazione dei compiti ai vari membri del gruppo e attenendosi ai costi preventivati;
		}
		\item{\textbf{Budget}: controllo e verifica della pianificazione monetaria cercando di attenersi ai costi preventivati;
		}
		\item{\textbf{Standard}: attenersi a standard ben precisi, affinché il gruppo abbia sempre una linea guida di supporto;
		}
		\item{\textbf{Preparazione del personale}: assicurarsi che ciascun membro del personale abbia familiarità con le tecnologie richieste in ogni \gl{task}, in modo tale da garantire produttività dell'ambiente lavorativo;
		}
	\end{itemize}

Le metriche utilizzate, definite nel documento \emph{Norme di Progetto v2.0.0}, sono le seguenti:
\begin{itemize}
			\item{\emph{Schedule Variance} (\gl{SV});}
			\item{\emph{Budget Variance} (\gl{BV}).}
\end{itemize}

\subsubsection{Gestione dei rischi} 
Questo processo ha come scopo il riconoscimento dei rischi a cui il progetto può incorrere. Il gruppo 7DOS mira, così facendo, a ridurre il più possibile il verificarsi di essi.
\paragraph{Obiettivi} \Spazio
Gli obiettivi da rispettare durante lo sviluppo del progetto sono i seguenti:
\begin{itemize}
	\item{\textbf{Identificazione nuovi rischi}: ad ogni fase del progetto il gruppo identificherà i possibili nuovi rischi;}
	\item{\textbf{Analisi dei rischi}: i rischi individuati verranno analizzati in modo da poter fornire procedure automatiche per prevenire che uno di essi si verifichi.}
\end{itemize}
Le metriche utilizzate, definite nel documento \emph{Norme di Progetto v2.0.0}, sono le seguenti:
\begin{itemize}
	\item\emph{Numero rischi non previsti;}
	\item\emph{Indisponibilità servizi esterni.}
\end{itemize}

\subsubsection{Gestione dei test} 
Questo processo ha come scopo 
\paragraph{Obiettivi} \Spazio
Gli obiettivi da rispettare durante lo sviluppo del progetto sono i seguenti:
\begin{itemize}
	\item{}
\end{itemize}
Le metriche utilizzate, definite nel documento \emph{Norme di Progetto v2.0.0}, sono le seguenti:
\begin{itemize}
	\item{}
\end{itemize}

\subsubsection{Versionamento e build} 
Questo processo ha come scopo la quantificazione e il controllo dei commit, in modo tale da verificare l'effettivo sviluppo di versioni sempre più complete e stabili del prodotto. Inoltre, lo strumento di integrazione continua \gl{Travis CI} permetterà di eseguire le \gl{build} in modo automatico, permettendo una verifica della correttezza del codice rispetto alle norme.
\paragraph{Obiettivi} \Spazio
Gli obiettivi da rispettare durante lo sviluppo del progetto sono i seguenti:
\begin{itemize}
	\item{\textbf{Commit frequenti}: ogni membro del gruppo deve garantire una buona frequenza di commit, affinché il prodotto sia sempre aggiornato con le ultime modifiche;
	\item{\textbf{Modifiche di piccole dimensioni}: ogni commit deve comportare una quantità di modifiche limitata. In questo modo, ogni versione del prodotto è monitorabile e verificabile;}
	\item{\textbf{Buona riuscita delle build}: ogni membro deve accertarsi che ogni commit comporti una build positiva, in modo tale evitare la diffusione di errori.}
	}
\end{itemize}
Le metriche utilizzate, definite nel documento \emph{Norme di Progetto v2.0.0}, sono le seguenti:
\begin{itemize}
	\item\emph{Media commit a settimana;}
	\item\emph{Media build a settimana;}
	\item\emph{Percentuale build superate.}
\end{itemize}

\subsubsection{Tabella riassuntiva delle metriche relative ai processi}
\renewcommand{\arraystretch}{1.5}
\begin{table}[H]
	\centering
	\begin{tabular}{|C{5cm}|C{4cm}|C{4cm}|}
		\hline
		\rowcolor{title_row}
		\textbf{\color{title_text}{Nome Metrica}} &  \textbf{\color{title_text}{Range accettabile}} & \textbf{\color{title_text}{Range ottimale}} \\ \hline
		Schedule Variance & $\geq$-5 giorni  & $\geq$0 giorni \\ \hline
		Budget Variance&$\geq$-10\% &$\geq$0\% \\ \hline
		Numero rischi non previsti&$\geq$-3 &$\geq$0 \\ \hline
		Indisponibilità servizi esterni&$\geq$-5 &$\geq$0 \\ \hline
		Media commit a settimana&$\geq$20 &$\geq$30 \\ \hline
		Media build a settimana&$\geq$20 &$\geq$30 \\ \hline
		Percentuale build superate& $\geq$70\% &$\geq$80\% \\ \hline

	\end{tabular}
	\caption{Riassunto delle metriche dei test sui processi}
	\label{tabella:riassunto metriche dei test sui processi}
\end{table}
\renewcommand{\arraystretch}{1}

\subsubsection{Gestione}
	Il gruppo 7DOS ha deciso di seguire l'approccio a maturità di processo per i seguenti motivi:
	\begin{itemize}
	\item Predisposizione alla cura della qualità del prodotto e dei processi;
	\item Previsto nelle buone pratiche di \gl{management};
	\item Adatto per i neofiti della programmazione progettuale.
	\end{itemize}
	
\subsection{Qualità di prodotto}
Per poter garantire che il prodotto realizzato sia di alta qualità, è necessario definire un modello per la valutazione di quest'ultima; il team 7DOS, per questo motivo, ha scelto di adottare il modello di qualità delineato nello standard \gl{ISO/IEC 25010}), anche noto come \gl{SQuaRE}.

Tale modello comprende 8 caratteristiche (ciascuna divisa in sotto-caratteristiche, per un totale di 31) che vanno prese in considerazione durante lo sviluppo del progetto per garantire un'elevata qualità complessiva del prodotto finale.

Per praticità e rilevanza ai fini del prodotto, sono state selezionate 5 caratteristiche da considerare e per ciascuna sono state individuate le sotto-caratteristiche più rilevanti al progetto, da perseguire come obiettivi prioritari. In particolare, sono state scartate: \gl{Compatibility}, in quanto andando a realizzare un plug-in (di natura integrato in un sistema preesistente) è stata giudicata superflua; \gl{Security}, in quanto il plug-in non dovrà gestire autenticazione o raccolta di dati; ed infine \gl{Portability} in quanto essendo il prodotto un plug-in per un determinato sistema, non è rilevante la sua portabilità ad altri ambienti. 

\subsubsection{Qualità dei documenti}
I documenti prodotti devono possedere caratteristiche consistenti. In particolare, essi devono essere leggibili, comprensibili e corretti a livello ortografico e sintattico.
\paragraph{Obiettivi} \Spazio
Gli obiettivi da rispettare durante lo sviluppo del progetto sono i seguenti:
	\begin{itemize}
		\item{\textbf{Leggibilità}: i documenti prodotti dovranno essere leggibili e comprensibili da persone con almeno una licenza di istruzione media;}
		\item{\textbf{Correttezza ortografica}: i documenti prodotti dovranno essere privi di errori ortografici.}
	\end{itemize}

Le metriche utilizzate, definite nel documento \emph{Norme di Progetto v2.0.0}, sono le seguenti:

\begin{itemize}
		\item{\emph{Gunning fog index};}
		\item{\emph{Indice di Gulpease};}	
		\item{\emph{Numero di errori grammaticali}.}
\end{itemize}

\subsubsection{Qualità del software}

\begin{itemize}
		\item{\textbf{Functional Suitability}: caratteristica che esprime il grado di soddisfacimento dei requisiti espliciti ed impliciti da parte di un prodotto o servizio.\\ L'insieme delle funzioni rispettano le aspettative e i risultati ottenuti sono corretti e presentano livelli di precisione desiderati;
	}
		\item{\textbf{Performance Efficiency}: 		caratteristica che esprime le prestazioni relative al sistema come quantità di risorse utilizzate per eseguire una determinata funzionalità. Quest'ultima deve richiedere la minor quantità di risorse disponibili e, al tempo stesso, deve essere eseguita nel minor tempo possibile; 
	}
		\item{\textbf{Usability}: caratteristica che esprime il grado con cui un prodotto o sistema può essere usato da un determinato utente. \\ Il prodotto è comprensibile dall'utente e il tempo per conoscere a pieno le sue funzionalità è accettabile. Inoltre, queste ultime devono essere coerenti con ciò che l'utente si aspetta;
	}
		\item{\textbf{Reliability}: caratteristica che esprime il grado con cui un prodotto esegue correttamente determinate funzioni  mentre è in uso. Questo comporta che vengano evitati il più possibile malfunzionamenti in caso di \gl{fault}. Se questi si verificano, vengono attivate procedure di gestione dell'errore in modo da non influenzare le prestazioni;
	}
	\item{\textbf{Maintainability}: caratteristica che esprime il grado di efficacia ed efficienza con cui il prodotto può essere modificato, tramite azioni di correzione, o aggiornato, tramite azioni di miglioramento. In particolare, deve essere possibile risalire rapidamente alle cause che hanno generato un malfunzionamento, eseguire cambiamenti di componenti originali e testare il sistema per validare le varie modifiche.
	}
\end{itemize}

Le metriche utilizzate, definite nel documento \emph{Norme di Progetto v2.0.0}, sono le seguenti:
\begin{itemize}
	\item{\textbf{Functional Suitability}:
		\begin{itemize}
			\item{\emph{Functional Implementation Completeness}.}
			\item{\emph{Average Functional Implementation Correctness}.}			
		\end{itemize}	
	}
	\item{\textbf{Performance Efficiency}:
		\begin{itemize}
			\item{\emph{Tempo di risposta}.}
		\end{itemize}
	}
	\item{\textbf{Usability}: 
		\begin{itemize}
			\item{\emph{Average Learning Time}.}
		\end{itemize}	
	}
	\item{\textbf{Reliability}: 
		\begin{itemize}
			\item{\emph{Failure Density};}
			\item{\emph{Blocco di operzioni non corrette}.}
		\end{itemize}	
	}
	\item{\textbf{Maintainability}: 
		\begin{itemize}
			\item{\emph{Failure Analysis};}
			\item{\emph{Comment Ratio}.}
		\end{itemize}	
	}
\end{itemize}

\subsection{Tabella riassuntiva delle metriche relative ai prodotti}
\renewcommand{\arraystretch}{1.5}
\begin{table}[H]
	\centering
	\begin{tabular}{|C{4cm}|C{3cm}|C{3cm}|C{3cm}|}
		\hline
		\rowcolor{title_row}
		\textbf{\color{title_text}{Nome Metrica}} & \textbf{\color{title_text}{Intervallo limite}} & \textbf{\color{title_text}{Range accettabile}} & \textbf{\color{title_text}{Range ottimale}} \\ \hline
		Gunning fog index & / &12-15&0-12 \\ \hline
		Indice di Gulpease & 0-100& 40-100& 60-100 \\ \hline
		Numero di errori grammaticali & / & 0 & 0 \\ \hline
		Functional Implementation Completeness & 0-100 & 75-100 & 100 \\ \hline
		Average Functional Implementation Correctnes&0-100&80-100&95-100 \\ \hline
		Tempo di risposta & / &0-8 sec & 0-3 sec \\ \hline
		Average Learning Time& / &0-15 & 0-30 \\ \hline
		Failure Density&0-100&0-10&0\\ \hline
		Blocco di operazioni non corrette &0-100&80-10&95-100\\ \hline
		Failure Analysis&0-100&60-10&80-100\\ \hline
		Comment Ratio&0-100&60-10&80-100\\ \hline
	\end{tabular}
	\caption{Riassunto delle metriche dei test sui prodotti}
	\label{tabella:riassunto metriche dei test sui prodotti}
\end{table}
\renewcommand{\arraystretch}{1}
\pagebreak