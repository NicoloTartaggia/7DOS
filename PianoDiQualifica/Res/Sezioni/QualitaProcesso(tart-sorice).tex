%\newsection{Strategia generale per la verifica}
%Il Piano di Qualifica impone che vengano delineati obiettivi da portare a termine seguendo strategia e metriche ben definite.\\
%La qualità di un qualsiasi prodotto o sistema è strettamente legata alla qualità dei processi che portano al suo sviluppo. Pertanto, nelle seguenti sezioni, verranno descritti gli obiettivi da perseguire nell'intento di garantire la massima qualità di processi e prodotti nella realizzazione del progetto.

\newsection{Qualità di processo}
	È impossibile creare prodotti di alta qualità se il proprio \gl{way of working} è scadente: risulta quindi fondamentale che i processi attuati, \emph{in primis}, garantiscano un elevato livello qualitativo. Il gruppo 7DOS ha deciso di adottare la normativa \gl{ISO/IEC 15504}, anche nota come \gl{SPICE}, e di seguire il principio di miglioramento continuo (\gl{PDCA}).
	
Per il controllo della qualità di processo il gruppo 7DOS utilizzerà l'approccio a maturità di processo, in quanto previsto dalle buone pratiche di \gl{management} e più adatto ad un gruppo inesperto.


\subsection{Pianificazione e controllo}
Questo processo ha come scopo la pianificazione dello sviluppo del progetto, andando a definire suddivisione delle attività da svolgere, organizzazione oraria del lavoro, pianificazione e controllo dei costi e standard di supporto e accertandosi che il team abbia padronanza delle tecnologie utilizzate.
\subsubsection{Obiettivi}
Gli obiettivi da rispettare durante lo sviluppo del progetto sono i seguenti:
	\begin{itemize}
		\item{\textbf{Pianificazione}: organizzazione oraria del lavoro, prestando attenzione nell'assegnazione dei compiti ai vari membri del gruppo e attenendosi ai costi preventivati;
		}
		\item{\textbf{Budget}: controllo e verifica della pianificazione monetaria cercando di attenersi ai costi preventivati;
		}
		\item{\textbf{Standard}: attenersi a standard ben precisi, affinché il gruppo abbia sempre una linea guida di supporto;
		}
		\item{\textbf{Preparazione del personale}: assicurarsi che ciascun membro del personale abbia familiarità con le tecnologie richieste in ogni \gl{task}, in modo tale da garantire produttività dell'ambiente lavorativo.
		}
	\end{itemize}
\subsubsection{Metriche}
Le metriche utilizzate, definite nel documento \emph{Norme di Progetto v2.0.0}, sono le seguenti:
\begin{itemize}
			\item{\emph{Schedule Variance} (\gl{SV});}
			\item{\emph{Budget Variance} (\gl{BV}).}
\end{itemize}

\subsection{Gestione dei rischi}
Questo processo ha come scopo il riconoscimento dei rischi a cui il progetto può incorrere. Il gruppo 7DOS mira, così facendo, a ridurre il più possibile il verificarsi di essi.
\subsubsection{Obiettivi} 
Gli obiettivi da rispettare durante lo sviluppo del progetto sono i seguenti:
\begin{itemize}
	\item{\textbf{Identificazione nuovi rischi}: ad ogni fase del progetto il gruppo identificherà i possibili nuovi rischi;}
	\item{\textbf{Analisi dei rischi}: i rischi individuati verranno analizzati in modo da poter fornire procedure automatiche per prevenire che uno di essi si verifichi.}
\end{itemize}
\subsubsection{Metriche}
Le metriche utilizzate, definite nel documento \emph{Norme di Progetto v2.0.0}, sono le seguenti:
\begin{itemize}
	\item\emph{Numero rischi non previsti;}
	\item\emph{Indisponibilità servizi esterni.}
\end{itemize}

\subsection{Gestione dei test}
Questo processo ha come scopo la definizione di misurazioni sull'esecuzione dell'analisi dinamica, in modo tale da garantire una gestione efficace di essa.
\subsubsection{Obiettivi} 
Gli obiettivi da rispettare durante lo sviluppo del progetto sono i seguenti:
\begin{itemize}
	\item{\textbf{Efficienza dei test}: i test devono essere progettati per individuare il maggior numero di difetti possibili. In questo modo è possibile realizzare un prodotto di qualità che soddisfa i requisiti specificati;}
	\item{\textbf{Risoluzione tempestiva dei problemi riscontrati}: le problematiche evidenziate dai test devono essere risolte nel minor tempo possibile dal gruppo;}
	\item{\textbf{Eseguire tutti i test necessari}: il sistema si evolve. Molti dei difetti precedentemente segnalati vengono corretti. Ogni volta che viene risolto un errore o è stata aggiunta una nuova funzionalità, è necessario eseguire tutti i test per assicurarsi che il nuovo software modificato abbia effettivamente corrette gli errori noti e non abbia introdotto nuovi errori.}
\end{itemize}
\subsubsection{Metriche}
Le metriche utilizzate, definite nel documento \emph{Norme di Progetto v2.0.0}, sono le seguenti:
\begin{itemize}
	\item{\emph{Percentuale test case passati};}
	\item{\emph{Percentuale test case falliti};}
	\item{\emph{Tempo medio necessario al team per risolvere un errore};}
	\item{\emph{Efficienza nella progettazione dei test};}
	\item{\emph{Percentuale di difetti corretti};}
	\item{\emph{Percentuale test eseguiti}.}

\end{itemize}

\subsection{Versionamento e build} 
Questo processo ha come scopo la quantificazione e il controllo dei \gl{Commit}, in modo tale da verificare l'effettivo sviluppo di versioni sempre più complete e stabili del prodotto. Inoltre, lo strumento di integrazione continua \gl{Travis CI} permetterà di eseguire le \gl{build} in modo automatico, permettendo una verifica della correttezza del codice rispetto alle norme.
\subsubsection{Obiettivi} 
Gli obiettivi da rispettare durante lo sviluppo del progetto sono i seguenti:
\begin{itemize}
	\item{\textbf{Commit frequenti}: ogni membro del gruppo deve garantire una buona frequenza di commit, affinché il prodotto sia sempre aggiornato con le ultime modifiche;
	\item{\textbf{Modifiche di piccole dimensioni}: ogni commit deve comportare una quantità di modifiche limitata. In questo modo, ogni versione del prodotto è monitorabile e verificabile;}
	\item{\textbf{Buona riuscita delle build}: ogni membro deve accertarsi che ogni commit comporti una build positiva, in modo tale evitare la diffusione di errori.}
	}
\end{itemize}
\subsubsection{Metriche}
Le metriche utilizzate, definite nel documento \emph{Norme di Progetto v2.0.0}, sono le seguenti:
\begin{itemize}
	\item\emph{Media commit a settimana;}
	\item\emph{Media build Travis a settimana;}
	\item\emph{Percentuale build Travis superate.}
\end{itemize}

\subsection{Riassunto delle metriche di processo}
\renewcommand{\arraystretch}{1.5}
\begin{table}[H]
	\centering
	\begin{tabular}{|C{5cm}|C{4cm}|C{4cm}|}
		\hline
		\rowcolor{title_row}
		\textbf{\color{title_text}{Nome Metrica}} &  \textbf{\color{title_text}{Range accettabile}} & \textbf{\color{title_text}{Range ottimale}} \\ \hline
		Schedule Variance & $\geq$-5 giorni  & $\geq$0 giorni \\ \hline
		Budget Variance&$\geq$-10\% &$\geq$0\% \\ \hline
		Numero rischi non previsti&$\leq$3 & 0 \\ \hline
		Indisponibilità servizi esterni&$\geq$-5 &$\geq$0 \\ \hline
		Percentuale test case passati & 100\%  & 100\% \\ \hline
		Percentuale test case falliti & 0\% & 0\% \\ \hline
		Tempo medio necessario al team per risolvere un errore & 0h-4h & 0h-2h \\ \hline
		Efficienza nella progettazione dei test & 0.5h-3h & 0.5h-2h \\ \hline
		Percentuale dei difetti corretti & 90\%-100\% & 95\%-100\% \\ \hline
		Percentuale dei test eseguiti & 85\%-100\% & 95\%-100\% \\ \hline
		Media commit a settimana&$\geq$20 &$\geq$30 \\ \hline
		Media build Travis a settimana&$\geq$20 &$\geq$30 \\ \hline
		Percentuale build Travis superate& $\geq$70\% &$\geq$80\% \\ \hline
	\end{tabular}
	\caption{Riassunto delle metriche di processo}
	\label{tabella:riassunto metriche di processo}
\end{table}
\renewcommand{\arraystretch}{1}

\pagebreak
