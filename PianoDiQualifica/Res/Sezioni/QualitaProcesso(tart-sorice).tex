\newsection{Qualità di processo}

\subsection{Scopo}
	Per poter raggiungere gli obiettivi prefissati è necessario che i processi che portano al loro compimento garantiscano un buon livello  	di qualità. Il gruppo 7DOS, per questo motivo, ha deciso di adottare la normativa ISO/IEC 15504 (chiamata anche SPICE) e di 	seguire il principio di miglioramento continuo (\gl{PDCA}). \\ Per ogni \gl{processo} lo standard definisce una scala di maturità a cinque livelli (più il livello base, detto "livello 0"), riportati di seguito:
	
	\begin{itemize}
	\item \textbf {Livello 0 - Incomplete process}: il processo riporta \gl{performance} e risultati incompleti, inoltre è gestito in modo caotico.
	\item \textbf {Livello 1 - Performed process}:  il processo raggiunge i risultati attesi ma viene eseguito in modo non controllato. Gli attributi di tale processo sono:
		\begin{itemize}
		\item \textbf{1.1 - Process performance}: 
		\end{itemize}
	\item \textbf {Livello 2 - Managed process}: il processo è pianificato e tracciato secondo standard prefissati, dunque il suo prodotto è controllato, manutenuto e soddisfa determinati criteri di qualità. Gli attributi di tale processo sono:
		\begin{itemize}
		\item \textbf{2.1 - Performance management}:
		\item \textbf{2.2 - Work product management}:
		\end{itemize}
	\item \textbf {Livello 3 - Established process}: il processo possiede specifici standard organizzativi che includono linee guida personalizzate, il tutto è consolidato tramite una politica di feedback del prodotto. Gli attributi di tale processo sono:
		\begin{itemize}
		\item \textbf{3.1 - Process definition}:
		\item \textbf{3.2 - Process deployment}:
		\end{itemize}
	\item \textbf {Livello 4 - Predictable process}: il processo è quantitativamente misurato e statisticamente analizzato per permettere di prendere decisioni oggettive e per assicurare che le prestazioni rimangano all'interno di limiti definiti. Gli obiettivi sono, di conseguenza, supportati in maniera consistente. Gli attributi di tale processo sono:
		\begin{itemize}
		\item \textbf{4.1 - Process measurement}:
		\item \textbf{4.2 - Process control}:
		\end{itemize}
	\item \textbf {Livello 5 - Optimizing process}: il processo è in continuo miglioramento per raggiungere adeguatamente gli obiettivi prefissati. Gli attributi di tale processo sono:
		\begin{itemize}
		\item \textbf{5.1 - Process innovation}:
		\item \textbf{5.2 - Process optimization}:
		\end{itemize}
	\end{itemize}
	
	Lo standard SPICE offre una scala di valutazione per ogni processo, in modo da misurare il livello di raggiungimento degli stessi:
	\begin{itemize}
	\item \textbf{N - Not achieved}: 0 - 15\%;
	\item \textbf{P - Partially achieved}: >15\% - 50\%;
	\item \textbf{L - Largely achieved}: >50\% - 85\%;
	\item \textbf{F - Fully Achieved}: >85\% - 100\%;
	\end{itemize}

	Il ciclo di miglioramento continuo (PDCA, \emph{Plan-Do-Check-Act)} prevede quattro fasi iterative che permettono di controllare costantemente lo sviluppo di un processo, in modo da poter perseguire la miglior qualità di quest'ultimo:
	\begin{itemize}
	\item \textbf{Plan}: in questa fase vengono definiti elementi estremamente importanti che riguardano il ciclo di migliormento continuo. In particolare vengono fissati obiettivi, processi da utilizzare, risultati da ottenere, personale incaricato per i vari processi e scadenze da rispettare;
	\item \textbf{Do}: in questa fase vengono avviate tutte le attvità previste da completare entro la data stabilita;
	\item \textbf{Check}: in quest fase vengono confrontati i risultati ottenuti dalle varie attività con quelli ipotizzzati durante la fase Plan;
	\item \textbf{Act}: in questa fase vengono individuate le possibili problematiche che hanno prodotto risultati differenti da quelli attesi. Di conseguenza, verranno determinate tutte le attività da revisionare per migliorare la qualità del processo. 
	\end{itemize}

\pagebreak