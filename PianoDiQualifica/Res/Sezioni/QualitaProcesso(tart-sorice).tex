\newsection{Strategia generale per la verifica}
Il Piano di Qualifica prevede che vengano delineati obiettivi da portare a termine seguendo strategia e metriche ben definite.
\subsection{Obiettivi}
La qualità di un qualsiasi prodotto o sistema è strettamente legata alla qualità dei processi che portano al suo sviluppo. Pertanto, nelle seguenti sezioni, verranno descritti gli obiettivi da perseguire nell'intento di garantire la massima qualità di processi e prodotti nella realizzazione del progetto.
\subsection{Qualità di processo}
	È impossibile creare prodotti di alta qualità se il proprio \gl{way of working} è scadente: risulta quindi fondamentale che i processi attuati, \emph{in primis}, garantiscano un elevato livello qualitativo. Il gruppo 7DOS ha deciso di adottare la normativa \gl{ISO/IEC 15504}, anche nota come \gl{SPICE}, e di seguire il principio di miglioramento continuo (\gl{PDCA}). 

\subsubsection{Gestione}
	Il gruppo 7DOS ha deciso di seguire l'approccio a maturità di processo per i seguenti motivi:
	\begin{itemize}
	\item Predisposizione alla cura della qualità del prodotto e dei processi;
	\item Previsto nelle buone pratiche di \gl{management};
	\item Adatto per i neofiti della programmazione progettuale.
	\end{itemize}
	
\subsection{Qualità di prodotto}
Per poter garantire che il prodotto realizzato sia di alta qualità, è necessario definire un modello per la valutazione di quest'ultima; il team 7DOS, per questo motivo, ha scelto di adottare il modello di qualità delineato nello standard \gl{ISO/IEC 25010}), anche noto come \gl{SQuaRE}.

Tale modello comprende 8 caratteristiche (ciascuna divisa in sotto-caratteristiche, per un totale di 31) che vanno prese in considerazione durante lo sviluppo del progetto per garantire un'elevata qualità complessiva del prodotto finale.

Per praticità e rilevanza ai fini del prodotto, sono state selezionate 5 caratteristiche da considerare e per ciascuna sono state individuate le sotto-caratteristiche più rilevanti al progetto, da perseguire come obiettivi prioritari. In particolare, sono state scartate: \gl{Compatibility}, in quanto andando a realizzare un plug-in (di natura integrato in un sistema preesistente) è stata giudicata superflua; \gl{Security}, in quanto il plug-in non dovrà gestire autenticazione o raccolta di dati; ed infine \gl{Portability} in quanto essendo il prodotto un plug-in per un determinato sistema, non è rilevante la sua portabilità ad altri ambienti. 

\pagebreak