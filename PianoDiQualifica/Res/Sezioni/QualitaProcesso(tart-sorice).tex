\newsection{Qualità di processo}

\subsection{Scopo}
	Per poter raggiungere gli obiettivi prefissati è necessario che i processi che portano al loro compimento garantiscano un buon livello  	di qualità. Il gruppo 7DOS, per questo motivo, ha deciso di adottare la normativa ISO/IEC 15504 (chiamata anche SPICE) e di 	seguire il principio di miglioramento continuo (\gl{PDCA}). \\ Per ogni \gl{processo} lo standard definisce una scala di maturità a cinque livelli (più il livello base, detto "livello 0"), riportati di seguito:
	
	\begin{itemize}
	\item \textbf {Livello 0 - Incomplete process}: il processo riporta \gl{performance} e risultati incompleti, inoltre è gestito in modo caotico.
	\item \textbf {Livello 1 - Performed process}:  il processo raggiunge i risultati attesi ma viene eseguito in modo non controllato. Gli attributi di tale processo sono:
		\begin{itemize}
		\item \textbf{1.1 - Process performance}: 
		\end{itemize}
	\item \textbf {Livello 2 - Managed process}: il processo è pianificato e tracciato secondo standard prefissati, dunque il suo prodotto è controllato, manutenuto e soddisfa determinati criteri di qualità. Gli attributi di tale processo sono:
		\begin{itemize}
		\item \textbf{2.1 - Performance management}:
		\item \textbf{2.2 - Work product management}:
		\end{itemize}
	\item \textbf {Livello 3 - Established process}: il processo possiede specifici standard organizzativi che includono linee guida personalizzate, il tutto è consolidato tramite una politica di feedback del prodotto. Gli attributi di tale processo sono:
		\begin{itemize}
		\item \textbf{3.1 - Process definition}:
		\item \textbf{3.2 - Process deployment}:
		\end{itemize}
	\item \textbf {Livello 4 - Predictable process}: il processo è quantitativamente misurato e statisticamente analizzato per permettere di prendere decisioni oggettive e per assicurare che le prestazioni rimangano all'interno di limiti definiti. Gli obiettivi sono, di conseguenza, supportati in maniera consistente. Gli attributi di tale processo sono:
		\begin{itemize}
		\item \textbf{4.1 - Process measurement}:
		\item \textbf{4.2 - Process control}:
		\end{itemize}
	\item \textbf {Livello 5 - Optimizing process}: il processo è in continuo miglioramento per raggiungere adeguatamente gli obiettivi prefissati. Gli attributi di tale processo sono:
		\begin{itemize}
		\item \textbf{5.1 - Process innovation}:
		\item \textbf{5.2 - Process optimization}:
		\end{itemize}
	\end{itemize}
	
	Lo standard SPICE offre una scala di valutazione per ogni processo, in modo da misurare il livello di raggiungimento degli stessi:
	\begin{itemize}
	\item \textbf{N - Not achieved}: 0 - 15\%;
	\item \textbf{P - Partially achieved}: >15\% - 50\%;
	\item \textbf{L - Largely achieved}: >50\% - 85\%;
	\item \textbf{F - Fully Achieved}: >85\% - 100\%;
	\end{itemize}


\pagebreak