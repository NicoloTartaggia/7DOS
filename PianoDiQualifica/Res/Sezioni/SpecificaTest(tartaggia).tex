\subsection{Pianificazione dei test}
La consistenza dei test in esecuzioni ripetute è una condizione imperativa, dunque è necessario che essi siano \gl{ripetibili}: un dato test eseguito in un ambiente specifico deve produrre, se fornito un determinato input, sempre gli stessi output. Al fine di garantire ciò, il team 7DOS ha scelto di basarsi sullo standard \gl{ISO/IEC/IEEE 29119}\footnote{ISO/IEC/IEEE 29119 parte 3 sezione 7, IEEE 2013.} per quanto riguarda la pianificazione e documentazione dei test dinamici. Tale standard prevede di definire, per ciascun test o suite di test, una \emph{specifica di test} composta di:
\begin{itemize}
	\item {\textbf{Specifica di progettazione:} definisce le funzionalità del prodotto da testare e le condizioni di test, ovvero l'ambiente di esecuzione e le pre-condizioni (particolari eventi o stati pregressi) necessarie al suo svolgimento;}
	\item {\textbf{Specifica di caso:} definisce l'insieme degli input che si desidera testare, e l'insieme dei risultati attesi per ogni (gruppo di) input per una o più funzionalità testate;}
	\item {\textbf{Specifica di procedura:} definisce l'ordine di esecuzione dei test (nel caso di una suite di test), la modalità di svolgimento, ovvero le azioni da compiere e gli input da inserire in modo ordinato, eventuali azioni necessarie per il raggiungimento delle pre-condizioni e la modalità di analisi dei risultati ottenuti.}
\end{itemize}

Per quanto concerne la \emph{specifica di progettazione}, il team ha scelto di associarvi uno o più requisiti funzionali a seconda della specificità del test, descritti nell'\emph{Analisi dei Requisiti v3.0.0}. Questo permette di descrivere più precisamente le funzionalità oggetto di test e di tracciare al meglio il soddisfacimento dei requisiti. \\

Ogni test, per essere identificato in modo univoco, segue la seguente sintassi:
\begin{center}
	T[Tipo]-[ID]
\end{center}
\begin{itemize}
	\item \textbf{Tipo}: specifica il tipo del test	e viene indicato con una delle seguenti lettere:
	\begin{itemize}
		\item A: accettazione;
		\item S: sistema;
		\item I: integrazione;
		\item U: unità.
	\end{itemize}
	\item \textbf{ID}: indica il codice del test rispettando una struttura gerarchica.	
\end{itemize}
\definecolor{darkgreen}{rgb}{0.0, 0.5, 0.0}
Per ogni test viene indicato lo stato in cui si trova, con una delle seguenti sigle:
\begin{itemize}
	\item \textbf{NI}: non implementato;
	\item \textcolor{orange}{\textbf{I}}: implementato;
	\item \textcolor{red}{\textbf{NS}}: non superato;
	\item \textcolor{darkgreen}{\textbf{S}}: superato.
\end{itemize}
\subsubsection{Test di accettazione}
\normalsize
\renewcommand{\arraystretch}{1}
\begin{longtable}{|C{2.5cm}|C{2.5cm}|C{6.5cm}|C{1.5cm}|}
	\hline
	\rowcolor{title_row}
	\textbf{\color{title_text}{Test}} & \textbf{\color{title_text}{Requisito}} & \textbf{\color{title_text}{Descrizione}} & \textbf{\color{title_text}{Stato}} \\
	\hline
	\endhead
	{TS-1} & {R0F1} & prova  & {NI}\\
	\hline
	{TS-2} & {R0F2} & prova & {NI}\\
	\hline
	{TS-3} & {R0F3} & prova & {NI}\\
	\hline
	{TS-3} & {R0F3} & prova & {NI}\\
	\hline
	{TS-4} & {R0F4} & prova & {NI}\\
	\hline
	{TS-5} & {R0F5} & prova & {NI}\\
	\hline
	\caption{Riassunto dei test di accettazione}
	\label{tabella:riassunto ta}
\end{longtable}
\normalsize
\renewcommand{\arraystretch}{1}
\subsubsection{Test di sistema}
\begin{longtable}{|C{2.5cm}|C{2.5cm}|C{6.5cm}|C{1.5cm}|}
			\hline
			\rowcolor{title_row}
			\textbf{\color{title_text}{Test}} & \textbf{\color{title_text}{Requisito}} & \textbf{\color{title_text}{Descrizione}} & \textbf{\color{title_text}{Stato}} \\
			\hline
			\endhead
			{TS-0F1} & {R0F1} & Verifica che sia possibile e leggere la
			definizione della rete Bayesiana da un
			file in formato JSON.  & {NI}\\
			\hline
			{TS-0F1.1} & {R1F1.1} & Verifica che sia possibile validare un
			file .json. & {NI}\\
			\hline
			{TS-2} & {R0F2} & Verifica che sia possibile gestire la
			connessione tra i nodi della rete ai
			rispettivi flussi di dati. & {NI}\\
			\hline
			{TS-2.1} & {R0F2.1} & Verifica che sia possibile connettere un
			nodo della rete ad un flusso di dati. & {NI}\\
			\hline
			{TS-2.2} & {R0F2.2} & Verifica che sia possibile disconnettere un
			nodo della rete da un flusso di dati. & {NI}\\
			\hline
			{TS-2.3} & {R1F2.3} & Verifica che sia essere possibile modificare il
			flusso di dati connesso ad un nodo. & {NI}\\
			\hline
			{TS-3} & {R0F3} & Verifica che sia possibile applicare il
			ri-calcolo delle probabilità della rete
			secondo regole temporali prestabilite. & {NI}\\
			\hline
			{TS-3.1} & {R1F3.1} & Verifica che sia possibile modificare le
			suddette regole temporali. & {NI}\\
			\hline
			{TS-4} & {R0F4} & Verifica che sia possibile fornire nuovi dati
			al sistema di Grafana derivati dai nodi
			della rete non collegati al flusso di
			monitoraggio. & {NI}\\
			\hline
			{TS-4.1} & {R1F4.1} & Verifica che sia possibile aggiornare i dati
			in base alla frequenza stabilita. & {NI}\\
			\hline
			{TS-5} & {R0F5} & Verifica che i dati siano disponibili al sistema di
			creazione di grafici e dashboard per la
			loro visualizzazione. & {NI}\\
			\hline
			{TS-5.1} & {R1F5.1} & Verifica che sia e possibile aggiornare la
			dashboard in base alla frequenza
			stabilita. & {NI}\\
			\hline
			{TS-5.2} & {R1F5.2} & Verifica che sia possibile creare un panel. & {NI}\\
			\hline
			{TS-5.3} & {R1F5.3} & Verifica che sia possibile spostare un
			panel. & {NI}\\
			\hline
			{TS-5.4} & {R1F5.4} & Verifica che sia possibile cancellare un
			panel. & {NI}\\
			\hline
			{TS-5.5} & {R1F5.5} & Verifica che sia possibile minimizzare un
			panel. & {NI}\\
			\hline
			{TS-5.6} & {R1F5.6} & Verifica che sia possibile configurare un
			panel. & {NI}\\
			\hline
			{TS-5.6.1} & {R0F5.6.1} & Verifica che sia possibile selezionare un
			flusso dati. & {NI}\\
			{TS-5.6.2} & {R1F5.6.2} & Verifica che sia selezionare un
			nodo della rete. & {NI}\\
			\hline
			{TS-5.6.3} & {R1F5.6.3} & Verifica che sia possibile selezionare un
			intervallo di tempo.  & {NI}\\
			\hline
			{TS-5.7} & {R1F5.7} & Verifica che sia possibile modificare un
			panel.  & {NI}\\
			\hline
			{TS-5.7.1} & {R1F5.7.1} & Verifica che sia possibile usare le
			modifiche standard di Grafana su un
			panel.
  & {NI}\\
			\hline
			{TS-6} & {R1F6} & Verifica che sia  possibile definire alert in
			base a livelli di soglia raggiunti dai
			nodi non collegati al flusso dei dati.  & {NI}\\
			\hline
			{TS-6.1} & {R1F6.1} & Verifica che sia possibile configurare i
			parametri di un alert.  & {NI}\\
			\hline
			{TS-6.1.1} & {R1F6.1.1} & Verifica che sia possibile inserire il nome
			di un alert.  & {NI}\\
			\hline
			{TS-6.1.2} & {R1F6.1.2} & Verifica che sia  possibile inserire
			l'intervallo di verifica di un alert.  & {NI}\\
			\hline
			{TS-6.1.3} & {R1F6.1.3} & Verifica che sia possibile inserire la
			condizione di attivazione di un alert.  & {NI}\\
			\hline
			{TS-6.2} & {R1F6.2} & Verifica che sia possibile impostare il
			modo in cui viene notificata
			l'attivazione di un alert.  & {NI}\\
			\hline
			{TS-7} & {R1F7} & Verifica che sia possibile disegnare la rete
			Bayesiana con un editor
			grafico specializzato.  & {NI}\\
			\hline
			{TS-7.1} & {R1F7.1} & Verifica che sia possibile creare un nodo
			della rete.  & {NI}\\
			\hline
			{TS-7.11} & {R1F7.1.1} & Verifica che sia  possibile inizializzare la lista di predecessori
			del nodo.  & {NI}\\
			\hline
			{TS-7.1.2} & {R1F7.1.2} & Verifica che sia possibile inizializzare la lista di successori del
			nodo.  & {NI}\\
			\hline
			{TS-7.1.3} & {R1F7.1.3} & Verifica che sia possibile inizializzare il nome del nodo.  & {NI}\\
			\hline
			{TS-7.1.4} & {R1F7.1.4} & Verifica che sia possibile inizializzare la CPT associata al
			nodo.  & {NI}\\
			\hline
			{TS-7.1.4.1} & {R1F7.1.4.1} & Verifica che sia possibile inizializzare la lista degli stati
			associata alla CPT del nodo.  & {NI}\\
			\hline
			{TS-7.1.4.2} & {R1F7.1.4.2} & Verifica che sia possibile inizializzare la lista delle
			combinazioni degli stati dei nodi
			predecessori associata alla CPT del
			nodo.  & {NI}\\
			\hline
			{TS-7.1.4.3} & {R1F7.1.4.3} & Verifica che sia possibile inizializzare le celle della CPT.  & {NI}\\
			\hline
			{TS-7.2} & {R1F7.2} & Verifica che sia possibile modificare i
			parametri di un nodo della rete.  & {NI}\\
			\hline
			{TS-7.2.1} & {R1F7.2.1} & Verifica che sia  possibile modificare il
			nome di un nodo della rete.  & {NI}\\
			\hline
			{TS-7.2.2} & {R1F7.2.2} & Verifica che sia possibile modificare la
			CPT associata ad un nodo della rete.  & {NI}\\
			\hline
			{TS-7.2.2.1} & {R1F7.2.2.1} & Verifica che sia possibile aggiungere uno
			stato alla CPT associata ad un nodo
			della rete.
  & {NI}\\
			\hline
			{TS-7.2.2.2} & {R1F7.2.2.2} & Verifica che sia possibile eliminare uno
			stato dalla CPT associata ad un nodo
			della rete.  & {NI}\\
			\hline
			{TS-7.2.2.3} & {R1F7.2.2.3} & Verifica che sia possibile modificare i
			parametri associati ad uno stato della
			CPT associata ad un nodo della rete.  & {NI}\\
			\hline
			{TS-7.2.2.4} & {R1F7.2.2.4} & Verifica che sia possibile modificare una
			cella della CPT.  & {NI}\\
			\hline
			{TS-7.3} & {R1F7.3} & Verifica che sia possibile eliminare un
			nodo dalla rete.  & {NI}\\
			\hline
			{TS-7.4} & {R1F7.4} & Verifica che sia possibile creare un
			collegamento tra due nodi della rete.  & {NI}\\
			\hline
			{TS-7.4.1} & {R1F7.4.1} & Verifica che sia possibile indicare il nodo
			di partenza del collegamento.  & {NI}\\
			\hline
			{TS-7.4.2} & {R1F7.4.2} & Verifica che sia possibile indicare il nodo
			di arrivo del collegamento.  & {NI}\\
			\hline
			{TS-7.5} & {R1F7.5} & Verifica che sia possibile eliminare un
			collegamento dalla rete.  & {NI}\\
			\hline
			{TS-7.6} & {R1F7.6} & Verifica che sia possibile salvare la rete su
			file JSON.  & {NI}\\
			\hline
			{TS-7.6.1} & {R1F7.6.1} & Verifica che sia possibile indicare il nome
			del file JSON su cui si vuole salvare la
			struttura della rete.  & {NI}\\
			\hline
			{TS-7.6.2} & {R1F7.6.2} & Verifica che sia possibile indicare il
			percorso del file system in cui si vuole
			salvare il file JSON contenente la
			struttura della rete.  & {NI}\\
			\hline
			{TS-7.7} & {R1F7.7} & Verifica che sia possibile gestire errori
			relativi alla modifica di un nodo.  & {NI}\\
			\hline
			{TS-7.7.1} & {R1F7.7.1} & Verifica che sia  possibile gestire
			l'inserimento di valori non validi per il
			nome di un nodo.  & {NI}\\
			\hline
			{TS-7.7.2} & {R1F7.7.2} & Verifica che sia possibile gestire
			l'inserimento di valori non validi per il
			nome di uno stato associato alla CPT
			di un nodo della rete  & {NI}\\
			\hline
			{TS-7.7.3} & {R1F7.7.3} & Verifica che sia possibile gestire
			l'inserimento di valori non validi per
			l'intervallo associato ad uno stato del
			nodo.
  & {NI}\\
			\hline
			{TS-7.7.4} & {R1F7.7.4} & Verifica che sia e possibile gestire
			l'inserimento di valori non validi per
			una cella della tabella.  & {NI}\\
			\hline
			{TS-8} & {R2F8} & Verifica che sia possibile applicare più reti
			Bayesiane in oggetti di monitoraggio
			diversi.  & {NI}\\
			\hline
			{TS-} & {RF} & Verifica che sia   & {NI}\\
			\hline
			{TS-} & {RF} & Verifica che sia   & {NI}\\
			\hline
			{TS-} & {RF} & Verifica che sia   & {NI}\\
			\hline
	\caption{Riassunto dei test di sistema}
	\label{tabella:riassunto TS-}
\end{longtable}

\renewcommand{\arraystretch}{1}
\subsubsection{Test di integrazione}
Questa sezione verrà sviluppata in futuro, quando sorgerà la necessità di definire dei test di integrazione per il prodotto.
\subsubsection{Test di unità}
Questa sezione verrà sviluppata in futuro, quando sorgerà la necessità di definire dei test di unità per il prodotto.



\pagebreak