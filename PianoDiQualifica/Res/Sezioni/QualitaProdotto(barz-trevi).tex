\newsection{Qualità di prodotto}
Per poter garantire che il prodotto realizzato sia di alta qualità, è necessario definire un modello per la valutazione di quest'ultima; il team 7DOS, per questo motivo, ha scelto di adottare il modello di qualità delineato nello standard ISO/IEC 25010, anche noto come \gl{SQuaRE}.

Tale modello comprende 8 caratteristiche (ciascuna divisa in sotto-caratteristiche, per un totale di 31) che vanno prese in considerazione durante lo sviluppo del progetto per garantire un'elevata qualità complessiva del prodotto finale.

Per praticità e rilevanza ai fini del prodotto, sono state selezionate 6 caratteristiche da considerare e per ciascuna sono state individuate le sotto caratteristiche più rilevanti al progetto. 

Tutte le metriche per la valutazione della qualità elencate nel capitolo successivo saranno basate sui seguenti principi cardine.

\subsection{Functional Suitability}
Questa caratteristica esprime il grado di soddisfacimento dei requisiti espliciti ed impliciti da parte di un prodotto o servizio, quando utilizzato sotto determinate condizioni.
\subsubsection{Sotto-caratteristiche notevoli}
In riguardo al prodotto che si intende realizzare, sono state individuate le seguenti sotto-caratteristiche da perseguire come obiettivi prioritari: 
\begin{itemize}
	\item{\textbf{Functional Completeness}}: esprime il grado in cui l'insieme di funzioni copre i compiti specificati e gli obiettivi dell'utente;
	\item{\textbf{Functional Correctness}}: esprime il grado in cui il prodotto restituisce risultati corretti, entro il livello di precisione desiderato.
\end{itemize}
\subsubsection{Metriche adottate} 
\paragraph{Functional Implementation Completeness} ~\\ ~\\
Misurazione in percentuale del grado in cui le funzionalità offerte dalla corrente implementazione del software coprono l'insieme di funzioni specificate nei requisiti.\\
Questa metrica è stata scelta per valutare il grado di completezza del prodotto; l'obiettivo è implementare tutte le funzionalità richieste.\\
Viene utilizzata la seguente formula:
$$FI_{Comp}=\frac{NF_i}{NF_r}*100$$
dove FI\textsubscript{Comp} è il valore della metrica, NF\textsubscript{i} è numero di funzioni attualmente implementate e NF\textsubscript{r} è numero di funzioni specificate dai requisiti.
\begin{itemize}
	\item{\textbf{Intervallo possibile}: 0-100;}
	\item{\textbf{Intervallo accettabile}: 75-100;}
	\item{\textbf{Intervallo obiettivo}: 100.}
\end{itemize}

\paragraph{Average Functional Implementation Correctness} ~\\ ~\\
Misurazione in percentuale del grado in cui le funzionalità offerte dalla corrente implementazione del software, in media, rispettano il livello di precisione indicato nei requisiti.\\
Questa metrica è stata scelta per valutare il grado di accuratezza e garantire la qualità dei risultati restituiti dal prodotto, in quanto andrà a fare previsioni sulla verosimiglianza di alcuni eventi in base ai dati forniti, ed è necessario che tali previsioni siano sufficientemente accurate.\\
Viene utilizzata la seguente formula:
$$aFI_{Corr}=\frac{\sum\limits_{i=1}^N\frac{iPF_i}{rPF_i}}{N}*100$$
dove aFI\textsubscript{Corr} è il valore della metrica, iPF\textsubscript{i} è il livello di precisione della i-esima funzione implementata, rPF\textsubscript{i} è il livello di precisione della i-esima funzione secondo i requisiti, e N è il numero totale di funzioni considerate.
\begin{itemize}
	\item{\textbf{Intervallo possibile}: 0-100;}
	\item{\textbf{Intervallo accettabile}: 80-100;}
	\item{\textbf{Intervallo obiettivo}: 95-100.}
\end{itemize}

\subsection{Performance Efficiency}
Definisce le prestazioni relative al sistema come, quantità di risorse utilizzate per eseguire una determinata funzionalità del sistema sotto delle specifiche condizioni. 
\subsubsection{Sotto-caratteristiche notevoli}
\begin{itemize}
	\item{\textbf{Time Behaviour}}: esprime il grado in cui i tempi di risposta ed elaborazione e i volumi di produzione di un prodotto o sistema, durante l'esecuzione delle sue funzionalità, rispettano i requisiti.
	\item{\textbf{Resource Utilization}}: esprime il grado in cui il numero e tipo di risorse utilizzate da un prodotto o sistema, durante l'esecuzione delle sue funzionalità, rispetta i requisiti.
	\item{\textbf{Capacity}}: esprime il grado in cui i limiti massimi di un prodotto o sistema rispettano quelli definiti dai requisiti.
\end{itemize}
\subsubsection{Metriche adottate}
\paragraph{Average Response Time}
\begin{flushleft}
Tempo medio richiesto per completare l'esecuzione di una funzionalità richiesta ed esporre il risultato
$$ARM=\frac{\sum\limits_{i=1}^N TF{\small{i}}}{NF}$$
Dove:
\begin{itemize}
\item{\textbf{ARM}} :Average Response Time
\item{\textbf{TF${\small{i}}$}}: Tempo trascorso dal momento in cui viene richiesta l'esecuzione dell'i-esima funzionalita' fino al momento in cui la funzionalità viene effettivamente completata ed il risultato viene esposto
\item{\textbf{NF}}: Numero di funzionalità eseguite durante una singola esecuzione del prodotto software
\end{itemize}
\textbf{Range Ottimale}:0.3s-0.5s \\
\textbf{Range Accettabile}:0.5s-1.0s
\end{flushleft}
\paragraph{Average CPU Usage}
\begin{flushleft}
Percentuale media di utilizzo del processore durante una singola esecuzione del prodotto SW(Non so come rappresentarla sotto forma di formula)
\end{flushleft}
\paragraph{Average Memory Usage}
\begin{flushleft}
Percentuale media di utilizzo della Memoria RAM durante una singola esecuzione del prodotto SW(Non so come rappresentarla sotto forma di formula)
\end{flushleft}

\subsection{Usability}
Questa caratteristica esprime il grado in cui un prodotto o sistema può essere usato da un determinato utente per raggiungere determinati scopi con efficacia, efficienza e soddisfazione in uno specifico contesto d'uso.
\subsubsection{Sotto-caratteristiche notevoli}
In riguardo al prodotto che si intende realizzare, sono state individuate le seguenti sotto-caratteristiche da perseguire come obiettivi prioritari.
\begin{itemize}
	\item{\textbf{Learnability}}: esprime il grado in cui un prodotto o sistema può essere usato da determinati utenti per raggiungere determinati obiettivi di imparare ad usare il prodotto o sistema  con efficacia, efficienza, sicurezza da rischi e soddisfazione in uno specifico contesto d'uso.
	\item{\textbf{Operability}}: esprime il grado in cui un prodotto o sistema ha attributi che lo rendono facile da operare e controllare.
	\item{\textbf{User Error Protection}}: esprime il grado in cui un sistema protegge gli utenti dal commettere errori.
\end{itemize}
\subsubsection{Metriche adottate}
\paragraph{Average Learning Time}  ~\\ ~\\
Misurazione in minuti del tempo medio impiegato da un utente per imparare ad utilizzare una singola funzionalità del prodotto.
\\Questa metrica è stata scelta poiché, trattandosi di un prodotto che verrà reso disponibile pubblicamente, è stato ritenuto importante renderlo semplice da imparare per permetterne l'uso ad una vasta gamma di utenti.
\\Viene utilizzata la seguente formula:
$$aLT=\frac{\sum\limits_{i=1}^N{LT_i}}{N}$$
dove aLT è il valore della metrica, LT\textsubscript{i} è il tempo necessario ad imparare ad utilizzare la i-esima funzione implementata, espresso in minuti, e N è il numero totale di funzioni considerate.
\begin{itemize}
	\item{\textbf{Intervallo possibile}: 0-$\infty$;}
	\item{\textbf{Intervallo accettabile}: 0-30;}
	\item{\textbf{Intervallo obiettivo}: 0-15.}
\end{itemize}

\subsection{Reliability}
Questa caratteristica esprime il grado in cui un sistema, prodotto o componente esegue determinate funzioni sotto determinate condizioni per un dato periodo di tempo.
\subsubsection{Sotto-caratteristiche notevoli}
In riguardo al prodotto che si intende realizzare, sono state individuate le seguenti sotto-caratteristiche da perseguire come obiettivi prioritari: 
\begin{itemize}
	\item{\textbf{Maturity}}: esprime il grado in cui un sistema, prodotto o componente raggiunge i requisiti di affidabilità in normali condizioni operative;
	\item{\textbf{Fault Tolerance}}: esprime il grado in cui un sistema, prodotto o componente opera come previsto nonostante la presenza di malfunzionamenti hardware o software.
\end{itemize}
\subsubsection{Metriche adottate}
\paragraph{Failure Density}  ~\\ ~\\
Misurazione in percentuale della quantità di test falliti rispetto alla quantità di test eseguiti.
\\Questa metrica è stata scelta per garantire che il prodotto sia generalmente stabile e non risulti poco utilizzabile o inutilizzabile a causa di eccessive failure. Valutazioni più precise saranno effettuate in base ai singoli risultati dei test.\\
Viene utilizzata la seguente formula:
$$FD=\frac{T_f}{T_c}*100$$
dove FD è il valore della metrica, T\textsubscript{f} è il numero di test falliti e T\textsubscript{e} è il numero di test eseguiti.
\begin{itemize}
	\item{\textbf{Intervallo possibile}: 0-100;}
	\item{\textbf{Intervallo accettabile}: 0-10;}
	\item{\textbf{Intervallo obiettivo}: 0.}
\end{itemize}

\subsection{Maintainability}
Definisce il grado di efficacia ed efficienza con cui un prodotto o un sistema può essere modificato per migliorarlo, correggerlo o adattarlo a dei cambiamenti all'ambiente. 
\subsubsection{Sotto-caratteristiche notevoli}
\begin{itemize}
	\item{\textbf{Modularity}}:Grado di scomposizione del sistema in parti minimali tali che un cambiamento ad uno specifica componente ha il minimo impatto su tutte le altri componenti.
	\item{\textbf{Reusability}}:Grado con cui una determinata componente del sistema può essere adattata ed utilizzata in altri sistemi o per realizzare altre componenti.
	\item{\textbf{Analisability}}:Grado di efficacia ed efficienza con cui e' possibile analizzare l'impatto nel sistema di uno specifico cambiamento ad una o più delle sue parti, ai fini di rilevare eventuali casi di fallimento.
	\item{\textbf{Modificability}}:Grado con il quale un prodotto o sistema può essere modificato efficacemente ed efficientemente senza introdurre difetti che ne possano intaccare la qualita' complessiva.
	\item{\textbf{Testability}}:Grado di efficacia ed efficienza con cui e' possibile stabilire ed eseguire dei test per valutare la qualità del sistema
\end{itemize}
\subsubsection{Metriche Adottate}
\paragraph{Average Time Spent per Feature Development}
\begin{flushleft}
Tempo medio speso dall'intero team di sviluppo per lo sviluppo di nuove funzionalita'.
	$$ATF=\frac{\sum\limits_{i=1}^N TF{\small{i}}}{NF}$$
Dove
\begin{itemize}
	\item{\textbf{ATF}}:Average time spent per feature development
	\item{\textbf{TF${\small{i}}$}}:Tempo uomo speso dall'intero team di sviluppo per lo sviluppo dell'i-esima funzionalita'.
	\item{\textbf{NU}}:Numero di funzionalità sviluppate
\end{itemize}
\textbf{Range Ottimale}:3h-4h 30m \\
\textbf{Range Accettabile}:4h 30m-5h 30m
\end{flushleft}
\paragraph{Average time spent per error correction}
\begin{flushleft}
Tempo medio speso dall'intero team di sviluppo per la risoluzione di una singola funzionalità.
	$$ATE=\frac{\sum\limits_{i=1}^N TE{\small{i}}}{NE}$$
Dove
\begin{itemize}
	\item{\textbf{ATE}}:Average time spent per error correction
	\item{\textbf{TE${\small{i}}$}}:Tempo uomo speso per risolvere l'i-esimo errore rilevato
	\item{\textbf{NE}}:Numero complessivo di errori rilevati
\end{itemize}
\textbf{Range Ottimale}:45m-1h 30m \\
\textbf{Range Accettabile}:1h 30m-2h 30m
\end{flushleft}
\paragraph{Average time spent per unit testing}
\begin{flushleft}
Tempo medio speso dall'intero team di sviluppo per la scrittura di test corretti e completi per una singola unita'.
	$$ATU=\frac{\sum\limits_{i=1}^N TU{\small{i}}}{NU}$$
Dove
\begin{itemize}
	\item{\textbf{ATU}}:Average time spent per unit testing
	\item{\textbf{TU${\small{i}}$}}:Tempo uomo speso dall'intero team di sviluppo per la scrittura di test corretti e completi per l'i-esima unita'.
	\item{\textbf{NU}}:Numero unita' testate
\end{itemize}
\textbf{Range Ottimale}:1h 30m-2h \\
\textbf{Range Accettabile}:2h-2h 30m
\end{flushleft}
\pagebreak