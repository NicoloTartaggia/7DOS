\newsection{Qualità di prodotto}
Per la valutazione della qualita' del prodotto il team 7DOS ha deciso di fare affidamento allo standard ISO 25010.
Quest' ultimo definisce 8 caratteristiche e 31 sotto caratteristiche che devono essere prese in considerazione durante lo sviluppo del progetto, per garantire un  elevata qualita' complessiva del prodotto.
Per questione di praticita' si e' deciso di adottare solamente un sottoinsieme ristretto delle 8 macrocaratteristiche, più nello specifico le 6 elencate di seguito.
Tutte le metriche per la valutazione della qualita' del prodotto adottate dal team 7DOS, elencate nel capitolo successivo saranno basate sui seguenti principi cardine.
\subsection{Functional Suitability} (trevisin)
Questa caratteristica esprime il grado di soddisfacimento dei requisiti espliciti ed impliciti da parte di un prodotto o servizio, quando utilizzato sotto determinate condizioni.
\subsubsection{Sottocaratteristiche notevoli}
In riguardo al prodotto che si intende realizzare, sono state individuate le seguenti sottocaratteristiche da perseguire come obiettivi prioritari: 
\begin{itemize}
	\item{\textbf{Functional Completeness}}: esprime il grado in cui l'insieme di funzioni copre i compiti specificati e gli obiettivi dell'utente;
	\item{\textbf{Functional Correctness}}: esprime il grado in cui il prodotto restituisce risultati corretti, entro il livello di precisione desiderato.
\end{itemize}

\subsection{Performance Efficiency}
Definisce le prestazioni relative al sistema come, quantita' di risorse utilizzate per eseguire una determinata funzionalita' del sistema sotto delle specifiche condizioni. 
\subsubsection{Sottocaratteristiche notevoli}
\begin{itemize}
	\item{\textbf{Time behaviour}}:Grado con il quale i tempi di risposta del sistema ad una determinata azione, ed i tempi di processamento della funzionalita' correlata, rispettano i requisiti.
	\item{\textbf{Resource utilization}}:Grado con il quale il numero di risorse utilizzate dal sistema durante l'esecuzione delle sue funzionalita' rispetta i requisiti.
	\item{\textbf{Capacity}}:Grado con il quale i limiti massimi dettati dal sistema rispettano quelli definiti dai requisiti.
\end{itemize}
\subsubsection{Metriche adottate}
\paragraph{Average Response Time}
Tempo medio richiesto per completare l'esecuzione di una funzionalita' richiesta ed esporre il risultato
$$ARM=\frac{\sum\limits_{i=1}^N TF{\small{i}}}{NF}$$
Dove
\begin{itemize}
\item{\textbf{ARM}}:Average Response Time
\item{\textbf{TF${\small{i}}$}}:Tempo trascorso dal momento in cui viene richiesta l'esecuzione dell'i-esima funzionalita' fino al momento in cui la funzionalita' viene effettivamente completata ed il risultato viene esposto
\item{\textbf{NF}}:Numero di funzionalita' eseguite durante una singola esecuzione del prodotto software
\end{itemize}
\paragraph{Average CPU Usage}
Percentuale media di utilizzo del processore durante una singola esecuzione del prodotto SW(Non so come rappresentarla sotto forma di formula)
\paragraph{Average Memory Usage}
Percentuale media di utilizzo della Memoria RAM durante una singola esecuzione del prodotto SW(Non so come rappresentarla sotto forma di formula)


\subsection{Usability} (trevisin)
Questa caratteristica esprime il grado in cui un prodotto o sistema può essere usato da un determinato utente per raggiungere determinati scopi con efficacia, efficienza e soddisfazione in uno specifico contesto d'uso.
\subsubsection{Sottocaratteristiche notevoli}
In riguardo al prodotto che si intende realizzare, sono state individuate le seguenti sottocaratteristiche da perseguire come obiettivi prioritari:
\begin{itemize}
	\item{\textbf{Learnability}}: esprime il grado in cui un prodotto o sistema può essere usato da determinati utenti per raggiungere determinati obiettivi di imparare ad usare il prodotto o sistema  con efficacia, efficienza, sicurezza da rischi e soddisfazione in uno specifico contesto d'uso.
	\item{\textbf{Operability}}: esprime il grado in cui un prodotto o sistema ha attributi che lo rendono facile da operare e controllare.
	\item{\textbf{User Error Protection}}: esprime il grado in cui un sistema protegge gli utenti dal commettere errori.
	.
\end{itemize}

\subsection{Reliability} (trevisin)
Questa caratteristica esprime il grado in cui un sistema, prodotto o componente esegue determinate funzioni sotto determinate condizioni per un dato periodo di tempo.
\subsubsection{Sottocaratteristiche notevoli}
In riguardo al prodotto che si intende realizzare, sono state individuate le seguenti sottocaratteristiche da perseguire come obiettivi prioritari: 
\begin{itemize}
	\item{\textbf{Maturity}}: esprime il grado in cui un sistema, prodotto o componente raggiunge i requisiti di affidabilità in normali condizioni operative;
	\item{\textbf{Fault Tolerance}}: esprime il grado in cui un sistema, prodotto o componente opera come previsto nonostante la presenza di malfunzionamenti hardware o software.
	\item{\textbf{Recoverability???}}: non so se considerare anche questa caratteristica.
\end{itemize}

\subsection{Maintainability}
Definisce il grado di efficacia ed efficenza con cui un prodotto od un sistema può essere modificato per migliorarlo, correggerlo od adattarlo a dei cambiamenti all'ambiente. 
\subsubsection{Sottocaratteristiche notevoli}
\begin{itemize}
	\item{\textbf{Modularity}}:Grado di scomposizione del sistema in parti minimali tali che un cambiamento ad uno specifica componente ha il minimo impatto su tutte le altri componenti.
	\item{\textbf{Reusability}}:Grado con cui una determinata componente del sistema può essere adattata ed utilizzata in altri sistemi o per realizzare altre componenti.
	\item{\textbf{Analisablity}}:Grado di efficacia ed efficienza con cui e' possibile analizzare l'impatto nel sistema di uno specifico cambiamento ad una o più delle sue parti, ai fini di rilevare eventuali casi di fallimento.
	\item{\textbf{Modificability}}:Grado con il quale un prodotto o sistema può essere modificato efficaciemente ed efficentemente senza introdurre difetti che ne possano intaccare la qualita' complessiva.
	\item{\textbf{Testability}}:Grado di efficacia ed efficenza con cui e' possibile stabilire ed eseguire dei test per valutare la qualita' del sistema
\end{itemize}
\subsubsection{Metriche Adottate}
\paragraph{Average time spent per error}
Tempo medio speso dall'intero team di sviluppo per la risoluzione di una singola funzionalità.
	$$ATE=\frac{\sum\limits_{i=1}^N TTE{\small{i}}}{NE}$$
Dove
\begin{itemize}
	\item{\textbf{ATE}}:Average time spent per error
	\item{\textbf{TTE${\small{i}}$}}:Tempo complessivo speso dall'intero team nella risoluzione di errori
	\item{\textbf{NE}}:Numero complessivo di errori rilevati
\end{itemize}
\subsection{Portability}
Grado di efficacia ed efficenza con cui un sistema, prodotto o componente può essere trasferito da una specifica piattaforma hardware o software ad un altro ambiente completamente differente.
\subsubsection{Sottocaratteristiche notevoli}
\begin{itemize}
	\item{\textbf{Adaptability}}:Grado di efficacia ed efficenza con cui un prodotto può essere adattato a hardware diverso.
	\item{\textbf{Installability}}:Grado di efficacia ed efficenza con un prodotto può essere installato o disinstallato da uno specifico ambiente.
	\item{\textbf{Replaceability}}:Grado di efficacia ed efficenza con cui un prodotto e'in grado di rimpiazzare un altro prodotto con lo stesso scopo.
\end{itemize}
\pagebreak