\subsection{UC5: Gestione degli alert}
\hypertarget{UC5}{}
\begin{figure} [H]
 	\centering
 	\includegraphics[scale=0.45]{Img/UC5}
 	\caption{UC5: Gestione degli alert}\label{}
\end{figure}
\begin{itemize}
	\item \textbf{Attori}: Utente;
	\item \textbf{Descrizione}: L'attore configura le opzioni relative ad alert personalizzati;
	\item \textbf{Precondizione}: Il plug-in deve leggere un flusso di dati;
	\item \textbf{Flusso principale degli eventi}:
		\begin{itemize}
			\item Configurazione dei parametri (UC5.1);
			\item Gestione delle notifiche (UC5.2).
		\end{itemize}
	\item \textbf{Postcondizione}: Gli alert e le notifiche di attivazione sono stati configurati.
\end{itemize}

\subsection{UC5.1: Configurazione dei parametri}
\hypertarget{UC5.1}{}
\begin{figure} [H]
	\centering
	\includegraphics[scale=0.45]{Img/UC5-1}
	\caption{UC5.1: Configurazione dei parametri}\label{}
\end{figure}
\begin{itemize}
	\item \textbf{Attori}: Utente;
	\item \textbf{Descrizione}: L'attore configura i parametri che definiscono un alert;
	\item \textbf{Precondizione}: Il sistema permette la configurazione dei parametri di un alert;
	\item \textbf{Flusso principale degli eventi}:
		\begin{itemize}
			\item Inserimento del nome (UC5.1.1);
			\item Inserimento intervallo di notifica (UC5.1.2);
			\item Inserimento della condizione di attivazione (UC5.1.3).
		\end{itemize}
	\item \textbf{Postcondizione}: L'alert è stato configurato.
\end{itemize}

\subsection{UC5.1.1: Inserimento del nome}
\hypertarget{UC5.1.1}{}
\begin{itemize}
	\item \textbf{Attori}: Utente;
	\item \textbf{Descrizione}: L'attore inserisce il nome dell'alert;
	\item \textbf{Precondizione}: Il sistema permette l'inserimento del nome di un alert;
	\item \textbf{Postcondizione}: Il nome dell'alert è stato inserito.
\end{itemize}

\subsection{UC5.1.2: Inserimento intervallo di verifica}
\hypertarget{UC5.1.2}{}
\begin{itemize}
	\item \textbf{Attori}: Utente;
	\item \textbf{Descrizione}: L'attore inserisce l'intervallo di verifica ed una eventuale durata minima della condizione di attivazione per notificare l'alert;
	\item \textbf{Precondizione}: Il sistema permetta l'inserimento dell'intervallo di verifica di un alert;
	\item \textbf{Postcondizione}: L'intervallo di verifica dell'alert è stato inserito.
\end{itemize}

\subsection{UC5.1.3: Inserimento della condizione di attivazione}
\hypertarget{UC5.1.3}{}
\begin{itemize}
	\item \textbf{Attori}: Utente;
	\item \textbf{Descrizione}: L'attore inserisce la condizione necessaria per l'attivazione dell'alert;
	\item \textbf{Precondizione}: Il sistema permette l'inserimento di una condizione di attivazione di un alert;
	\item \textbf{Postcondizione}: La condizione di attivazione dell'alert è stata inserita.
\end{itemize}

\subsection{UC5.2: Gestione delle notifiche}
\hypertarget{UC5.2}{}
\begin{itemize}
	\item \textbf{Attori}: Utente;
	\item \textbf{Descrizione}: L'attore seleziona il modo in cui viene notificata l'attivazione di un alert;
	\item \textbf{Precondizione}: Il sistema permette di notificare l'attivazione di un alert;
	\item \textbf{Postcondizione}: Il modo in cui viene notificata l'attivazione di un alert è stato selezionato.
\end{itemize}
