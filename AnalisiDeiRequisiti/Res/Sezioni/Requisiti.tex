\newsection{Requisiti}
\newcolumntype{H}{>{\centering\arraybackslash}m{7cm}}
\subsection{Requisiti Funzionali}
\normalsize
\definecolor{dosblue}{rgb}{0.88,1,1}
\begin{longtable}{|c|H|c|}
	\hline
	\rowcolor{dosblue}
	\textbf{Id Requisito} & \textbf{Descrizione} & \textbf{Fonte}\\
	\hline
	\endhead
	\hypertarget{R0F1}{R0F1} & Deve essere possibile leggere la definizione della rete Bayesiana da un file in formato json. & Capitolato \\ \hline 
	\hypertarget{R1F1.1}{R1F1.1} & Deve essere possibile verificare che il file .json sia valido. & Interno \\ \hline 
	\hypertarget{R0F2}{R0F2} & Deve essere possibile gestire la connessione tra i nodi della rete ai rispettivi flussi di dati. & Capitolato \\ \hline 
	\hypertarget{R1F2.1}{R1F2.1} & Deve essere possibile connettere un nodo della rete ad un flusso di dati. & Capitolato \\ \hline 
	\hypertarget{R1F2.2}{R1F2.2} & Deve essere possibile disconnettere un nodo della rete da un flusso di dati. & Capitolato \\ \hline 
	\hypertarget{R1F2.3}{R1F2.3} & Deve essere possibile modificare il flusso di dati connesso ad un nodo. & Capitolato \\ \hline 
	\hypertarget{R0F3}{R0F3} & Deve essere possibile applicare il ri-calcolo delle probabilità della rete secondo regole temporali prestabilite. & Capitolato \\ \hline 
	\hypertarget{R1F3.1}{R1F3.1} & Deve essere possibile modificare le suddette regole temporali. & Interno \\ \hline 
	\hypertarget{R0F4}{R0F4} & Deve essere possibile fornire nuovi dati al sistema di Grafana derivati dai nodi della rete non collegati al flusso di monitoraggio. & Capitolato \\ \hline 
	\hypertarget{R1F4.1}{R1F4.1} & Deve essere possibile aggiornare i dati in base alla frequenza stabilita. & Interno \\ \hline 
	\hypertarget{R0F5}{R0F5} & Rendere disponibili i dati al sistema di creazione di grafici e dashboard per la loro visualizzazione. & Capitolato \\ \hline 
	\hypertarget{R1F5.1}{R1F5.1} & Deve essere possibile aggiornare la dashboard in base alla frequenza stabilita. & Interno \\ \hline 
	\hypertarget{R1F5.2}{R1F5.2} & Deve essere possibile creare un panel. & Interno \\ \hline 	\hypertarget{R1F5.3}{R1F5.3} & Deve essere possibile spostare un panel. & Interno \\ \hline 	
	\hypertarget{R1F5.4}{R1F5.4} & Deve essere possibile cancellare  un panel. & Interno \\ \hline 	
	\hypertarget{R1F5.5}{R1F5.5} & Deve essere possibile minimizzare  un panel. & Interno \\ \hline 	
	\hypertarget{R1F5.6}{R1F5.6} & Deve essere possibile configurare un panel. & Interno \\ \hline 	
	\hypertarget{R1F5.6.1}{R1F5.6.1} & Deve essere possibile selezionare un flusso dati. & Interno \\ \hline 	
	\hypertarget{R1F5.6.2}{R1F5.6.2} & Deve essere possibile selezionare un nodo della rete. & Interno \\ \hline 	
	\hypertarget{R1F5.6.3}{R1F5.6.3} & Deve essere possibile selezionare un intervallo di tempo. & Interno \\ \hline 	
	\hypertarget{R1F5.7}{R1F5.7} & Deve essere possibile modificare un panel. & Interno \\ \hline 	
	\hypertarget{R1F5.7.1}{R1F5.7.1} & Deve essere possibile usare le modifiche standard di Grafana su un panel. & Interno \\ \hline 
	\hypertarget{R1F6}{R1F6} & Deve essere possibile definire alert in base a livelli di soglia raggiunti dai nodi non collegati al flusso dei dati. & Capitolato \\ \hline 
	\hypertarget{R1F6.1}{R1F6.1} & Deve essere possibile configurare i parametri di un alert. & Interno \\ \hline 
	\hypertarget{R1F6.1.1}{R1F6.1.1} & Deve essere possibile inserire il nome di un alert. & Interno \\ \hline 
	\hypertarget{R1F6.1.2}{R1F6.1.2} & Deve essere possibile inserire l’intervallo di verifica di un alert. & Interno \\ \hline 
	\hypertarget{R1F6.1.3}{R1F6.1.3} & Deve essere possibile inserire la condizione di attivazione di un alert. & Interno \\ \hline 
	\hypertarget{R1F6.2}{R1F6.2} & Deve essere possibile impostare il modo in cui viene notificata l’attivazione di un alert. & Interno \\ \hline 
	\hypertarget{R1F7}{R1F7} & Deve essere possibile disegnare la rete Bayesiana con un piccolo editor grafico specializzato. & Capitolato \\ \hline 
	\hypertarget{R1F7.1}{R1F7.1} & Deve essere possibile creare un nodo all’interno della rete. & Interno \\ \hline 
	\hypertarget{R1F7.2}{R1F7.2} & Deve essere possibile modificare la tabella delle probabilità condizionate associata ad un nodo della rete. & Interno \\ \hline 
	\hypertarget{R1F7.3}{R1F7.3} & Deve essere possibile eliminare un nodo dalla rete. & Interno \\ \hline 
	\hypertarget{R1F7.4}{R1F7.4} & Deve essere possibile creare un collegamento tra due nodi della rete. & Interno \\ \hline 
	\hypertarget{R1F7.5}{R1F7.5} & Deve essere possibile eliminare un collegamento tra due nodi della rete. & Interno \\ \hline 
	\hypertarget{R1F7.6}{R1F7.6} & Deve essere possibile salvare la rete su un file .json. & Interno \\ \hline 
	\hypertarget{R2F8}{R2F8} & Deve essere possibile applicare più reti Bayesiane in oggetti di monitoraggio diversi. & Capitolato \\ \hline 
	\hypertarget{R2F9}{R2F9} & Deve essere possibile creare una rete Bayesiana a partire dai dati raccolti sul campo anziché svilupparla con la collaborazione degli esperti del settore. & Capitolato \\ \hline 
	\hypertarget{R2F10}{R2F10} & Deve essere possibile condividere un grafico. & Interno \\ \hline 
	\hypertarget{R2F10.1}{R2F10.1} & Deve essere possibile visualizzare il link diretto ad una dashboard o ad un panel. & Interno \\ \hline 
	\hypertarget{R2F10.2}{R2F10.2} & Deve essere possibile visualizzare visualizzare il codice per l’inclusione di un panel in una pagina web. & Interno \\ \hline 
	\hypertarget{R2F10.3}{R2F10.3} & Deve essere possibile selezionare le opzioni di visualizzazione per la condivisione dei grafici. & Interno \\ \hline 
	\hypertarget{R2F10.3.1}{R2F10.3.1} & Deve essere possibile selezionare la visualizzazione dell’intervallo di tempo corrente in un grafico. & Interno \\ \hline 
	\hypertarget{R2F10.3.2}{R2F10.3.2} & Deve essere possibile deselezionare la visualizzazione dell’intervallo di tempo corrente in un grafico. & Interno \\ \hline 
	\hypertarget{R2F10.3.3}{R2F10.3.3} & Deve essere possibile selezionare la visualizzazione di variabili di template in un grafico. & Interno \\ \hline 
	\hypertarget{R2F10.3.4}{R2F10.3.4} &  Deve essere possibile deselezionare la visualizzazione di variabili di template in un grafico. & Interno \\ \hline 
	\hypertarget{R2F10.3.5}{R2F10.3.5} & Deve essere possibile selezionare il tema di un grafico. & Interno \\ \hline 
	\hypertarget{R2F10.4}{R2F10.4} & Deve essere possibile condividere uno snapshot di una dashboard o di un panel. & Interno \\ \hline 
	\hypertarget{R2F10.4.1}{R2F10.4.1} & Deve essere possibile pubblicare uno snapshot sull’istanza locale dell’utente. & Interno \\ \hline 
	\hypertarget{R2F10.4.2}{R2F10.4.2} & Deve essere possibile pubblicare uno snapshot su Raintank. & Interno \\ \hline 
	\hypertarget{R2F10.4.3}{R2F10.4.3} & Deve essere possibile configurare le opzioni di visualizzazione di uno snapshot. & Interno \\ \hline 
	\hypertarget{R2F10.4.3.1}{R2F10.4.3.1} & Deve essere possibile inserire il nome di uno snapshot. & Interno \\ \hline 
	\hypertarget{R2F10.4.3.2}{R2F10.4.3.2} & Deve essere possibile selezionare il tempo di permanenza di uno snapshot. & Interno \\ \hline 
	\hypertarget{R2F10.4.3.3}{R2F10.4.3.3} & Deve essere possibile inserire il tempo massimo per il caricamento dei dati in uno snapshot. & Interno \\ \hline 
	
	\caption[Requisiti Funzionali]{Requisiti Funzionali}
	\label{tabella:req0}
\end{longtable}
\clearpage
\newcolumntype{H}{>{\centering\arraybackslash}m{7cm}}
\subsection{Requisiti di Qualità}
\normalsize
\begin{longtable}{|c|H|c|}
	\hline
	\rowcolor{dosblue}
	\textbf{Id Requisito} & \textbf{Descrizione} & \textbf{Fonte}\\
	\hline
	\endhead
	\hypertarget{R0Q1}{R0Q1} & La progettazione e il codice devono seguire le norme riportate nel documento allegato Norme di Progetto. & Interno \\ \hline 
	\hypertarget{R0Q2}{R0Q2} & Tutti i documenti e il codice prodotto devono rispettare le metriche riportate nel documento Piano di Qualifica & Interno \\ \hline 
	\hypertarget{R0Q3}{R0Q3} & Deve essere prodotto un manuale utente & Interno \\ \hline 
	\hypertarget{R0Q4}{R0Q4} & Deve essere prodotto un manuale sviluppatore. & Interno \\ \hline 
	\caption[Requisiti di Qualità]{Requisiti Di Qualità}
	\label{tabella:req1}
\end{longtable}
\clearpage
\newcolumntype{H}{>{\centering\arraybackslash}m{7cm}}
\subsection{Requisiti di Vincolo}
\normalsize
\begin{longtable}{|c|H|c|}
	\hline
	\rowcolor{dosblue}
	\textbf{Id Requisito} & \textbf{Descrizione} & \textbf{Fonte}\\
	\hline
	\endhead
	\hypertarget{R2V1}{R2V1} & Identificare altri metodi di Intelligenza Artificiale oltre alla rete Bayesiana che siano applicabili all’analisi del flusso di dati di monitoraggio al fine di aiutare la linea di produzione del software a scoprire i punti migliorabili dalla propria procedura. & Capitolato  \\ \hline 
	\hypertarget{R0V2}{R0V2} & Il plugin deve utilizzare JavaScript. & Capitolato  \\ \hline 
	\hypertarget{R0V3}{R0V3} & Il codice sorgente prodotto deve essere rilasciato in un repository pubblico con licenza open source. & Interno  \\ \hline 
	\hypertarget{R1V4}{R1V4} & Il plugin può utilizzare la libreria Jbayes.js. & Capitolato  \\ \hline 
	\hypertarget{R0V5}{R0V5} & Il plugin deve essere utilizzabile nell’ambiente Grafana. & Capitolato  \\ \hline 
	\caption[Requisiti Di Vincolo]{Requisiti di Vincolo}
	\label{tabella:req3}
\end{longtable}
\clearpage



\subsection{Tracciamento Requisiti - Use case}
\normalsize
\begin{longtable}{|c|c|}
	\hline
\textbf{Codice Requisiti} & \textbf{Codice Use case} \\
\hline
\endhead
\hyperlink{R0F1}{R0F1} & \hyperlink{UC4}{UC4}\\
\hline
\hyperlink{R1F1.1}{R1F1.1} & \hyperlink{UC1.9}{UC1.9}\\
\hline
\hyperlink{R0F2}{R0F2} & \hyperlink{UC2.1}{UC2.1}\\
\hline
\hyperlink{R1F2.1}{R1F2.1} & \hyperlink{UC2.1.1}{UC2.1.1}\\
\hline
\hyperlink{R1F2.2}{R1F2.2} & \hyperlink{UC2.1.2}{UC2.1.2}\\
\hline
\hyperlink{R1F2.3}{R1F2.3} & \hyperlink{UC2.1.3}{UC2.1.3}\\
\hline
\hyperlink{R1F3.1}{R1F3.1} & \hyperlink{UC2.2}{UC2.2}\\
\hline
\hyperlink{R0F4}{R0F4} & \hyperlink{UC3.1}{UC3.1}\\
\hline
\hyperlink{R1F4.1}{R1F4.1} & \hyperlink{UC3.2}{UC3.2}\\
\hline
\hyperlink{R1F5.2}{R1F5.2} & \hyperlink{UC6.1}{UC6.1}\\
\hline
\hyperlink{R1F5.3}{R1F5.3} & \hyperlink{UC6.2}{UC6.2}\\
\hline
\hyperlink{R1F5.4}{R1F5.4} & \hyperlink{UC6.3}{UC6.3}\\
\hline
\hyperlink{R1F5.5}{R1F5.5} & \hyperlink{UC6.4}{UC6.4}\\
\hline
\hyperlink{R1F5.6}{R1F5.6} & \hyperlink{UC7.1}{UC7.1}\\
\hline
\hyperlink{R1F5.6.1}{R1F5.6.1} & \hyperlink{UC7.1.1}{UC7.1.1}\\
\hline
\hyperlink{R1F5.6.2}{R1F5.6.2} & \hyperlink{UC7.1.2}{UC7.1.2}\\
\hline
\hyperlink{R1F5.6.3}{R1F5.6.3} & \hyperlink{UC7.1.3}{UC7.1.3}\\
\hline
\hyperlink{R1F5.7}{R1F5.7} & \hyperlink{UC7.2}{UC7.2}\\
\hline
\hyperlink{R1F5.7.1}{R1F5.7.1} & \hyperlink{UC7.2.1}{UC7.2.1}\\
\hline
\hyperlink{R1F6}{R1F6} & \hyperlink{UC5}{UC5}\\
\hline
\hyperlink{R1F6.1}{R1F6.1} & \hyperlink{UC5.1}{UC5.1}\\
\hline
\hyperlink{R1F6.1.1}{R1F6.1.1} & \hyperlink{UC5.1.1}{UC5.1.1}\\
\hline
\hyperlink{R1F6.1.2}{R1F6.1.2} & \hyperlink{UC5.1.2}{UC5.1.2}\\
\hline
\hyperlink{R1F6.1.3}{R1F6.1.3} & \hyperlink{UC5.1.3}{UC5.1.3}\\
\hline
\hyperlink{R1F6.2}{R1F6.2} & \hyperlink{UC5.2}{UC5.2}\\
\hline
\hyperlink{R1F7}{R1F7} & \hyperlink{UC1}{UC1}\\
\hline
\hyperlink{R1F7.1}{R1F7.1} & \hyperlink{UC1.1}{UC1.1}\\
\hline
\hyperlink{R1F7.2}{R1F7.2} & \hyperlink{UC1.2.2}{UC1.2.2}\\
\hline
\hyperlink{R1F7.3}{R1F7.3} & \hyperlink{UC1.3}{UC1.3}\\
\hline
\hyperlink{R1F7.4}{R1F7.4} & \hyperlink{UC1.4}{UC1.4}\\
\hline
\hyperlink{R1F7.5}{R1F7.5} & \hyperlink{UC1.5}{UC1.5}\\
\hline
\hyperlink{R1F7.6}{R1F7.6} & \hyperlink{UC1.6}{UC1.6}\\
\hline
\hyperlink{R2F10}{R2F10} & \hyperlink{UC8}{UC8}\\
\hline
\hyperlink{R2F10.1}{R2F10.1} & \hyperlink{UC8.1}{UC8.1}\\
\hline
\hyperlink{R2F10.2}{R2F10.2} & \hyperlink{UC8.2}{UC8.2}\\
\hline
\hyperlink{R2F10.3}{R2F10.3} & \hyperlink{UC8.3}{UC8.3}\\
\hline
\hyperlink{R2F10.3.1}{R2F10.3.1} & \hyperlink{UC8.3.1}{UC8.3.1}\\
\hline
\hyperlink{R2F10.3.2}{R2F10.3.2} & \hyperlink{UC8.3.2}{UC8.3.2}\\
\hline
\hyperlink{R2F10.3.3}{R2F10.3.3} & \hyperlink{UC8.3.3}{UC8.3.3}\\
\hline
\hyperlink{R2F10.3.4}{R2F10.3.4} & \hyperlink{UC8.3.4}{UC8.3.4}\\
\hline
\hyperlink{R2F10.3.5}{R2F10.3.5} & \hyperlink{UC8.3.5}{UC8.3.5}\\
\hline
\hyperlink{R2F10.4}{R2F10.4} & \hyperlink{UC8.4}{UC8.4}\\
\hline
\hyperlink{R2F10.4.1}{R2F10.4.1} & \hyperlink{UC8.4.1}{UC8.4.1}\\
\hline
\hyperlink{R2F10.4.2}{R2F10.4.2} & \hyperlink{UC8.4.2}{UC8.4.2}\\
\hline
\hyperlink{R2F10.4.3}{R2F10.4.3} & \hyperlink{UC8.4.3}{UC8.4.3}\\
\hline
\hyperlink{R2F10.4.3.1}{R2F10.4.3.1} & \hyperlink{UC8.4.3.1}{UC8.4.3.1}\\
\hline
\hyperlink{R2F10.4.3.2}{R2F10.4.3.2} & \hyperlink{UC8.4.3.2}{UC8.4.3.2}\\
\hline
\hyperlink{R2F10.4.3.3}{R2F10.4.3.3} & \hyperlink{UC8.4.3.3}{UC8.4.3.3}\\
\hline
	\caption[Tracciamento Requisiti-Use case]{Tracciamento Requisiti-Use case}
	\label{tabella:requi-usecase}
\end{longtable}
\clearpage

\subsection{Tracciamento Use case - Requisiti}
\normalsize
\begin{longtable}{|c|c|}
	\hline\\
	\rowcolor{dosblue}
	\textbf{Codice Use case} & \textbf{Codice Requisiti} \\
	\hline
	\endhead
	\hyperlink{UC1}{UC1} & \hyperlink{R0F1}{R0F1}\\
	\hline
	\hyperlink{UC1.1}{UC1.1} & \hyperlink{R0F1.6}{R0F1.6}\\
	\hline
	\hyperlink{UC1.1.1}{UC1.1.1} & \hyperlink{R0F1.6.1}{R0F1.6.1}\\
	\hline
	\caption[Tracciamento Use case-Requisiti]{Tracciamento Use case-Requisiti}
	\label{tabella:requi-usecase}
\end{longtable}
\clearpage



