\newsection{Descrizione generale}

\subsection{Prospettive del prodotto}
Il prodotto di questo progetto vede la realizzazione di un plug-in per la piattaforma Grafana in grado di sfruttare la tecnologia delle reti Bayesiane associando i nodi della rete ad un flusso di dati costantemente monitorato. Correlato al mondo \gl{DevOps}, il prodotto si pone l'obiettivo di  determinare quali eventi non visibili siano i più probabili, segnalando in modo tempestivo quelli che superano precise soglie limite, attivando degli allarmi personalizzati. 

\subsection{Funzioni del prodotto}
Il plug-in fornisce un'interfaccia che permette all'utente di:

\begin{itemize}
	\item Importare la definizione di una rete Bayesiana da un file JSON;
	\item Utilizzare un editor grafico per:
	\begin{itemize}
		\item Creare una rete Bayesiana;
		\item Modificare una rete Bayesiana;
		\item Esportare la definizione della rete in un file JSON.
	\end{itemize} 
	\item Configurare la connessione tra i nodi della rete ed il flusso dati;
	\item Leggere i dati raccolti dalla rete;
	\item Visualizzare i dati raccolti tramite grafici;
	\item Modificare la disposizione dei panel su una dashboard;
	\item Creare \gl{alert} personalizzati;
	\item Condividere i grafici.
\end{itemize}

\subsection{Caratteristiche degli utenti}
Come specificato dal proponente nel \emph{Verbale 2018-12-12 v1.0.0} il prodotto si rivolge ad un utente generico e non ad uno nello specifico, in quanto chiunque potrà usufruire dei servizi offerti dal plug-in attraverso Grafana. Non emerge quindi il requisito di avere alcuna gerarchia di utenti, o utenti con privilegi differenziati. Pertanto il prodotto prevede una sola tipologia di utente, ovvero l’utilizzatore finale.

\subsection{Vincoli generali}
Il prodotto deve essere compatibile con Grafana e deve essere sviluppato in JavaScript. L'utente deve essere registrato alla piattaforma per ottenere l'accesso all'area \emph{Plugins} da cui effettuare il download e l'installazione del prodotto. Non sono richiesti particolari requisiti hardware, anche se gli stessi possono influenzarne la velocità di esecuzione.

\pagebreak