\subsection{UC2: Configurazione della connessione tra rete Bayesiana e sorgente dati}
\hypertarget{UC2}{}
\begin{figure} [H]
	\centering
	\includegraphics[scale=0.45]{Img/UC2}
	\caption{UC2 - Configurazione della connessione tra rete Bayesiana e sorgente dati}\label{}
\end{figure}
\begin{itemize}
	\item \textbf{Attori}: Utente;
	\item \textbf{Scopo e descrizione}: l'attore configura la connessione dei nodi della rete ai rispettivi flussi di dati provenienti dalla \gl{sorgente dati};
	\item \textbf{Precondizione}: è stata creata o caricata una rete Bayesiana adeguata. Grafana riceve correttamente informazioni dalla sorgente dati;
	\item \textbf{Scenario principale}:
	\begin{itemize}
		\item Gestione della connessione tra un nodo ed un flusso di dati (UC2.1);
		\item Impostazione delle regole temporali per il ricalcolo (UC2.2);
		\item Salvataggio della configurazione attuale (UC2.3);
		\item Caricamento di una configurazione salvata (UC2.4).
	\end{itemize}
	\item \textbf{Postcondizione}: la connessione tra la rete Bayesiana e la sorgente dati è configurata correttamente.
\end{itemize}

\subsection{UC2.1: Gestione della connessione tra un nodo ed un flusso di dati}
\hypertarget{UC2.1}{}
\begin{figure} [H]
	\centering
	\includegraphics[scale=0.45]{Img/UC2-1}
	\caption{UC2.1 - Gestione della connessione tra un nodo ed un flusso di dati}\label{}
\end{figure}
\begin{itemize}
	\item \textbf{Attori}: Utente;
	\item \textbf{Scopo e descrizione}: l'attore modifica il modo in cui un nodo è connesso ad un flusso di dati;
	\item \textbf{Precondizione}: l'attore ha selezionato un nodo della rete Bayesiana;
	\item \textbf{Scenario principale}:
	\begin{itemize}
		\item Connessione di un nodo ad un flusso di dati (UC2.1.1);
		\item Disconnessione di un nodo ad un flusso di dati (UC2.1.2);
		\item Modifica del flusso di dati connesso ad un nodo (UC2.1.3).
	\end{itemize}
	\item \textbf{Postcondizione}: il nodo selezionato è connesso al, oppure disconnesso dal, flusso di dati designato.
\end{itemize}

\subsection{UC2.1.1: Connessione di un nodo ad un flusso di dati}
\hypertarget{UC2.1.1}{}
\begin{itemize}
	\item \textbf{Attori}: Utente;
	\item \textbf{Scopo e descrizione}: l'attore connette il nodo selezionato ad un flusso di dati;
	\item \textbf{Precondizione}: il nodo selezionato non è connesso ad un flusso dati;
	\item \textbf{Scenario principale}: un nodo viene connesso al flusso dati dall'attore;
	\item \textbf{Postcondizione}: il nodo selezionato è connesso al flusso di dati desiderato.
\end{itemize}

\subsection{UC2.1.2: Disconnessione di un nodo da un flusso di dati}
\hypertarget{UC2.1.2}{}
\begin{itemize}
	\item \textbf{Attori}: Utente;
	\item \textbf{Scopo e descrizione}: l'attore disconnette il nodo selezionato da un flusso di dati;
	\item \textbf{Precondizione}: il nodo selezionato è connesso ad un flusso dati;
	\item \textbf{Scenario principale}: un nodo viene disconnesso dal flusso dati dall'attore;
	\item \textbf{Postcondizione}: il nodo selezionato è disconnesso dal flusso di dati.
\end{itemize}

\subsection{UC2.1.3: Modifica del flusso di dati connesso ad un nodo}
\hypertarget{UC2.1.3}{}
\begin{itemize}
	\item \textbf{Attori}: Utente;
	\item \textbf{Scopo e descrizione}: l'attore modifica il flusso di dati a cui un nodo è connesso;
	\item \textbf{Precondizione}: il nodo selezionato è connesso ad un flusso dati diverso da quello desiderato;
	\item \textbf{Scenario principale}: un flusso dati viene modificato dall'attore;
	\item \textbf{Postcondizione}: il nodo selezionato è connesso al flusso di dati desiderato.
\end{itemize}
\subsection{UC2.2: Impostazione delle regole temporali per il ricalcolo}
\hypertarget{UC2.2}{}
\begin{itemize}
	\item \textbf{Attori}: Utente;
	\item \textbf{Scopo e descrizione}: l'attore imposta le regole temporali per il ricalcolo delle probabilità della rete;
	\item \textbf{Precondizione}: è possibile inserire delle regole temporali;
	\item \textbf{Scenario principale}: vengono impostate le regole temporali per il ricalcolo delle probabilità dall'attore;
	\item \textbf{Postcondizione}: l'attore inserisce le regole temporali desiderate.
\end{itemize}

\subsection{UC2.3: Salvataggio della configurazione attuale}
\hypertarget{UC2.3}{}
\begin{figure} [H]
	\centering
	\includegraphics[scale=0.5]{Img/UC2-3}
	\caption{UC2.3 - Salvataggio della configurazione attuale}\label{}
\end{figure}
\begin{itemize}
	\item \textbf{Attori}: Utente;
	\item \textbf{Scopo e descrizione}: l'attore salva l'attuale configurazione della connessione della rete Bayesiana al flusso dati in un file JSON per un futuro riutilizzo;
	\item \textbf{Precondizione}: il sistema permette di leggere un file JSON, l'attore ha indicato di voler salvare la configurazione attuale su file;
	\item \textbf{Scenario principale}:
	\begin{itemize}
		\item Inserimento del nome (UC2.3.1);
		\item Inserimento del percorso (UC2.3.2).
	\end{itemize}
	\item \textbf{Postcondizione}: viene salvato un file contenente la configurazione attuale;
	\item \textbf{Estensioni}:
	\begin{itemize}
		\item L'utente prova a salvare una configurazione non valida, il sistema rimane nello stato precedente all'azione (UC2.5).
	\end{itemize}
\end{itemize}

\subsection{UC2.3.1: Inserimento nome} 
\hypertarget{UC2.3.1}{} 
\begin{itemize} 
	\item{\textbf{Attori primari}: Utente;} 
	\item{\textbf{Scopo e descrizione}: l'attore inserisce il nome con cui vuole salvare l'attuale configurazione di rete;} 
	\item{\textbf{Precondizione}: l'attore ha indicato di volere salvare l'attuale configurazione di rete;}
	\item{\textbf{Scenario principale}: viene inserito il nome dell'attuale configurazione di rete che l'attore vuole salvare;}
	\item{\textbf{Postcondizione}: l'attuale configurazione di rete è stata nominata correttamente.}
\end{itemize} 

\subsection{UC2.3.2: Inserimento percorso} 
\hypertarget{UC2.3.2}{} 
\begin{itemize} 
	\item{\textbf{Attori primari}: Utente;} 
	\item{\textbf{Scopo e descrizione}: l'attore inserisce il percorso in cui vuole salvare l'attuale configurazione di rete;} 
	\item{\textbf{Precondizione}: l'attore ha indicato di volere salvare l'attuale configurazione di rete;}
	\item{\textbf{Scenario principale}: viene inserito il percorso in cui l'attore vuole salvare l'attuale configurazione di rete;} 
	\item{\textbf{Postcondizione}: il percorso in cui l'attuale configurazione di rete è stato definito correttamente.}
\end{itemize}

\subsection{UC2.4: Caricamento di una configurazione salvata}
\hypertarget{UC2.4}{}
\begin{itemize}
	\item \textbf{Attori}: Utente;
	\item \textbf{Scopo e descrizione}: l'attore configura la connessione tra la rete Bayesiana e la sorgente dati secondo le impostazioni descritte da un file salvato su disco;
	\item \textbf{Precondizione}: il sistema permette di leggere un file JSON, esiste un file JSON contenente la configurazione;
	\item \textbf{Scenario principale}: viene caricata una configurazione di rete dall'attore;
	\item \textbf{Postcondizione}: la connessione tra rete e sorgente dati viene configurata secondo le informazioni salvati nel file;
	\item \textbf{Estensioni}: 
	\begin{itemize}
		\item  Viene caricato un file dal contenuto non valido e/o in un formato non valido, il sistema rimane nello stato precedente all'azione (UC2.6).
	\end{itemize}
\end{itemize}

\subsection{UC2.5: Errore salvataggio configurazione attuale}
\hypertarget{UC2.5}{}
\begin{itemize}
	\item \textbf{Attori}: Utente;
	\item \textbf{Scopo e descrizione}: il salvataggio della configurazione di rete attuale non è terminato con successo, ciò può essere causato da diverse ragioni, le più comuni possono essere: l'inserimento di un nome non valido, l'inserimento di un percorso non valido o problemi hardware;
	\item \textbf{Precondizione}: l'attore tenta di salvare una configurazione dell'attuale rete;
	\item \textbf{Scenario principale}: il sistema notifica un messaggio d'errore causato da un salvataggio non andato a buon fine dell'attuale configurazione di rete;
	\item \textbf{Postcondizione}: l'attore viene notificato dell'errore, il sistema rimane nello stato precedente all'azione.
\end{itemize}

\subsection{UC2.6: Caricamento di una configurazione non valida}
\hypertarget{UC2.6}{}
\begin{itemize}
	\item \textbf{Attori}: Utente;
	\item \textbf{Scopo e descrizione}: l'attore tenta di caricare un file non valido, e ne viene notificato;
	\item \textbf{Precondizione}: i contenuti del file, e/o il suo formato, non sono validi;
	\item \textbf{Scenario principale}: il sistema notifica un messaggio d'errore causato da un caricamento non andato a buon fine della configurazione di rete;
	\item \textbf{Postcondizione}: l'attore viene notificato dell'errore, il sistema rimane nello stato precedente all'azione.
\end{itemize}

\subsection{UC3: Visualizzazione dell'output della rete Bayesiana}
\hypertarget{UC3}{}
\begin{figure} [H]
	\centering
	\includegraphics[scale=0.45]{Img/UC3}
	\caption{UC3: Visualizzazione dell'output della rete Bayesiana}\label{}
\end{figure}
\begin{itemize}
	\item \textbf{Attori}: Utente;
	\item \textbf{Scopo e descrizione}: l'utente richiede di visualizzare dei dati calcolati dalla rete nella dashboard, tali dati vengono resi disponibili tramite panel;
	\item \textbf{Precondizione}: è stata caricata e configurata una rete Bayesiana adeguata; Grafana riceve correttamente informazioni dalla sorgente dati;
	\item \textbf{Scenario principale}:
	\begin{itemize}
		\item Lettura dei dati dalla rete;
		\item Aggiornamento della dashboard (UC3.1).
	\end{itemize}
	\item \textbf{Postcondizione}: l'output della rete Bayesiana viene letto e reso disponibile all'utente tramite panel correttamente.
\end{itemize}
\subsection{UC3.1: Aggiornamento della dashboard}
\hypertarget{UC3.1}{}
\begin{itemize}
	\item \textbf{Attori}: Utente;
	\item \textbf{Scopo e descrizione}: l'utente imposta la frequenza di aggiornamento della dashboard; Grafana aggiorna la visualizzazione dei panel;
	\item \textbf{Precondizione}: il sistema permette di impostare la frequenza di aggiornamento;
	\item \textbf{Scenario principale}:
	\begin{itemize}
		\item Impostazione della frequenza di aggiornamento della dashboard;
		\item Aggiornamento dei panel.
	\end{itemize}
	\item \textbf{Postcondizione}: Grafana aggiorna la dashboard con nuovi dati letti dalla rete, secondo la frequenza desiderata.
\end{itemize}

\subsection{UC4: Caricamento di una rete Bayesiana da file JSON}
\hypertarget{UC4}{}
\begin{itemize}
	\item \textbf{Attori}: Utente;
	\item \textbf{Scopo e descrizione}: l'attore carica la rete Bayesiana da un file JSON salvato su disco;
	\item \textbf{Precondizione}: il sistema permette di leggere un file JSON, esiste un file JSON contenente la rete;
	\item \textbf{Scenario principale}: viene caricata una rete Bayesiana presente in un file JSON dall'attore;
	\item \textbf{Postcondizione}: la rete Bayesiana viene creata secondo le informazioni salvate nel file;
	\item \textbf{Estensioni}:
	\begin{itemize}
		\item Viene caricato un file dal contenuto non valido e/o in un formato non valido, il sistema rimane nello stato precedente all'azione (UC11).
	\end{itemize}
\end{itemize}