\newsection{Casi d'uso}
\subsection{Struttura}
Ogni caso d'uso è descritto dalla seguente struttura:
\begin{itemize}
	\item Codice identificativo: $$ \textbf{UC \{codice\_padre\}.\{codice\_figlio\}  } $$
	\begin{itemize}
		\item UC specifica che si tratta di un caso d'uso;
		\item Codice\_padre identifica univocamente i casi d'uso;
		\item Codice\_figlio è un numero progressivo che identifica i sottocasi.
	\end{itemize}
	\item Titolo;
	\item Diagramma UML;
	\item Attori;
	\item Scopo e descrizione;
	\item Precondizioni;
	\item Flusso base degli eventi;
	\item Postcondizioni;
	\item Inclusioni (se presenti);
	\item Estensioni (se presenti).
\end{itemize}
\Spazio
Inoltre sono stati descritti due casi d'uso generali (UCG1 e UCG2) la cui nomenclatura non segue quella appena riportata. Essi rappresentano i casi d'uso più ad alto livello, insieme anche ad UC5, con cui l'utente può interagire con il prodotto.
Sono state quindi individuate due macro aree, una relativa alle operazioni che si possono effettuare sulla rete Bayesiana e l'altra relativa alle operazioni che comprendono la visualizzazione dei dati acquisiti.

\subsection{Attori}
L'attore che può interagire con il prodotto è soltanto uno:
\begin{itemize}
	\item \textbf{Utente}: si fa riferimento ad un utente generico, che ha intenzione di associare una rete Bayesiana ad un flusso dati per monitorare un'applicazione.
\end{itemize}

\newpage
\subsection{Caso d'uso generale UCG1: Operazioni sulla rete Bayesiana}
\begin{figure} [H]
	\centering
	\includegraphics[scale=0.75]{Img/UCG1}
	\caption{UCG1 - Operazioni sulla rete Bayesiana}\label{}
\end{figure}
\begin{itemize} 
	\item{\textbf{Attori primari}: Utente;}
	\item{\textbf{Scopo e descrizione}: l'attore vuole eseguire delle operazioni sulla rete Bayesiana;} 
	\item{\textbf{Precondizione}: l'applicazione deve essere avviata correttamente e pronta all'uso;} 
	\item{\textbf{Flusso base degli eventi}: 
		\begin{itemize} 
			\item{L'attore utilizza l'editor grafico (UC1);} 
			\item{L'attore configura la connessione tra la rete e la sorgente dati (UC2);} 
			\item{L'attore legge i dati acquisiti dalla rete Bayesiana (UC3)}; 
			\item{L'attore carica la definizione della rete da un file JSON (UC4).} 
		\end{itemize} 
	} 
	\item{\textbf{Postcondizione}: l'attore ha eseguito le operazioni sulla rete Bayesiana;} 
	\item{\textbf{Estensioni}: il file caricato non è valido (UC2.4.1).} 
\end{itemize}	

\subsection{Caso d'uso generale UCG2: Operazioni sulla visualizzazione dei dati}
\begin{figure} [H]
	\centering
	\includegraphics[scale=0.5]{Img/UCG2}
	\caption{UCG2 - Operazioni sulla visualizzazione dei dati}
\end{figure}
\begin{itemize} 
	\item{\textbf{Attori primari}: Utente;}
	\item{\textbf{Scopo e descrizione}: l'attore vuole eseguire delle operazioni relative alla visualizzazione dei dati acquisiti;} 
	\item{\textbf{Precondizione}: l'applicazione deve essere avviata correttamente ed in grado di leggere un flusso dati;} 
	\item{\textbf{Flusso base degli eventi}: 
		\begin{itemize} 
			\item{L'attore modifica la struttura della dashboard (UC6);} 
			\item{L'attore modifica la struttura di un panel (UC7);} 
			\item{L'attore condivide i grafici ottenuti (UC8)}. 
		\end{itemize} 
	} 
	\item{\textbf{Postcondizione}: l'attore ha eseguito le operazioni sulla visualizzazione dei dati.} 
\end{itemize} 
\newpage
