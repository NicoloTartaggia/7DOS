\newsection{Introduzione}
\subsection{Scopo del documento}
Il presente documento analizza e classifica i requisiti e i casi d'uso che sono stati individuati grazie all'analisi del \gl{capitolato} d'appalto C3 \emph{G\&B} e agli incontri con il \gl{proponente}.
Questo documento rappresenta un vincolo tra il fornitore, che si impegna a sviluppare un
software conforme alle caratteristiche riportate di seguito, e il proponente, che riconosce tali
requisiti come le caratteristiche ricercate.
In fase di collaudo la conformità ai requisiti concordati costituirà il criterio per l'accettazione
del \gl{prodotto} da parte del \gl{committente}.
\subsection{Glossario}
Per rendere la lettura del documento più semplice, chiara e comprensibile viene allegato il \emph{Glossario v1.0.0} nel quale sono contenute le definizioni dei termini tecnici, dei vocaboli ambigui, degli acronimi e delle abbreviazioni. La presenza di un termine all'interno del Glossario è segnalata con una "g" posta come pedice (esempio: $Glossario_{g}$).  
\subsection{Riferimenti}
\subsubsection{Normativi}
\begin{itemize}
	\item \textbf{Norme di Progetto:} \emph{Norme di Progetto v1.0.0.};
	\item \textbf{Capitolato d'appalto C3}: G\&B monitoraggio intelligente di processi DevOps \\
	\url{https://www.math.unipd.it/~tullio/IS-1/2018/Progetto/C3.pdf};
	\item \textbf{ISO/IEC 12207}:\\ \url{https://www.math.unipd.it/~tullio/IS-1/2009/Approfondimenti/ISO_12207-1995.pdf};
	\item \textbf{Da verbali}.
\end{itemize}
\subsubsection{Informativi}
\begin{itemize}
	\item \textbf{\gl{Grafana} Code Styleguide:} \\
	\url{http://docs.grafana.org/plugins/developing/code-styleguide/};
	\item \textbf{\gl{Angular TypeScript} Code Styleguide:} \\
	\url{https://angular.io/guide/styleguide};
	TODO: aggiungere rif al libro di swe
\end{itemize}
\pagebreak