%generare il pdf con il comando: pdflatex main.tex
\documentclass[12pt]{letter} % Copiata dal nostro stile

\usepackage{setspace}
\usepackage{geometry} % Per modificare margini e dimensioni varie
\usepackage{graphicx} % Per inserire immagini

\geometry{
	paper=a4paper, % Change to letterpaper for US letter
	top=3cm, % Top margin
	bottom=1.5cm, % Bottom margin
	left=2.2cm, % Left margin
	right=2.2cm, % Right margin
	%showframe, % Uncomment to show how the type block is set on the page
}

\usepackage[T1]{fontenc} % Output font encoding for international characters
\usepackage[utf8]{inputenc} % Required for inputting international characters
\usepackage{charter} % stix di default, copiato dagli answer credo
\usepackage{scrextend} % indentazione per il corpo della lettera
\usepackage{microtype} % Improve justification
\usepackage{eurosym} % wanna see the muney

\date{11/01/2019} % comment to show today date

\signature{
	Project manager 7DOS \\
	\includegraphics[width=6cm]{../PianoDiProgetto/Res/Firme/andrea.png}
} % firma a piè di pagina

%\address{123 Broadway \\ City, State 12345 \\ (000) 111-1111} % Your address and phone number











%\usepackage{calc} %introduce la notazione infissa per le op. aritmetiche interne a LaTeX

\usepackage[utf8]{inputenc}
\usepackage[T1]{fontenc}
\usepackage[italian]{babel} %il documento è in italiano
%\usepackage{textcomp} %The pack­age sup­ports the Text Com­pan­ion fonts, which pro­vide many text sym­bols
%(such as baht, bul­let, copy­right, mu­si­cal­note, onequar­ter, sec­tion, and yen), in the TS1 en­cod­ing.

\usepackage{graphicx}       %permette di inserire delle immagini
%\usepackage{caption}        %numerazione figure e loro descrizione testuale
\usepackage[labelformat=empty]{caption}
\usepackage{subcaption}     %sottofigure numerabili
\usepackage{float}  %permette di inserire un # qualsiasi di figure fluttuanti
\usepackage{xcolor}
\usepackage{rotating} %permette di ruotare le immagini
%\usepackage{changepage} %utile se c'è bisogno di aggiustare margini per centrare figure

%package utili per la math mode ( $ ... $ o \[ ... \] )
\usepackage{amsmath}
\usepackage{amssymb}
\usepackage{amsfonts}
%\usepackage{euler}    %font 'ams euler', lo stesso di 'Concrete Mathematics' di Knuth
\usepackage{amsthm}
\usepackage{mathtools}

% package utili per tabelle(\thead in particolare)
\usepackage{array, booktabs, caption}
\usepackage{makecell}
\renewcommand\theadfont{\bfseries}
\usepackage{boldline}

\usepackage{listings} %permette di inserire degli spezzoni di codice

\usepackage{tikz} %disegno di immagini vettoriali a schermo. Utile per grafi
\usetikzlibrary{arrows.meta}
\usetikzlibrary{graphs}
\usetikzlibrary{arrows}
%\usepackage{tikz-uml} %serve per disgnare l'UML, fantastica guida:
%https://perso.ensta-paristech.fr/~kielbasi/tikzuml/var/files/doc/tikzumlmanual.pdf
%download package: http://perso.ensta-paristech.fr/~kielbasi/tikzuml/

%package per le tabelle
\usepackage{booktabs} %permette di poter usare delle liste nelle tabelle
\usepackage{tabularx} 
\usepackage{longtable} %una tabella può continuare su più pagine
\usepackage{multirow} %utile per visualizzare una cella su più righe
%\usepackage{multicolumn} %cella su più colonne
%\usepackage[table]{xcolor} %rende disponibile l'utilizzo di un colore per lo sfondo
                        %delle celle di una tabella

%crea una cella per le tabelle in grado di andare a capo con \newline
%https://tex.stackexchange.com/questions/12703/how-to-create-fixed-width-table-columns-with-text-raggedright-centered-raggedlef
\usepackage{array}
\newcolumntype{L}[1]{>{\raggedright\let\newline\\\arraybackslash\hspace{0pt}}m{#1}}
\newcolumntype{C}[1]{>{\centering\let\newline\\\arraybackslash\hspace{0pt}}m{#1}}
\newcolumntype{R}[1]{>{\raggedleft\let\newline\\\arraybackslash\hspace{0pt}}m{#1}}


%indice con i puntini
\usepackage{tocloft}

%http://ctan.mirror.garr.it/mirrors/CTAN/macros/latex/contrib/appendix/appendix.pdf
\usepackage{appendix} %aggiunge dei comandi per l'appendice
\usepackage{parskip} %aiuta LaTeX a trovare il miglior stile per i page break
\setcounter{secnumdepth}{5} % numera i sottoparagrafi
\setcounter{tocdepth}{5} %aggiunge all'indice i sottoparagrafi
%\usepackage{titlesec} %\begin{paragraph} si può usare come subsubsubsection!


\usepackage{breakurl}%\url{...} può continare alla linea successiva. (si può andare a capo)

%Pacchetto per il simbolo degli euro
\usepackage{eurosym}

%Pacchetto per i colori delle tabelle
\usepackage{color, colortbl}

\definecolor{Maroon}{cmyk}{0, 0.87, 0.68, 0.32}
\usepackage[colorlinks=true]{hyperref}
\hypersetup{
    colorlinks=true,
    citecolor=black,
    filecolor=black,
    linkcolor=black, % colore dei link interni
    urlcolor=blue  % colore dei link interniesterni
}


% Creazione della copertina
\newcommand{\copertina}{
  \newgeometry{top=4cm}
  
  \begin{titlepage}
  \begin{center}

  \begin{center}
    %% qui metteteci l'immagine di copertina. Io ho messo quella dell'uni,
    %voi mettete quella del vostro grupo
    \centerline{\includegraphics[scale=0.1]{../logo}}
  \end{center}
  
  \vspace{1cm}

  \begin{Huge}
    \textbf{\Titolo{}} \\
  \end{Huge}

  \vspace{9pt}  
  
  \begin{large}
  \Gruppo{}\ - \Data{}
  \end{large}	  
  
  \vspace{15pt}

  \bgroup
  \def\arraystretch{1.3}
   \centering
   \begin{tabular}{c|L{5cm}}
      \multicolumn{2}{c}{\textbf{Informazioni sul documento} } \\ \hline
      \textbf{Versione} &  \Versione{}\\
      \textbf{Responsabile} & \Responsabile{}\\
      \textbf{Verifica} & \Verifica{}\\
      \textbf{Redazione} & \Redazione{} \\
      \textbf{Stato} & \Stato{} \\
      \textbf{Uso} & \Uso{} \\
      \textbf{Destinato a} & \Destinatoa{} \\
      \textbf{Email} & \Email{} \\
    \end{tabular}
  \egroup
  
  \vspace{15pt}

  \begin{center}
    \textbf{Descrizione\\}
    \DescrizioneDoc{}
  \end{center}

  \end{center}
  \end{titlepage}
  
  \restoregeometry
}

\newcommand{\code}[1]{\flextt{\texttt{#1}}}

\newcommand{\gl}[1]{\textit{#1}\ped{g}}

% VARIABILI

\newcommand{\sectiontitle}{}
\newcommand{\newsection}[1]{\renewcommand{\sectiontitle}{#1}\section{#1}}
%\newcommand{\titleAppendice}[1]{\renewcommand{\sectiontitle}{#1}\section*{#1}}

\newcommand{\Titolo}{Norme di Progetto}

\newcommand{\Gruppo}{7DOS}

\newcommand{\Versione}{3.0.0}

\newcommand{\Responsabile}{Giacomo Barzon}

\newcommand{\Verifica}{Giovanni Sorice \newline Nicolò Tartaggia}

\newcommand{\Redazione}{Lorenzo Busin \newline Andrea Trevisin \newline Michele Roverato}

\newcommand{\Destinatoa}{Prof. Tullio Vardanega \newline Prof. Riccardo Cardin \newline 7DOS}

\newcommand{\Uso}{Interno}

\newcommand{\Stato}{Approvato}

\newcommand{\Data}{22 Marzo 2019}

\newcommand{\Email}{\url{7dos.swe@gmail.com}}

\newcommand{\DescrizioneDoc}{Questo documento descrive le regole, gli strumenti e le convenzioni adottate durante la realizzazione del progetto \emph{G\&B}.}


\usepackage{listings}
\usepackage{color}
\definecolor{lightgray}{rgb}{.9,.9,.9}
\definecolor{darkgray}{rgb}{.4,.4,.4}
\definecolor{purple}{rgb}{0.65, 0.12, 0.82}

\lstdefinelanguage{JavaScript}{
	keywords={typeof, new, true, false, catch, function, return, null, catch, switch, var, if, in, while, do, else, case, break, for},
	keywordstyle=\color{blue}\bfseries,
	ndkeywords={class, export, boolean, throw, implements, import, this},
	ndkeywordstyle=\color{darkgray}\bfseries,
	identifierstyle=\color{black},
	sensitive=false,
	comment=[l]{//},
	morecomment=[s]{/*}{*/},
	commentstyle=\color{purple}\ttfamily,
	stringstyle=\color{red}\ttfamily,
	morestring=[b]',
	morestring=[b]"
}

\lstset{
	language=JavaScript,
	backgroundcolor=\color{lightgray},
	extendedchars=true,
	basicstyle=\footnotesize\ttfamily,
	showstringspaces=false,
	showspaces=false,
	numbers=left,
	numberstyle=\footnotesize,
	numbersep=9pt,
	tabsize=2,
	breaklines=true,
	showtabs=false,
	captionpos=b
}


\begin{document}
\copertina
\newpage 
\section*{\centering{Diario delle modifiche}}
\begin{table}[H]
	\centering
	\begin{tabular}{p{4cm}|C{3cm}|C{3cm}|R{2.5cm}|R{2cm}}
		\hlineB{3}
		
		\thead{Modifica} &\thead{Autore} &\thead{Ruolo} &\thead{Data} &\thead{Versione} \\
		
		\hlineB{3}
		
		\emph{Approvazione del documento} & Giacomo Barzon & Responsabile & 2019-01-06 & 1.0.0 \\
		\hline
		
		\emph{Verifica del documento} & Lorenzo Busin & Verificatore & 2019-01-05 & 0.8.0 \\
		\hline
		
		\emph{Verifica del documento} & Giovanni Sorice & Verificatore & 2019-01-02 & 0.7.0 \\
		\hline
		
		\emph{Verifica del documento} & Andrea Trevisin & Verificatore & 2018-12-28 & 0.6.0 \\
		\hline
		
		\emph{Stesura Meccanismi di controllo e e Consuntivi} & Nicolò Tartaggia & Analista & 2018-12-24 & 0.5.0 \\
		\hline
		
		\emph{Completamento stesura Pianificazione} & Nicolò Tartaggia & Analista & 2018-12-22 & 0.4.3 \\
		\hline
		
		\emph{Completamento stesura Suddivisione risorse} & Michele Roverato & Analista & 2018-12-17 & 0.4.2 \\
		\hline
		
		\emph{Completamento stesura Analisi dei Rischi} & Marco Costantino & Analista & 2018-12-14 & 0.4.1 \\
		\hline
		
		\emph{Inizio stesura Suddivisione risorse} & Michele Roverato & Analista & 2018-12-07 & 0.4.0 \\
		\hline
		
		\emph{Stesura Modello di Sviluppo} & Michele Roverato & Analista & 2018-12-05 & 0.3.0 \\
		\hline
		
		\emph{Inizio stesura Pianificazione} & Nicolò Tartaggia & Analista & 2018-12-04 & 0.2.0 \\
		\hline
		
		\emph{\textit{Inizio stesura Analisi dei Rischi}} & Marco Costantino & Analista & 2018-12-02 & 0.1.0 \\
		\hline
		
		\emph{Stesura della sezione Introduzione} & Marco Costantino & Analista & 2018-11-30 & 0.0.2 \\
		\hline
		
		\emph{Stesura dello scheletro del documento} & Nicolò Tartaggia & Analista & 2018-11-28 & 0.0.1 \\
		
	\end{tabular}
	
\end{table}


\clearpage

\tableofcontents
\pagebreak
\listoffigures

% SEZIONI DEL DOCUMENTO
% qui vanno presentate in ordine di apparizione le sezioni che compongono il documento
\newsection{Introduzione}
\subsection{Scopo del documento}
Questo documento rappresenta il Manuale dello Sviluppatore relativo al prodotto software \emph{G\&B} sviluppato dal gruppo 7DOS. Il suo scopo principale è fornire tutte le informazioni necessarie per usufruire delle funzionalità fino ad ora implementate ed, eventualmente, per estendere e migliorare il prodotto.
\subsection{Scopo del prodotto}
\emph{G\&B} è un plug-in per il software di monitoraggio Grafana. Il suo scopo è estenderne le funzionalità permettendo di monitorare il sistema desiderato con una o più reti Bayesiane definita ad-hoc dall'utente, consentendo poi la visualizzazione dei dati calcolati mediante i panel di Grafana.
\\
Il plug-in può essere utilizzato interamente in locale, e non dipende da servizi esterni per il suo funzionamento: una volta installato, è sufficiente impostare le \gl{datasource} necessarie, caricare e configurare la rete Bayesiana scelta e avviare il monitoraggio.

\subsection{Glossario}
In appendice al documento è presente un glossario con tutti i termini necessari per la piena comprensione del documento. Il prodotto è diretto a operatori IT e addetti al monitoraggio di sistemi informatici, pertanto vocaboli valutati come di conoscenza comune o appartenenti a competenze informatiche di base sono stati ignorati.\\
La presenza di un termine all'interno del glossario è segnalata con una "g" posta come pedice (esempio: \gl{Glossario}).

\subsection{Riferimenti}
\subsubsection{Installazione}
\begin{itemize}
	\item{\textbf{Git}\\
		\url{https://git-scm.com/}};
	\item{\textbf{Grafana}\\
		\url{https://grafana.com/docs/installation/}};
	\item{\textbf{Grafana Plug-in}\\
		\url{https://grafana.com/docs/plugins/installation/}};
	\item{\textbf{Node.js}\\
		\url{https://nodejs.org/}};
	\item{\textbf{NPM}\\
		\url{http://www.npmjs.com/}};
	\item{\textbf{InfluxDB}\\
		\url{https://www.influxdata.com/}};
	\item{\textbf{JetBrains - WebStorm}\\
		\url{https://jetbrains.com/webstorm}};
	\item{\textbf{JetBrains - WebStorm for students}\\
		\url{https://jetbrains.com/student}};
	\item{\textbf{VSCode}\\
		\url{https://code.visualstudio.com/}};
	\item{\textbf{Coveralls}\\
		\url{https://coveralls.io/}}.
\end{itemize}

\subsubsection{Tecnologie e librerie}
\begin{itemize}
	\item{\textbf{Writing data with the HTTP API}\\
		\url{https://docs.influxdata.com/influxdb/v1.7/guides/writing_data/}};
	\item{\textbf{Querying data with the HTTP API}\\
		\url{https://docs.influxdata.com/influxdb/v1.7/guides/querying_data/}};
	\item{\textbf{JsBayes}\\
		\url{https://github.com/vangj/jsbayes}};
	\item{\textbf{node-influx}\\
		\url{https://github.com/node-influx/node-influx}}.
	\item{\textbf{rx-http-request}\\
		\url{https://www.npmjs.com/package/@akanass/rx-http-request}};
	\item{\textbf{Stream-http}\\
		\url{https://github.com/node-influx/stream-http}}.
\end{itemize}

\subsubsection{Legali}
\begin{itemize}
	\item{\textbf{Licenza MIT}\\
		\url{https://opensource.org/licenses/MIT}}.
\end{itemize}
\subsubsection{Informativi}
\begin{itemize}
	\item{\textbf{Reti Bayesiane}\\
			\url{https://en.wikipedia.org/wiki/Bayesian_network}}.
\end{itemize}

\subsection{Maturità del documento}
Il presente documento potrebbe essere soggetto ad incrementi futuri.
\pagebreak
\newsection{Installazione}
\subsection{Node.js}
\subsection{Installazione Grafana}
\subsection{Installazione plugin}
\subsection{Installazione e configurazione InfluxDB}
\newsection{Configurazione dell'ambiente di lavoro}
Il seguente capitolo descrive come configurare l'ambiente di lavoro in modo corretto.

\subsection{WebStorm}
Per lo sviluppo del plug-in il team ha scelto di utilizzare l'\gl{IDE} WebStorm, sviluppato da JetBrains. L'IDE è a pagamento, tuttavia esso può essere scaricato gratuitamente dal loro sito ufficiale in prova gratuita per trenta giorni oppure se si è in possesso di un'e-mail personale universitaria ottenere gratuitamente una licenza valida un anno.\\
Il software è disponibile per Microsoft Windows, Linux e MacOS.
\subsection{VSCode}
VSCode è una valida alternativa a WebStorm, che il team consiglia di utilizzare nel caso in cui quest'ultimo non fosse reperibile.\\
Il software può essere scaricato molto facilmente dal sito ufficiale ed è disponibile per Microsoft Windows, Linux e MacOs.
\subsection{TSLint e ESLint}
TSLint e ESLint verranno automaticamente installati con l'esecuzione del comando:\\[0.2cm]
\hspace*{10mm}
\begin{ttfamily}
	npm install
\end{ttfamily}.\\[0.2cm]
Una volta installati correttamente, WebStorm li rileverà automaticamente tra le dipendenze presenti all'interno del package.json e procederà a segnalare tutti gli errori relativi all'analisi statica rilevati, senza la necessità di dover eseguire il comando:
\\[0.2cm]
\hspace*{10mm}
\begin{ttfamily}
npm run build
\end{ttfamily}.
\\[0.2cm]

\pagebreak
\newsection{Test}
\subsection{Test sul codice TypeScript}
\subsection{Test sul codice HTML/CSS}
Verrà fatto per la RA
\subsection{Code Coverage}


 


\pagebreak

\end{document}