\newsection{Processi di supporto}
\subsection{Documentazione}
Questa sezione descrive in modo dettagliato le procedure adottate dal gruppo in merito alla redazione, verifica e approvazione di tutta la documentazione prodotta. Tali norme devono essere rispettate in modo tassativo, al fine di realizzare documenti formali, coerenti e non ambigui.

\subsubsection{Fasi di sviluppo}
Ogni documento deve attraversare le seguenti fasi per essere considerato formale:
\begin{itemize}
	\item \textbf{Redazione}: un documento si trova in questa fase dal momento in cui viene creato fino alla sua approvazione. I \emph{Redattori} si occupano della stesura e della modifica delle sezioni che gli sono state assegnate dal \emph{Responsabile di Progetto};
	\item \textbf{Verifica}: un documento entra in questa fase al termine del lavoro dei \emph{Redattori}. Il \emph{Responsabile di Progetto} deve assegnare ai \emph{Verificatori} la procedura di verifica e validazione del documento. In caso di esito positivo esso passa allo stato \emph{Approvato}, nel caso contrario il \emph{Responsabile di Progetto} affida ai \emph{Redattori} il compito di apportare eventuali correzioni;
	\item \textbf{Approvazione}: un documento entra in questa fase una volta superata la fase di verifica in modo positivo ed è compito del \emph{Responsabile di Progetto} approvarlo in maniera ufficiale.
\end{itemize}

\subsubsection{Template}
Per uniformare la struttura dei documenti abbiamo creato un template \gl{LaTeX} che implementa la formattazione e l'impaginazione degli stessi. Il contenuto di ogni documento è costituito di più file, uno per ogni sezione, la cui stesura è stata incaricata ai \emph{Redattori}.

\subsubsection{Struttura dei documenti}
Tutta la documentazione deve rispettare la medesima struttura.

\paragraph{Frontespizio} \Spazio
Questa sezione contiene tutti gli elementi che dovranno essere presenti nella prima pagina di ogni documento.
\begin{itemize}
	\item \textbf{Logo del gruppo};
	\item \textbf{Titolo del documento};
	\item \textbf{Nome del gruppo};
	\item \textbf{Data di approvazione};
	\item \textbf{Informazioni sul documento}:
	\begin{itemize}
		\item {Versione corrente};
		\item {Nome e cognome del \emph{Responsabile di Progetto}};
		\item {Nome e cognome dei \emph{Verificatori}};
		\item {Nome e cognome dei \emph{Redattori}};
		\item {Destinazione d'uso};
		\item {Destinatari del documento};
		\item {Indirizzo email del gruppo}.
	\end{itemize}
	\item \textbf{Descrizione del documento}.
\end{itemize}



\paragraph{Diario delle modifiche} \Spazio
Questa sezione contiene le modifiche apportate al documento, organizzate in modo tabulare ed ordinate in modo decrescente dall'alto verso il basso secondo la versione dello stesso. Ogni colonna descrive le seguenti informazioni:
\begin{itemize}
	\item \textbf{Versione}: versione del documento;
	\item \textbf{Data}: data di esecuzione;
	\item \textbf{Descrizione}: tipo e soggetto di ogni modifica;
	\item \textbf{Autore}: nome e cognome dell'autore;
	\item \textbf{Ruolo}: ruolo dell'autore in quel momento.	
\end{itemize}

\paragraph{Indice} \Spazio
Ogni documento, esclusi i verbali, contiene un indice, il quale consente una visione generale del suo contenuto, ordinato e numerato rispetto alle sezioni presenti.
Ci possono essere al più tre tipologie: indice delle sezioni, indice delle tabelle e indice delle immagini.

\paragraph{Intestazione}
\begin{itemize}
	\item Logo del gruppo a sinistra;
	\item Titolo della sezione corrente a destra.
\end{itemize}


\paragraph{Piè di pagina}
\begin{itemize}
	\item Nome del documento e nome del gruppo a sinistra;
	\item Numero progressivo della pagina corrente a destra.
\end{itemize}

\subsubsection{Norme tipografiche}
\paragraph{Stile del testo}
	\begin{itemize}
		\item \textbf{Corsivo}: è usato per termini specifici o poco comuni, per indicare ruoli all’interno del progetto, per le citazioni e per i riferimenti ai documenti. Nel Glossario tutti i termini poco comuni in lingua inglese sono in corsivo;
		\item \textbf{Grassetto}: è usato per evidenziare concetti e parole chiave;
		\item \textbf{Maiuscolo}: è usato per indicare gli acronimi;
		\item \textbf{Sottolineato}: è usato nel Glossario per evidenziare link a termini presenti nel documento che vengono impiegati nella descrizione di altri termini;
		\item \textbf{Azzurro}: è usato per indicare collegamenti ipertestuali;
		\item \textbf{Verde}: è usato per includere frammenti di codice all'interno della documentazione.
	\end{itemize}
\paragraph{Elenchi puntati} \Spazio
	Ogni punto dell'elenco deve avere la lettera maiuscola e terminare con il carattere punto e virgola, tranne l'ultimo che deve terminare con il carattere punto.

\paragraph{Note a piè di pagina} \Spazio
In caso di presenza in una pagina interna di note da esplicare, esse vanno indicate nella pagina corrente, in basso a sinistra. Ogni nota deve riportare un numero e una descrizione.

\paragraph{Formati comuni}
\begin{itemize}
	\item \textbf{Orari}: \textbf{HH:MM}
	\begin{itemize}
		\item \textbf{HH}: indica le ore scritte con due cifre;
		\item \textbf{MM}: indica i minuti scritti con due cifre.
	\end{itemize}
	\item \textbf{Date}: \textbf{AAAA-MM-GG}
	\begin{itemize}
		\item \textbf{AAAA}: indica l'anno scritto con quattro cifre;
		\item \textbf{MM}: indica il mese scritto con due cifre;
		\item \textbf{GG}: indica il giorno scritto con due cifre.
	\end{itemize}
	\item \textbf{Termini ricorrenti}
	\begin{itemize}
		\item \textbf{Nomi propri}: vanno scritti usando il formato \textbf{Nome Cognome};
		\item \textbf{Ruoli di progetto}: vanno scritti in corsivo e con la lettera maiuscola;
		\item \textbf{Nomi dei documenti}: vanno scritti con la lettera maiuscola;
		\item \textbf{Riferimenti ai documenti}: vanno scritti in corsivo e con la lettera maiuscola.
	\end{itemize}
\end{itemize}

\subsubsection{Elementi grafici}
\paragraph{Immagini} \Spazio
Le immagini devono essere separate dal testo lasciando una spaziatura per facilitarne la lettura e vanno centrate orizzontalmente. Possono essere organizzate in modo affiancato e per ogni figura deve essere presente una breve didascalia, oltre ad un identificativo che permetta il relativo inserimento all’interno dell’indice delle
immagini. I formati consentiti sono \textbf{PNG} e \textbf{JPG}.
\paragraph{Tabelle} \Spazio
Le tabelle devono essere separate dal testo lasciando una spaziatura per facilitarne la lettura e vanno centrate orizzontalmente. Possono essere organizzate in modo affiancato e per ogni tabella deve essere presente una breve didascalia, oltre ad un identificativo che permetta il relativo inserimento all’interno dell’indice delle
tabelle. L'intestazione di ogni colonna deve essere in grassetto ed avere la lettera maiuscola.


\subsubsection{Nomenclatura dei documenti}
Il formato usato per la nomenclatura dei documenti, tranne i \emph{Verbali}, è il seguente: \Spazio \centerline{\textbf{NomeDocumento\_vX.Y.Z}}
\begin{itemize}
	\item \textbf{NomeDocumento}: indica il nome del documento, scritto senza spazi e con la lettera maiuscola in ogni parola.
	\item \textbf{vX.Y.Z}: indica la versione del documento secondo il seguente criterio:
	\begin{itemize}
		\item \textbf{X}: viene incrementato in seguito ad un'approvazione ufficiale del documento da parte del \emph{Responsabile di Progetto} e ciò comporta l'azzeramento di Y e Z;
		\item \textbf{Y}: viene incrementato in seguito ad un'azione di verifica o di stesura di una parte corposa del documento e ciò comporta l'azzeramento di Z;
		\item \textbf{Z}: viene incrementato in seguito ad un'azione di stesura di una parte esigua del documento.
	\end{itemize}
\end{itemize}
Il formato usato per la nomenclatura dei \emph{Verbali} invece è \textbf{Verbale\_AAAA-MM-GG}.

\subsubsection{Classificazione dei documenti}
\paragraph{Documenti informali} \Spazio
Un documento viene considerato informale se non è stato approvato dal \emph{Responsabile di Progetto}, pertanto è concesso esclusivamente ad uso interno al gruppo.

\paragraph{Documenti formali} \Spazio
Un documento viene considerato formale dopo aver superato con esito positivo l'attività di verifica e in seguito alla sua appovazione da parte del \emph{Responsabile di Progetto}, pertanto può essere destinato ad una distribuzione esterna al gruppo.

\paragraph{Verbali} \Spazio
Ogni verbale deve essere redatto dal segretario durante le riunioni, sia interne che esterne, tenute dal gruppo e deve rispettare il seguente contenuto:
\begin{itemize}
	\item \textbf{Informazioni incontro}: informazioni generali riguardo la riunione descritte secondo la seguente struttura:
	\begin{itemize}
		\item Luogo;
		\item Data;
		\item Ora;
		\item Partecipanti del gruppo;
		\item Partecipanti esterni.
	\end{itemize}
	\item \textbf{Argomenti affrontati}: descrizione dei temi discussi dal gruppo ed eventuali decisioni prese a riguardo.
\end{itemize}



\subsubsection{Sigle usate}
\begin{itemize}
	\item \textbf{AR}: Analisi dei Requisiti;
	\item \textbf{GL}: Glossario;
	\item \textbf{NdP}: Norme di Progetto;
	\item \textbf{PdP}: Piano di Progetto;
	\item \textbf{PdQ}: Piano di Qualifica;
	\item \textbf{SF}: Studio di Fattibilità;
	\item \textbf{RA}: Revisione di Accettazione;
	\item \textbf{RP}: Revisione di Progettazione;
	\item \textbf{RQ}: Revisione di Qualifica;
	\item \textbf{RR}: Revisione dei Requisiti.
\end{itemize}


\subsubsection{Glossario}
Il Glossario deve contenere tutti quei termini che possono risultare ambigui o che possono essere fraintesi. Le definizioni, elencate in modo alfabetico, devono essere chiare e concise.
L'inserimento nel Glossario deve avvenire parallelamente alla stesura della documentazione.
Deve essere segnalata soltanto la prima occorrenza del termine in un documento e deve essere impiegato l'apposito comando personalizzato che lo formatta in corsivo aggiungendo una lettera 'g' come pedice(eg. \gl{Termine nel Glossario}).

\subsubsection{Strumenti di supporto}
\paragraph{\LaTeX} \Spazio
L'intera documentazione deve essere redatta usando il linguaggio di markup \LaTeX \emph{ }che offre la possibilità di personalizzare comandi e variabili da usare all'interno dei documenti facilitandone la gestione e l'aggiornamento. Permette anche di dividere il contenuto di un documento dalla rispettiva struttura, suddividendolo in sezioni per facilitarne la stesura da parte dei \emph{Redattori} e di mantenere la stessa formattazione in tutti i documenti mediante l'uso di un template comune.


\paragraph{TexStudio} \Spazio
Per redigere i documenti in \LaTeX \emph{ }deve essere impiegato l'editor TexStudio. L'ambiente mette a disposizione un'interfaccia per la scrittura dei documenti, organizzati secondo una struttura gerarchica e la visualizzazione dell'anteprima del risultato ottenuto in seguito alla compilazione avvenuta con successo e in caso contrario segnalando all'utente eventuali errori.


\subsection{Versionamento}
Il servizio di hosting scelto per la repository è Github. La repository è
privata e sarà resa pubblica a fine progetto. Seguono le norme che ne riguardano gli aspetti
principali.
\subsubsection{Norme sui file}
I file devono rispettare le norme di nomenclatura già specificate in 3.1.6 Nomenclatura dei documenti.
Tramite il file .gitignore vengono esclusi i file intermedi di compilazione dei file latex.
Gli unici file ammessi nella repository sono .tex .jpg .png .pdf.

\subsubsection{Norme sui \gl{Commit}}
Ogni modifica sostanziale effettuata ai file dev'essere seguita da un commit. Il commento associato ad ogni commit deve essere nella forma: [nome file] descrizione.
La descrizione deve essere concisa e quanto più esaustiva possibile. Ad ogni commit deve corrispondere un aggiornamento del diario delle modifiche.

\subsubsection{Norme su branching e merging}
Le norme usate per branching e merging si rifanno al Git Feauture Branch Workflow:
master è il branch in cui risiederà documentazione e codice verificati e approvati.
Le modifiche o le nuove feature vanno implementate su un feature branch che saranno verificati
prima di effettuare una pull request verso master.
La pull request permette di non modificare master fintanto che non viene verificato il branch.
Ogni nuovo feature branch deve:
\begin{itemize}
	\item Essere creato a partire da master;
	\item Avere un nome che permetta di capire a quale feature è associato.
\end{itemize}
Queste norme saranno rivisitate all'inizio del processo di sviluppo per accertarsi che il feature branch workflow sia il più opportuno da usare. Una possibile alternativa con cui verrà confrontato è GitFlow.

\subsubsection{Strumenti di supporto}
\paragraph{Github}\Spazio
Github è il servizio di hosting scelto per la repository. Funziona utilizzando git, un software di \gl{versionamento} descritto nella sezione immediatamente successiva. Oltre all'hosting del codice sorgente, offre altre caratteristiche utili, tra le quali:
\begin{itemize}
	\item Un \gl{Issue tracking System}, con etichette e \gl{milestone};
	\item Una lista dei commit passati e delle relative modifiche ai file semplice da consultare;
	\item Account Educational gratuito per gli studenti, il quale permette la creazione di repository private, normalmente creabili solo con un account a pagamento.
\end{itemize}

\paragraph{Git}\Spazio
Git è un software open-source di controllo \gl{versione} distribuito.
Viene usato con Github, in quanto è un servizio già conosciuto dai membri del team e molto comodo da utilizzare.

\paragraph{Client Git}\Spazio
A seconda delle preferenze, ogni membro del team deciderà se usare come client GitKraken o Github Desktop. Entrambi sono client desktop che facilitano l'utilizzo di git, rendendo molto più semplice inviare le modifiche locali al repository remoto, o viceversa.

\subsubsection{Metriche per il versionamento}
\paragraph{Media commit a settimana} \Spazio
Media dei commit effettuati settimanalmente sulle diverse repository del gruppo. Utile per tenere traccia dell'impegno dedicato settimanalmente dal team sul progetto. Per questa metrica verranno utilizzati i risultati dei grafici generati automaticamente nella sezione \emph{Insights/Commits} di GitHub.

\subsection{Verifica}
Il processo di verifica ha lo scopo di esaminare un prodotto al fine di accertare che le attività produttive non abbiano introdotto errori sullo stesso, fornendo una prova oggettiva della sua correttezza o nel caso contrario, segnalando le eventuali problematiche riscontrate. Le metriche ed i metodi per effettuare verifica sono ampiamente e dettagliatamente descritti nel \emph{Piano di Qualifica v3.0.0}.

\subsubsection{Analisi dei processi}
La qualità dei processi è garantita dall'uso del ciclo di miglioramento continuo \gl{PDCA}, descritto nell'appendice B. Con l'applicazione di questo metodo è possibile perseguire una migliore qualità dei processi, incluso quello di verifica; ciò porta ad ottenere una migliore qualità dei prodotti che ne derivano. \\
Per ottenere una migliore qualità, quindi, bisogna che un processo venga definito, controllato e valutato.

\paragraph{Analisi PDCA} \Spazio
Attraverso lo studio della rappresentazione grafica del \gl{PDCA} relativo alla fase di progetto in esame, verranno tratte delle considerazioni riguardo lo svolgimento dei processi.
Da esso si possono identificare in modo visuale, quindi generico e non numericamente \gl{quantificabile}, dei dati sulla fase in esame ed osservarne la tendenza:
\begin{itemize}
	\item Errori di pianificazione, rappresentati da variazioni delle attività nello stato
	Plan;
	\item Velocità con cui il gruppo di lavoro porta avanti i processi tra i vari stati del
	ciclo PDCA.
\end{itemize}


\subsubsection{Analisi dei documenti}
 Descrizione delle procedure da eseguire per verificare adeguatamente i documenti.

\paragraph{Controllo del periodo} \Spazio
Per verificare la presenza di errori di sintassi, di parole grammaticalmente corrette ma presenti in un contesto sbagliato e di periodi difficili da comprendere è necessario l'intervento umano. Per questo motivo ogni documento deve essere sottoposto a verifica da parte dei \emph{Verificatori} incaricati, secondo la strategia walkthrough.

\paragraph{Rispetto delle Norme di Progetto} \Spazio
La verifica del rispetto delle norme descritte in \emph{Norme di Progetto v3.0.0} è compito dei \emph{Verificatori} incaricati. Non essendo possibile impiegare strumenti automatici per verificare la corretta applicazione di tutte le norme spetta quindi ai \emph{Verificatori} sottoporre i documenti a verifica, secondo la strategia inspection.

\paragraph{Metriche per i prodotti documentali} \Spazio
In seguito ad una qualsiasi modifica di un documento, si devono calcolare tutti i relativi indici di quel specifico documento. Inoltre dovrà essere ricalcolata la media complessiva di tutti gli indici calcolati fino a quel momento del documento in questione.
Di seguito verranno descritte in modo dettagliato tutte le metriche relative ai documenti adottate. Per una spiegazione dettagliata riguardo gli obbiettivi che il team si è imposto di raggiungere per le suddette metriche consultare il documento \emph{Piano di Qualifica v3.0.0}.
\subparagraph{Numero di errori grammaticali}\Spazio
Misura del numero di errori ortografici presenti all'interno di un documento.
Per verificare la presenza di errori ortografici nella documentazione deve essere usato lo strumento di controllo ortografico offerto dall'editor \emph{TexStudio} che integra i dizionari di OpenOffice per segnalare potenziali errori presenti nel testo.

\subparagraph{Gunning fog index}\Spazio
Indice utilizzato per misurare la facilità di lettura e comprensione di un testo. Il numero risultante è un indicatore del numero di anni di educazione formale necessari al fine di leggere il testo con facilità. \\
L'indice di Gunning fog è calcolabile tramite la seguente formula:
$$
0.4*((\frac{n^{\circ}\:parole}{n^{\circ}\:frasi})+100*(\frac{n^{\circ}\:parole\:complesse}{n^{\circ}\:parole}))
$$
Per ogni documento i \emph{Verificatori} devono calcolare il Gunning fog index e se questo dovesse risultare troppo alto, dovrà essere eseguita la verifica del documento con l'obiettivo di ricercare frasi troppo prolisse o complesse. Il calcolo deve essere effettuato attraverso uno \gl{script} scritto in Perl che trasforma i documenti dal formato .pdf al formato .txt per effettuare un calcolo più preciso, eliminando le tabelle ma non il loro contenuto.

\subparagraph{Indice di Gulpease}\Spazio
Utilizzato per misurare la leggibilità di un testo in lingua italiana.\\
L'indice di Gulpease è calcolabile tramite la seguente formula:
$$
89+\frac{(numero\:delle\:frasi)-10*(numero\:delle\:lettere)}{numero\:delle\:parole}
$$
I risultati sono compresi tra 0 e 100, dove 0 indica la leggibilità più bassa e 100 la leggibilità più alta. In generale risulta che i testi con indice:
\begin{itemize}
	\item{\textbf{Inferiore	a 80}}: sono difficili da leggere per chi ha la licenza elementare;
	\item{\textbf{Inferiore	a 60}}: sono difficili da leggere per chi ha la licenza media;
	\item{\textbf{Inferiore	a 40}}: sono difficili da leggere per chi ha la licenza superiore.
\end{itemize}
Per ogni documento i \emph{Verificatori} devono calcolare l'indice di Gulpease e se questo dovesse risultare troppo basso, dovrà essere eseguita la verifica del documento con l'obiettivo di ricercare frasi troppo prolisse o complesse. Il calcolo deve essere effettuato attraverso uno \gl{script} scritto in Perl che trasforma i documenti dal formato .pdf al formato .txt per effettuare un calcolo più preciso, eliminando le tabelle ma non il loro contenuto.


\subsubsection{Analisi dei prodotti software}
Descrizione delle procedure da eseguire per verificare adeguatamente i prodotti software.
\paragraph{Analisi statica}\Spazio
Il processo di analisi statica consiste nello svolgere dei test che non richiedono l'esecuzione del sistema software ma effettuano controlli sul codice sorgente. 
Può essere svolta nei seguenti modi:
\begin{itemize}
	\item \textbf{Walkthrough}: tecnica applicata quando non si sanno le tipologie di errori o problemi che si stanno cercando; è un'attività di lettura integrale e approfondita del testo o codice del prodotto, principalmente utilizzata durante le prime fasi del progetto in quanto permette una verifica più attenta e precisa dei prodotti. Ricade tra
	i compiti del \emph{Verificatore}, che si occuperà anche di stilare una
	lista degli errori riscontrati per facilitare la discussione di eventuali modifiche e permettere di individuare gli errori più frequenti. La fase finale consiste nell'applicare e registrare le modifiche correttive approvate;
	\item \textbf{Inspection}: attività di analisi mirata di parti specifiche del prodotto documentale o software che sono ritenute sezioni critiche, ovvero con grande concentrazione, potenziale o effettiva, di errori
	o anomalie. La lista degli errori da controllare va compilata prima dell'inizio dell'attività, in quanto maturata dall'esperienza acquisita durante le precedenti attività di \emph{Walkthrough}.
	Essendo limitata ad un'area specifica, risulta più veloce nell'esecuzione e nell'attuazione delle modifiche necessarie.
\end{itemize}

\subparagraph{Strumenti di supporto}\Spazio
Ai fini di svolgere l'analisi statica del codice lo strumento usato è: \gl{TSLint} un \gl{linter} che mette a disposizione numerose regole e la possibilità di creare regole personalizzate. TSLint è inoltre integrabile in numerosi IDE e nella piattaforma TravisCI che permette di eseguire test automatici su repository GitHub.
\paragraph{Analisi dinamica}\Spazio
Il processo di analisi dinamica consiste nell'esecuzione di test di varia tipologia i quali richiedono l'esecuzione del sistema software o solamente di alcune sue componenti. \\
La consistenza dei test in esecuzioni ripetute è una condizione imperativa, dunque è necessario che essi siano \gl{ripetibili}: un dato test eseguito in un ambiente specifico deve produrre, se fornito un determinato input, sempre gli stessi output. Al fine di garantire ciò, il team 7DOS ha scelto di basarsi sullo standard \gl{ISO/IEC/IEEE 29119}\footnote{ISO/IEC/IEEE 29119 parte 3 sezione 7, IEEE 2013.} per quanto riguarda la pianificazione e documentazione dei test dinamici. Tale standard prevede di definire, per ciascun test o suite di test, una \emph{specifica di test} composta di:
\begin{itemize}
	\item {\textbf{Specifica di progettazione:} definisce le funzionalità del prodotto da testare e le condizioni di test, ovvero l'ambiente di esecuzione e le pre-condizioni (particolari eventi o stati pregressi) necessarie al suo svolgimento;}
	\item {\textbf{Specifica di caso:} definisce l'insieme degli input che si desidera testare, e l'insieme dei risultati attesi per ogni input per una o più funzionalità testate;}
	\item {\textbf{Specifica di procedura:} definisce l'ordine di esecuzione dei test (nel caso di una suite di test), la modalità di svolgimento, ovvero le azioni da compiere e gli input da inserire in modo ordinato, eventuali azioni necessarie per il raggiungimento delle pre-condizioni e la modalità di analisi dei risultati ottenuti.}
\end{itemize}

Ogni test, per essere identificato in modo univoco, deve seguire la seguente sintassi:
\begin{center}
	T[Tipo]-[ID]
\end{center}
\begin{itemize}
	\item \textbf{Tipo}: specifica il tipo del test	e viene indicato con una delle seguenti lettere:
	\begin{itemize}
		\item A: accettazione;
		\item S: sistema;
		\item I: integrazione;
		\item U: unità.
	\end{itemize}
	\item \textbf{ID}: indica il codice del test rispettando una struttura gerarchica.	
\end{itemize}

\definecolor{darkgreen}{rgb}{0.0, 0.6, 0.0}
Per ogni test viene indicato lo stato in cui si trova, con una delle seguenti sigle:
\begin{itemize}
	\item \textbf{NI}: non implementato;
	\item \textcolor{orange}{\textbf{I}}: implementato;
	\item \textcolor{red}{\textbf{NS}}: non superato;
	\item \textcolor{darkgreen}{\textbf{S}}: superato.
\end{itemize}
Le principali tipologie di test che è possibile eseguire durante l'analisi dinamica sono le seguenti:
\begin{itemize}
	\item {\textbf{Test di unità}: mirano alla verifica della parte più piccola di lavoro prodotta da un \emph{Programmatore}, equivalente all'unità logica più piccola del prodotto, che può essere una singola classe, un metodo o funzione oppure un insieme di essi;}
	\item {\textbf{Test di integrazione}: mirano alla verifica di due o più unità già testate che vengono aggregate \textbf{incrementalmente} in una struttura più grande, rappresentando l’estensione logica del test di unità. In questo modo si può testare se il comportamento atteso dell'aggregato rispetta le previsioni. In caso negativo, è possibile che le singole unità contengano difetti residui da correggere oppure che i software utilizzati siano poco conosciuti e abbiano comportamenti inaspettati.
	La strategia di integrazione dei vari moduli scelta è quella \gl{bottom-up}, la quale prevede che vengano testate per prime le procedure più a basso livello e passando, progressivamente, a procedure di più alto livello;}
	\item {\textbf{Test di sistema}: mirano alla verifica del prodotto software completo di tutte le sue componenti. Un software a cui vengono applicati questi tipi di test deve essere giunto ad una versione ritenuta definitiva.
	Essendo test significativi, sarà richiesta la supervisione dei \emph{Verificatori} incaricati;}
	\item {\textbf{Test di regressione}: mirano ad eseguire nuovamente i test di unità e di integrazione su componenti software che hanno subito modifiche, in modo da controllare che i cambiamenti apportati non abbiano inserito nuovi errori sia nelle componenti modificate che nelle componenti non modificate e che prima non erano soggette ad errori;}
	\item {\textbf{Test di accettazione}: mirano al collaudo del prodotto software in presenza del \gl{proponente}. Essi sono test finali il cui superamento comporta la validazione e il rilascio del prodotto.}
\end{itemize}

\subparagraph{Strumenti di supporto} \Spazio
Per l'analisi dinamica vengono usati Mocha e Chai. Il primo è un framework  per l'esecuzione di test JavaScript che si basa su Node.js, il secondo è una libreria che contiene numerose interfacce per facilitare lo sviluppo dei test.

\paragraph{Metriche per i test}  \Spazio
Di seguito verranno descritte in modo dettagliato tutte le metriche adottate relative ai test. Per una spiegazione dettagliata riguardo gli obiettivi che il team si è imposto di raggiungere per le suddette metriche consultare il \emph{Piano di Qualifica v3.0.0.}

\subparagraph{Media \gl{Build} \gl{Travis} a settimana}\Spazio
Media delle build Travis effettuate settimanalmente sulle diverse repository del gruppo. Utile per tenere traccia dell'impegno dedicato settimanalmente dal team sul progetto.
\subparagraph{Percentuale di build Travis superate}\Spazio
Percentuale delle build Travis effettuate che hanno avuto esito positivo sul numero totale delle build effettuate. Utile per tenere traccia dei progressi raggiunti dal team.
\subparagraph{Percentuale di test eseguiti}\Spazio
Indica la percentuale di test eseguiti sul totale di test da eseguire, serve al team di verificatori per monitorare lo svolgimento delle loro attività.
Viene utilizzata la seguente formula:
$$PTE=\frac{TE}{TT}*100$$
Dove:
\begin{itemize}
	\item{\textbf{PTE}: è il valore della metrica;}
	\item{\textbf{TE}: è il numero di test eseguiti;}
	\item{\textbf{TT}: è il numero di totale di test.}
\end{itemize}

\subparagraph{Percentuale test case passati}\Spazio
Indica la percentuale di test passati rispetto ai test totali.\\
La percentuale test case passati è calcolabile tramite la seguente formula:
$$PTP=\frac{TP}{NT}*100$$
Dove:
\begin{itemize}
	\item{\textbf{PTP}: è il valore della metrica;}
	\item{\textbf{TP}: è il numero di test case passati;}
	\item{\textbf{NT}: è il numero di test case totale.}
\end{itemize}

\subparagraph{Percentuale test case falliti}\Spazio
Indica la percentuale di test falliti rispetto ai test totali.\\
La percentuale test case falliti è calcolabile tramite la seguente formula:
$$PTF=\frac{TF}{NT}*100$$
Dove:
\begin{itemize}
	\item{\textbf{PTF}: è il valore della metrica;}
	\item{\textbf{TF}: è il numero di test case falliti;}
	\item{\textbf{NT}: è il numero di test case totale.}
\end{itemize}

\subparagraph{Tempo medio necessario al team per risolvere un errore}\Spazio
Indica la quantità di tempo media spesa dal team per risolvere una criticità, utile a comprendere
l'impatto che l'introduzione di bug ha sui tempi di sviluppo. La formula usata per calcolarla è:
$$TMRE=\frac{TRE}{NE}$$
Dove:
\begin{itemize}
	\item{\textbf{TMRE}: è il valore della metrica;}
	\item{\textbf{TRE}: è il quantitativo di tempo speso a risolvere errori;}
	\item{\textbf{NE}: è il numero di errori risolti.}
\end{itemize}

\subparagraph{Efficienza nella progettazione dei test}\Spazio
Indica il tempo medio per la scrittura dei test. Un valore troppo basso potrebbe indicare la scrittura di test banali o poco efficaci.
Al contrario un valore troppo alto potrebbe indicare la scrittura di test troppo complessi che rischiano di contenere essi stessi errori.
La formula usata per calcolare questa metrica è:
$$TDE=\frac{NTP}{TST}$$
Dove:
\begin{itemize}
	\item{\textbf{TDE}: è il valore della metrica;}
	\item{\textbf{NTP}: è il numero di test scritti;}
	\item{\textbf{TST}: è il quantitativo di tempo speso a scrivere test.}
\end{itemize}

\subparagraph{Percentuale di errori corretti} \Spazio
Indica la percentuale di criticità (bug, difetti in genere) risolte sul totale di quelle note.
Se questa percentuale è troppo bassa, il team sarà costretto a fermare i processi di sviluppo per risolvere le criticità esistenti.
Per il calcolo viene utilizzata la seguente formula:
$$PCR=\frac{CR}{CN}*100$$
Dove:
\begin{itemize}
	\item{\textbf{PCR}: è il valore della metrica;}
	\item{\textbf{CR}: è il numero di criticità risolte;}
	\item{\textbf{CN}: è il numero di criticità note.}
\end{itemize}

\subparagraph{Code coverage}\Spazio
Per assicurarsi che la più grande porzione possibile di codice venga testata, si adotta questa metrica, che indica in percentuale, quanto del codice sorgente
è soggetto a test. Più alta tale percentuale, minore il rischio che vi siano criticità non rilevate.
Per il suo calcolo si farà uso degli strumenti automatici descritti nella sezione §2.2.2 di questo documento, ovvero TravisCI e Coveralls.

\subsubsection{Analisi diagrammi UML}
La verifica dei diagrammi UML prodotti è compito dei \emph{Verificatori}, che devono accertarsi che rispettino lo standard UML e che siano semanticamente corretti.

\subsubsection{Miglioramento del processo di verifica}
Per migliorare e ottimizzare il processo di verifica i \emph{Verificatori} devono riportare gli errori riscontrati più frequentemente in una \emph{checklist}, al fine di rendere più efficiente ed efficace l'attività di verifica, prestando maggiore attenzione agli errori più comuni.

\subsection{Validazione}
Il processo di validazione deve verificare che il prodotto sia conforme a quanto pianificato e sia capace di gestire e minimizzare gli effetti degli errori.
\subsubsection{Procedura di validazione}
Per completare l'attività di validazione il prodotto deve seguire, nel giusto ordine, i seguenti passi:
\begin{enumerate}
	\item Esecuzione dei test sul prodotto;
	\item Verifica dei risultati ad opera dei \emph{Verificatori}, che ne stendono un resoconto;
	\item Lettura del resoconto da parte del \emph{Responsabile di Progetto}. Se i test non danno risultati soddisfacenti, il \emph{Responsabile} chiede di eseguire nuovamente i test e, se lo ritiene necessario, può aggiungere nuove indicazioni in merito. In caso di esito negativo, ripetere tutti i passi a partire dal punto 1;
	\item Invio dei risultati alla proponente.
\end{enumerate}

\pagebreak
