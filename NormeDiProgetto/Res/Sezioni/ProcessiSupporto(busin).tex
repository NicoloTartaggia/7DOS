\newsection{Processi di supporto}
\subsection{Documentazione}
Questa sezione descrive in modo dettagliato le procedure adottate dal gruppo in merito alla redazione, verifica e approvazione di tutta la documentazione prodotta. Tali norme devono essere rispettate in modo tassativo, al fine di realizzare documenti formali, coerenti e non ambigui.

\subsubsection{Fasi di sviluppo}
Ogni documento deve attraversare le seguenti fasi per essere considerato formale:
\begin{itemize}
	\item \textbf{Redazione}: un documento si trova in questa fase dal momento in cui viene creato fino alla sua approvazione. I \emph{Redattori} si occupano della stesura e della modifica delle sezioni che gli sono state assegnate dal \emph{Responsabile di Progetto};
	\item \textbf{Verifica}: un documento entra in questa fase al termine del lavoro dei \emph{Redattori}. Il \emph{Responsabile di Progetto} deve assegnare ai \emph{Verificatori} la procedura di verifica e validazione del documento. In caso di esito positivo esso passa allo stato \emph{Approvato}, nel caso contrario il \emph{Responsabile di Progetto} affida ai \emph{Redattori} il compito di apportare eventuali correzioni;
	\item \textbf{Approvazione}: un documento entra in questa fase una volta superata la fase di verifica in modo positivo ed è compito del \emph{Responsabile di Progetto} approvarlo in maniera ufficiale.
\end{itemize}

\subsubsection{Template}
Per uniformare la struttura dei documenti è stato creato un template \gl{LaTeX} che implementa la formattazione e l'impaginazione degli stessi. Il contenuto di ogni documento è costituito di più file, uno per ogni sezione, la cui stesura è stata incaricata ai \emph{Redattori}.

\subsubsection{Struttura dei documenti}
Tutta la documentazione deve rispettare la medesima struttura.

\paragraph{Frontespizio} \Spazio
Questa sezione contiene tutti gli elementi che dovranno essere presenti nella prima pagina di ogni documento.
\begin{itemize}
	\item \textbf{Logo del gruppo};
	\item \textbf{Titolo del documento};
	\item \textbf{Nome del gruppo};
	\item \textbf{Data di approvazione};
	\item \textbf{Informazioni sul documento}:
	\begin{itemize}
		\item {Versione corrente};
		\item {Nome e cognome del \emph{Responsabile di Progetto}};
		\item {Nome e cognome dei \emph{Verificatori}};
		\item {Nome e cognome dei \emph{Redattori}};
		\item {Destinazione d'uso};
		\item {Destinatari del documento};
		\item {Indirizzo email del gruppo}.
	\end{itemize}
	\item \textbf{Descrizione del documento}.
\end{itemize}



\paragraph{Diario delle modifiche} \Spazio
Questa sezione contiene le modifiche apportate al documento, organizzate in modo tabulare ed ordinate in modo decrescente dall'alto verso il basso secondo la versione dello stesso. Ogni colonna descrive le seguenti informazioni:
\begin{itemize}
	\item \textbf{Modifica}: tipo e soggetto di ogni modifica;
	\item \textbf{Autore}: nome e cognome dell'autore;
	\item \textbf{Ruolo}: ruolo dell'autore in quel momento;
	\item \textbf{Data}: data di esecuzione;
	\item \textbf{Versione}: versione del documento.
\end{itemize}

\paragraph{Indice} \Spazio
Ogni documento, esclusi i verbali, contiene un indice, il quale consente una visione generale del suo contenuto, ordinato e numerato rispetto alle sezioni presenti.
Ci possono essere al più tre tipologie: indice delle sezioni, indice delle tabelle e indice delle immagini.

\paragraph{Intestazione}
\begin{itemize}
	\item Logo del gruppo a sinistra;
	\item Titolo della sezione corrente a destra.
\end{itemize}


\paragraph{Piè di pagina}
\begin{itemize}
	\item Nome del documento e nome del gruppo a sinistra;
	\item Numero progressivo della pagina corrente a destra.
\end{itemize}

\subsubsection{Norme tipografiche}
\paragraph{Stile del testo}
	\begin{itemize}
		\item \textbf{Corsivo}: è usato per termini specifici o poco comuni, per indicare ruoli all’interno del progetto, per le citazioni e per i riferimenti ai documenti. Nel Glossario tutti i termini poco comuni in lingua inglese sono in corsivo;
		\item \textbf{Grassetto}: è usato per evidenziare concetti e parole chiave;
		\item \textbf{Maiuscolo}: è usato per indicare gli acronimi;
		\item \textbf{Sottolineato}: è usato nel Glossario per evidenziare link a termini presenti nel documento che vengono impiegati nella descrizione di altri termini;
		\item \textbf{Azzurro}: è usato per indicare collegamenti ipertestuali.
	\end{itemize}
\paragraph{Elenchi puntati} \Spazio
	Ogni punto dell'elenco deve avere la lettera maiuscola e terminare con il carattere punto e virgola, tranne l'ultimo che deve terminare con il carattere punto.

\paragraph{Note a piè di pagina} \Spazio
In caso di presenza in una pagina interna di note da esplicare, esse vanno indicate nella pagina corrente, in basso a sinistra. Ogni nota deve riportare un numero e una descrizione.

\paragraph{Formati comuni}
\begin{itemize}
	\item \textbf{Orari}: \textbf{HH:MM}
	\begin{itemize}
		\item \textbf{HH}: indica le ore scritte con due cifre;
		\item \textbf{MM}: indica i minuti scritti con due cifre.
	\end{itemize}
	\item \textbf{Date}: \textbf{AAAA-MM-GG}
	\begin{itemize}
		\item \textbf{AAAA}: indica l'anno scritto con quattro cifre;
		\item \textbf{MM}: indica il mese scritto con due cifre;
		\item \textbf{GG}: indica il giorno scritto con due cifre.
	\end{itemize}
	\item \textbf{Termini ricorrenti}
	\begin{itemize}
		\item \textbf{Nomi propri}: vanno scritti usando il formato \textbf{Nome Cognome};
		\item \textbf{Ruoli di progetto}: vanno scritti in corsivo e con la lettera maiuscola;
		\item \textbf{Nomi dei documenti}: vanno scritti con la lettera maiuscola;
		\item \textbf{Riferimenti ai documenti}: vanno scritti in corsivo e con la lettera maiuscola.
	\end{itemize}
\end{itemize}

\subsubsection{Elementi grafici}
\paragraph{Immagini} \Spazio
Le immagini devono essere separate dal testo lasciando una spaziatura per facilitarne la lettura e vanno centrate orizzontalmente. Possono essere organizzate in modo affiancato e per ogni figura deve essere presente una breve didascalia, oltre ad un identificativo che permetta il relativo inserimento all’interno dell’indice delle
immagini. I formati consentiti sono \textbf{PNG} e \textbf{JPG}.
\paragraph{Tabelle} \Spazio
Le tabelle devono essere separate dal testo lasciando una spaziatura per facilitarne la lettura e vanno centrate orizzontalmente. Possono essere organizzate in modo affiancato e per ogni tabella deve essere presente una breve didascalia, oltre ad un identificativo che permetta il relativo inserimento all’interno dell’indice delle
tabelle. L'intestazione di ogni colonna deve essere in grassetto ed avere la lettera maiuscola.


\subsubsection{Nomenclatura dei documenti}
Il formato usato per la nomenclatura dei documenti, tranne i \emph{Verbali}, è il seguente: \Spazio \centerline{\textbf{NomeDocumento\_vX.Y.Z}}
\begin{itemize}
	\item \textbf{NomeDocumento}: indica il nome del documento, scritto senza spazi e con la lettera maiuscola in ogni parola.
	\item \textbf{vX.Y.Z}: indica la versione del documento secondo il seguente criterio:
	\begin{itemize}
		\item \textbf{X}: viene incrementato in seguito ad un'approvazione ufficiale del documento da parte del \emph{Responsabile di Progetto} e ciò comporta l'azzeramento di Y e Z;
		\item \textbf{Y}: viene incrementato in seguito ad un'azione di verifica o di stesura di una parte corposa del documento e ciò comporta l'azzeramento di Z;
		\item \textbf{Z}: viene incrementato in seguito ad un'azione di stesura di una parte esigua del documento.
	\end{itemize}
\end{itemize}
Il formato usato per la nomenclatura dei \emph{Verbali} invece è \textbf{Verbale\_AAAA-MM-GG}.

\subsubsection{Classificazione dei documenti}
\paragraph{Documenti informali} \Spazio
Un documento viene considerato infomale se non è stato approvato dal \emph{Responsabile di Progetto}, pertanto è concesso esclusivamente ad uso interno al gruppo.

\paragraph{Documenti formali} \Spazio
Un documento viene considerato formale dopo aver superato con esito positivo l'attività di verifica e in seguito alla sua appovazione da parte del \emph{Responsabile di Progetto}, pertanto può essere destinato ad una distribuzione esterna al gruppo.

\paragraph{Verbali} \Spazio
Ogni verbale deve essere redatto dal segretario durante le riunioni, sia interne che esterne, tenute dal gruppo e deve rispettare il seguente contenuto:
\begin{itemize}
	\item \textbf{Informazioni incontro}: informazioni generali riguardo la riunione descritte secondo la seguente struttura:
	\begin{itemize}
		\item Luogo;
		\item Data;
		\item Ora;
		\item Partecipanti del gruppo;
		\item Partecipanti esterni.
	\end{itemize}
	\item \textbf{Argomenti affrontati}: descrizione dei temi discussi dal gruppo ed eventuali decisioni prese a riguardo.
\end{itemize}



\subsubsection{Sigle usate}
\begin{itemize}
	\item \textbf{AR}: Analisi dei Requisiti;
	\item \textbf{GL}: Glossario;
	\item \textbf{NdP}: Norme di Progetto;
	\item \textbf{PdP}: Piano di Progetto;
	\item \textbf{PdQ}: Piano di Qualifica;
	\item \textbf{SF}: Studio di Fattibilità;
	\item \textbf{RA}: Revisione di Accettazione;
	\item \textbf{RP}: Revisione di Progettazione;
	\item \textbf{RQ}: Revisione di Qualifica;
	\item \textbf{RR}: Revisione dei Requisiti.
\end{itemize}


\subsubsection{Glossario}
Il Glossario deve contenere tutti quei termini che possono risultare ambigui o che possono essere fraintesi. Le definizioni, elencate in modo alfabetico, devono essere chiare e concise.
L'inserimento nel Glossario deve avvenire parallelamente alla stesura della documentazione.
Deve essere segnalata soltanto la prima occorrenza del termine in un documento e deve essere impiegato l'apposito comando personalizzato che lo formatta in corsivo aggiungendo una lettera 'g' come pedice(eg. \gl{Termine nel Glossario}).

\subsubsection{Strumenti di supporto}
\paragraph{\LaTeX} \Spazio
L'intera documentazione deve essere redatta usando il linguaggio di markup \LaTeX \emph{ }che offre la possibilità di personalizzare comandi e variabili da usare all'interno dei documenti facilitandone la gestione e l'aggiornamento. Permette anche di dividere il contenuto di un documento dalla rispettiva struttura, suddividendolo in sezioni per facilitarne la stesura da parte dei \emph{Redattori} e di mantenere la stessa formattazione in tutti i documenti mediante l'uso di un template comune.


\paragraph{TexStudio} \Spazio
Per redigere i documenti in \LaTeX \emph{ }deve essere impiegato l'editor TexStudio. L'ambiente mette a disposizione un'interfaccia per la scrittura dei documenti, organizzati secondo una struttura gerarchica e la visualizzazione dell'anteprima del risultato ottenuto in seguito alla compilazione avvenuta con successo e in caso contrario segnalando all'utente eventuali errori.


\subsection{Verifica}
Il processo di verifica ha lo scopo di esaminare un prodotto al fine di accertare che le attività produttive non abbiano introdotto errori sullo stesso, fornendo una prova oggettiva della sua correttezza o nel caso contrario, segnalando le eventuali problematiche riscontrate. Le metriche ed i metodi per effettuare verifica sono ampiamente e dettagliatamente descritti nel \emph{Piano di Qualifica v1.0.0} . Per questa prima parte del progetto il processo di verifica viene eseguito sulla documentazione.

\subsubsection{Analisi dei processi}
Descrizione delle procedure per verificare adeguatamente lo svolgimento dei processi.

\paragraph{Controllo delle metriche} \Spazio
Alla conclusione di ogni fase del progetto, per ogni macro-attività , si devono calcolare i relativi indici. Al fine di avere un indice complessivo di fase deve essere inoltre calcolato il valore medio
di tali indici. Di seguito verranno descritte in modo dettagliato tutte le metriche relative ai processi adottate. Per una spiegazione dettagliata riguardo gli obbiettivi che il team si è imposto di raggiungere per le suddette metriche consultare il piano di qualifica.
\subparagraph{\gl{Schedule Variance (SV)}}\Spazio
Metrica che indica se si è in linea, in anticipo o in ritardo rispetto alle attività pianificate nella \gl{baseline}. Essa è un indicatore di efficacia. \\
Se SV > 0 significa che si sta procedendo più velocemente rispetto a quanto pianificato. Viceversa, se SV < 0 allora il gruppo è in ritardo. 
\subparagraph{\gl{Budget Variance (BV)}}\Spazio
Metrica che indica se la spesa complessiva fino a quel momento è maggiore o minore rispetto a quanto pianificato. Essa è un indicatore con valore contabile e finanziario. \\
Se BV > 0 il budget complessivo sta diminuendo con velocità minore rispetto a quanto pianificato. Viceversa se BV < 0.

\paragraph{Analisi PDCA} \Spazio
Attraverso lo studio della rappresentazione grafica del \gl{PDCA} relativo alla fase di progetto in esame, verranno tratte delle considerazioni riguardo lo svolgimento dei processi.
Da esso si possono identificare in modo visuale, quindi generico e non numericamente \gl{quantificabile}, dei dati sulla fase in esame ed osservarne la tendenza:
\begin{itemize}
	\item Errori di pianificazione, rappresentati da variazioni delle attività nello stato
	Plan;
	\item Velocità con cui il gruppo di lavoro porta avanti i processi tra i vari stati del
	ciclo PDCA.
\end{itemize}



\subsubsection{Analisi dei documenti}
 Descrizione delle procedure da eseguire per verificare adeguatamente i documenti.
\paragraph{Controllo delle metriche} \Spazio
In seguito ad una qualsiasi modifica di un documento, si devono calcolare tutti i relativi indici di quel specifico documento. Inoltre dovrà essere ricalcolata la media complessiva di tutti gli indici calcolati fino a quel momento del documento in questione.
Di seguito verranno descritte in modo dettagliato tutte le metriche relative ai documenti adottate. Per una spiegazione dettagliata riguardo gli obbiettivi che il team si è imposto di raggiungere per le suddette metriche consultare il piano di qualifica.

\subparagraph{Numero di errori ortografici}\Spazio
Misura del numero di errori ortografici presenti all'interno di un documento.
Per verificare la presenza di errori ortografici nella documentazione deve essere usato lo strumento di controllo ortografico offerto dall'editor \emph{TexStudio} che integra i dizionari di OpenOffice per segnalare potenziali errori presenti nel testo.

\subparagraph{Gunning fog index}\Spazio
Indice utilizzato per misurare la facilità di lettura e comprensione di un testo. Il numero risultante è un indicatore del numero di anni di educazione formale necessari al fine di leggere il testo con facilità. \\
L'indice di Gunning fog è calcolabile tramite la seguente formula:
$$
0.4*((\frac{n^{\circ}\:parole}{n^{\circ}\:frasi})+100*(\frac{n^{\circ}\:parole\:complesse}{n^{\circ}\:parole}))
$$
Per ogni documento i \emph{Verificatori} devono calcolare il Gunning fog index e se questo dovesse risultare troppo alto, dovrà essere eseguita la verifica del documento con l'obiettivo di ricercare frasi troppo prolisse o complesse. Il calcolo deve essere effettuato attraverso uno \gl{script} scritto in Perl che trasforma i documenti dal formato .pdf al formato .txt per effettuare un calcolo più preciso, eliminando le tabelle ma non il loro contenuto.
\subparagraph{Indice di Gulpease}\Spazio
Utilizzato per misurare la leggibilità di un testo in lingua italiana.\\
L'indice di Gulpease è calcolabile tramite la seguente formula:
$$
89+\frac{(numero\:delle\:frasi)-10*(numero\:delle\:lettere)}{numero\:delle\:parole}
$$
I risultati sono compresi tra 0 e 100, dove 0 indica la leggibilità più bassa e 100 la leggibilità più alta. In generale risulta che i testi con indice:
\begin{itemize}
	\item{\textbf{Inferiore	a 80}}: sono difficili da leggere per chi ha la licenza elementare;
	\item{\textbf{Inferiore	a 60}}: sono difficili da leggere per chi ha la licenza media;
	\item{\textbf{Inferiore	a 40}}: sono difficili da leggere per chi ha la licenza superiore.
\end{itemize}
Per ogni documento i \emph{Verificatori} devono calcolare l'indice di Gulpease e se questo dovesse risultare troppo basso, dovrà essere eseguita la verifica del documento con l'obiettivo di ricercare frasi troppo prolisse o complesse. Il calcolo deve essere effettuato attraverso uno \gl{script} scritto in Perl che trasforma i documenti dal formato .pdf al formato .txt per effettuare un calcolo più preciso, eliminando le tabelle ma non il loro contenuto.
\paragraph{Controllo del periodo} \Spazio
Per verificare la presenza di errori di sintassi, di parole grammaticalmente corrette ma presenti in un contesto sbagliato e di periodi difficili da comprendere è necessario l'intervento umano. Per questo motivo ogni documento deve essere sottoposto a verifica da parte dei \emph{Verificatori} incaricati, secondo la strategia walkthrough.

\paragraph{Rispetto delle norme di progetto} \Spazio
La verifica del rispetto delle norme descritte in \emph{Norme di Progetto v1.0.0} è compito dei \emph{Verificatori} incaricati. Non essendo possibile impiegare strumenti automatici per verificare la corretta applicazione di tutte le norme spetta quindi ai \emph{Verificatori} sottoporre i documenti a verifica, secondo la strategia inspection.

\paragraph{Miglioramento del processo di verifica} \Spazio
Per migliorare e ottimizzare il processo di verifica i \emph{Verificatori} devono riportare gli errori riscontrati più frequentemente al fine di rendere più efficiente ed efficace la verifica di un documento, prestando maggiore attenzione agli errori più comuni.
\subsubsection{Analisi dei prodotti software}
Descrizione delle procedure da eseguire per verificare adeguatamente i prodotti software.
\paragraph{Analisi statica}\Spazio
Esecuzione di test che non richiedono l'esecuzione del sistema software ma effettuano controlli sul codice sorgente del prodotto software.
[Inserire test statici che eseguiamo]
\paragraph{Analisi dinamica}\Spazio
Il processo di analisi dinamica consiste nell'esecuzione di test di varia tipologia i quali richiedono l'esecuzione del sistema software o solamente di alcune sue componenti.
Le principali tipologie di test che è possibile eseguire durante l'analisi dinamica sono le seguenti:
\begin{itemize}
	\item {\textbf{Test d'unità}: verificano un unità, il più piccolo sottosistema possibile (nella programmazione ad oggetti comunemente una classe o od un metodo) che può essere testato separatamente;}
	\item {\textbf{Test d'integrazione}: verificano se sono rispettati i contratti di interfaccia tra più moduli o sub-system;}
	\item {\textbf{Test di sistema}: verificano il comportamento dell’intero sistema software.}
\end{itemize}

\paragraph{Controllo delle metriche} \Spazio
Di seguito viene riportato solamente un piccolo sottoinsieme di metriche relative ai prodotti software ritenute come fondamentali dal team 7DOS.
Questa sezione verrà ampliata successivamente qualora fosse necessario introdurre nuove metriche per la valutazione della qualità dei prodotti software.

\subparagraph{Functional Implementation Completeness}\Spazio
Misurazione in percentuale del grado con cui le funzionalità offerte dalla corrente implementazione del software coprono l'insieme di funzioni specificate nei requisiti.\\
Questa metrica è stata scelta per valutare il grado di completezza del prodotto; l'obiettivo è implementare tutte le funzionalità richieste.\\
Viene utilizzata la seguente formula:
$$FI_{Comp}=\frac{NF_i}{NF_r}*100$$
dove FI\textsubscript{Comp} è il valore della metrica, NF\textsubscript{i} è numero di funzioni attualmente implementate e NF\textsubscript{r} è numero di funzioni specificate dai requisiti.

\subparagraph{Average Functional Implementation Correctness}\Spazio
Misurazione in percentuale del grado in cui le funzionalità offerte dalla corrente implementazione del software, in media, rispettano il livello di precisione indicato nei requisiti.\\
Questa metrica è stata scelta per valutare il grado di accuratezza e garantire la qualità dei risultati restituiti dal prodotto, in quanto andrà a fare previsioni sulla verosimiglianza di alcuni eventi in base ai dati forniti, ed è necessario che tali previsioni siano sufficientemente accurate.\\
Viene utilizzata la seguente formula:
$$aFI_{Corr}=\frac{\sum\limits_{i=1}^N\frac{iPF_i}{rPF_i}}{N}*100$$
dove aFI\textsubscript{Corr} è il valore della metrica, iPF\textsubscript{i} è il livello di precisione della i-esima funzione implementata, rPF\textsubscript{i} è il livello di precisione della i-esima funzione secondo i requisiti, e N è il numero totale di funzioni considerate.

\subparagraph{Average Learning Time}\Spazio
Misurazione in minuti del tempo medio impiegato da un utente per imparare ad utilizzare una singola funzionalità del prodotto.
\\Questa metrica è stata scelta poiché, trattandosi di un prodotto che verrà reso disponibile pubblicamente, è stato ritenuto importante renderlo semplice da imparare per permetterne l'uso ad una vasta gamma di utenti.
\\Viene utilizzata la seguente formula:
$$aLT=\frac{\sum\limits_{i=1}^N{LT_i}}{N}$$
dove aLT è il valore della metrica, LT\textsubscript{i} è il tempo necessario ad imparare ad utilizzare la i-esima funzione implementata, espresso in minuti, e N è il numero totale di funzioni considerate.

\subparagraph{Failure Density}\Spazio
Misurazione in percentuale della quantità di test falliti rispetto alla quantità di test eseguiti.
\\Questa metrica è stata scelta per garantire che il prodotto sia generalmente stabile e non risulti poco utilizzabile o inutilizzabile a causa di eccessive \gl{failure}. Valutazioni più precise saranno effettuate in base ai singoli risultati dei test.\\
Viene utilizzata la seguente formula:
$$FD=\frac{T_f}{T_c}*100$$
dove FD è il valore della metrica, T\textsubscript{f} è il numero di test falliti e T\textsubscript{e} è il numero di test eseguiti.
\pagebreak
