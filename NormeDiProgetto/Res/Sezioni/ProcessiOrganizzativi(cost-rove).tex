\newsection{Processi organizzativi}
\subsection{Gestione delle comunicazioni}
\subsubsection{Comunicazioni interne al gruppo}
Le comunicazioni interne sono gestite con Discord, un software di messaggistica istantanea e di VoIP con altri membri di uno stesso server. In alternativa, ove necessario, gli utenti possono scambiarsi messaggi privati anche al di fuori di un server.
Un server può essere creato gratuitamente, ed è possibile organizzare le conversazione creando vari canali vocali e testuali associati ad uno scopo e ad alcuni membri del team.\newline
Oltre a qualche canale vocale per discutere di vari argomenti, i canali testuali principali sono: \verb|#|general per le discussioni generali e i canali nella categoria "Notification Center" dove apparirà una notifica inviata da un bot in caso di evento rilevante, per esempio l'arrivo di una nuova e-mail o un commit nel repository.
\newline
Il maggior vantaggio rispetto a Slack sta nell'essere un software con tutte le funzionalità di base disponibili gratuitamente, rendendo molto vantaggioso avere la possibilità di trovarsi a discutere con tutti gli altri membri del gruppo, cosa non possibile nella versione gratuita di Slack che offre solo chiamate 1-1.

\subsubsection{Comunicazioni esterne}
Le comunicazioni esterne sono compito del Project Manager, che utilizzerà l'indirizzo e-mail del gruppo \href{mailto:7dos.swe@gmail.com}{7dos.swe@gmail.com} facendo così apparire una notifica nel server Discord del team una volta inviata la mail.

\subsection{Gestione degli incontri}
\subsubsection{Incontri interni}
Il Project Manager ha il compito di decidere il giorno e la data di un incontro, discutendone con gli altri membri del team. Per comodità, è stato scelto il martedì come giorno di riferimento da proporre inizialmente ai membri del team.
\newline
Una volta raggiunto un accordo su un giorno con un numero consistente di partecipanti (almeno 4), tutti i membri che hanno accettato dovranno presentarsi in orario nel luogo prestabilito per la riunione, comunicando eventuali ritardi o impegni insormontabili.
Ad ogni incontro, tutti devono partecipare attivamente alla discussione offrendo la loro opinione sui punti da decidere e discutere, mentre un membro del team scelto dal Project Manager prenderà appunti sui contenuti da inserire appena possibile in un verbale.
\newline
Se non è possibile trovare un giorno in cui effettuare una riunione in un luogo fisico, si tenterà di scegliere un giorno in cui trovarsi a discutere su Discord.

\subsubsection{Incontri esterni}
Gli incontri esterni saranno anch'essi organizzati dal Project Manager, che si metterà in contatto con il committente o il proponente tramite email come descritto al punto 4.1.2 per accordarsi con essi sulla data, l'ora e il luogo dell'incontro. Una volta che questi ultimi saranno stati decisi e confermati da entrambe le parti, il Project Manager avviserà i membri utilizzando Discord.
In caso di impossibilità da una delle due parti ad organizzare un incontro fisico, si considererà l'opzione di un incontro utilizzando un servizio VoIP come Skype, Hangout o simili.

\subsection{Strumenti di gestione e coordinamento}
In questa sezione si riassumono brevemente i principali strumenti utilizzati nella comunicazione e nella collaborazione del team.

\subsubsection{Discord}
Discord è il software che si è scelto di utilizzare per le comunicazioni interne al gruppo, come spiegato al punto 4.1.1. Permette la creazione gratuita di un server nel quale comunicare con gli altri membri mediante canali testuali e vocali. Permette inoltre la creazione di un numero illimitato di Webhooks, utili per integrare molti altri strumenti, come Github.

\subsubsection{Github}
Github è il servizio di hosting scelto per la repository. Funziona utilizzando git, un software di versionamento descritto al punto 4.3.3. Oltre all'hosting del codice sorgente, offre altre caratteristiche utili, tra le quali:
\begin{itemize}  
\item Un Issue tracking System, con etichette e milestone;
\item Una lista dei commit passati e delle relative modifiche ai file semplice da consultare;
\item Account Educational gratuito per gli studenti, il quale permette la creazione di repository private, normalmente creabili solo con un account a pagamento.
\end{itemize}

\subsubsection{Git}
Git è un software open-source di controllo versione distribuito. 
//TODO

\subsubsection{Client Git}
A seconda delle preferenze, ogni membro del team deciderà se usare come client GitKraken o GitHub Desktop. Entrambi sono client desktop che facilitano l'utilizzo di git, rendendo molto più semplice inviare le modifiche locali al repository remoto, o viceversa.

\subsubsection{Software di ticketing?}
//TODO
\pagebreak