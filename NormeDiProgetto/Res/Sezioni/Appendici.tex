\appendix
\addcontentsline{toc}{section}{Appendici}
\section*{Appendice}
\section{ISO/IEC 15504(SPICE)}
Per ogni \gl{processo} lo standard ISO/IEC 15504 definisce una scala di maturità a cinque livelli (più il livello base, detto "livello 0"), riportati di seguito:
\begin{itemize}
	\item \textbf {Livello 0 - Incomplete Process}: il processo riporta \gl{performance} e risultati incompleti, inoltre è gestito in modo caotico;
	\item \textbf {Livello 1 - Performed Process}:  il processo raggiunge i risultati attesi ma viene eseguito in modo non controllato. Gli attributi di tale processo sono:
	\begin{itemize}
		\item \textbf{1.1 - Process Performance}: capacità di un processo di raggiungere gli obiettivi definiti.
	\end{itemize}
	\item \textbf {Livello 2 - Managed Process}: il processo è pianificato e tracciato secondo standard prefissati, dunque il suo prodotto è controllato, manutenuto e soddisfa determinati criteri di qualità. Gli attributi di tale processo sono:
	\begin{itemize}
		\item \textbf{2.1 - Performance Management}: capacità di un processo di identificare gli obiettivi e di definire, monitorare e modificare le sue performance;
		\item \textbf{2.2 - Work Product Management}: capacità di un processo di identificare, elaborare, documentare e controllare i propri risultati.
	\end{itemize}
	\item \textbf {Livello 3 - Established Process}: il processo possiede specifici standard organizzativi che includono linee guida personalizzate, il tutto è consolidato tramite una politica di feedback del prodotto. Gli attributi di tale processo sono:
	\begin{itemize}
		\item \textbf{3.1 - Process Definition}: il processo specifico si basa su processi standard, individuando e incorporando caratteristiche fondamentali di questi;
		\item \textbf{3.2 - Process Deployment}: sono stati definiti ed assegnati dei ruoli a ciascun membro del team, ogni risorsa necessaria per l'esecuzione del processo è disponibile ed utilizzabile.
	\end{itemize}
	\item \textbf {Livello 4 - Predictable Process}: il processo è quantitativamente misurato e statisticamente analizzato per permettere di prendere decisioni oggettive e per assicurare che le prestazioni rimangano all'interno di limiti definiti. Gli obiettivi sono, di conseguenza, supportati in maniera consistente. Gli attributi di tale processo sono:
	\begin{itemize}
		\item \textbf{4.1 - Process Measurement}: i risultati di misurazione dei processi vengono utilizzati per garantire che le prestazioni del processo supportino il raggiungimento degli obiettivi determinati;
		\item \textbf{4.2 - Process Control}: il processo è gestito quantitativamente in modo da renderlo stabile, capace e prevedibile entro limiti definiti.
	\end{itemize}
	\item \textbf {Livello 5 - Optimizing Process}: il processo è in continuo miglioramento per raggiungere adeguatamente gli obiettivi prefissati. Gli attributi di tale processo sono:
	\begin{itemize}
		\item \textbf{5.1 - Process Innovation}: gli obiettivi di miglioramento del processo supportano gli obiettivi aziendali rilevanti;
		\item \textbf{5.2 - Process Optimization}: le modifiche alla definizione, gestione e le prestazioni del processo si traducono in un impatto efficace per il raggiungimento degli obiettivi.
	\end{itemize}
\end{itemize}

Lo standard SPICE offre una scala di valutazione per ogni processo, in modo da misurare il livello di raggiungimento degli stessi:
\begin{itemize}
	\item \textbf{N - Not Achieved}: 0 - 15\%;
	\item \textbf{P - Partially Achieved}: >15\% - 50\%;
	\item \textbf{L - Largely Achieved}: >50\% - 85\%;
	\item \textbf{F - Fully Achieved}: >85\% - 100\%.
\end{itemize}

\section{Ciclo di miglioramento continuo (PDCA)}
Il ciclo di miglioramento continuo (PDCA, \emph{Plan-Do-Check-Act)} prevede quattro fasi iterative che permettono di controllare costantemente lo sviluppo di un processo, in modo da poter perseguire la miglior qualità di quest'ultimo:
\begin{itemize}
	\item \textbf{Plan}: in questa fase vengono definiti elementi estremamente importanti che riguardano il ciclo di miglioramento continuo. In particolare vengono fissati obiettivi, processi da utilizzare, risultati da ottenere, personale incaricato per i vari processi e scadenze da rispettare;
	\item \textbf{Do}: in questa fase vengono avviate tutte le attività previste da completare entro la data stabilita;
	\item \textbf{Check}: in questa fase vengono confrontati i risultati ottenuti dalle varie attività con quelli ipotizzati durante la fase Plan;
	\item \textbf{Act}: in questa fase vengono individuate le possibili problematiche che hanno prodotto risultati differenti da quelli attesi. Di conseguenza, verranno determinate tutte le attività da revisionare per migliorare la qualità del processo.
\end{itemize}
\section{ISO/IEC 25010 (SQuaRE)}
\subsection{Functional Suitability}
Questa caratteristica esprime il grado di soddisfacimento dei requisiti espliciti ed impliciti da parte di un prodotto o servizio, quando utilizzato sotto determinate condizioni.\\ \\
\textbf{Sotto-caratteristiche notevoli:}
\begin{itemize}
	\item{\textbf{Functional Completeness}}: esprime il grado con cui l'insieme di funzioni copre i compiti specificati e gli obiettivi dell'utente;
	\item{\textbf{Functional Correctness}}: esprime il grado con cui il prodotto restituisce risultati corretti, entro il livello di precisione desiderato.
\end{itemize}

\subsection{Performance Efficiency}
Questa caratteristica esprime le prestazioni relative al sistema come, quantità di risorse utilizzate per eseguire una determinata funzionalità del sistema sotto specifiche condizioni.  \\ \\
\textbf{Sotto-caratteristiche notevoli:}
\begin{itemize}
	\item{\textbf{Time Behaviour}}: esprime il grado con cui i tempi di risposta ed elaborazione e i volumi di produzione di un prodotto o sistema, durante l'esecuzione delle sue funzionalità, rispettano i requisiti;
	\item{\textbf{Resource Utilization}}: esprime il grado con cui il numero e tipo di risorse utilizzate da un prodotto o sistema, durante l'esecuzione delle sue funzionalità, rispetta i requisiti.
\end{itemize}

\subsection{Usability}
Questa caratteristica esprime il grado con cui un prodotto o sistema può essere usato da un determinato utente per raggiungere determinati scopi con efficacia, efficienza e soddisfazione in uno specifico contesto d'uso.\\ \\
\textbf{Sotto-caratteristiche notevoli:}
\begin{itemize}
	\item{\textbf{Learnability}}: esprime il grado con cui determinati utenti sono in grado di imparare ad utilizzare il prodotto o sistema efficacemente, efficientemente, con sicurezza da rischi e soddisfazione in un dato contesto d'uso;
	\item{\textbf{Operability}}: esprime il grado con cui un prodotto o sistema ha attributi che lo rendono facile da operare e controllare;
	\item{\textbf{User Error Protection}}: esprime il grado di efficacia ed efficienza con cui un sistema protegge gli utenti dal commettere errori.
\end{itemize}

\subsection{Reliability}
Questa caratteristica esprime il grado con cui un sistema, prodotto o componente esegue determinate funzioni sotto specifiche condizioni per un dato periodo di tempo. \\ \\
\textbf{Sotto-caratteristiche notevoli:}
\begin{itemize}
	\item{\textbf{Maturity}}: esprime il grado con cui un sistema, prodotto o componente raggiunge i requisiti di affidabilità in normali condizioni operative;
	\item{\textbf{Fault Tolerance}}: esprime il grado con cui un sistema, prodotto o componente opera come previsto nonostante la presenza di malfunzionamenti hardware o software.
\end{itemize}

\subsection{Maintainability}
Questa caratteristica esprime il grado di efficacia ed efficienza con cui un prodotto, o componente può essere modificato per migliorarlo, correggerlo o adattarlo a dei cambiamenti all'ambiente. \\ \\
\textbf{Sotto-caratteristiche notevoli}
\begin{itemize}
	\item{\textbf{Modularity}}: esprime il grado di scomposizione del sistema in parti minimali tali che un cambiamento ad una specifica componente ha il minimo impatto su tutte le altre componenti;
	\item{\textbf{Analyzability}}: esprime il grado di efficacia ed efficienza con cui è possibile analizzare l'impatto nel sistema di uno specifico cambiamento ad una o più delle sue parti, ai fini di rilevare eventuali casi di fallimento;
	\item{\textbf{Modifiability}}: esprime il grado con cui un prodotto o sistema può essere modificato efficacemente ed efficientemente senza introdurre difetti che ne possano intaccare la qualità complessiva;
	\item{\textbf{Testability}}: esprime il grado di efficacia ed efficienza con cui è possibile stabilire ed eseguire dei test per valutare la qualità del sistema.
\end{itemize}

\section{Metriche}
In questa sezione vengono riportate le metriche adottate.
\subsection{Per la qualità di processo}
Alla conclusione di ogni fase del progetto, per ogni macro-attività , si devono calcolare i relativi indici. Al fine di avere un indice complessivo di fase deve essere inoltre calcolato il valore medio di tali indici. Di seguito verranno descritte in modo dettagliato tutte le metriche relative ai processi adottate. Per una spiegazione dettagliata riguardo gli obbiettivi che il team si è imposto di raggiungere per le suddette metriche consultare il \emph{Piano di Qualifica v2.0.0.}
\subsubsection{\gl{Schedule Variance (SV)}}
Metrica che indica se il progetto è in linea, in anticipo o in ritardo rispetto alle attività pianificate nella \gl{baseline}. Essa è un indicatore di efficacia. \\
Calcolabile attraverso la seguente formula:
\[ SV=BCWP-BCWS\]
Dove:
\begin{itemize}
	\item {\textbf{SV}: corrisponde al valore della metrica;}
	\item {\textbf{BCWP}: corrisponde al valore, in giorni od euro, prodotto dalle attività del progetto realizzate fino alla data corrente, od in altre parole la somma complessiva del valore di tutte le componenti completate fino ad ora.}
	\item {\textbf{BCWS}: corrisponde al numero di giorni od euro pianificati inizialmente per realizzare le attività fino alla data corrente.}
\end{itemize}
Se SV > 0 significa che si sta procedendo più velocemente rispetto a quanto pianificato. Viceversa, se SV < 0 allora il gruppo è in ritardo.
\subsubsection{\gl{Budget Variance (BV)}}
Metrica che indica se la spesa complessiva fino a quel momento è maggiore o minore rispetto a quanto pianificato. Essa è un indicatore con valore contabile e finanziario. \\
Calcolabile attraverso la seguente formula:
\[BV=\frac{BCWS-ACWP}{BCWS}\]
\begin{itemize}
	\item {\textbf{BV}: corrisponde al valore della metrica;}
	\item {\textbf{BCWS}: corrisponde al costo pianificato in giorni od euro per realizzare le attività del progetto fino alla data corrente;}
	\item {\textbf{ACWP}: corrisponde al costo, in giorni od euro, effettivamente sostenuto fino alla data corrente.}
\end{itemize}
Se BV > 0 il budget complessivo sta diminuendo con velocità minore rispetto a quanto pianificato. Viceversa se BV < 0.
\subsubsection{Numero rischi non previsti}
Indica la quantità di rischi rilevati durante lo svolgimento delle attività, non preventivati all'interno dell'analisi dei rischi durante l'attuale fase del progetto.
\subsubsection{Indisponibilità servizi esterni}
Numero complessivo di giorni in cui uno specifico servizio esterno, tra tutti quelli utilizzati, è rimasto indisponibile per la maggior parte del tempo rendendone quindi impossibile l'utilizzo. Sarà necessario utilizzare un contatore separato per ogni servizio utilizzato.
\subsubsection{Media commit a settimana}
Media dei commit effettuati settimanalmente sulle diverse repository del gruppo. Utile per tenere traccia dell'impegno dedicato settimanalmente dal team sul progetto. \textbf{Misurazione}: verranno utilizzati i risultati dei grafici generati automaticamente nella sezione \emph{Insights/Commits} di GitHub.
\subsubsection{Media numero \gl{Build} \gl{Travis} a settimana}
Media delle build Travis effettuate settimanalmente sulle diverse repository del gruppo. Utile per tenere traccia dell'impegno dedicato settimanalmente dal team sul progetto.
\subsubsection{Percentuale di build Travis superate}
Percentuale delle build Travis effettuate che hanno avuto esito positivo sul numero totale delle build effettuate. Utile per tenere traccia dei progressi raggiunti dal team.

\subsubsection{Controllo delle metriche di progetto}
Nella presente sezione sono riportate le metriche \gl{Budget Variance (BV)} e \gl{Schedule Variance (SV)}. Grazie ad esse lo stato di avanzamento delle attività può essere controllato e \gl{quantificato} e i possibili problemi di costo/schedulazione individuati prima che possano diventare critici. In particolare, attraverso le metriche Budget Variance e Schedule Variance, è possibile rispettivamente:
\begin{itemize}
\item{Verificare se le spese sono state maggiori a quanto previsto dal budget;}
\item{Verificare se un'attività è in linea, in anticipo o in ritardo rispetto alla schedulazione delle attività.}
\end{itemize}

\subsection{Per i test}
Di seguito verranno descritte in modo dettagliato tutte le metriche relative ai test adottate. Per una spiegazione dettagliata riguardo gli obbiettivi che il team si è imposto di raggiungere per le suddette metriche consultare il \emph{Piano di Qualifica v2.0.0.}

\subsubsection{Percentuale di test eseguiti}
Indica la percentuale di test eseguiti sul totale di test da eseguire, serve al team di verificatori per monitorare lo svolgimento delle loro attività.
Viene utilizzata la seguente formula:
$$PTE=\frac{TE}{TT}*100$$
Dove:
\begin{itemize}
	\item{\textbf{PTE}: è il valore della metrica;}
	\item{\textbf{TE}: è il numero di test eseguiti;}
	\item{\textbf{TT}: è il numero di totale di test.}
\end{itemize}

\subsubsection{Percentuale test case passati}
Indica la percentuale di test passati rispetto ai test totali.\\
La percentuale test case passati è calcolabile tramite la seguente formula:
$$PTP=\frac{TP}{NT}*100$$
Dove:
\begin{itemize}
	\item{\textbf{PTP}: è il valore della metrica;}
	\item{\textbf{TP}: è il numero di test case passati;}
	\item{\textbf{NT}: è il numero di test case totale.}
\end{itemize}

\subsubsection{Percentuale test case falliti}
Indica la percentuale di test falliti rispetto ai test totali.\\
La percentuale test case falliti è calcolabile tramite la seguente formula:
$$PTF=\frac{TF}{NT}*100$$
Dove:
\begin{itemize}
	\item{\textbf{PTF}: è il valore della metrica;}
	\item{\textbf{TF}: è il numero di test case falliti;}
	\item{\textbf{NT}: è il numero di test case totale.}
\end{itemize}

\subsubsection{Tempo medio necessario al team per risolvere un errore}
Indica la quantità di tempo media spesa dal team per risolvere una criticità, utile a comprendere
l'impatto che l'introduzione di bug ha sui tempi di sviluppo. La formula usata per calcolarla è:
$$TMRE=\frac{TRE}{NE}$$
Dove:
\begin{itemize}
	\item{\textbf{TMRE}: è il valore della metrica;}
	\item{\textbf{TRE}: è il quantitativo di tempo speso a risolvere errori;}
	\item{\textbf{NE}: è il numero di errori risolti.}
\end{itemize}

\subsubsection{Efficienza nella progettazione dei test}
Indica il tempo medio per la scrittura dei test. Un valore troppo basso potrebbe indicare la scrittura di test banali o poco efficaci.
Al contrario un valore troppo alto potrebbe indicare la scrittura di test troppo complessi che rischiano di contenere essi stessi errori.
La formula usata per calcolare questa metrica è:
$$TDE=\frac{NTP}{TST}$$
Dove:
\begin{itemize}
	\item{\textbf{TDE}: è il valore della metrica;}
	\item{\textbf{NTP}: è il numero di test scritti;}
	\item{\textbf{TST}: è il quantitativo di tempo speso a scrivere test.}
\end{itemize}

\subsubsection{Percentuale di errori corretti}
Indica la percentuale di criticità (bug, difetti in genere) risolte sul totale di quelle note.
Se questa percentuale è troppo bassa, il team sarà costretto a fermare i processi di sviluppo per risolvere le criticità esistenti.
Per il calcolo viene utilizzata la seguente formula:
$$PCR=\frac{CR}{CN}*100$$
Dove:
\begin{itemize}
	\item{\textbf{PCR}: è il valore della metrica;}
	\item{\textbf{CR}: è il numero di criticità risolte;}
	\item{\textbf{CN}: è il numero di criticità note.}
\end{itemize}

\subsection{Per la qualità del prodotto}
\subsubsection{Metriche per i prodotti documentali}
In seguito ad una qualsiasi modifica di un documento, si devono calcolare tutti i relativi indici di quel specifico documento. Inoltre dovrà essere ricalcolata la media complessiva di tutti gli indici calcolati fino a quel momento del documento in questione.
Di seguito verranno descritte in modo dettagliato tutte le metriche relative ai documenti adottate. Per una spiegazione dettagliata riguardo gli obbiettivi che il team si è imposto di raggiungere per le suddette metriche consultare il piano di qualifica.

\paragraph{Numero di errori grammaticali}\Spazio
Misura del numero di errori ortografici presenti all'interno di un documento.
Per verificare la presenza di errori ortografici nella documentazione deve essere usato lo strumento di controllo ortografico offerto dall'editor \emph{TexStudio} che integra i dizionari di OpenOffice per segnalare potenziali errori presenti nel testo.

\paragraph{Gunning fog index}\Spazio
Indice utilizzato per misurare la facilità di lettura e comprensione di un testo. Il numero risultante è un indicatore del numero di anni di educazione formale necessari al fine di leggere il testo con facilità. \\
L'indice di Gunning fog è calcolabile tramite la seguente formula:
$$
0.4*((\frac{n^{\circ}\:parole}{n^{\circ}\:frasi})+100*(\frac{n^{\circ}\:parole\:complesse}{n^{\circ}\:parole}))
$$
Per ogni documento i \emph{Verificatori} devono calcolare il Gunning fog index e se questo dovesse risultare troppo alto, dovrà essere eseguita la verifica del documento con l'obiettivo di ricercare frasi troppo prolisse o complesse. Il calcolo deve essere effettuato attraverso uno \gl{script} scritto in Perl che trasforma i documenti dal formato .pdf al formato .txt per effettuare un calcolo più preciso, eliminando le tabelle ma non il loro contenuto.

\paragraph{Indice di Gulpease}\Spazio
Utilizzato per misurare la leggibilità di un testo in lingua italiana.\\
L'indice di Gulpease è calcolabile tramite la seguente formula:
$$
89+\frac{(numero\:delle\:frasi)-10*(numero\:delle\:lettere)}{numero\:delle\:parole}
$$
I risultati sono compresi tra 0 e 100, dove 0 indica la leggibilità più bassa e 100 la leggibilità più alta. In generale risulta che i testi con indice:
\begin{itemize}
	\item{\textbf{Inferiore	a 80}}: sono difficili da leggere per chi ha la licenza elementare;
	\item{\textbf{Inferiore	a 60}}: sono difficili da leggere per chi ha la licenza media;
	\item{\textbf{Inferiore	a 40}}: sono difficili da leggere per chi ha la licenza superiore.
\end{itemize}
Per ogni documento i \emph{Verificatori} devono calcolare l'indice di Gulpease e se questo dovesse risultare troppo basso, dovrà essere eseguita la verifica del documento con l'obiettivo di ricercare frasi troppo prolisse o complesse. Il calcolo deve essere effettuato attraverso uno \gl{script} scritto in Perl che trasforma i documenti dal formato .pdf al formato .txt per effettuare un calcolo più preciso, eliminando le tabelle ma non il loro contenuto.


\subsubsection{Metriche per i prodotti software}
Di seguito viene riportato solamente un piccolo sottoinsieme di metriche relative ai prodotti software ritenute come fondamentali dal team 7DOS.
Questa sezione verrà ampliata successivamente qualora fosse necessario introdurre nuove metriche per la valutazione della qualità dei prodotti software.

\paragraph{Functional Implementation Completeness}\Spazio
Misurazione in percentuale del grado con cui le funzionalità offerte dalla corrente implementazione del software coprono l'insieme di funzioni specificate nei requisiti.\\
Abbiamo scelto questa metrica per valutare il grado di completezza del prodotto; l'obiettivo è implementare tutte le funzionalità richieste.\\
Viene utilizzata la seguente formula:
$$FI_{Comp}=\frac{NF_i}{NF_r}*100$$
Dove:
\begin{itemize}
	\item{\textbf{FI\textsubscript{Comp}}: è il valore della metrica;}
	\item{\textbf{NF\textsubscript{i}}: è il numero di funzioni attualmente implementate;}
	\item{\textbf{NF\textsubscript{r}}: è il numero di funzioni specificate dai requisiti.}
\end{itemize}

\paragraph{Average Functional Implementation Correctness}\Spazio
Misurazione in percentuale del grado in cui le funzionalità offerte dalla corrente implementazione del software, in media, rispettano il livello di precisione indicato nei requisiti.\\
Abbiamo scelto questa metrica per valutare il grado di accuratezza e garantire la qualità dei risultati restituiti dal prodotto, in quanto andrà a fare previsioni sulla verosimiglianza di alcuni eventi in base ai dati forniti, ed è necessario che tali previsioni siano sufficientemente accurate.\\
Viene utilizzata la seguente formula:
$$aFI_{Corr}=\frac{\sum\limits_{i=1}^N\frac{iPF_i}{rPF_i}}{N}*100$$
Dove:
\begin{itemize}
	\item{\textbf{aFI\textsubscript{Corr}}: è il valore della metrica;}
	\item{\textbf{iPF\textsubscript{i}}: è il livello di precisione della i-esima funzione implementata;}
	\item{\textbf{rPF\textsubscript{i}}: è il livello di precisione della i-esima funzione secondo i requisiti;}
	\item{\textbf{N}: è il numero totale di funzioni considerate.}
\end{itemize}

\paragraph{Tempo di risposta}\Spazio
Misurazione in secondi che indica il tempo medio che intercorre fra la richiesta software di una determinata funzionalità e la restituzione del risultato all'utente. \\
Viene utilizzata la seguente formula:
$$TR=\frac{\sum\limits_{i=1}^N{T_i}}{N}$$
Dove:
\begin{itemize}
	\item{\textbf{TR}: è il valore della metrica;}
	\item{\textbf{T\textsubscript{i}}: è il tempo intercorso fra la richiesta \emph{i} di una funzionalità ed il comportamento delle operazioni necessarie a restituire un risultato a tale richiesta;}
	\item{\textbf{N}: è il numero totale di funzioni considerate.}
\end{itemize}

\paragraph{Average Learning Time}\Spazio
Misurazione in minuti del tempo medio impiegato da un utente per imparare ad utilizzare una singola funzionalità del prodotto.
\\Abbiamo scelto questa metrica poiché, trattandosi di un prodotto che verrà reso disponibile pubblicamente, abbiamo ritenuto importante renderlo semplice da imparare per permetterne l'uso ad una vasta gamma di utenti.
\\Viene utilizzata la seguente formula:
$$aLT=\frac{\sum\limits_{i=1}^N{LT_i}}{N}$$
Dove:
\begin{itemize}
	\item{\textbf{aLT}: è il valore della metrica;}
	\item{\textbf{LT\textsubscript{i}}: è il tempo necessario ad imparare ad utilizzare la i-esima funzione implementata, espresso in minuti;}
	\item{\textbf{N}: è il numero totale di funzioni considerate.}
\end{itemize}

\paragraph{Failure Density}\Spazio
Misurazione in percentuale della quantità di failure rilevati rispetto alla quantità di test eseguiti.
\\Abbiamo scelto questa metrica per garantire che il prodotto sia generalmente stabile e non risulti poco utilizzabile o inutilizzabile a causa di eccessive \gl{failure}. Valutazioni più precise saranno effettuate in base ai singoli risultati dei test.\\
Viene utilizzata la seguente formula:
$$FD=\frac{T_f}{T_e}*100$$
Dove:
\begin{itemize}
	\item{\textbf{FD}: è il valore della metrica;}
	\item{\textbf{T\textsubscript{f}}: è il numero di test falliti;}
	\item{\textbf{T\textsubscript{e}}: è il numero di test eseguiti.}
\end{itemize}

\paragraph{Blocco di operazioni non corrette}\Spazio
Misurazione in percentuale della quantità di funzionalità in grado di gestire in modo corretto possibili errori.\\
Viene utilizzata la seguente formula:
$$BONC=\frac{NF_E}{NO_T}*100$$
Dove:
\begin{itemize}
	\item{\textbf{BONC}: è il valore della metrica;}
	\item{\textbf{NF\textsubscript{E}}: è il numero di failure evitate;}
	\item{\textbf{NO\textsubscript{T}}: è il numero di test con possibile esecuzione di operazioni non corrette.}
\end{itemize}

\paragraph{Failure Analysis}\Spazio
Misurazione in percentuale delle modifiche effettuate in risposta a failure che hanno portato all'introduzione di nuove failure in altre componenti del sistema. \\
Viene utilizzata la seguente formula:
$$FA=\frac{NF_I}{NF_R}*100$$
Dove:
\begin{itemize}
	\item{\textbf{FA}: è il valore della metrica;}
	\item{\textbf{NF\textsubscript{I}}: è il numero di failure delle quali sono state individuate le cause;}
	\item{\textbf{NF\textsubscript{R}}: è il numero di failure rilevate.}
\end{itemize}

\paragraph{Comment Ratio}\Spazio
Misurazione in percentuale che indica il rapporto tra il numero di righe in cui è presente del codice ed il numero di righe in cui sono presenti dei commenti(o annotazioni). \\
Viene utilizzata la seguente formula:
$$CR=\frac{NR_C}{NR_A}*100$$
Dove:
\begin{itemize}
	\item{\textbf{CR}: è il valore della metrica;}
	\item{\textbf{NR\textsubscript{C}}: è il numero di righe in cui è presente del codice;}
	\item{\textbf{NR\textsubscript{A}}: è il numero di righe in cui sono presenti dei commenti.}
\end{itemize}

\paragraph{Code coverage}\Spazio
Per assicurarsi che la più grande porzione possibile di codice venga testata, si adotta questa metrica, che indica in percentuale, quanto del codice sorgente
è soggetto a test. Più alta tale percentuale, minore il rischio che vi siano criticità non rilevate.
Per il suo calcolo si farà uso degli strumenti automatici descritti nella sezione §2.2.2.4 di questo documento, ovvero TravisCI e Coveralls.
