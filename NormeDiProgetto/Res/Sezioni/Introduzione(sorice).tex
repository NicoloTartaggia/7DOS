\newsection{Introduzione}
\subsection{Scopo del documento}
Il presente documento descrive e fissa tutte le norme, le convenzioni e gli strumenti che verranno adottati dal nostro gruppo per assicurarci un modus operandi, nel \gl{progetto}, comune a tutti i componenti dello stesso. Questo suppone che tutti i componenti del gruppo abbiano preso visione del documento e ne abbiano concordato e accettato i modi per garantire la massima omogeneità e collaborazione per tutto il progetto. 
\subsection{Glossario}
Per rendere la lettura del documento più semplice, chiara e comprensibile viene allegato il \emph{Glossario v1.0.0} nel quale sono contenute le definizioni dei termini tecnici, dei vocaboli ambigui, degli acronimi e delle abbreviazioni. La presenza di un termine all'interno del Glossario è segnalata con una "g" posta come pedice (esempio: $Glossario_{g}$).  
\subsection{Riferimenti}
\subsubsection{Normativi}
\begin{itemize}
	\item \textbf{ISO/IEC 12207:} \url{https://www.math.unipd.it/~tullio/IS-1/2009/Approfondimenti/ISO_12207-1995.pdf} (Ultima consultazione effettuata: TODO da inserire);
	\item \textbf{Da verbali} \emph{.}
\end{itemize}
\subsubsection{Informativi}
\begin{itemize}
	\item \textbf{Norme di Progetto:} \emph{Norme di Progetto v1.0.0.}
	\item \textbf{Piano di Progetto:} \emph{Piano di Progetto v1.0.0.}
	\item \textbf{Capitolato C6:} Soldino: piattaforma \gl{Ethereum} per pagamenti IVA \\ \url{https://www.math.unipd.it/~tullio/IS-1/2018/Progetto/C6.pdf};
	
\end{itemize}
\pagebreak