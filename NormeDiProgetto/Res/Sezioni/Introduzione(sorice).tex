\newsection{Introduzione}
\subsection{Scopo del documento}
Il presente documento descrive e fissa tutte le norme, le convenzioni e gli strumenti che verranno adottati dal nostro team per assicurare un modus operandi comune a tutti i membri nello sviluppo del \gl{progetto}. Questo suppone che tutti i componenti del gruppo abbiano preso visione del documento e ne abbiano concordato e accettato i modi per garantire la massima omogeneità e collaborazione per tutto il progetto.
\subsection{Glossario}
Per rendere la lettura del documento più semplice, chiara e comprensibile viene allegato il documento \emph{Glossario v2.0.0} nel quale sono contenute le definizioni dei termini tecnici, dei vocaboli ambigui, degli acronimi e delle abbreviazioni. La presenza di un termine all'interno del Glossario è segnalata con una "g" posta come pedice (esempio: $Glossario_{g}$).
\subsection{Maturità del documento}
Il presente documento sarà soggetto ad incrementi futuri. Per questo motivo, non si pone l'obiettivo di risultare completo già in questa fase del progetto.
Tale decisione è dovuta al fatto che sono state trattate le esigenze di attività di progetto più impellenti e ricorrenti.
Tutto ciò che riguarda la pianificazione degli incrementi, può essere trovato nel \emph{Piano di Progetto v1.0.0} all'interno della sezione 4.
\subsection{Riferimenti}
\subsubsection{Normativi}
\begin{itemize}
	\item \textbf{ISO/IEC 12207}:\\ \url{https://www.math.unipd.it/~tullio/IS-1/2009/Approfondimenti/ISO_12207-1995.pdf};
	\item \textbf{Verbali:} \emph{Verbale del 2018-12-04}.
\end{itemize}
\subsubsection{Informativi}
\begin{itemize}
	\item \textbf{\gl{ JavaScript Style Guide and Coding Conventions:}} \\
	\url{https://www.w3schools.com/js/js_conventions.asp};
	\item \textbf{\gl{Grafana} Code Styleguide:} \\
	\url{http://docs.grafana.org/plugins/developing/code-styleguide/};
	\item \textbf{\gl{Angular TypeScript} Code Styleguide:} \\
	\url{https://angular.io/guide/styleguide};
		\item \textbf{Trender} \\
	\url{https://github.com/campagna91/Trender/tree/master}.
		\item \textbf{Software Engineering - Ian Sommerville - 10th Edition}(Capitolo 2).
\end{itemize}
\pagebreak
