\newsection{Processi primari}
\subsection{Fornitura}
Il fine di questa sezione è quello di definire le norme che i membri del gruppo 7DOS sono invitati a rispettare con l'obiettivo di proporsi e diventare fornitori nei confronti dell'azienda proponente Zucchetti srl. e dei committenti Prof. Tullio Vardanega e Prof. Riccardo Cardin per quanto concerne il prodotto \emph{G\&B}.
Con la finalità di raggiungere questa meta nel miglior modo possibile, abbiamo intenzione di collaborare in modo efficiente e efficace con i referenti dell'azienda. 
I punti fondamentali che verranno affrontati insieme al proponente saranno:
\begin{itemize}
\item Determinare gli aspetti cruciali al fine di soddisfare l'azienda proponente;
\item Concordare la qualifica del prodotto;
\item Determinare vincoli sui processi e sui requisiti;
\item Stimare i costi del prodotto finale.
\end{itemize}
\subsubsection{Attività}
\subparagraph{Studio di fattibilità}
Successivamente alla presentazione dei Capitolati d'appalto, ogno componente del gruppo ha svolto un'attenta analisi delle proposte presentate per poi decidere durante la riuone del 27 novembre 2017 il capitolato per il quale partecipare alla gara d'appalto. In seguito, gli analisti hanno svolto ulteriori analisi dei rischi e delle opportunità di ogni capitolato che sono sfociate nella redazione del Documento di fattibilità v 1.0.0, nel documento vi troviamo le motivazioni che hanno portato il nostro gruppo a favorire la scelta del prodotto per cui proporci come fornitori. Inoltre, riporta per ogni capitolato le seguenti informazioni:
\begin{itemize}
 	\item\textbf{{Descrizione}}: riporta una breve sintesi del prodotto da sviluppare;
 	\item\textbf{{Studio del dominio}}: riporta un'analisi del dominio applicativo, in cui vi è una più corposa descrizione del prodotto da sviluppare con l'aggiunta di una generale contestualizzazione, e un'analisi del dominio tecnologico, in cui vengono elencate le maggiori tecnologie coinvolte per ogni prodotto secondo la descizione del capitolato e dalle esperienze pregresse dei componenti del gruppo;
	\item\textbf{{Valutazione generale}}:composta dagli aspetti positivi, dai fattori di rischio e da una Valutazione finale in cui vi si può trovare in breve la motivazione della scelta presa per ogni capitolato in base a ciò che è stato riportato nelle due sezioni precedenti; 	
 	
\end{itemize}
\subparagraph{Piano di progetto}

\subparagraph{Piano di qualifica}

\subsection{Sviluppo}
\pagebreak