\newsection{Processi primari}
\subsection{Fornitura}
Il fine di questa sezione è quello di definire le norme che i membri del gruppo 7DOS sono invitati a rispettare con l'obiettivo di proporsi e diventare fornitori nei confronti dell'azienda proponente Zucchetti srl. e dei committenti Prof. Tullio Vardanega e Prof. Riccardo Cardin per quanto concerne il prodotto \emph{G\&B}.
Con la finalità di raggiungere questa meta nel miglior modo possibile, abbiamo intenzione di collaborare in modo efficiente e efficace con i referenti dell'azienda. 
I punti fondamentali che verranno affrontati insieme al proponente saranno:
\begin{itemize}
\item Determinare gli aspetti cruciali al fine di soddisfare l'azienda proponente;
\item Concordare la qualifica del prodotto;
\item Determinare vincoli sui processi e sui requisiti;
\item Stimare i costi del prodotto finale.
\end{itemize}
\subsubsection{Attività}
\subparagraph{Studio di fattibilità}
Successivamente alla presentazione dei Capitolati d'appalto, ogno componente del gruppo ha svolto un'attenta analisi delle proposte presentate per poi decidere durante la riuone del 27 novembre 2017 il capitolato per il quale partecipare alla gara d'appalto. In seguito, gli analisti hanno svolto ulteriori analisi dei rischi e delle opportunità di ogni capitolato che sono sfociate nella redazione del Documento di fattibilità v 1.0.0, nel documento vi troviamo le motivazioni che hanno portato il nostro gruppo a favorire la scelta del prodotto per cui proporci come fornitori. Inoltre, riporta per ogni capitolato le seguenti informazioni:
\begin{itemize}
 	\item\textbf{{Descrizione}}: riporta una breve sintesi del prodotto da sviluppare;
 	\item\textbf{{Studio del dominio}}: riporta un'analisi del dominio applicativo, in cui vi è una più corposa descrizione del prodotto da sviluppare con l'aggiunta di una generale contestualizzazione, e un'analisi del dominio tecnologico, in cui vengono elencate le maggiori tecnologie coinvolte per ogni prodotto secondo la descizione del capitolato e dalle esperienze pregresse dei componenti del gruppo;
	\item\textbf{{Valutazione generale}}:composta dagli aspetti positivi, dai fattori di rischio e da una Valutazione finale in cui vi si può trovare in breve la motivazione della scelta presa per ogni capitolato in base a ciò che è stato riportato nelle due sezioni precedenti; 		
\end{itemize}
\subparagraph{Piano di progetto}
La redazione di un piano da seguire durante la realizzazione del progetto spetta al Responsabile, aiutato nelle scelte dagli Amministratori. Il documento dovrà dovrà coprire le seguenti tematiche:
 \begin{itemize}
 	\item\textbf{{Analisi dei rishi}}: riporta una dettagliata analisi dei rischi che si potrebbero incontrare durante la realizzazione del progetto, determinadone, in base alle conoscenze pregresse e alle nuove acquisite, la probabilità che essi accadano e la loro gravità. Inoltre, quando possibile, verranno analizzati i possibili metodi per affrontarli;
 	\item\textbf{{Pianificazione}}: viene presentata una pianificazione delle attività da svolgere nel corso del progetto, fornendo delle scadenze temporali il più possibile precise e veritiere;
 	\item\textbf{{Preventivo}}: sulla base della pianificazione, viene stimata la quantità di lavoro necessaria per portare a termine ogni attivita (e quindi ogni fase) del progetto, per arrivare infine ad avere una valutazione complessiva per tutto il progetto e proporre un preventivo finale con il costo del lavoro precedentemente stimato; 		
 \end{itemize}
\subparagraph{Piano di qualifica}
Il compito di \gl{verifica} e \gl{validazione} viene svolto da parte dei Verificatori, questa masione verrà svolta secondo un preciso metodo che deve coprire le seguenti tematiche:
\begin{itemize}
	\item\textbf{{Metodo di verifica}}: riporta le procedure di controllo sulla qualità di processo e di prodotto, considerando i mezzi e le risorse a disposizione; 	
	\item\textbf{{Misure e metriche}}: presenta criteri oggettivi per i documenti, i processi e il software; 	
	\item\textbf{{Gestione della revisione}}: precisa nel dettaglio le metodologie di comunicazione delle procedure di controllo per la qualità di processo e delle anomalie; 	
	\item\textbf{{Pianificazione del collaudo}}: definisce dettagliatamente le metodologie di collaudo a cui sarà sottoposto il progetto realizzato; 	
	\item\textbf{{Resoconto dell'attività di verifica}}: riporta le metriche calcolate e il resoconto sul collaudo delle attività sottoposte a verifica e validazione; 			
\end{itemize}
\subsection{Sviluppo}
Il processo in questione affronta le attività ed i compiti svolti dal gruppo con l'obiettivo di sviluppare il software richiesto dal proponente. Per una corretta implementazione è fondamentale:
\begin{itemize}
		\item realizzare un prodotto finale conforme alle richeste del proponente; 	
		\item realizzare un prodotto fianle che soddisfa i test di verifica e validazione; 	
		\item fissare gli obiettivi di sviluppo; 	
		\item fissare i vincoli tecnologici e di design; 	
\end{itemize}
Inoltre, il gruppo ha deciso di seguire le linee guida dettate dallo standard ISO/IEC 12207, per questo motivo le attività alla base del progetto di sviluppo saranno le seguenti:
\begin{itemize}
	\item Analisi dei requisiti; 	
	\item Progettazione;	
	\item Codifica; 	
\end{itemize}
\subsubsection{Attività}
\subparagraph{Analisi dei requisiti}
\subparagraph{Progettazione}
\subparagraph{Codifica}
\subparagraph{Procedure}
\subsection{Strumenti}
\pagebreak