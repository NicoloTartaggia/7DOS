\newsection{Processi primari}
\subsection{Fornitura}
Il fine di questa sezione è quello di definire le norme che i membri del gruppo 7DOS sono invitati a rispettare con l'obiettivo di proporsi e diventare fornitori nei confronti dell'azienda proponente Zucchetti s.r.l. e dei committenti Prof. Tullio Vardanega e Prof. Riccardo Cardin per quanto concerne il prodotto \emph{G\&B}.
Per raggiungere questa meta nel miglior modo possibile, abbiamo intenzione di collaborare in modo \gl{efficiente} e \gl{efficace} con i referenti dell'azienda.
I punti fondamentali che verranno affrontati insieme al proponente saranno:
\begin{itemize}
\item Determinare gli aspetti cruciali al fine di soddisfare l'azienda proponente;
\item Concordare la qualifica del prodotto;
\item Determinare vincoli sui processi e sui requisiti;
\item Stimare i costi del prodotto finale.
\end{itemize}
Di seguito verranno riportate tutte le attività che compongono questo processo:
\subsubsection{Studio di Fattibilità}
Nel documento \emph{Studio di Fattibilità v1.0.0.} troviamo le motivazioni che hanno portato il nostro gruppo a favorire la scelta del prodotto per cui proporci come fornitori. Inoltre, si riportano, per ogni capitolato, le seguenti informazioni:
\begin{itemize}
 	\item\textbf{{Descrizione}}: riporta una breve sintesi del prodotto da sviluppare;
 	\item\textbf{{Studio del dominio}}: riporta un'analisi del dominio applicativo, in cui vi è una più corposa descrizione del prodotto da sviluppare con l'aggiunta di una generale contestualizzazione, e un'analisi del dominio tecnologico, in cui vengono elencate le maggiori tecnologie coinvolte per ogni prodotto secondo la descrizione del capitolato e dalle esperienze pregresse dei componenti del gruppo;
 	\item\textbf{{Valutazione generale}}: composta dagli aspetti positivi e dagli aspetti negativi trovati e discussi dal gruppo riguardo al capitolato in esame;
 	\item\textbf{{Valutazione finale}}: vi si può trovare in breve la motivazione della scelta presa per ogni capitolato in base a ciò che è stato riportato nelle tre sezioni precedenti.
\end{itemize}
\pagebreak
\subsubsection{Piano di Progetto}
La redazione di un \gl{piano} da seguire durante la realizzazione del progetto spetta al \emph{Responsabile}, aiutato nelle scelte dagli \emph{Amministratori}. Il documento dovrà coprire le seguenti tematiche:
 \begin{itemize}
 	\item\textbf{{Analisi dei rischi}}: riporta una dettagliata analisi dei rischi che si potrebbero incontrare durante la realizzazione del progetto, determinandone, in base alle conoscenze pregresse e alle nuove acquisite, la probabilità che essi accadano e la loro gravità. Inoltre, quando possibile, verranno analizzati i possibili metodi per affrontarli;
 	\item\textbf{{Pianificazione}}: viene presentata una pianificazione delle \gl{attività} da svolgere nel corso del progetto, fornendo delle scadenze temporali il più possibile precise e veritiere;
 	\item {\textbf{Suddivisione risorse}: riporta la suddivisione oraria delle risorse, tenendo conto dei ruoli ricoperti da tutti i membri. In questo modo si ha una visione chiara su come il gruppo sta procedendo e quante ore vengono impiegate per ogni attività;}
 	\item\textbf{{Preventivo}}: sulla base della pianificazione, viene stimata la quantità di lavoro necessaria per portare a termine ogni attività (e quindi ogni fase) del progetto, per arrivare infine ad avere una valutazione complessiva per tutto il progetto e proporre un preventivo finale con il costo del lavoro precedentemente stimato.
 \end{itemize}
\paragraph{Strumenti di supporto}
\subparagraph{Gantt Project}\Spazio
Per quanto riguarda la creazione dei diagrammi di Gantt è stato deciso di utilizzare Gantt Project data la sua completezza, facilità di utilizzo e la possibilità di esportare i diagrammi sotto forma di \gl{PNG}. Inoltre, i file di Gantt Project sono salvati in formato \gl{XML}, quindi facilmente versionabili.
\subparagraph{Microsoft Excel}\Spazio
Per la realizzazione dei consuntivi è stato deciso di utilizzare Microsoft Excel come strumento per creare i grafici, data la sua efficacia e facilità di utilizzo.
\subsubsection{Piano di Qualifica}
Il \gl{compito} di \gl{verifica} e \gl{validazione} viene svolto da parte dei \emph{Verificatori}. Questa mansione verrà svolta secondo un preciso metodo che deve coprire le seguenti tematiche:
\begin{itemize}
	\item\textbf{{Metodo di verifica}}: riporta le procedure di controllo sulla \gl{qualità} di \gl{processo} e di \gl{prodotto}, considerando i mezzi e le risorse a disposizione;
	\item\textbf{{Misure e metriche}}: presenta criteri oggettivi per i documenti, i processi e il software;
	\item\textbf{{Gestione della revisione}}: precisa nel dettaglio le metodologie di comunicazione delle procedure di controllo per la qualità di processo e delle anomalie;
	\item\textbf{{Pianificazione del collaudo}}: definisce dettagliatamente le metodologie di collaudo a cui sarà sottoposto il progetto realizzato;
	\item\textbf{{Resoconto dell'attività di verifica}}: riporta le metriche calcolate e il resoconto sul collaudo delle attività sottoposte a verifica e validazione.
\end{itemize}
\subsection{Sviluppo}
Il processo in questione affronta le attività ed i compiti svolti dal gruppo con l'obiettivo di sviluppare il software richiesto dal proponente. Per una corretta implementazione è fondamentale:
\begin{itemize}
		\item Realizzare un prodotto finale conforme alle richieste del proponente;
		\item Realizzare un prodotto finale che soddisfa i test di verifica e validazione;
		\item Fissare gli obiettivi di sviluppo;
		\item Fissare i vincoli tecnologici e di design.
\end{itemize}
Inoltre, il gruppo ha deciso di seguire le linee guida dettate dallo standard \gl{ISO}/\gl{IEC} 12207. Per questo motivo le attività scelte alla base del progetto di sviluppo saranno le seguenti:
\begin{itemize}
	\item Analisi dei Requisiti;
	\item Progettazione;
	\item Codifica.
\end{itemize}
\subsubsection{Analisi dei requisiti}
\paragraph{Scopo}\Spazio
Determinare con precisione i requisiti del progetto ed elencarli in modo formale. Essi vengono estrapolati da varie fonti:
\begin{itemize}
	\item Documenti di specifica del \gl{capitolato};
	\item Incontri con l'azienda proponente;
	\item Verbali interni ed esterni;
	\item Casi d'uso.
\end{itemize}
\paragraph{Casi d'uso}\Spazio
Ogni caso d'uso è descritto dalla seguente struttura:
\begin{itemize}
	\item codice identificativo: $$ \textbf{UC \{codice\_padre\}.\{codice\_figlio\}  } $$
		\begin{itemize}
				\item UC specifica che si tratta di un caso d'uso;
				\item Codice\_padre identifica univocamente i casi d'uso;
				\item Codice\_figlio è un numero progressivo che identifica i sottocasi.
		\end{itemize}
	\item Titolo;
	\item Diagramma UML;
	\item Attori;
	\item Scopo e descrizione;
	\item Precondizione;
	\item Flusso base degli eventi;
	\item Postcondizioni;
	\item Inclusioni (se presenti);
	\item Estensioni (se presenti).
\end{itemize}
\paragraph{Requisiti}\Spazio
Ogni requisito è descritto dalla seguente struttura:
\begin{itemize}
	\item Nome;
	\item Tipo;
	\item Importanza;
	\item Stato implementazione;
	\item Fonti.
\end{itemize}

Inoltre, a ciascun requisito corrisponde un codice identificativo cosi composto:
$$ \textbf{R \{importanza\}.\{tipo\}.\{identificativo\}  } $$
\begin{itemize}
	\item R specifica che si tratta di un requisito ;
	\item importanza identifica la rilevanza del requisito e può assumere 3 valori:
	\begin{itemize}
		\item 0: indica che il requisito è obbligatorio e il suo soddisfacimento dovrà necessariamente avvenire;
		\item 1: indica che il requisito è desiderabile, cioè il suo soddisfacimento può portare maggiore completezza al sistema ma non è fondamentale per lo stesso;
		\item 2: indica che il requisito è opzionale, e quindi la decisione di implementarlo o meno verrà presa dopo le dovute considerazioni;
	\end{itemize}
	\item Tipo distingue se si tratta di un requisito funzionale (F), di qualità (Q), di prestazione (P) o di vincolo (V);
	\item Identificativo è un numero progressivo che identifica i sottocasi.
\end{itemize}
\paragraph{UML}\Spazio
I diagrammi UML devono essere realizzati usando la versione del linguaggio \emph{v2.0}
\paragraph{Strumenti di supporto}
\subparagraph{Trender}\Spazio
Per quanto riguarda il tracciamento dei requisiti, il gruppo ha deciso di utilizzare inizialmente uno strumento non totalmente adatto a questo compito, cioè un foglio di calcolo. Essendo a conoscenza delle limitate funzionalità offerte da questa tecnologie per lo scopo a noi d'interesse, il team 7DOS ha deciso di migrare i dati raccolti in uno strumento creato appositamente per questo scopo. Lo strumento scelto è Trender e rende il tracciamento dei requisiti semi-automatico, inoltre offre la creazione di file .tex per il tracciamento dei requisiti.
\subparagraph{Astah UML}\Spazio
Per quanto concerne la creazione e la modellazione dei diagrammi UML per i casi d'uso, è stato deciso di utilizzare Astah UML, data la sua facilità di utilizzo e compatibilità con UML 2.x. Per la successiva fase di modellazione delle classi, potrebbe verificarsi una scelta diversa riguardo al software da utilizzare.
\subsubsection{Progettazione}
\paragraph{Scopo}\Spazio
Esplicita concretamente una prima forma ad alto livello del design del software pensata per il progetto, determinandone le caratteristiche più in evidenza in modo da:
	\begin{itemize}
	\item \textbf{{Comunicare con gli stakeholder}} e dare delle informazioni chiare, così da intraprendere le prime discussioni sul progetto;
	\item \textbf{{Analizzare il sistema}} in modo esplicito nei primi passi del progetto può portare ad importanti considerazioni riguardo a performance, affidabilità e manutenibilità;
	\item Capire come il sistema è organizzato e come i componenti interoperano tra di loro sottolineando la possibilità di \textbf{{riutilizzo in larga scala}} di parte del codice, dato che si è notata una frequente ripetizione dei requisiti in un sistema complesso come quello affrontato dal nostro team.
	\end{itemize}
\paragraph{Attività}\Spazio
Il design del software non è composta da una singola attività, bensì da un insieme preciso di sotto-attività progettuali che portano al design finale.
Le sotto-attività che il nostro gruppo ha scelto di svolgere sono:
	\begin{itemize}
	\item\textbf{{Progettazione architetturale}}: i progettisti identificheranno una struttura globale del sistema, quindi i componenti principali, le loro relazioni e come saranno distribuiti;
	\item\textbf{{Progettazione dell'interfaccia}}: i progettisti definiranno le interfacce tra i \gl{moduli} di sistema. Ciò deve essere fatto assolutamente in modo non ambiguo dato che, grazie a queste specifiche, ogni componente potrà usare in modo appropriato le funzionalità degli altri componenti del progetto senza sapere l'effettiva implementazione;
	\item\textbf{{Selezione e progettazione dei componenti}}: i progettisti cercheranno dei componenti riutilizzabili e valuteranno se inserirli nel progetto, specificandone gli appropriati utilizzi e aggiustamenti, o se sarà più opportuno costruire dei nuovi componenti.
\end{itemize}
\paragraph{Codifica}\Spazio
Le convenzioni stilistiche definite in \gl{Grafana Plugin Code Styleguide} e, più in particolare per il linguaggio Typescript, in \gl{Angular TypeScript} Styleguide verranno seguite dai programmatori per lo sviluppo dell'intero progetto. \\
Soltanto il Responsabile di progetto, dopo un'attenta analisi e valutazione, potrà ammettere modifiche alle convinzioni stabilite.\\
L’unica lingua ammessa per i nomi di variabili, classi e funzioni è l’inglese.
\pagebreak