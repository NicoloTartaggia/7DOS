\section{Valutazione sul capitolato scelto}
	\subsection{Capitolato C3 - G\&B: monitoraggio intelligente di processi DevOps}
	
		\subsubsection{Descrizione}
		Il capitolato C3 ci introduce al mondo dei \gl{DevOps} in cui gli operatori che erogano i servizi e coloro che fabbricano il software sono a stretto contatto. Per far si che la collaborazione sia efficace, è necessario un intenso e completo monitoraggio dei sistemi. Da qui nasce la necessità di rendere “intelligente” il controllo sulle risorse, così da massimizzare l’efficienza e la qualità degli interventi al sistema.

	\subsubsection{Studio del dominio}
		\paragraph{Dominio applicativo} \Spazio
		Il proposito del progetto è la realizzazione di un \gl{plug-in} per il sistema di monitoraggio \gl{Grafana}. Si vuole che esso possa, tramite l’utilizzo dell’intelligenza artificiale, monitorare la \gl{livellines} del sistema e riesca a consigliare degli interventi strategici da effettuare sullo stesso (o almeno a delineare la zona di intervento). Tutto ciò tramite l’utilizzo delle \gl{reti Bayesiane}, dato che è un ottimo strumento per coniugare le conoscenze e competenze degli esperti, con la flessibilità dei sistemi basati sulla probabilità.
		\paragraph{Dominio tecnologico}
		\begin{itemize}
		\item\textbf{{Grafana}}: sistema di monitoraggio e analisi opensource;
		\item\textbf{{Reti Bayesiane}} 
		\item\textbf{{\gl{JavaScript}}}
		\item\textbf{{\gl{JSON}}}: formato adatto per la lettura dei nodi della rete Bayesiana.
		\end{itemize}
	
	\subsubsection{Conclusioni}
		\paragraph{Aspetti positivi}
		\begin{itemize}
			\item{Interesse comune per tecnologie moderne come le reti Bayesiane e l'intelligenza artificiale;}
			\item{Collaborazione con un'importante azienda del territorio italiano;}
			\item{Interesse per lo sviluppo di un prodotto open source.}
		\end{itemize}
		\paragraph{Aspetti negativi}	
		\begin{itemize}
			\item{Nessuno dei componenti del gruppo ha mai lavorato in ambiente Grafana;}
			\item{La conoscenza del linguaggio JavaScript e delle sue librerie deve essere approfondita.}
		\end{itemize} 
		\paragraph{Valutazione finale} \Spazio
		L'uso di tecnologie innovative legate al mondo dell'intelligenza artificiale come le reti Bayesiane è risultato di interesse comune a tutti i membri e ciò ha portato l'intero gruppo a scegliere questo capitolato, sebbene nessuno dei componenti abbia conoscenze approfondite riguardo gli strumenti che si andranno ad usare.  
				
		 
\section{Valutazione sugli altri capitolati}
	\subsection{Capitolato C1 - Butterfly}
		\subsubsection{Descrizione}
	    Il primo capitolato, denominato \emph{Butterfly: monitor per processi \gl{CI/CD}} propone la creazione di un sistema publisher subscriber per facilitare e automatizzare l'invio e la ricezione delle segnalazioni provenienti dai vari strumenti utilizzati durante i processi di continuous integration e continuous delivery.
		\subsubsection{Studio del dominio}
			\paragraph{Dominio applicativo} \Spazio
			Il sistema che si richiede venga sviluppato in particolare dovrà essere composto da 4 elementi:
			\begin{itemize}	 
\item \textbf{{Producers}}: hanno il compito di recuperare le segnalazioni e ripubblicarle sotto forma di messaggi all'interno dei
topic adeguati;
\item \textbf{{Consumers}}: hanno il compito di abbonarsi a dei topic e inoltrare ai destinatari finali tutte le segnalazioni
provenienti dal suddetto topic;
\item \textbf{{Broker}}: strumento utile per la gestione ed istanziazione dei topic;
\item \textbf{{Componenti aggiuntive custom}}: si richiede, in particolare, la realizzazione di una componente che riesca a determinare la persona più adatta a cui inoltrare la segnalazione e la invii solo ed esclusivamente a lei.
			\end{itemize}
			\paragraph{Dominio tecnologico}
			\begin{itemize}
				\item \textbf{\gl{Java8+}}; 
				\item  \textbf{\gl{Apache Kafka}}: piattaforma a bassa latenza per la gestione dei \gl{feed dati} in tempo reale;
				\item  \textbf{\gl{Docker}}: progetto open source per l'automatizzazione del deployment (consegna al cliente);
				\item  \textbf{\gl{SonarQube}}: piattaforma per l'analisi statica del codice;
				\item  \textbf{\gl{Redmine}}: \gl{Issue Tracking System} e \gl{Project Management Tool};
				\item  \textbf{\gl{GitLab}}: software di versionamento.
			\end{itemize}
			
<<<<<<< Updated upstream
			\paragraph{Aspetti positivi}
=======
			\paragraph{Aspetti positivi} 
>>>>>>> Stashed changes
				\begin{itemize}
				\item {Linguaggio di programmazione (Java8+) affrontato durante l'anno accademico nel corso di Programmazione Concorrente e Distribuita. Di conseguenza i componenti del team hanno già familiarità con esso;}
				\item{Ampio dominio tecnlogico che permette di allargare le proprie conoscenze, utile anche per un utilizzo futuro in ambito lavorativo;}
			\end{itemize}
			\paragraph{Aspetti negativi}
			\begin{itemize}
				\item {Alcune tecnologie non sono conosciute dal gruppo, come SonarQube, Apache Kafka. Ciò comporta uno studio preventivo prima dell'analisi dei requisiti;}
			\end{itemize} 
			\paragraph{Conclusioni}
			L'idea di sviluppare un'applicazione che possa connettere tra loro molteplici strumenti per CI/CD largamente utilizzati in ambito di sviluppo e' stata accolta positivamente dal team. Tuttavia il capitolato non è stato scelto visto il numero di tecnologie sconosciute e l'interesse per altre tecnologie da parte del gruppo.
			
	\subsection{Capitolato C2 - Colletta: piattaforma raccolta dati di analisi di testo}
	\subsubsection{Descrizione}
	L'obbiettivo del progetto è quello di realizzare una piattaforma collaborativa di raccolta dati in cui gli utenti possano avere a disposizione e svolgere piccoli esercizi di grammatica, come l'analisi grammaticale.
	I dati raccolti inoltre dovranno essere facilmente accessibili dagli sviluppatori con lo scopo di ottimizzare un software per l'analisi grammaticale mediante tecniche di apprendimento automatico.   
	\subsubsection{Studio del dominio}
<<<<<<< Updated upstream
	\paragraph{Dominio applicativo} \Spazio
	All'interno della piattaforma è dunque possibile distinguere tre attori principali:
	\begin{itemize} 
	\item \textbf{Insegnati}: hanno interesse a preparare in modo semplice e rapido degli esercizi per gli allievi, l'insegnante non dovrà scrivere direttamente le soluzioni dell'esercizio, esse verranno proposte automaticamente da un software per
	l'analisi grammaticale. L'insegnante dunque dovrà solamente correggere eventuali errori presenti nelle soluzioni proposte automaticamente;
	\item \textbf{Allievi}: hanno interesse ad eseguire gli esercizi in modo rapido ed immediato;
	\item \textbf{Sviluppatori}: hanno interesse ad accedere ai dati in modo facile ed immediato ai fini di migliorare il software per l'analisi grammaticale.
	\end{itemize}
	\paragraph{Dominio tecnologico} \Spazio
	Il committente non impone l'uso di specifiche tecnologie e lascia una discreta libertà sulla scelta di esse. Di seguito vengono riportate quelle suggerite nel capitolato.
=======
	\paragraph{Dominio applicativo} 
	Il software è utilizzabile nel mercato online e molti utenti potrebbero usufruirne sia per cercare lavoro, sia per avere un CV online sempre accessibile e aggiornato con il tempo con la conferma delle competenze da entità verificate.
	\paragraph{Dominio tecnologico} 
>>>>>>> Stashed changes
	\begin{itemize}
		\item \textbf{HTML5, CSS3 e JavaScript}: linguaggi per lo sviluppo di siti web;
		\item \textbf{\gl{Hunpos} o \gl{FreeLing}}: software open source per il \gl{part-of-speech tagging};
		\item  \textbf{\gl{Firebase}}: database \gl{NoSQL};
	\end{itemize}
	
	\paragraph{Aspetti positivi}
<<<<<<< Updated upstream
	\begin{itemize}
		\item {da aggiungere;}
	\end{itemize} 
	\paragraph{Aspetti negativi} 
=======
>>>>>>> Stashed changes
	\begin{itemize}
	\item {da aggiungere;}
	\end{itemize} 
<<<<<<< Updated upstream
	\paragraph{Conclusioni} \Spazio
	Il capitolato non è stato ben accolto dai membri del team a causa della tematica scarsamente interessante e dell'assenza di tecnologie realmente innovative che possano fornire nuove capacità e conoscenze.
=======
	\paragraph{Aspetti negativi} 
	Per alcuni dei componenti del gruppo le tecnologie da utilizzare e alcuni framework imposti come vincolo obbligatorio dall'azienda sono sconosciuti.
	\paragraph{Conclusioni}
	Per via degli aspetti precedentemente elencati il gruppo ha preferito concentrarsi su un capitolato diverso.
>>>>>>> Stashed changes
	
	
	\subsection{Capitolato C4 - MegAlexa: arricchitore di skill di Amazon Alexa}
	\subsubsection{Descrizione}	
		Il capitolato C4 ci introduce al nuovo mondo delle user interface. Ciò avviene mettendoci davanti all’interfaccia più in voga del momento, cioè \gl{Amazon Alexa}. Infatti i dispositivi che fanno uso di Alexa sfruttano il modo più semplice e veloce per l’uomo di esprimere i propri bisogni e desideri, la voce. 
	\subsubsection{Studio del dominio}
	
<<<<<<< Updated upstream
	\paragraph{Dominio applicativo} \Spazio
     L’obiettivo che si pone il progetto è quello di creare un’interfaccia web o una mobile app per la customizzazione dei \gl{workflow} che poi verranno avviati tramite una skill creata ad hoc dal team di sviluppo.
     Ciò vuol dire che un utente potrà utilizzare la sua immaginazione per dare vita ai workflow più consoni alle sue abitudine e bisogni in base a dei connettori precedentemente creati dal nostro team che legano insieme più funzioni (ad esempio, leggi l'ora - leggi il meteo - leggi posta).
=======
	\paragraph{Dominio applicativo} 
     Il capitolato è posizionato nell'ambito della sintesi vocale, tecnologia molto presente nel mercato ordierno. 
>>>>>>> Stashed changes
	\paragraph{Dominio tecnologico}
	\begin{itemize}
		\item {\textbf{\gl{AWS}} con le relative \gl{API Getway}, \gl{Lamda} e \gl{Aurora Serverless}};
		\item  \textbf{\gl{NodejS}}: piattaforma open source per l'esecuzione di codice JavaScript server-side;
		\item \textbf{HTML5, CSS3 e JavaScript}: linguaggi per lo sviluppo di siti web;
		\item  \textbf{Swift o Kotlin} linguaggi per lo sviluppo di mobile app.
	\end{itemize}
	\paragraph{Aspetti positivi}
	\begin{itemize}
	\item {Le tecnologie impiegate sono risultate innovative e stimolanti;}
	\end{itemize}
<<<<<<< Updated upstream
=======
	\paragraph{Aspetti positivi} 
	Interessante la sintesi vocale in quanto è una tecnologia innovativa e ancora in fase di sviluppo in molti settori.
>>>>>>> Stashed changes
	\paragraph{Aspetti negativi}
	\begin{itemize}
		\item Lo sviluppo di un'interfaccia web o mobile non rappresenta un argomento d'interesse per il gruppo;
		\item Il capitolato non descrive in modo chiaro e preciso l'obiettivo del progetto
	\end{itemize}
<<<<<<< Updated upstream
	\paragraph{Conclusioni} \Spazio
	Il capitolato in esame ha suscitato interesse da parte del team. L'utilizzo di tecnologie giovani e innovative come Alexa e AWS rappresentano una nuova generazione di interfacce, di cui è previsto uno sviluppo intenso nei prossimi anni, tuttavia l'implementazione di una applicazione mobile o di una interfaccia web non rappresenta una sfida abbastanza impegantiva in quanto tecnologie diffuse e poco stimolanti.
=======
	\paragraph{Conclusioni} 
	Per i fattori appena elencati, il gruppo si è orientato in un capitolato diverso.
>>>>>>> Stashed changes
	
	\subsection{Capitolato C5 - P2PCS: piattaforma di peer-to-peer car sharing}
		\subsubsection{Descrizione}
		Il capitolato C5 prevede la realizzazione di un applicativo mobile finalizzato allo sharing \gl{p2p} di un proprio mezzo di trasporto.
		\subsubsection{Studio del dominio}
			\paragraph{Dominio applicativo} \Spazio
			L'applicativo ha lo scopo di far incontrare domanda e offerta di noleggio auto p2p. L'affittuario dovrà poter indicare in quali giorni e in quali orari il suo veicolo è disponibile per un'eventuale prenotazione, il luogo in cui ritirarlo ed eventuali altre condizioni alle quali il cliente dovrà sottostare per affittare il mezzo.
			Il cliente, invece, avrà la possibilità di cercare e prenotare veicoli per una determinata data e luogo. Il tutto dovrà essere accompagnato da un sistema che coinvolga l'utente tramite sistemi di \gl{Gamification theory}.
			\paragraph{Dominio tecnologico} \Spazio
				\begin{itemize}
					\item  \textbf{NodejS}: piattaforma open source per l'esecuzione di codice JavaScript server-side;
					\item \textbf{\gl{Android}}: sistema operativo per smartphone;
					\item \textbf{\gl{Octalysis}}: \gl{Gamification framework}:
				\end{itemize}
		\subsubsection{Aspetti positivi}
		\begin{itemize}
			\item da inserire
		\end{itemize}
		\subsubsection{Aspetti negativi}
		\begin{itemize}
			\item da inserire
		\end{itemize}
		\subsubsection{Conclusioni}
		Il capitolato non ha colto l'interesse del team per via del contesto poco interessante e della, quantomeno apparente, semplicità del progetto. Infatti, lo sviluppo di una piattaforma web è stato visto poco interessante rispetto alle altre proposte e non innovativo dal punto di vista delle tecnolgie interessate. 
		Il capitolato C5 \emph{P2PCS:  piattaforma di peer-to-peer car sharing} , prevede la realizzazione di un
applicativo mobile finalizzato allo sharing p2p di un proprio mezzo di trasporto.

		\subsubsection{Studio del dominio}
<<<<<<< Updated upstream
			\paragraph{Dominio applicativo} \Spazio
			L'applicativo ha lo scopo di far incontrare domanda e offerta di noleggio auto p2p.
L'affittuario, dovrà poter indicare in quali giorni ed orari il suo veicolo è disponibile
per una eventuale prenotazione, il luogo in cui ritirarlo, ed eventuali altre
condizioni alle quali il cliente dovrà sottostare per affittare il mezzo.
Il cliente avrà la possibilità di cercare e di prenotare veicoli per data e luogo.
Il tutto dovrà essere accompagnato da un sistema che coinvolga l'utente tramite sistemi di
Gamification theory.
			\paragraph{Dominio tecnologico} \Spazio
			E' richiesta la conoscenza delle seguenti tecnologie:
=======
			\paragraph{Dominio applicativo} 
			L'applicazione deve essere di aiuto per la progettazione di un buon software: deve permettere la realizzazione di diagrammi UML da cui deve essere possibile la generazione di codice. Potranno essere disegnati diagrammi di robustezza seguendo le regole con cui i tre tipi di oggetti rappresentabili dai diagrammi di robustezza (le interfacce, le procedure e le entità persistenti) possono reagire tra di loro.
			\paragraph{Dominio tecnologico} 
			E' richiesta la conoscenza delle seguenti tecnologie per la realizzazione dell'applicazione web:
>>>>>>> Stashed changes
				\begin{itemize}
					\item \textbf{\gl{Node.JS}}: piattaforma per l'esecuzione di codice JS server-side;
					\item \textbf{\gl{Android}}: OS per smartphone;
					\item \textbf{\gl{Framework Octalysis}}: Gamification Framework.
				\end{itemize}
		\subsubsection{Aspetti positivi}
		\begin{itemize}
			\item Il \gl{proponente} fornisce componenti software per le parti più complesse del capitolato.
		\end{itemize}
		\subsubsection{Aspetti negativi}
		\begin{itemize}
			\item Le tecnologie coinvolte non hanno colto l'interesse del team;
			\item Il capitolato è apparso troppo semplice.
		\end{itemize}
		\subsubsection{Conclusioni}
		Questo capitolato non ha generato interesse nel team per via della sua semplicità e per la mancanza di interesse nel contesto da parte del team. Pertanto non è stato scelto.
		
		
	\subsection{Capitolato C6 - Soldino: piattaforma Ethereum per pagamenti IVA}
		\subsubsection{Descrizione}
		Il capitolato \emph{Marvin, dimostratore di Uniweb su Ethereum}, proposto dall'azienda RedBabel, propone di realizzare una versione di Uniweb come una \gl{Dapp} che giri su \gl{Ethereum Virtual Machine}. \\  Infatti, come su Uniweb, devono poter interagire:
		\begin{itemize} 
			\item studenti, per aver accesso alla propria carriera universitaria, registrarsi ad esami, accettare e rifiutare voti; 
			\item professori, per pubblicare liste di esami, pubblicare voti;
			\item Università, per gestire corsi, orario, spazi, e altro.
		\end{itemize}
	 	Le interazioni tra questi tre attori vengono tradotte con una serie di smart contracs.
		\subsubsection{Studio del dominio}
			\paragraph{Dominio applicativo} 
			Questo capitolato si pone l'obbiettivo unire le tecnologie ad oggi in forte ascesa con il mondo universitario. Come per Uniweb saranno studenti, professori ed Università ad utilizzare il prodotto risultante per ottimizzare studio e lavoro.
			\paragraph{Dominio tecnologico}
			Per comprendere a fondo il dominio e per realizzare il progetto è richiesta la conoscenza delle seguenti tecnologie:
			\begin{itemize}
				\item Ethereum;
				\item \gl{Truffle};
				\item \gl{Etherscan.io};
				\item Javascript;
				\item \gl{ESLint};
				\item \gl{React};
				\item \gl{SCSS}.
			\end{itemize}
		\subsubsection{Aspetti positivi}
		\begin{itemize}
			\item Particolare interesse da parte di vari membri del team verso il dominio del capitolato;
			\item Le tecnologie utilizzate sono sempre più richieste nel mondo del lavoro e una loro conoscenza approfondita gioverebbe ad ognuno dei componenti del gruppo.
		\end{itemize}
		\subsubsection{Aspetti negativi}
		\begin{itemize}
			\item Il dominio applicativo del software è molto vasto e la conoscenze di esso da parte dei membri del team non sono sufficienti;
			\item Può risultare scomodo contattare il proponente dato che ha sede ad Amsterdam.
		\end{itemize}
		\subsubsection{Conclusioni}
		Sebbene le conoscenze acquisite dallo sviluppo di questo capitolato siano direttamente spendibili nel mondo del lavoro, una buona progettazione richiederebbe uno studio approfondito di tutti i suoi campi applicativi, il quale risulterebbe troppo oneroso rispetto al tempo a disposizione.