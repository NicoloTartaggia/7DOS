\section{Valutazione sul capitolato scelto}
	\subsection{Capitolato C3 - G\&B: monitoraggio intelligente di processi DevOps}
	
		\subsubsection{Descrizione}
		Il capitolato C3 ci introduce al mondo dei “DevOps” in cui gli operatori che erogano i servizi e coloro che fabbricano il software sono a stretto contatto. Per far si che la collaborazione sia efficace, è necessario un intenso e completo monitoraggio dei sistemi. Da qui nasce la necessità di rendere “intelligente” il controllo sulle risorse, così da massimizzare l’efficienza e la qualità degli interventi al sistema.

	\subsubsection{Studio del dominio}
		\paragraph{Dominio applicativo} \Spazio
		Il proposito del progetto è la realizzazione di un plug-in per il sistema di monitoraggio \gl{Grafana}. Si vuole che esso possa, tramite l’utilizzo dell’intelligenza artificiale, monitorare la \gl{livellines} del sistema e riesca a consigliare degli interventi strategici da effettuare sullo stesso (o almeno a delineare la zona di intervento). Tutto ciò tramite l’utilizzo delle \gl{reti Bayesiane}, dato che è un ottimo strumento per coniugare le conoscenze e competenze degli esperti, con la flessibilità dei sistemi basati sulla probabilità.
		\paragraph{Dominio tecnologico}
		\begin{itemize}
		\item\textbf{{Grafana}}: sistema di monitoraggio e analisi opensource;
		\item\textbf{{Reti Bayesiane}} 
		\item\textbf{{\gl{Javascript}}}
		\item\textbf{{\gl{JSON}}}: formato adatto per la lettura dei nodi della rete Bayesiana.
		\end{itemize}
		
		\paragraph{Aspetti positivi}
		\begin{itemize}
			\item {Il capitolato in oggetto ha subito suscitato grande interesse nel nostro team per le tematiche moderne e interessanti, come le reti Bayesiane e l’intelligenza artificiale, soprattutto per la possibilità di dare forma nella pratica a molti concetti teorici visti in alcune occasioni.
				Per questo motivo, e anche per l’importanza dell’azienda proponente e del suo noto know-how, abbiamo deciso di scegliere questo capitolato.}
			\item{}
		\end{itemize}
		\paragraph{Aspetti negativi}	
		\begin{itemize}
			\item {Dallo studio di fattibilità del capitolato è emerso il fatto che lo sviluppo di un plug-in per un applicativo già esistente e consolidato come Grafana, il quale non è mai stato utilizzato da alcun componente del gruppo, impone certi vincoli tecnologici e può limitare di conseguenza le scelte del team.}
			\item{nessun componente del gruppo ha mai lavorato con Grafana}
		\end{itemize} 
		\paragraph{Conclusioni} \Spazio
				
		 




\section{Valutazione sugli altri capitolati}
	\subsection{Capitolato C1 - Butterfly}
		\subsubsection{Descrizione}
	    Il primo capitolato, denominato \emph{Butterfly: monitor per processi CI/CD} propone la creazione di un sistema publisher subscriber per facilitare e 		automatizzare l'invio e la ricezione delle segnalazioni provenienti dai vari strumenti utilizzati durante i processi di continuous integration e continuous delivery.
		\subsubsection{Studio del dominio}
			\paragraph{Dominio applicativo} \Spazio
			 Il sistema che si richiede venga sviluppato in particolare dovra' essere composto da 4 elementi:
\\-producers: hanno il compito di recuperare le segnalazioni e ripubblicarle sotto forma di messaggi all'interno dei
topic adeguati;
\\-consumers: hanno il compito di abbonarsi a dei topic e inoltrare ai destinatari finali tutte le segnalazioni
provenienti dal suddetto topic;
\\-broker: strumento utile per la gestione ed istanziazione dei topic;
\\-componenti aggiuntive custom: si richiede, in particolare, la realizzazione di una componente che riesca a determinare la persona più adatta a cui inoltrare la segnalazione e la invii solo ed esclusivamente a lei.
			\paragraph{Dominio tecnologico}
			\begin{itemize}
				\item \textbf{\gl{Java8+}}; 
				\item  \textbf{\gl{Apache Kafka}}: piattaforma a bassa latenza per la gestione dei \gl{feed dati} in tempo reale;
				\item  \textbf{\gl{Docker}}: progetto open source per l'automatizzazione del deployment (consegna al cliente);
				\item  \textbf{\gl{SonarQube}}: piattaforma per l'analisi statica del codice;
				\item  \textbf{\gl{Redmine}}: \gl{Issue Tracking System} e \gl{Project Management Tool};
				\item  \textbf{\gl{GitLab}}: software di versionamento.
			\end{itemize}
			
			\paragraph{Aspetti positivi}
				\begin{itemize}
				\item {Linguaggio di programmazione (Java8+) affrontato durante l'anno accademico nel corso di Pogrammazione Concorrente e Distribuita. Di conseguenza i componenti del team hanno già familiarità con esso;}
				\item{Ampio dominio tecnlogico che permette di allargare le proprie conoscenze, utile anche per un utilizzo futuro in ambito lavorativo;}
			\end{itemize}
			\paragraph{Aspetti negativi}
			\begin{itemize}
				\item {Alcune tecnologie non sono conosciute dal gruppo, come SonarQube, Apache Kafka. Ciò comporta uno studio preventivo prima dell'analisi dei requisiti;}
			\end{itemize} 
			\paragraph{Conclusioni} \Spazio
			L'idea di sviluppare un'applicazione che possa connettere tra loro molteplici strumenti per CI/CD largamente utilizzati in ambito di sviluppo e' stata accolta positivamente dal team. Tuttavia il capitolato non è stato scelto visto il numero di tecnologie sconosciute e l'interesse per altre tecnologie da parte del gruppo.
			
	\subsection{Capitolato C2 - Colletta: piattaforma raccolta dati di analisi di testo}
	\subsubsection{Descrizione}
	Il progetto dal titolo \emph{BlockCV: blockchain per gestione di CV certificati} prevede la creazione di una piattaforma distribuita per la pubblicazione di Curriculum Vitae con la possibilità di ricercare offerte di lavoro. Il sistema dev'essere basato sulla tecnologia Blockchain.
	Il software deve essere integrabile nell'attuale sistema lavorativo, quindi dalla pubblicazione del CV, all'arricchimento dello stesso grazie a maggiori esperienze acquisite dall'utente e deve dare la possibilità alle varie realtà che hanno possibilità di assunzione di confermare le competenze degli utenti.    
	\subsubsection{Studio del dominio}
	\paragraph{Dominio applicativo} \Spazio
	Il software è utilizzabile nel mercato online e molti utenti potrebbero usufruirne sia per cercare lavoro, sia per avere un CV online sempre accessibile e aggiornato con il tempo con la conferma delle competenze da entità verificate.
	\paragraph{Dominio tecnologico} \Spazio
	\begin{itemize}
		\item \textbf{Blockchain};
		\item \textbf{Hyperledger Fabric};
		\item  \textbf{Java EE};
		\item  \textbf{Play} framework per l'interfaccia grafica;
		\item  \textbf{\gl{MongoDB}} per il database;
		\item  \textbf{\gl{HTML} e \gl{CSS}} per realizzare l'interfaccia utente.
	\end{itemize}
	
	\paragraph{Aspetti positivi} \Spazio
	\begin{itemize}
		\item {Il blockchain è una tecnologia innovativa e molto attuale nell'ultimo periodo anche grazie al collegamento con la \gl{criptovaluta} Bitcoin;}
		\item {Esperienza da parte di diversi membri del team nello sviluppare interfacce web.}
	\end{itemize} 
	\paragraph{Aspetti negativi} \Spazio
	Per alcuni dei componenti del gruppo le tecnologie da utilizzare e alcuni framework imposti come vincolo obbligatorio dall'azienda sono sconosciuti.
	\paragraph{Conclusioni} \Spazio
	Per via degli aspetti precedentemente elencati il gruppo ha preferito concentrarsi su un capitolato diverso.
	
	
	\subsection{Capitolato C4 - MegAlexa: arricchitore di skill di Amazon Alexa}
	\subsubsection{Descrizione}	
		Il progetto \emph{DeSpeect: interfaccia grafica per Speect} prevede la realizzazione di un'interfaccia grafica di una libreria Open Source per lo sviluppo di \gl{frontend} e \gl{backend} di un sistema in sintesi vocale.
		L'esigenza nasce dalla necessità di monitorare il comportamento dei plugin della libreria in questione.
	\subsubsection{Studio del dominio}
	
	\paragraph{Dominio applicativo} \Spazio
     Il capitolato è posizionato nell'ambito della sintesi vocale, tecnologia molto presente nel mercato ordierno. 
	\paragraph{Dominio tecnologico}
	\begin{itemize}
		\item \textbf{Linux} richiesta la compatibilità del software con questo sistema operativo;
		\item  \textbf{Linguaggio C} necessario per integrarsi con la libreria sulla quale si basa il progetto;
		\item  \textbf{Speect} tecnologia madre del progetto.
	\end{itemize}
	\paragraph{Aspetti positivi} \Spazio
	Interessante la sintesi vocale in quanto è una tecnologia innovativa e ancora in fase di sviluppo in molti settori.
	\paragraph{Aspetti negativi}
	\begin{itemize}
		\item L'azienda richiede che venga realizzata un'interfaccia grafica, così facendo si ha poco modo di prendere dimestichezza con la sintesi vocale;
		\item Poca chiarezza nella descrizione del capitolato.
	\end{itemize}
	\paragraph{Conclusioni} \Spazio
	Per i fattori appena elencati, il gruppo si è orientato in un capitolato diverso.
	
	\subsection{Capitolato C5 - P2PCS: piattaforma di peer-to-peer car sharing}
		\subsubsection{Descrizione}
		Il capitolato C5 \emph{IronWorks, utilità per la costruzione di software robusto} è proposto dall'azienda Zucchetti S.r.l. e pone come obbiettivo la generazione automatica del codice da diagrammi \gl{UML} per rendere più facile seguire le buone regole di programmazione. In particolare chiede la realizzazione di un'\gl{editor} per la costruzione di diagrammi UML (l'interesse è rivolto verso i soli diagrammi di robustezza) con la relativa generazione di codice Java per le entità persistenti e per i metodi di scrittura e lettura verso un database relazionale. 
		\subsubsection{Studio del dominio}
			\paragraph{Dominio applicativo} \Spazio
			L'applicazione deve essere di aiuto per la progettazione di un buon software: deve permettere la realizzazione di diagrammi UML da cui deve essere possibile la generazione di codice. Potranno essere disegnati diagrammi di robustezza seguendo le regole con cui i tre tipi di oggetti rappresentabili dai diagrammi di robustezza (le interfacce, le procedure e le entità persistenti) possono reagire tra di loro.
			\paragraph{Dominio tecnologico} \Spazio
			E' richiesta la conoscenza delle seguenti tecnologie per la realizzazione dell'applicazione web:
				\begin{itemize}
					\item Per la parte server:
					\begin{itemize}
						\item Java;
						\item \gl{TomCat};
						\item JavaScript;
						\item Node.js.
					\end{itemize}
					\item Per la parte client:
					\begin{itemize}
						\item HTML5;
						\item CSS.
					\end{itemize}
					\item Per l'archiviazione dei dati su file di testo o su database:
					\begin{itemize}
						\item \gl{XML};
						\item JSON;
						\item SQL.
					\end{itemize}
				\end{itemize}
		\subsubsection{Aspetti positivi}
		\begin{itemize}
			\item Il team presenta già conoscenze sulle tecnologie da utilizzare per il lato client dell'applicazione web;
			\item Il \gl{proponente} fornisce dei software di riferimento, alcuni open source.
		\end{itemize}
		\subsubsection{Aspetti negativi}
		\begin{itemize}
			\item I diagrammi di robustezza sono poco conosciuti e poco utilizzati;
			\item Nel mercato sono già presenti molti software con le caratteristiche richieste.
		\end{itemize}
		\subsubsection{Conclusioni}
		Questo capitolato, sebbene non presentasse eccessive criticità, non ha suscitato particolare interesse rispetto ad altri. La limitazione alla rappresentazione dei soli diagrammi di robustezza è stata ritenuta troppo vincolante per la creazione di un prodotto completo e veramente utilizzabile nello sviluppo software.
		
		
	\subsection{Capitolato C6 - Soldino: piattaforma Ethereum per pagamenti IVA}
		\subsubsection{Descrizione}
		Il capitolato \emph{Marvin, dimostratore di Uniweb su Ethereum}, proposto dall'azienda RedBabel, propone di realizzare una versione di Uniweb come una \gl{Dapp} che giri su \gl{Ethereum Virtual Machine}. \\  Infatti, come su Uniweb, devono poter interagire:
		\begin{itemize} 
			\item studenti, per aver accesso alla propria carriera universitaria, registrarsi ad esami, accettare e rifiutare voti; 
			\item professori, per pubblicare liste di esami, pubblicare voti;
			\item Università, per gestire corsi, orario, spazi, e altro.
		\end{itemize}
	 	Le interazioni tra questi tre attori vengono tradotte con una serie di smart contracs.
		\subsubsection{Studio del dominio}
			\paragraph{Dominio applicativo} \Spazio
			Questo capitolato si pone l'obbiettivo unire le tecnologie ad oggi in forte ascesa con il mondo universitario. Come per Uniweb saranno studenti, professori ed Università ad utilizzare il prodotto risultante per ottimizzare studio e lavoro.
			\paragraph{Dominio tecnologico} \Spazio
			Per comprendere a fondo il dominio e per realizzare il progetto è richiesta la conoscenza delle seguenti tecnologie:
			\begin{itemize}
				\item Ethereum;
				\item \gl{Truffle};
				\item \gl{Etherscan.io};
				\item Javascript;
				\item \gl{ESLint};
				\item \gl{React};
				\item \gl{SCSS}.
			\end{itemize}
		\subsubsection{Aspetti positivi}
		\begin{itemize}
			\item Particolare interesse da parte di vari membri del team verso il dominio del capitolato;
			\item Le tecnologie utilizzate sono sempre più richieste nel mondo del lavoro e una loro conoscenza approfondita gioverebbe ad ognuno dei componenti del gruppo.
		\end{itemize}
		\subsubsection{Aspetti negativi}
		\begin{itemize}
			\item Il dominio applicativo del software è molto vasto e la conoscenze di esso da parte dei membri del team non sono sufficienti;
			\item Può risultare scomodo contattare il proponente dato che ha sede ad Amsterdam.
		\end{itemize}
		\subsubsection{Conclusioni}
		Sebbene le conoscenze acquisite dallo sviluppo di questo capitolato siano direttamente spendibili nel mondo del lavoro, una buona progettazione richiederebbe uno studio approfondito di tutti i suoi campi applicativi, il quale risulterebbe troppo oneroso rispetto al tempo a disposizione.