\section{Introduzione}
	\subsection{Scopo del documento}
	Il presente documento nasce con l'intento di descrivere, analizzare e valutare le caratteristiche, le criticità e le potenzialità di tutti i progetti proposti. In particolare si descrivono le motivazioni e le considerazioni che hanno portato il gruppo alla scelta del \gl{capitolato} d'appalto C3 \emph{"G\&B"} e allo scarto degli altri.
	\subsection{Glossario}
	Volendo evitare incomprensioni  ed equivoci per rendere la lettura del documento più semplice e chiara viene allegato il \emph{Glossario v1.0.0} nel quale sono contenute le definizioni dei termini tecnici, dei vocaboli ambigui, degli acronimi e delle abbreviazioni. Questi termini sono evidenziati nel presente documento con una g al pedice (esempio: $Glossario_{g}$).  
	\subsection{Riferimenti}
		\subsubsection{Normativi}
		\begin{itemize}
			\item \textbf{Norme di Progetto:} \emph{Norme di Progetto v1.0.0.}
		\end{itemize}
		
		\subsubsection{Informativi}
		\begin{itemize}
			\item \textbf{Capitolato scelto C3:} G\&B: monitoraggio intelligente di processi DevOps \\ 
			\url{https://www.math.unipd.it/~tullio/IS-1/2018/Progetto/C3.pdf};
 			\item \textbf{Capitolato  C1:} Butterfly: monitor per processi CI/CD \\ \url{https://www.math.unipd.it/~tullio/IS-1/2018/Progetto/C1.pdf};
 			\item \textbf{Capitolato C2:} Colletta: piattaforma raccolta dati di analisi di testo \\ 
 			\url{https://www.math.unipd.it/~tullio/IS-1/2018/Progetto/C2.pdf};
 			\item \textbf{Capitolato C4:} MegAlexa: arricchitore di skill di \gl{Amazon Alexa} \\ \url{https://www.math.unipd.it/~tullio/IS-1/2018/Progetto/C4.pdf};
 			\item \textbf{Capitolato C5:} P2PCS: piattaforma di peer-to-peer car sharing \\ \url{https://www.math.unipd.it/~tullio/IS-1/2018/Progetto/C5.pdf};
 			\item \textbf{Capitolato C6:} Soldino: piattaforma \gl{Ethereum} per pagamenti IVA \\ \url{https://www.math.unipd.it/~tullio/IS-1/2018/Progetto/C6.pdf};
 			
		\end{itemize}
		


