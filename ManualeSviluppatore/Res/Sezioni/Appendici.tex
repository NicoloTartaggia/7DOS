\newsection{Appendici}
\subsection{Glossario}
\subsubsection{D}

\paragraph{Datasource} \Spazio
\label{sec:datasource} 
Una datasource è una sorgente di dati in Grafana. Solitamente si tratta di un database. Grafana supporta nativamente:
	\begin{itemize}
	\item CloudWatch;
	\item Elasticsearch;
	\item Graphite;
	\item InfluxDB;
	\item Microsoft SQL;
	\item MySql;
	\item OpenTSDB;
	\item PostgreSQL;
	\item Prometheus;
	\item Stackdriver.
	\end{itemize}


\subsubsection{F}

	\paragraph{Framework} \Spazio
	\label{sec:ide}
Un framework è un architettura di supporto che ha lo scopo di ospitare e facilitare lo sviluppo di software.


\subsubsection{I}

	\paragraph{IDE} \Spazio
	\label{sec:ide}
IDE: Integradeted Development Environment è un applicazione software che fornisce tutta una serie di funzionalità, tool e features utili allo sviluppo software.

	\paragraph{InfluxDB} \Spazio
	\label{sec:influxdb}
InfluxDB è un database basato sul concetto di serie temporale. InfluxDB è specializzato e ottimizzato per il salvataggio e la lettura di serie temporali: record salvati in ordine temporale e caratterizzati dal timestamp, ovvero un campo che indica una data. InfluxDB è utilizzato per lo più in ambiti in cui è necessario salvare valori generati da sensori oppure analytics in tempo reale.
	
\subsubsection{J}

	\paragraph{JSbayes} \Spazio
	\label{sec:influxdb}
JSbayes è una libreria javascript che implementa un tool per l'inferenza di reti bayesiane. 


\subsubsection{N}

	\paragraph{Node} \Spazio
	\label{sec:node}
Node.js è un ambiente run-time che permette di eseguire codice JavaScript al di fuori del browser.

	\paragraph{Npm} \Spazio
	\label{sec:npm}
Npm: Node.js package manager è il package manager di default di Node.js. Consiste in un applicativo client a riga di comando e un database online pubblico chiamato npm registry. Npm registry può essere navigato tramite il client a riga di comando per cercare ed installare packages.


\subsubsection{P}

	\paragraph{Package.json} \Spazio
	\label{sec:package}
Il package.json è un file, in formato json, utilizzato principalmente in progetti Javascript o Node.js che permettere di configurare il package o l'applicazione. Permette inoltre a npm o yarn di salvare nomi e versioni dei package installati.

	\paragraph{Panel} \Spazio
	\label{sec:panel}
Un \emph{panel} è una rappresentazione visiva di dati elaborati da un sistema mostrati sotto-forma di grafici.


\subsubsection{R}

	\paragraph{Root directory} \Spazio
	\label{sec:rootdirectory}
La root directory di una qualsiasi partizione, è la cartella più "alta" nella gerarchia delle cartelle.




