\appendix
\addcontentsline{toc}{section}{Appendice}
%\section*{Appendice}
\newsection{Glossario}
\subsection*{C}
\subsubsection*{Campionamento a pesatura di verosimiglianza}
\label{sec:campionamento}
Tecnica di inferenza delle probabilità dei nodi di una rete Bayesiana in cui i risultati di ogni campionamento vengono valutati in base alla verosimiglianza degli esiti: se un esito è difficilmente verificato, conta di meno nel calcolo della media delle probabilità.

\subsection*{D}

\subsubsection*{Datasource}
\label{sec:datasource} 
Una datasource è una sorgente di dati in Grafana. Solitamente si tratta di un database. Grafana supporta nativamente:
	\begin{itemize}
	\item CloudWatch;
	\item Elasticsearch;
	\item Graphite;
	\item InfluxDB;
	\item Microsoft SQL;
	\item MySql;
	\item OpenTSDB;
	\item PostgreSQL;
	\item Prometheus;
	\item Stackdriver.
	\end{itemize}

\subsubsection*{Dual core}
\label{sec:dualcore}
CPU composta da due core, ovvero da due nuclei di processori. Questa struttura permette di dimezzare il carico di lavoro di ogni processore e di svolgere operazioni in parallelo.

\subsection*{F}

	\subsubsection*{Framework} 
	\label{sec:ide}
Un framework è un architettura di supporto che ha lo scopo di ospitare e facilitare lo sviluppo di software.


\subsection*{I}

	\subsubsection*{IDE} 
	\label{sec:ide}
IDE: Integradeted Development Environment è un applicazione software che fornisce tutta una serie di funzionalità, tool e features utili allo sviluppo software.

	\subsubsection*{InfluxDB} 
	\label{sec:influxdb}
InfluxDB è un database basato sul concetto di serie temporale. InfluxDB è specializzato e ottimizzato per il salvataggio e la lettura di serie temporali: record salvati in ordine temporale e caratterizzati dal timestamp, ovvero un campo che indica una data. InfluxDB è utilizzato per lo più in ambiti in cui è necessario salvare valori generati da sensori oppure analytics in tempo reale.
	
\subsection*{J}

	\subsubsection*{JSbayes} 
	\label{sec:influxdb}
JSbayes è una libreria javascript che implementa un tool per l'inferenza di reti bayesiane. 


\subsection*{N}

	\subsubsection*{Node} 
	\label{sec:node}
Node.js è un ambiente run-time che permette di eseguire codice JavaScript al di fuori del browser.

	\subsubsection*{Npm} 
	\label{sec:npm}
Npm: Node.js package manager è il package manager di default di Node.js. Consiste in un applicativo client a riga di comando e un database online pubblico chiamato npm registry. Npm registry può essere navigato tramite il client a riga di comando per cercare ed installare packages.


\subsection*{P}

	\subsubsection*{Package.json} 
	\label{sec:package}
Il package.json è un file, in formato json, utilizzato principalmente in progetti Javascript o Node.js che permettere di configurare il package o l'applicazione. Permette inoltre a npm o yarn di salvare nomi e versioni dei package installati.

	\subsubsection*{Panel} 
	\label{sec:panel}
Un \emph{panel} è una rappresentazione visiva di dati elaborati da un sistema mostrati sotto-forma di grafici.


\subsection*{R}

	\subsubsection*{Root directory}
	\label{sec:rootdirectory}
La root directory di una qualsiasi partizione, è la cartella più "alta" nella gerarchia delle cartelle.




