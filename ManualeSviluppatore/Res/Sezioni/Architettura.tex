\pagebreak
\newsection{Architettura}
All'interno della seguente sezione verrà descritta l'architettura dell'intero sistema.\\
Ogni componente logico del plug-in viene eseguito all'interno dello stesso, e quindi all'interno del browser, senza appoggiarsi a server esterni per l'esecuzione. Per questo motivo ogni panel è un componente a sé stante e indipendente. \\
Per la realizzazione dell'architettura del plug-in il team ha deciso di adottare uno stile architetturale a layer.\\
I principali layer che sono stati individuati nel corso dello sviluppo dell'applicazione sono i seguenti:
\begin{itemize}
	\item{\textbf{Presentation layer}: si occupa di gestire la componente grafica del panel;}
	\item{\textbf{Persistence layer}: si occupa di mantenere la struttura della rete;}
	\item{\textbf{Business layer}: incorpora tutta la logica relativa all’intero processo di ricalcolo delle probabilità;}
	\item{\textbf{Database layer}: composto dall’insieme di tutti i database esterni forniti dall’utente per il reperimento delle informazioni.}
\end{itemize}
\begin{figure} [H]
	\centerline{
	\includegraphics[scale=0.3]{Img/Diagramma_Package}}
	\caption{Diagramma dei package}\label{}
\end{figure}
\begin{figure} [H]
	\centering
	\includegraphics[scale=0.07]{Img/Diagramma_Classi}
	\caption{Diagramma delle classi complessivo}\label{}
\end{figure}
Analizziamo ora più in dettaglio ciascuno dei layer precedentemente descritti.
\subsection{Persistence Layer}
L'intera nostra applicazione è basata sulla libreria Javascript JSBayes consigliata dal proponente, la quale permette di mantenere l'intera struttura della rete ed esegue su di essa il calcolo delle probabilità di ogni stato associato ad ogni nodo della rete.
Questa libreria tuttavia presenta non pochi problemi, tra i quali una pressochè totale assenza della gestione degli errori, una difficile interazione con altri oggetti, ed una scarsa estendibilità.
Il persistence layer è composto principalmente da un insieme di classi che permettono di ovviare ai problemi precedentemente descritti.
\begin{figure} [H]
	\centerline{
	\includegraphics[scale=0.25]{Img/PersistenceLayer2}}
	\caption{Diagramma UML persistence layer}\label{}
\end{figure}
\subsubsection{NetworkAdapter}
La classe NetworkAdapter ha il compito di schermare tutte le operazioni non sicure di JSBayes, cioè quelle che vanno a modificare la struttura della rete rischiando quindi di minare la potenziale integrità dei dati.
Essa espone solamente i metodi necessari per effettuare il ricalcolo delle probabilità e che quindi non modificano in alcun modo lo stato.
\begin{figure} [H]
	\centering
	\includegraphics[scale=0.7]{Img/NetworkAdapter2}
	\caption{Diagramma UML NetworkAdapter}\label{}
\end{figure}
\subsubsection{NodeAdapter}
Per lo sviluppo delle funzionalità del plug-in il team ha avuto la necessità di aggiungere informazioni ai singoli nodi della rete bayesiana memorizzata da JSBayes.
Estendere dalle classi offerte dalla libreria era impensabile a causa dei molteplici problemi descritti all'inizio di questa sezione. Per questo motivo abbiamo deciso di realizzare un adapter anche per i singoli nodi di JSBayes ed includere all'interno di questi tutte le informazioni aggiuntive necessarie, come gli AbstractValue che verranno descritti in dettaglio all'interno della sezione seguente.
\begin{figure} [H]
	\centering
	\includegraphics[scale=0.7]{Img/NodeAdapter}
	\caption{Diagramma UML NodeAdapter}\label{}
\end{figure}
\subsubsection{AbstractValue}
Per poter determinare in quale stato il nodo si trova sulla base dei dati prelevati da una qualsiasi fonte dati, come un database, il semplice nome che JSBayes associa ad ogni stato non era sufficiente. Il compito degli AbstractValue è proprio quello di colmare questa lacuna.\\
Invocando su un oggetto AbstractValue il metodo
\begin{ttfamily}
	isValueType(value:string):bool
\end{ttfamily}
e passandogli come parametro il risultato di una lettura da una qualsiasi fonte dati l'AbstractValue ritorna un boolean indicando all'invocatore se lo stato associato è attualmente vero o meno.\\
Il team ha realizzato solamente tre implementazioni concrete di abstract value, rispettivamente RangeValue, StringValue e BoolValue, tuttavia è possibile realizzare altre implementazioni semplicemente estendendo dall'interfaccia AbstractValue.
\begin{figure} [H]
	\centering
	\includegraphics[scale=0.55]{Img/AbstractValue}
	\caption{Diagramma UML AbstractValue}\label{}
\end{figure}
\subsubsection{NetBuilder}
Per garantire l'integrità dei dati abbiamo deciso di esternare il processo di creazione della rete JSBayes all'interno di un builder esterno il quale ha il compito di effettuare tutti i controlli necessari e ritornare un riferimento ad un NetworkAdapter solo al termine dell'intero processo di creazione invocando il metodo
\begin{ttfamily}
	build():NetworkAdapter
\end{ttfamily}.
\begin{figure} [H]
	\centering
	\includegraphics[scale=0.7]{Img/NetBuilder}
	\caption{Diagramma UML NetBuilder}\label{}
\end{figure}
\subsubsection{NetParser}
Il NetParser ha il compito di realizzare un NetAdapter a partire dal contenuto di un file di testo, sfruttando le funzionalità esposte dal NetBuilder.
Il team fino ad ora ha realizzato un'unica implementazione, quella relativa a file di tipo JSON in quanto era l'unico formato richiesto esplicitamente dal proponente.
Nel caso in cui un giorno fosse necessario utilizzare formati differenti sarà sufficiente realizzare una differente implementazione dell'interfaccia NetParser.
\begin{figure} [H]
	\centering
	\includegraphics[scale=1]{Img/NetParser}
	\caption{Diagramma UML NetParser}\label{}
\end{figure}
\subsection{Business Layer}
Il business layer ingloba all'interno di esso tutta la logica relativa al processo di calcolo delle probabilità.
Dopo un attenta analisi del problema il team è riuscito a scomporre il processo di calcolo delle probabilità in 3 macro-fasi principali:
\begin{itemize}
	\item La lettura dei dati dai database associati ai nodi delle rete;
	\item Il processo di calcolo delle probabilità vero e proprio partendo dai dati letti precedentemente;
	\item La scrittura delle probabilità appena calcolate su un database che l'utente poi potrà utilizzare per fare la visualizzazione dei dati su Grafana.
\end{itemize}
Ci è sembrato dunque abbastanza naturale modellare l'architettura del business layer in tre oggetti distinti ognuno dei quali si occupa di gestire una specifica fase del processo dall'inizio alla fine.
\begin{figure} [H]
	\centering
	\includegraphics[scale=0.1]{Img/BuisnessLayer}
	\caption{Diagramma UML Business Layer}\label{}
\end{figure}
Analizziamo ora più in dettaglio gli oggetti precedentemente citati.
\subsubsection{NetReader}
Il NetReader ha il compito di associare i nodi della rete a delle sorgenti dati scelte dall'utente, come per esempio dei database. Quest'ultime vengono rappresentate all'interno della nostra applicazione da oggetti di tipo Flow.
\begin{figure} [H]
	\centering
		\includegraphics[scale=0.8]{Img/NetReader}
	\caption{Diagramma UML NetReader}\label{}
\end{figure}
\paragraph{Flow}\Spazio
Un oggetto Flow ha il compito ritornare un risultato prelevato da una qualsiasi sorgente dati.
Il team attualmente ha realizzato un'unica implementazione di un Flow il ClientFlow il quale utilizza dei ReadClient, di cui parleremo più approfonditamente all'interno della prossima sezione, per interfacciarsi con dei database.
\paragraph{ClientPool}\Spazio
Per non dover istanziare più client di lettura per la stessa sorgente dati o database, abbiamo implementato una Object Pool per i client di lettura: quando si connette un nodo ad un flusso di dati, se il client richiesto esiste viene estratto dalla pool, altrimenti viene creato.\\
Di seguito è possibile vedere un diagramma di sequenza il quale riassume il processo di associazione di un flusso dati ad un nodo nel caso in cui il client non sia già presente all'interno della pool.
\begin{figure} [H]
	\centerline{
	\includegraphics[scale=0.23]{Img/ConnessioneNodi}}
	\caption{Diagramma di sequenza connessione nodi}\label{}
\end{figure}
\subsubsection{NetUpdater}
NetUpdater ha il compito di effettuare il calcolo delle probabilità vero e proprio utilizzando i dati letti da NetReader e le funzionalità esposte dal NetworkAdapter.
\begin{figure} [H]
	\centering
	\includegraphics[scale=0.8]{Img/NetUpdater}
	\caption{Diagramma UML NetUpdater}\label{}
\end{figure}
\subsubsection{NetWriter}
Il NetWriter ha il compito di scrivere all'interno di un apposito database scelto dall'utente tutti i dati calcolati dal NetUpdater. Per fare ciò esso si appoggia ad uno o più WriteClient classe che analizzeremo più approfonditamente all'interno del database layer.
Il team, ai fini del progetto, attualmente ha realizzato un'unica implementazione del NetWriter, il SingleNetWriter il quale scrive tutti i dati ricevuti in input all'interno di un unico database. 
È importante sottolineare tuttavia come, grazie all'interfaccia NetWriter, sia relativamente semplice integrare all'interno del plug-in una versione differente del NetWriter il quale per esempio permette la scrittura dei dati ricevuti in input all'interno di molteplici database differenti.
\begin{figure} [H]
	\centering
	\includegraphics[scale=1]{Img/NetWriter}
	\caption{Diagramma UML NetWriter}\label{}
\end{figure}
\subsubsection{NetManager}
La classe NetManager ha un riferimento ad un'istanza di NetReader, NetUpdater e NetWriter. Esso ha il compito di coordinare l'intero processo di calcolo delle probabilità invocando sequenzialmente le operazioni giuste nell'ordine corretto.
\begin{figure} [H]
	\centering
	\includegraphics[scale=0.8]{Img/NetManager}
	\caption{Diagramma UML NetManager}\label{}
\end{figure}
\subsubsection{Results}
NetReader, NetUpdater e NetWriter comunicano tra loro fornendo i risultati delle proprie operazioni  all'interno di appositi oggetti “risultato” il cui cambiamento è altamente improbabile. In questo modo è possibile effettuare anche importanti modifiche alla struttura interna di queste macro-componenti senza che le interazioni con le altre ne risentano anche solo minimamente.\\
Inoltre abbiamo deciso di rendere questi oggetti risultato degli iterator per garantire un ordine nel loro scorrimento.\\
In particolare sono state realizzate due tipologie di result:
\begin{itemize}
	\item Gli InputResult ritornati dal NetReader al termine delle sue operazioni ed utilizzati dal NetUpdater per effettuare il calcolo delle probabilità;
	\begin{figure} [H]
		\centering
		\includegraphics[scale=0.65]{Img/InputResult}
		\caption{Diagramma UML InputResult}\label{}
	\end{figure}
	\item I CalcResult ritornati dal NetUpdater al termine delle sue operazioni ed utilizzati dal NetWriter per effettuare la scrittura sul database di output indicato dall'utente.
	\begin{figure} [H]
		\centering
		\includegraphics[scale=0.5]{Img/CalcResult}
		\caption{Diagramma UML CalcResult}\label{}
	\end{figure}
\end{itemize}
Di seguito è possibile vedere un diagramma di sequenza il quale riassume l'intero processo di calcolo delle probabilità descritto all'interno di questa sezione.
\begin{figure} [H]
	\centering
			\includegraphics[scale=0.17]{Img/ProcessoRicalcolo}
	\caption{Diagramma sequenza calcolo probabilità}\label{}
\end{figure}
\subsubsection{TimeBasedNetUpdater}
Il TimeBasedNetUpdater possiede un riferimento al NetManager ed ha il compito di effettuare il processo di calcolo regolarmente secondo regole temporali definite dall'utente.
\begin{figure} [H]
	\centering
		\includegraphics[scale=0.8]{Img/TimeBasedUpdater}
	\caption{Diagramma UML TimeBasedNetUpdater}\label{}
\end{figure}

\subsection{Database Layer}
Il database layer è composto da tutto l'insieme di classi che permettono all'applicazione di interfacciarsi con sorgenti dati esterne come per l'appunto dei database.
In particolare questo layer è suddivisibile in due tipologie di classi, i ReadClient ed i WriteClient i quali offrono rispettivamente funzionalità di scrittura e lettura su database.\\
Il team attualmente ha realizzato solo due implementazioni di ReadClient e WriteClient, quelle necessarie per interfacciarsi con database di tipo influx in quanto espressamente richiesto dal committente.\\
Tuttavia essendo i ReadClient e WriteClient degli strategy pattern è relativamente facile realizzare differenti implementazioni di quest'ultimi che permettono di interfacciarsi con database differenti.\\
Inoltre per semplificare la creazione dei client si è deciso di implementare un abstract factory.
\begin{figure} [H]
	\centering
	\includegraphics[scale=0.5]{Img/ReadClient}	
	\caption{Diagramma UML ReadClient}\label{}
\end{figure}
\begin{figure} [H]
	\centering
	\includegraphics[scale=0.5]{Img/WriteClient}
	\caption{Diagramma UML WriteClient}\label{}
\end{figure}
\subsection{Presentation Layer}
Il presentation layer ha il compito di gestire l'interfaccia grafica del plug-in.\\
Ad ogni tab di edit del plug-in è associata una view ed un controller, in particolare abbiamo:
\begin{itemize}
	\item Una tab per l’associazione dei database da cui prelevare i dati necessari per il processo di calcolo delle probabilità, la quale fa riferimento al NetReader come model;
	\item Una tab che permette di selezionare il database di output in cui si vogliono scrivere le probabilità calcolate, la quale fa riferimento al NetWriter come model;
	\item Una tab in cui l'utente può caricare un file JSON a partire dal quale verrà poi creata l'intera rete, il cui model è il NetParser;
	\item Una tab in cui l'utente può avviare e terminare il processo di ricalcolo continuo e scegliere la frequenza con cui quest'ultimo avverrà. Questa tab fa riferimento al TimeBasedUpdater come model.
\end{itemize}
\begin{figure} [H]
	\centerline{
	\includegraphics[scale=0.22]{Img/PresentationLayer}}
	\caption{Diagramma UML presentation layer}\label{}
\end{figure}
Di seguito è possibile vedere un diagramma di sequenza il quale riassume l'intero processo di creazione delle classi che avviene una volta avviato il plug-in
\begin{figure} [H]
	\centerline{
	\includegraphics[scale=0.25]{Img/Creazione}}
	\caption{Diagramma sequenza creazione classi plug-in}\label{}
\end{figure}
\pagebreak