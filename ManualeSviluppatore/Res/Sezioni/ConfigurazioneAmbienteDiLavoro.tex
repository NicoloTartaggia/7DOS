\newsection{Configurazione dell'ambiente di lavoro}
\subsection{Obbiettivo del capitolo}
Il seguente capitolo si impone l'obbiettivo di spiegare al lettore come configurare il proprio ambiente di lavoro in modo tale che sia lo stesso dei membri del team 7DOS.  
\subsection{Requisiti}

\subsubsection{WebStorm}
Per lo sviluppo del plug-in il team ha scelto di utilizzare l'\gl{IDE} WebStorm, sviluppato da JetBrains. L'IDE è a pagamento, tuttavia esso può essere scaricato gratuitamente dal loro sito ufficiale connettendo al proprio account un'e-mail universitaria.
Il software è disponibile per Microsoft Windows, Linux e MacOs.
\subsubsection{VSCode}
VSCode è una valida alternativa a WebStorm, che il team consiglia di utilizzare nel caso in cui quest'ultimo non fosse reperibile.
Il software può essere scaricato molto facilmente dal sito ufficiale ed è disponibile per Microsoft Windows, Linux e MacOs.
\subsubsection{TSLint e ESLint}
TSLint e ESLint verranno automaticamente installati con l'esecuzione del comando
\begin{verbatim}
	npm install
\end{verbatim}
Una volta installati correttamente WebStorm li rileverà automaticamente tra le dipendenze presenti all'interno del package.json e procederà a segnalare tutti gli errori relativi all'analisi statica rilevati senza la necessità di dover eseguire il comando
\begin{verbatim}
npm run build
\end{verbatim}

