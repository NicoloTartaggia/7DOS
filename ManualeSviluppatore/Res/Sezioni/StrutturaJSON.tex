\pagebreak
\newsection{Struttura del JSON}
I file JSON che contengono la definizione di una rete Bayesiana sono strutturati nel seguente modo:
\begin{itemize}
	\item \textbf{nodes}: contiene l'elenco dei nodi della rete;
	\begin{itemize}
		\item \textbf{name}: nome del nodo;
		\item \textbf{values}: contiene l'elenco delle values:
		\begin{itemize}
			\item name: nome della value;
			\item type: tipo della value;
			\item value/rangeMin/rangeMax: valore della value.
		\end{itemize}
		\item \textbf{parents}: contiene l'elenco dei nodi padri;
		\item \textbf{cpt}: contiene l'elenco delle probabilità associate.
	\end{itemize}
\end{itemize}
\Spazio
Ecco un esempio di rete con due nodi in formato JSON:
\begin{lstlisting}
{
	"nodes":[
		{
		"name":"Node1",
		"values":[
			{
				"name":"Example of string value",
				"type":"string",
				"value":"Value1"
			},
			{
				"name":"Another example of string value",
				"type":"string",
				"value":"Value2"
			}
		],
		"parents":["Node2"],
		"cpt":[[0.2,0.8],[0.5,0.5],[0.4,0.6]]
		},
		{
		"name":"Node2",
		"values":[
			{
				"name":"Low Range",
				"type":"range",
				"rangeMin":0,
				"rangeMax":10
			},
			{
				"name":"Normal Range",
				"type":"range",
				"rangeMin":11,
				"rangeMax":80
			},
			{
				"name":"Alert Range",
				"type":"range",
				"rangeMin":81,
				"rangeMax":100
			}
		],
		"parents":[],
		"cpt":[[0.6,0.2,0.2]]
		}
	]
}
\end{lstlisting}

\pagebreak


