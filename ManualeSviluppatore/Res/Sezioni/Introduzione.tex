\newsection{Introduzione}
\subsection{Scopo del documento}
Questo documento rappresenta il Manuale dello Sviluppatore relativo al \gl{prodotto software} G\&B sviluppato dal gruppo 7DOS. Il suo scopo principale è fornire tutte le informazioni necessarie per usufruire delle funzionalità fino ad ora implementate ed, eventualmente, per estendere e migliorare il prodotto.
\subsection{Scopo del prodotto}
Lo scopo del prodotto è la creazione di un \gl{plug-in} per il software di monitoraggio \gl{Grafana} in grado di collegare dati provenienti da una specifica \gl{datasource} ad un sistema probabilistico definito in una \gl{rete Bayesiana}. In seguito, i risultati ottenuti vengono memorizzati in un database, nel nostro caso \gl{influxDB}, e letti attraverso un \gl{panel} standard di Grafana. In questo modo è possibile evidenziare eventi non visibili ma con alta \gl{likelihood}.
\subsection{Riferimenti}
\subsubsection{Informativi}
\begin{itemize}
	\item{\textbf{Rete Bayesiana}\\
		\url{https://en.wikipedia.org/wiki/Bayesian_network}}
\end{itemize}
\subsubsection{Installazione}
\begin{itemize}
	\item{\textbf{TypeScript}\\
		\url{https://www.typescriptlang.org/index.html#download-links}}
	\item{\textbf{Node.js}\\
		\url{https://nodejs.org/it/}}
	\item{\textbf{Grafana}\\
		\url{http://docs.grafana.org/installation/}}
	\item{\textbf{Grafana Plug-in}\\
		\url{http://docs.grafana.org/plugins/installation/}}
	\item{\textbf{InfluxDB}\\
		\url{https://portal.influxdata.com/downloads/}}
	\item{\textbf{VSCode}\\
		\url{https://code.visualstudio.com/}}
	\item{\textbf{JetBrains WebStorm}\\
		\url{https://www.jetbrains.com/webstorm/}}
	\item{\textbf{JetBrains WebStorm for students}\\
		\url{https://www.jetbrains.com/student/}}
\end{itemize}	
\subsubsection{Legali}
\begin{itemize}
	\item{\textbf{Licenza MIT}\\
		\url{https://opensource.org/licenses/MIT}}
\end{itemize}
\subsection{Glossario}