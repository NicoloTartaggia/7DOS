\newsection{Introduzione}
\subsection{Scopo del documento}
Questo documento rappresenta il Manuale dello Sviluppatore relativo al prodotto software \emph{G\&B} sviluppato dal gruppo 7DOS. Il suo scopo principale è fornire tutte le informazioni necessarie per usufruire delle funzionalità fino ad ora implementate ed, eventualmente, per estendere e migliorare il prodotto.
\subsection{Scopo del prodotto}
\emph{G\&B} è un plug-in per il software di monitoraggio Grafana. Il suo scopo è estenderne le funzionalità permettendo di monitorare il sistema desiderato con una o più reti Bayesiane definita ad-hoc dall'utente, consentendo poi la visualizzazione dei dati calcolati mediante i panel di Grafana.
\\
Il plug-in può essere utilizzato interamente in locale, e non dipende da servizi esterni per il suo funzionamento: una volta installato, è sufficiente impostare le \gl{datasource} necessarie, caricare e configurare la rete Bayesiana scelta e avviare il monitoraggio.

\subsection{Glossario}
In appendice al documento è presente un glossario con tutti i termini necessari per la piena comprensione del documento. Il prodotto è diretto a operatori IT e addetti al monitoraggio di sistemi informatici, pertanto vocaboli valutati come di conoscenza comune o appartenenti a competenze informatiche di base sono stati ignorati.\\
La presenza di un termine all'interno del glossario è segnalata con una "g" posta come pedice (esempio: \gl{Glossario}).

\subsection{Riferimenti}
\subsubsection{Installazione}
\begin{itemize}
	\item{\textbf{Git}\\
		\url{https://git-scm.com/}}
	\item{\textbf{Grafana}\\
		\url{https://grafana.com/docs/installation/}}
	\item{\textbf{Grafana Plug-in}\\
		\url{https://grafana.com/docs/plugins/installation/}}
	\item{\textbf{Node.js}\\
		\url{https://nodejs.org/}}
	\item{\textbf{NPM}\\
		\url{http://www.npmjs.com/}}
	\item{\textbf{InfluxDB}\\
		\url{https://www.influxdata.com/}}
	\item{\textbf{JetBrains - WebStorm}\\
		\url{https://jetbrains.com/webstorm}}
	\item{\textbf{JetBrains - WebStorm for students}\\
		\url{https://jetbrains.com/student}}
	\item{\textbf{VSCode}\\
		\url{https://code.visualstudio.com/}}
\end{itemize}
\subsubsection{Legali}
\begin{itemize}
	\item{\textbf{Licenza MIT}\\
		\url{https://opensource.org/licenses/MIT}}
\end{itemize}
\subsubsection{Informativi}
\begin{itemize}
	\item{\textbf{Reti Bayesiane}\\
			\url{https://en.wikipedia.org/wiki/Bayesian_network}}
\end{itemize}

\subsection{Maturità del documento}
Il presente documento potrebbe essere soggetto ad incrementi futuri. Per questo motivo, non si pone l'obiettivo di risultare completo.
Tutto ciò che riguarda la pianificazione degli incrementi, può essere trovato nel \emph{Piano di Progetto v3.0.0} all'interno della §4.
\pagebreak