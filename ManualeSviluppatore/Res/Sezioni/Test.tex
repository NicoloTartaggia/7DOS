\newsection{Test}
\subsection{Obiettivo del capitolo}
Questo capitolo descrive le procedure per eseguire i test sul funzionamento e la sintassi del codice.
\subsection{Test sul codice TypeScript}
Per avviare i test sul funzionamento del codice TypeScript, è sufficiente eseguire da terminale il comando:
\begin{verbatim}
	npm run test
\end{verbatim}
Il comando avvierà l'esecuzione di tutti i test presenti nella cartella \texttt{/src/test}.
Per avviare i test sulla sintassi del codice è necessario lanciare i comandi:
\begin{itemize}
	\item \begin{verbatim}
	npm run eslint
	\end{verbatim} per verificare la sintassi JavaScript;
	\item \begin{verbatim}
	npm run tslint
	\end{verbatim}
	per verificare la sintassi TypeScript.
\end{itemize}
Alternativamente questi due comandi possono essere eseguiti assieme lanciando:
\begin{verbatim}
npm run eslint && npm run tslint
\end{verbatim}
È importante notare che il comando congiunto non è compatibile con \emph{Windows Powershell}.
Al termine verranno segnalate le parti di codice che non rispettano le linee guida.

\subsection{Test sul codice HTML/CSS}
Verrà fatto per la RA.
\subsection{Code Coverage}
Per avviare il calcolo del code coverage è sufficiente eseguire da terminale il comando:
\begin{verbatim}
	npm run coverage
\end{verbatim}
Il comando esegue prima i test generando un report di coverage (in formato text-lcov) e poi manda direttamente il risultato a coveralls.io.