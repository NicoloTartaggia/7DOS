\newsection{Test}
Questo capitolo descrive le procedure per eseguire i test sul funzionamento e sulla sintassi del codice.
\subsection{Test sul codice}
Per avviare i test sul funzionamento del codice TypeScript, è sufficiente eseguire da terminale il comando:
\\[0.2cm]
\hspace*{10mm}
\begin{ttfamily}
	npm run test
\end{ttfamily}.
\\[0.2cm]
Il comando avvierà l'esecuzione di tutti i test presenti nella cartella \texttt{/src/test}.
Per avviare i test sulla sintassi del codice è necessario lanciare i comandi:
\begin{itemize}
	\item \begin{ttfamily}npm run eslint\end{ttfamily} per verificare la sintassi JavaScript;
	\item \begin{ttfamily}npm run tslint\end{ttfamily}
	per verificare la sintassi TypeScript.
\end{itemize}
Alternativamente questi due comandi possono essere eseguiti assieme lanciando:\\[0.2cm]
\hspace*{10mm}
\begin{ttfamily}
npm run eslint \&\& npm run tslint
\end{ttfamily}.
\\[0.2cm]
È importante notare che il comando congiunto non è compatibile con \emph{Windows Powershell}.
Al termine verranno segnalate le parti di codice che non rispettano le linee guida.

\subsection{Code Coverage}
Per avviare il calcolo del code coverage è sufficiente eseguire da terminale il comando:
\\[0.2cm]
\hspace*{10mm}
\begin{ttfamily}
	npm run coverage
\end{ttfamily}.
\\[0.2cm]
Il comando esegue prima i test generando un report di coverage (in formato text-lcov) e poi manda direttamente il risultato a coveralls.io.
Se la repository non è pubblica è necessario generare e usare un token per l'account usato. La precisa procedura per fare ciò è presente nella documentazione di coveralls.io.