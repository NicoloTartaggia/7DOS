\newsection{Installazione}
\subsection{Installazione di Grafana}
Per poter installare e utilizzare il plug-in è necessario scaricare e configurare precedentemente Grafana, al link URL \url{http://docs.grafana.org/installation/}. \\
Il plug-in è stato sviluppato utilizzando la versione 5.4.3. Per evitare errori nell'esecuzione, si raccomanda di avvalersi della stessa versione.
Una volta terminata la configurazione descritta al riferimento precedente, in Windows avviare l'eseguibile\\ \begin{ttfamily}grafana-5.4.3/bin/grafana-server.exe\end{ttfamily}, mentre in Linux eseguire il comando\\ \begin{ttfamily}sudo service grafana-server start\end{ttfamily}.\\
A questo punto, aprire una finestra del browser e spostarsi all'indirizzo URL \url{localhost:8080}. Sottolineiamo che la porta 8080 è quella suggerita nella pagina di installazione di Grafana.

\subsection{Installazione Node.js}
Per usufruire delle funzionalità del plug-in è necessario installare il \gl{framework} Node.js e il relativo gestore dei pacchetti chiamato Node Package Manager (\gl{NPM}) attraverso l'apposita sezione al link URL \url{https://nodejs.org/it/download/}. In questo modo saranno disponibili i comandi \begin{ttfamily}npm install\end{ttfamily} e \begin{ttfamily}npm run build\end{ttfamily}, necessari rispettivamente per l'installazione di tutte le dipendenze specificate nel \gl{package.json} e per l'esecuzione del processo di \gl{build}.
Verificare, alla fine, la disponibilità le comando \begin{ttfamily}npm\end{ttfamily} tramite comando \begin{ttfamily}npm -v\end{ttfamily}. In caso negativo, scaricarlo dal link \url{https://www.npmjs.com/get-npm}.

\subsection{Installazione plug-in}
Il plug-in completo è disponibile al link \url{https://github.com/NicoloTartaggia/7DOS-plugin}, scaricabile attraverso il bottone "Clone or download" e, in seguito, "Download ZIP", oppure attraverso comando \begin{ttfamily}git clone https://github.com/NicoloTartaggia/7DOS-plugin\end{ttfamily}. Una volta scaricato ed, eventualmente, estratto dall'archivio complesso, spostare il contenuto all'interno del percorso specifico: 
\begin{itemize}
	\item{\begin{ttfamily}grafana-5.4.3/data/plugins\end{ttfamily}} in Windows; \item{\begin{ttfamily}var/lib/grafana/plugins\end{ttfamily} in Linux.}
\end{itemize}
Dopodiché, da terminale spostarsi all'interno della cartella del plug-in e eseguire il comando \begin{ttfamily}npm install\end{ttfamily} per installare tutti i moduli node necessari al funzionamento del plug-in. Terminata l'installazione, procedere con il comando \begin{ttfamily}npm run build\end{ttfamily} per eseguire il processo di build. 
Infine, riavviare il server di grafana. In Windows riaprire l'eseguibile \begin{ttfamily}grafana-server.exe\end{ttfamily}; in Linux eseguire il comando \begin{ttfamily}sudo service grafana-server restart\end{ttfamily}.

\subsection{Installazione e configurazione InfluxDB}
Grafana utilizza diverse \gl{datasource} per prelevare i dati da elaborare e monitorare. In particolare, il prodotto sviluppato si appoggia ad \gl{InfluxDB}.
Di conseguenza, sono necessari due passaggi:
\begin{itemize}
	\item{Installare InfluxDB su Grafana:  dirigersi alla sezione "Configuration" dal side-menu a sinistra e selezionare la voce "Data sources". Successivamente, aggiungere una datasource tramite il bottone "Add data source" e selezionare InfluxDB. Una volta installato, impostare InfluxDB come datasource di default spuntando il relativo checkbox nelle impostazioni;}
	\item{Scaricare InfluxDB al link \url{https://portal.influxdata.com/downloads/}. La versione di riferimento utilizzata durante lo sviluppo è la 1.7. Una volta scaricato, avviare il server. In Windows, eseguire \begin{ttfamily}influxd.exe\end{ttfamily}}, mentre in Linux eseguire il comando \begin{ttfamily}sudo service influxdb start\end{ttfamily}.\\
	L'istanza sarà interrogabile all'indirizzo \url{localhost:8086}.
\end{itemize}

\pagebreak