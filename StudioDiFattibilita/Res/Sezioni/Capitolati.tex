\newsection{Valutazione sul capitolato scelto}
	\subsection{Capitolato C3 - G\&B}
	
	\subsubsection{Descrizione}
	Il capitolato C3 \emph{G\&B: monitoraggio intelligente di processi DevOps}, presentato dall'azienda Zucchetti, richiede la realizzazione di un'estensione per un sistema di monitoraggio dei processi \gl{DevOps} che offra la possibilità di applicare metodi di \gl{intelligenza artificiale} al flusso dei dati raccolti, al fine di guidare eventuali interventi sulla linea di produzione del software.
	
	\subsubsection{Studio del dominio}
	\paragraph{Dominio applicativo} \Spazio 
	L'obiettivo del progetto è la realizzazione di un \gl{plugin} per il software di monitoraggio \gl{Grafana}. Tale plugin deve essere in grado di monitorare la \gl{liveliness} di un sistema, individuarne i punti critici e consigliare interventi strategici (o delineare la zona di intervento) per azioni migliorative, mediante l'impiego di tecniche di intelligenza artificiale. Nello specifico verranno impiegate delle \gl{reti Bayesiane}, i cui risultanti sistemi probabilistici, una volta collegati ai dati raccolti, permetteranno di evidenziare eventi non visibili ma con alta \gl{likelihood}. L'applicativo potrà essere impiegato dall'azienda nel monitoraggio del flusso di dati ricevuti per allarmi o segnalazioni tra operatori di servizi gestionali in \gl{Cloud} e la rispettiva linea di produzione software, ma si dovrà poter prestare anche ad altri utilizzi.
	\paragraph{Dominio tecnologico}
	\begin{itemize}
		\item\textbf{{Grafana}}: software per la visualizzazione dei dati tramite \gl{dashboard} e grafici;
		\item\textbf{{Reti Bayesiane}}: tecnica per l'analisi intelligente dei dati;
		\item\textbf{{\gl{JavaScript}}}: linguaggio per lo sviluppo del plugin;
		\item\textbf{{\gl{JSON}}}: formato dei file per la definizione delle reti Bayesiane.
	\end{itemize}
	
	\subsubsection{Valutazione generale}
	\paragraph{Aspetti positivi}
	\begin{itemize}
		\item{Sviluppo di un prodotto nell'ambito del metodo di sviluppo software DevOps;}
		\item{Applicazione di metodi di intelligenza artificiale a tecnologie moderne in ambito produttivo;}
		\item{Collaborazione con un'azienda di forte rilievo nel territorio italiano;}
		\item{Acquisizione di competenze su un software gratuito e open source;}
		\item{Sviluppo di un prodotto finale open source.}
	\end{itemize}
	\paragraph{Aspetti negativi}	
	\begin{itemize}
		\item{Nessun membro possiede competenze in ambiente Grafana, da cui l'onere di documentarsi su modalità di sviluppo e \gl{best practices} specifiche;} 
		\item{Necessità di approfondire la conoscenza del linguaggio JavaScript e delle sue librerie;}
		\item{Necessità di acquisire le conoscenze matematiche fondamentali per poter implementare le tecniche di intelligenza artificiale richieste.}
	\end{itemize} 
	\subsubsection{Valutazione finale}
	
	Il capitolato è stato ricevuto in modo molto positivo dall'intero gruppo, principalmente per l'utilizzo di metodi di intelligenza artificiale applicati all'analisi di dati provenienti da un sistema, che costituisce un dominio applicativo di forte interesse per il futuro del panorama tecnologico. Sebbene nessuno dei membri del gruppo abbia esperienza con le tecnologie richieste (implicando un onere non indifferente nell'acquisizione delle competenze necessarie), esse presentano un forte potenziale e sono largamente richieste nell'ambito lavorativo; ciò ha portato il gruppo a ritenere questo capitolato come scelta più stimolante e promettente.
	\pagebreak		
	
	\newsection{Valutazione sugli altri capitolati}
	\subsection{Capitolato C1 - Butterfly}
	\subsubsection{Descrizione}
	Il capitolato C1 \emph{Butterfly: monitor per processi \gl{CI/CD}}, presentato dall'azienda Imola Informatica, richiede la creazione di un applicativo che integri i sistemi di segnalazione offerti da vari software utilizzati durante i processi di \gl{Continuous Integration} e \gl{Continuous Delivery}, con l'obiettivo di centralizzare e automatizzare l'invio delle segnalazioni, e indirizzarne la ricezione in modo appropriato. 
	\subsubsection{Studio del dominio}
	\paragraph{Dominio applicativo} \Spazio
	L'obiettivo del progetto è la realizzazione di un applicativo che verrà utilizzato in un contesto di sviluppo software con processi di Continuous Integration e Continuous Delivery, in cui vengono generate frequenti segnalazioni dai vari \emph{tools} utilizzati (e.g. commit di \gl{git}) secondo i  meccanismi di esposizione proprietari. Esso dovrà recuperare o intercettare tali segnalazioni e, sfruttando il pattern \gl{publish-subscribe}, inoltrarle agli utenti di interesse.
	L'applicativo da sviluppare sarà formato da 4 componenti principali:
	\begin{itemize}	 
		\item \textbf{{Producers}}: hanno il compito di recuperare le segnalazioni dai vari strumenti e pubblicarle, sotto forma di messaggi, all'interno dei
		topic adeguati;
		\item \textbf{{Consumers}}: hanno il compito di abbonarsi a dei topic e inoltrare ai destinatari finali tutte le segnalazioni
		appartenenti al suddetto topic;
		\item \textbf{{Broker}}: strumento utile per la gestione ed istanziazione dei topic;
		\item \textbf{{Componenti aggiuntive custom}}: si richiede, in particolare, la realizzazione di una componente che riesca a determinare la persona più adatta a cui inoltrare la segnalazione e la invii solo ed esclusivamente a lei.
	\end{itemize}
	\paragraph{Dominio tecnologico} \Spazio
	L'applicativo dovrà potersi interfacciare con i seguenti strumenti:
	\begin{itemize}
		\item  \textbf{\gl{Redmine}}: \gl{Issue Tracking System} e \gl{Project Management Tool};
		\item  \textbf{\gl{GitLab}}: software di versionamento.
		\item  \textbf{\gl{SonarQube}}: piattaforma per l'analisi statica del codice;
		\item  \textbf{\gl{Telegram}}: sistema di messaggistica;
		\item  \textbf{\gl{Slack}}: software per gestire le comunicazioni in un gruppo di lavoro;
		\item  \textbf{\gl{Email}}: posta elettronica.
		
	\end{itemize}
	
	Le preferenze tecniche per lo sviluppo del progetto prevedono l'uso obbligatorio di determinate tecnologie e l'eventuale utilizzo di tecnologie non obbligatorie, ma fortemente consigliate:
	\subparagraph{Tecnologie obbligatorie}
	\begin{itemize}
		\item  \textbf{\gl{Docker}}: strumento per la \gl{containerizzazione} di applicativi software, da usare per l'istanziazione dei componenti;
		\item  \textbf{\gl{API} \gl{Rest}}: interfacce esposte dai componenti attraverso le quali usare l'applicativo.
	\end{itemize}
	\subparagraph{Tecnologie consigliate}
	\begin{itemize}
		\item \textbf{\gl{Java8+}, \gl{Python} o \gl{Node.js}}: linguaggi per lo sviluppo dei componenti applicativi;  
		\item  \textbf{\gl{Apache Kafka}}: piattaforma a bassa latenza per la gestione dei \gl{feed dati} in tempo reale, da usare come Broker.
	\end{itemize}
	\subsubsection{Valutazione generale}
	\paragraph{Aspetti positivi} 
	\begin{itemize}
		\item {Linguaggio di programmazione (Java8+) affrontato durante l'anno accademico nel corso di Programmazione Concorrente e Distribuita, con cui i componenti del team hanno già familiarità ;}
		\item{Ampio dominio tecnologico che permette di allargare le proprie conoscenze, utile anche per un utilizzo futuro in ambito lavorativo.}
	\end{itemize}
	\paragraph{Aspetti negativi}
	\begin{itemize}
		\item {Alcune tecnologie non sono conosciute dal gruppo, come Redmine, SonarQube e Apache Kafka. Ciò comporta l'onere di uno studio preventivo prima dell'analisi dei requisiti e dell'acquisizione delle competenze necessarie all'integrazione di tali strumenti.}
	\end{itemize} 
	\subsubsection{Valutazione finale}
	L'idea di sviluppare un'applicazione che possa connettere tra loro molteplici strumenti per CI/CD largamente utilizzati in ambito di sviluppo e' stata accolta positivamente dal team. Tuttavia il capitolato non è stato scelto visto l'eccessivo numero di tecnologie sconosciute e l'interesse per altre tecnologie da parte del gruppo.
	
	\subsection{Capitolato C2 - Colletta}
	\subsubsection{Descrizione}
	Il capitolato C2 \emph{Colletta: piattaforma raccolta dati di analisi di testo}, presentato dall'azienda Mivoq, richiede la realizzazione di una piattaforma per la raccolta di dati in cui gli utenti possono avere a disposizione e svolgere piccoli esercizi di grammatica, come l'analisi grammaticale.
	I dati raccolti dovranno essere facilmente accessibili dagli sviluppatori con l'obiettivo di ottimizzare un software per l'analisi grammaticale mediante tecniche di \gl{apprendimento automatico}.   
	
	\subsubsection{Studio del dominio}
	\paragraph{Dominio applicativo} \Spazio
	L'obiettivo del progetto è la realizzazione di una piattaforma per la raccolta dati che implementi la possibilità di eseguire e correggere in modo automatico esercizi di grammatica, salvarne il risultato e collezionare dati con l'obiettivo di migliorare il software di apprendimento automatico.
	Prendendo in considerazione una piattaforma per l'analisi grammaticale è possibile distinguere tre attori principali:
	\begin{itemize} 
	\item \textbf{Insegnanti}: preparano gli esercizi per gli allievi e correggono eventuali errori presenti nelle soluzioni elaborate in maniera automatica dal software;
	\item \textbf{Allievi}: eseguono gli esercizi proposti ricevendone una valutazione immediata;
	\item \textbf{Sviluppatori}: accedono ai dati collezionati ai fini di migliorare il software per l'analisi grammaticale.
	\end{itemize}
	\paragraph{Dominio tecnologico} \Spazio
	Il committente non impone l'uso di specifiche tecnologie e lascia una discreta libertà sulla scelta di esse:
	\begin{itemize}
		\item \textbf{HTML, CSS e JavaScript}: linguaggi per lo sviluppo della piattaforma web;
		\item \textbf{\gl{Hunpos} o \gl{FreeLing}}: software open source per il \gl{part-of-speech tagging};
		\item  \textbf{\gl{Firebase}}: database \gl{NoSQL}.
	\end{itemize}
	\subsubsection{Valutazione generale}
	\paragraph{Aspetti positivi}
	\begin{itemize}
		\item {La raccolta di dati con il fine di migliorare un prodotto software è risultata interessante grazie anche alle tecniche di apprendimento automatico necessarie per l'ottimizzazione.}
	\end{itemize} 
	\paragraph{Aspetti negativi} 
	\begin{itemize}
		\item {Le tecniche di apprendimento automatico sono sconosciute dal gruppo e ciò richiede l'onere di acquisire determinate competenze in materia;}
		\item {Lo scopo di realizzare una applicazione per esercizi di grammatica non è risultato un ambito di interesse comune.}
	\end{itemize} 
	\subsubsection{Valutazione finale}
	Il capitolato non è stato ricevuto in maniera positva dal gruppo a causa dell'ambito di applicazione del prodotto finale, che non ha suscitato interesse per la sua realizzazione.
	
\subsection{Capitolato C4 - MegAlexa}
\subsubsection{Descrizione}    
Il capitolato C4 \emph{MegAlexa: arricchitore di skill di Amazon Alexa}, presentato dall’azienda Zero12, richiede di creare un applicativo in grado di estendere le funzionalità di \gl{Amazon Alexa} con la possibilità da parte dell’utente di creare delle skill personalizzate a partire da quelle preesistenti, permettendo di eseguire una routine di eventi dato un solo comando vocale.
\subsubsection{Studio del dominio}

\paragraph{Dominio applicativo} \Spazio

L’obiettivo del progetto è la realizzazione di un applicativo web e mobile per la creazione di interazioni personalizzate con Amazon Alexa. Registrandosi alla piattaforma, l’utente potrà accedere all’applicativo che fornirà dei \gl{connettori} (o \gl{micro-funzioni}) da inserire all’interno di un \gl{workflow}, permettendo all’utente di definire una routine univoca di azioni avviabile mediante un comando vocale (e.g.: una routine composta da un messaggio di benvenuto, lettura di un \gl{feed RSS} e selezione delle notizie). Il progetto farà uso di tecnologie di sintesi e riconoscimento vocale con la previsione di realizzare un applicativo multilingue.
\paragraph{Dominio tecnologico}
\begin{itemize}
	\item {\textbf{\gl{AWS}} con le relative API  \gl{Getway}, \gl{Lambda} e \gl{Aurora Serverless}: per l’interazione con Alexa};
	\item  \textbf{\gl{Node.js}}: linguaggio di programmazione per la realizzazione dell’applicativo;
	\item \textbf{HTML, CSS e JavaScript}: linguaggi per lo sviluppo dell’interfaccia web;
	\item  \textbf{Swift o Kotlin}: linguaggi per lo sviluppo della mobile app.
\end{itemize}
\subsubsection{Valutazione generale}
\paragraph{Aspetti positivi}
\begin{itemize}
	\item{Le tecnologie impiegate sono risultate innovative e stimolanti;}
	\item{Interesse per la sintesi vocale in quanto è una tecnologia innovativa, con forte potenziale di sviluppo.}
\end{itemize}
\paragraph{Aspetti negativi}
\begin{itemize}
	\item{Lo sviluppo di un'interfaccia web o mobile non rappresenta un argomento d'interesse per il gruppo;}
	\item{Il capitolato non descrive in modo chiaro e preciso l'obiettivo del progetto.}
\end{itemize}
\subsubsection{Valutazione finale} \Spazio
Il capitolato in esame ha suscitato interesse da parte del team. L'utilizzo di tecnologie giovani e innovative come Alexa e AWS rappresentano una nuova generazione di interfacce, di cui è previsto uno sviluppo intenso nei prossimi anni, tuttavia l'implementazione di una applicazione mobile o di una interfaccia web non è risultata di interesse comune al gruppo in quanto fa uso di tecnologie diffuse e poco stimolanti.
Per i fattori appena elencati, il gruppo si è orientato verso un capitolato diverso.

	\subsection{Capitolato C5 - P2PCS}
\subsubsection{Descrizione}
Il capitolato C5 \emph{P2PCS: piattaforma di peer-to-peer car sharing}, presentato dall'azienda GaiaGo, prevede la realizzazione di un servizio per il car sharing \gl{Peer to Peer (P2P)} di uno o più mezzi di trasporto propri. L'applicativo andrà ad arricchire le funzionalità di un'applicazione mobile preesistente, chiamata \gl{GaiaGo}, la quale implementa un applicativo per il car sharing condominiale.
\subsubsection{Studio del dominio}
\paragraph{Dominio applicativo} \Spazio
L'obiettivo del progetto é la realizzazione di applicativo con lo scopo di far incontrare domanda e offerta di noleggio auto P2P. L'affittuario dovrà poter indicare in quali giorni e in quali orari il suo veicolo è disponibile per un'eventuale prenotazione, il luogo in cui ritirarlo ed eventuali altre condizioni alle quali il cliente dovrà sottostare per affittare il mezzo. Il cliente, invece, avrà la possibilità di cercare e prenotare veicoli per una determinata data e luogo. Il tutto dovrà essere accompagnato da un sistema che coinvolga l'utente tramite sistemi di \gl{Gamification theory}.
\paragraph{Dominio tecnologico} 
\begin{itemize}
	\item  \textbf{NodejS}: piattaforma open source per l'esecuzione di codice JavaScript server-side;
	\item \textbf{\gl{Android}}: sistema operativo per smartphone;
	\item \textbf{\gl{Octalysis}}: \gl{Gamification framework}.
\end{itemize}
\pagebreak
\subsubsection{Valutazione generale}
\paragraph{Aspetti positivi}
\begin{itemize}
	\item Il committente fornisce componenti software per le parti più complesse del capitolato.
\end{itemize}
\paragraph{Aspetti negativi}
\begin{itemize}
	\item Le tecnologie coinvolte non hanno colto l'interesse del team;
	\item Il capitolato è apparso troppo semplice.
\end{itemize}
\subsubsection{Valutazione finale}
Il capitolato non ha colto l'interesse del team per via del contesto poco interessante e della, quantomeno apparente, semplicità del progetto. Infatti, lo 				sviluppo di una piattaforma mobile è stato recepito come poco stimolante rispetto alle altre proposte e non innovativo dal punto di vista delle tecnologie 					interessate. 
		
		
	\subsection{Capitolato C6 - Soldino}
		\subsubsection{Descrizione}
		Il capitolato \emph{C6 Soldino: piattaforma Ethereum per pagamenti IVA}, presentato dall'azienda RedBabel, richiede la realizzazione di un sistema automatizzato basato su \gl{Ethereum} per il calcolo e il pagamento dell'\gl{IVA} tramite un'interfaccia web.
		\subsubsection{Studio del dominio}
			\paragraph{Dominio applicativo} \Spazio
			La finalità del progetto è la realizzazione di un’\gl{applicazione decentralizzata} basata su Ethereum per il calcolo automatico dell’IVA, in modo da agevolare le operazioni contabili degli esercenti. L'applicativo richiede che le attività commerciali siano registrate in una lista, così da tenere traccia della contabilità aziendale al fine di calcolare automaticamente l'importo dell'IVA su base trimestrale. Il progetto richiede anche la realizzazione di un sistema di \gl{e-commerce}.
			
			
			\paragraph{Dominio tecnologico}
			\begin{itemize}
				\item \textbf{Ethereum}: piattaforma per la creazione e pubblicazione di \gl{smart contracts};
				\item \textbf{\gl{Truffle}}: framework per Ethereum;
				\item \textbf{\gl{Raiden networks}}: soluzione per trasferimenti di \gl{token ECR-20} quasi istantanei;
				\item \textbf{HTML, CSS e Javascript}: linguaggi per lo sviluppo di siti web.
			\end{itemize}
		\subsubsection{Valutazione generale}
		\paragraph{Aspetti positivi}
		\begin{itemize}
			\item L'uso di tecnologie moderne ed in continua espansione ha attirato l'attenzione del gruppo;
			\item L'obiettivo del progetto è risultato valido ed interessante.
		\end{itemize}
		\paragraph{Aspetti negativi}
		\begin{itemize}
			\item Alcune tecnologie necessarie non sono conosciute dal gruppo e ciò comporta l'onere di acquisire specifiche conoscenze a riguardo;
			\item La realizzazione di un sistema di e-commerce è risultata dispendiosa e poco stimolante da realizzare.
		\end{itemize}
		\subsubsection{Valutazione finale }
		Il capitolato è stato ricevuto in maniera positiva da parte del gruppo, in quanto forniva molti stimoli riguardo alle tecnologie innovative impiegate, tuttavia è stato deciso di non sceglierlo a causa del sistema di e-commerce da realizzare, che è stata ritenuta una attività dispendiosa e poco stimolante.