\documentclass[12pt]{letter} % Copiata dal nostro stile

\usepackage{setspace}
\usepackage{geometry} % Per modificare margini e dimensioni varie
\usepackage{graphicx} % Per inserire immagini

\geometry{
	paper=a4paper, % Change to letterpaper for US letter
	top=3cm, % Top margin
	bottom=1.5cm, % Bottom margin
	left=2.2cm, % Left margin
	right=2.2cm, % Right margin
	%showframe, % Uncomment to show how the type block is set on the page
}

\usepackage[T1]{fontenc} % Output font encoding for international characters
\usepackage[utf8]{inputenc} % Required for inputting international characters
\usepackage{charter} % stix di default, copiato dagli answer credo
\usepackage{scrextend} % indentazione per il corpo della lettera
\usepackage{microtype} % Improve justification
\usepackage{eurosym} % wanna see the muney

\date{11/01/2019} % comment to show today date

\signature{
	Project manager 7DOS \\
	\includegraphics[width=6cm]{../PianoDiProgetto/Res/Firme/andrea.png}
} % firma a piè di pagina

%\address{123 Broadway \\ City, State 12345 \\ (000) 111-1111} % Your address and phone number











%\usepackage{calc} %introduce la notazione infissa per le op. aritmetiche interne a LaTeX

\usepackage[utf8]{inputenc}
\usepackage[T1]{fontenc}
\usepackage[italian]{babel} %il documento è in italiano
%\usepackage{textcomp} %The pack­age sup­ports the Text Com­pan­ion fonts, which pro­vide many text sym­bols
%(such as baht, bul­let, copy­right, mu­si­cal­note, onequar­ter, sec­tion, and yen), in the TS1 en­cod­ing.

\usepackage{graphicx}       %permette di inserire delle immagini
%\usepackage{caption}        %numerazione figure e loro descrizione testuale
\usepackage[labelformat=empty]{caption}
\usepackage{subcaption}     %sottofigure numerabili
\usepackage{float}  %permette di inserire un # qualsiasi di figure fluttuanti
\usepackage{xcolor}
\usepackage{rotating} %permette di ruotare le immagini
%\usepackage{changepage} %utile se c'è bisogno di aggiustare margini per centrare figure

%package utili per la math mode ( $ ... $ o \[ ... \] )
\usepackage{amsmath}
\usepackage{amssymb}
\usepackage{amsfonts}
%\usepackage{euler}    %font 'ams euler', lo stesso di 'Concrete Mathematics' di Knuth
\usepackage{amsthm}
\usepackage{mathtools}

% package utili per tabelle(\thead in particolare)
\usepackage{array, booktabs, caption}
\usepackage{makecell}
\renewcommand\theadfont{\bfseries}
\usepackage{boldline}

\usepackage{listings} %permette di inserire degli spezzoni di codice

\usepackage{tikz} %disegno di immagini vettoriali a schermo. Utile per grafi
\usetikzlibrary{arrows.meta}
\usetikzlibrary{graphs}
\usetikzlibrary{arrows}
%\usepackage{tikz-uml} %serve per disgnare l'UML, fantastica guida:
%https://perso.ensta-paristech.fr/~kielbasi/tikzuml/var/files/doc/tikzumlmanual.pdf
%download package: http://perso.ensta-paristech.fr/~kielbasi/tikzuml/

%package per le tabelle
\usepackage{booktabs} %permette di poter usare delle liste nelle tabelle
\usepackage{tabularx} 
\usepackage{longtable} %una tabella può continuare su più pagine
\usepackage{multirow} %utile per visualizzare una cella su più righe
%\usepackage{multicolumn} %cella su più colonne
%\usepackage[table]{xcolor} %rende disponibile l'utilizzo di un colore per lo sfondo
                        %delle celle di una tabella

%crea una cella per le tabelle in grado di andare a capo con \newline
%https://tex.stackexchange.com/questions/12703/how-to-create-fixed-width-table-columns-with-text-raggedright-centered-raggedlef
\usepackage{array}
\newcolumntype{L}[1]{>{\raggedright\let\newline\\\arraybackslash\hspace{0pt}}m{#1}}
\newcolumntype{C}[1]{>{\centering\let\newline\\\arraybackslash\hspace{0pt}}m{#1}}
\newcolumntype{R}[1]{>{\raggedleft\let\newline\\\arraybackslash\hspace{0pt}}m{#1}}


%indice con i puntini
\usepackage{tocloft}

%http://ctan.mirror.garr.it/mirrors/CTAN/macros/latex/contrib/appendix/appendix.pdf
\usepackage{appendix} %aggiunge dei comandi per l'appendice
\usepackage{parskip} %aiuta LaTeX a trovare il miglior stile per i page break
\setcounter{secnumdepth}{5} % numera i sottoparagrafi
\setcounter{tocdepth}{5} %aggiunge all'indice i sottoparagrafi
%\usepackage{titlesec} %\begin{paragraph} si può usare come subsubsubsection!


\usepackage{breakurl}%\url{...} può continare alla linea successiva. (si può andare a capo)

%Pacchetto per il simbolo degli euro
\usepackage{eurosym}

%Pacchetto per i colori delle tabelle
\usepackage{color, colortbl}

\definecolor{Maroon}{cmyk}{0, 0.87, 0.68, 0.32}
\usepackage[colorlinks=true]{hyperref}
\hypersetup{
    colorlinks=true,
    citecolor=black,
    filecolor=black,
    linkcolor=black, % colore dei link interni
    urlcolor=blue  % colore dei link interniesterni
}


% Creazione della copertina
\newcommand{\copertina}{
  \newgeometry{top=4cm}
  
  \begin{titlepage}
  \begin{center}

  \begin{center}
    %% qui metteteci l'immagine di copertina. Io ho messo quella dell'uni,
    %voi mettete quella del vostro grupo
    \centerline{\includegraphics[scale=0.1]{../logo}}
  \end{center}
  
  \vspace{1cm}

  \begin{Huge}
    \textbf{\Titolo{}} \\
  \end{Huge}

  \vspace{9pt}  
  
  \begin{large}
  \Gruppo{}\ - \Data{}
  \end{large}	  
  
  \vspace{15pt}

  \bgroup
  \def\arraystretch{1.3}
   \centering
   \begin{tabular}{c|L{5cm}}
      \multicolumn{2}{c}{\textbf{Informazioni sul documento} } \\ \hline
      \textbf{Versione} &  \Versione{}\\
      \textbf{Responsabile} & \Responsabile{}\\
      \textbf{Verifica} & \Verifica{}\\
      \textbf{Redazione} & \Redazione{} \\
      \textbf{Stato} & \Stato{} \\
      \textbf{Uso} & \Uso{} \\
      \textbf{Destinato a} & \Destinatoa{} \\
      \textbf{Email} & \Email{} \\
    \end{tabular}
  \egroup
  
  \vspace{15pt}

  \begin{center}
    \textbf{Descrizione\\}
    \DescrizioneDoc{}
  \end{center}

  \end{center}
  \end{titlepage}
  
  \restoregeometry
}

\newcommand{\code}[1]{\flextt{\texttt{#1}}}

\newcommand{\gl}[1]{\textit{#1}\ped{g}}

% VARIABILI

\newcommand{\sectiontitle}{}
\newcommand{\newsection}[1]{\renewcommand{\sectiontitle}{#1}\section{#1}}
%\newcommand{\titleAppendice}[1]{\renewcommand{\sectiontitle}{#1}\section*{#1}}

\newcommand{\Titolo}{Norme di Progetto}

\newcommand{\Gruppo}{7DOS}

\newcommand{\Versione}{3.0.0}

\newcommand{\Responsabile}{Giacomo Barzon}

\newcommand{\Verifica}{Giovanni Sorice \newline Nicolò Tartaggia}

\newcommand{\Redazione}{Lorenzo Busin \newline Andrea Trevisin \newline Michele Roverato}

\newcommand{\Destinatoa}{Prof. Tullio Vardanega \newline Prof. Riccardo Cardin \newline 7DOS}

\newcommand{\Uso}{Interno}

\newcommand{\Stato}{Approvato}

\newcommand{\Data}{22 Marzo 2019}

\newcommand{\Email}{\url{7dos.swe@gmail.com}}

\newcommand{\DescrizioneDoc}{Questo documento descrive le regole, gli strumenti e le convenzioni adottate durante la realizzazione del progetto \emph{G\&B}.}



\begin{document}
\copertina
\newpage 
\section*{\centering{Diario delle modifiche}}
\begin{table}[H]
	\centering
	\begin{tabular}{p{4cm}|C{3cm}|C{3cm}|R{2.5cm}|R{2cm}}
		\hlineB{3}
		
		\thead{Modifica} &\thead{Autore} &\thead{Ruolo} &\thead{Data} &\thead{Versione} \\
		
		\hlineB{3}
		
		\emph{Approvazione del documento} & Giacomo Barzon & Responsabile & 2019-01-06 & 1.0.0 \\
		\hline
		
		\emph{Verifica del documento} & Lorenzo Busin & Verificatore & 2019-01-05 & 0.8.0 \\
		\hline
		
		\emph{Verifica del documento} & Giovanni Sorice & Verificatore & 2019-01-02 & 0.7.0 \\
		\hline
		
		\emph{Verifica del documento} & Andrea Trevisin & Verificatore & 2018-12-28 & 0.6.0 \\
		\hline
		
		\emph{Stesura Meccanismi di controllo e e Consuntivi} & Nicolò Tartaggia & Analista & 2018-12-24 & 0.5.0 \\
		\hline
		
		\emph{Completamento stesura Pianificazione} & Nicolò Tartaggia & Analista & 2018-12-22 & 0.4.3 \\
		\hline
		
		\emph{Completamento stesura Suddivisione risorse} & Michele Roverato & Analista & 2018-12-17 & 0.4.2 \\
		\hline
		
		\emph{Completamento stesura Analisi dei Rischi} & Marco Costantino & Analista & 2018-12-14 & 0.4.1 \\
		\hline
		
		\emph{Inizio stesura Suddivisione risorse} & Michele Roverato & Analista & 2018-12-07 & 0.4.0 \\
		\hline
		
		\emph{Stesura Modello di Sviluppo} & Michele Roverato & Analista & 2018-12-05 & 0.3.0 \\
		\hline
		
		\emph{Inizio stesura Pianificazione} & Nicolò Tartaggia & Analista & 2018-12-04 & 0.2.0 \\
		\hline
		
		\emph{\textit{Inizio stesura Analisi dei Rischi}} & Marco Costantino & Analista & 2018-12-02 & 0.1.0 \\
		\hline
		
		\emph{Stesura della sezione Introduzione} & Marco Costantino & Analista & 2018-11-30 & 0.0.2 \\
		\hline
		
		\emph{Stesura dello scheletro del documento} & Nicolò Tartaggia & Analista & 2018-11-28 & 0.0.1 \\
		
	\end{tabular}
	
\end{table}


\clearpage

\tableofcontents


	% A	----------------------------------------------------------------------------------------------
	\newpage	
	\newsection{A}


	\subsection{Alert}
	\label{sec:alter}
	Notifica.


	\subsection{Analisi dinamica}
	\label{sec:analisidinamica}
	Tecnica di analisi di un \underline{\hyperref[sec:prodottosoftware]prodotto {software}} in cui il quest'ultimo viene eseguito in un ambiente reale o virtuale; coincide con i test:
	\begin{itemize}
		\item Di unità: verificano il corretto funzionamento dell'unità.
		\item Di integrazione: verifica la corretta interazione tra più unità che interagiscono dando luogo ad un \emph{build}.
		\item Di sistema: verificano la corretta interazione tra più \emph{build} che compongono un sistema complesso.
	\end{itemize}
	Il collaudo, test finale eseguito con il committente, è solo un attività formale se i test precedenti sono stati eseguiti rigorosamente. I test devono essere ripetibili e deterministici, perciò vanno documentati: le condizioni iniziali di esecuzione, l'esecuzione, il risultato atteso ed il risultato ottenuto. I test inoltre devono essere automatizzati per ridurre al minimo la possibilità di errore umano.


	\subsection{Analisi dei requisiti}
	\label{sec:analisirequisiti}
	L'obiettivo dell'analisi dei requisiti è quello di individuare e definire i \underline{\hyperref[sec:requisito]{requisiti}} di progetto. Il risultato di tale processo è un documento contrattuale che andrà fornito all'azienda appaltatrice nell'ambito di una gara d'appalto. Sulla base di questo documento l'azienda deciderà a chi affidare l'appalto.


	\subsection{Analisi statica}
	\label{sec:analisistatica}
	Tecnica di analisi di un \underline{\hyperref[sec:prodottosoftware]{prodotto software}} in cui questo non viene eseguito, bensì ne viene esaminato il codice.
	\emph{Walkthrough} e \emph{Inspection} sono le due tecniche di analisi statica prevalenti: \emph{Walkthrough} prevede un'analisi a tappeto del codice sorgente alla ricerca di possibili criticità, mentre \emph{Inspection} somiglia di più all'analisi a campione e prevede l'analisi specifica di parti considerate di importanza critica.
	
	
	\subsection{Angular TypeScript}
	\label{sec:angulartypescript}
	Framework costruito interamente in \underline{\hyperref[sec:typescript]{TypeScript}}.
	
	\subsection{Attività}
	\label{sec:attivita}
	L'attività è una componente essenziale di un \underline{\hyperref[sec:progetto]{progetto}}. Un'attività prevede dell'intenzionalità: le specifiche dell'attività sono determinate da chi la svolge.

	% B	----------------------------------------------------------------------------------------------
	\newpage
	\newsection{B}
	
	\subsection{Baseline}
	\label{sec:baseline}
	La \emph{baseline} è in generale un punto di partenza, il piano di progetto originale. Essa ha diverse declinazioni che indirizzano un obiettivo strategico e il cui scopo è aiutare a misurare l'avanzamento del processo nella direzione degli obiettivi prefissati. Questi vengono concordati con il committente, in modo di dimostrare l'avanzamento del progetto. La  \emph{baseline} è suddivisa in parti, definite nel modo migliore per aiutare a raggiungere gli obiettivi. Ogni parte evolve nel tempo ed ha quindi un numero di versioni, a cura dell'\emph{owner} (responsabile) della parte. Il mantenimento della  \emph{baseline} avviene tramite la \underline{\hyperref[sec:controlloconfigurazione]{gestione della configurazione}}.

	
	\subsection{Best Practice}
	\label{sec:bestpractice}
	Una \emph{best practice} è il modo migliore di approcciare un problema. Le \emph{best practice} prevedono l'applicazione di principi noti ed autorevoli. In ingegneria è richiesto conoscere ed usare le \emph{best practice} esistenti, non crearne di nuove.


	\subsection{Bottom-Up}
	\label{sec:bottomup}
	Tecnica di sviluppo che consiste nel comporre sistemi complessi partendo da moduli di dimensioni e complessità inferiori.
	
	
	\subsection{Brainstorming}
	\label{sec:brainstorming}
	Discussioni collaborative e creative, in cui viene data voce ad ogni persona a turno. Hanno l'intento di far emergere più spunti e concetti possibili, per poi operare una selezione di quelli più interessanti. Per garantire che idee e considerazioni non vadano perse dev'essere designato un membro per la stesura di un verbale.

	
	\subsection{Budget Variance}
	\label{sec:budgetvariance}
	La \gl{Budget Variance} è una metrica che indica se la spesa \textit{complessiva} attuale è maggiore o minore alla spesa pianificata. Viene calcolata: $BV = EV - AC$ Dove BV indica la budget variance, EV indica l'earned value ovvero il valore delle attività realizzate, AC indica l'actual cost ovvero il costo effettivamente sostenuto.
	
	% C	----------------------------------------------------------------------------------------------
	\newpage
	\newsection{C}

	
	\subsection{Capitolato d'appalto}
	\label{sec:capitolato}
	Documento, prodotto da un'azienda, che descrive un prodotto o un sistema che questa desidera sia realizzato. Il \emph{capitolato} è una chiamata ai fornitori, una notifica di bisogno che andrà esaminata per verificare se è il caso di partecipare al bando d'appalto per la realizzazione del prodotto richiesto. Costituisce la prima fonte di informazioni da analizzare per l'Analisi dei Requisiti. Il dominio dell'azienda appaltatrice influenzerà le informazioni contenute nel capitolato che quindi dovrà essere attentamente analizzato tenendo conto del contesto professionale da cui arriva; sarà necessario un dialogo con l'azienda appaltatrice per la corretta comprensione del capitolato.


	\subsection{Ciclo di vita del software}
	\label{sec:ciclodivita}
	In \underline{\hyperref[sec:swe]{ingegneria del software}}, modo in cui l'attività di realizzazione di un \underline{\hyperref[sec:prodottosoftware]{prodotto software}} viene scomposta in sotto-attività coordinate tra loro, e il cui risultato é il prodotto software e tutta la documentazione necessaria. Il ciclo di vita del software varia a seconda della metodologia di sviluppo adottata.


	
	\subsection{Cloud}
	\label{sec:cloud}
	Abbreviativo di \emph{cloud computing} ("computazione sulle nuvole"). Indica un paradigma di erogazione di servizi informatici da un fornitore ad un cliente attraverso Internet. In particolare, il fornitore permette al cliente di accedere e configurare delle risorse fisiche in remoto, \emph{on demand}.

	
	\subsection{CoCoMo}
	\label{sec:cocomo}
	Modello algoritmico per la stima di costi e risorse. Esso stima le risorse necessarie e le esprime in mesi/persona (MP). Ha come input:
	\begin{itemize}
	\item La complessità del progetto;
	\item Le dimensioni del SW da sviluppare;
	\item Il peso della complessità sullo sviluppo;
	\item Un coefficiente moltiplicativo (parte da 1);
	\item Un fattore di espansione del tempo (parte da 2.5);
	\item Un coefficiente di complessità.
	\end{itemize}

	
	\subsection{Code-'n-Fix}
	\label{sec:codenfix}
	Vecchia pratica nell'ambito di produzione del \underline{\hyperref[sec:prodottosoftware]{software}}, autodescrittiva: l'intera attività di produzione consisteva nel codificare e riparare software.


	
	\subsection{Compito}
	\label{sec:compito}
	Un \emph{compito} è una componente essenziale di un \underline{\hyperref[sec:progetto]{progetto}}. I compiti vengono assegnati e non lasciano spazio alla decisione di chi li riceve.

	
	\subsection{Conditional Probability Table(CPT)}
	\label{sec:CPT}
	Una \emph{Conditional Probability Table (CPT)} è una tabella in cui:
		\begin{itemize}
			\item{Ogni colonna rappresenta l'i-esimo stato in cui il nodo corrente può risiedere. Ad ogni stato è identificato da un nome ed un intervallo di valori associato. Un nodo non può possedere più stati con intervalli di valori sovrapposti tra loro;}
			\item{Ogni riga rappresenta la j-esima combinazione esistente di tutti i possibili stati dei nodi predecessori;}
			\item{Le celle interne indicano la probabilità, condizionata dai nodi predecessori, che il nodo corrente si trovi in uno specifico stato data una combinazione esistente dei possibili stati dei nodi predecessori.}
		\end{itemize}

	
	\subsection{Configurazione}
	\label{sec:configurazione}
	Insieme di regole che determina, per esempio, come assemblare le sezioni di un \underline{\hyperref[sec:prodottosoftware]{software}}, quali \underline{\hyperref[sec:versione]{versioni}} usare per ogni sezione, come le sezioni interagiscono o quali sezioni con quali versioni producono una \underline{\hyperref[sec:baseline]{baseline}}.


	\subsection{Controllo di configurazione}
	\label{sec:controlloconfigurazione}
	Gestione e controllo della \underline{\hyperref[sec:configurazione]{configurazione}} che permette di assemblare le varie componenti di un \underline{\hyperref[sec:prodottosoftware]{software}}.


	\subsection{Committente}
	\label{sec:committente}
	Figura che commissiona un lavoro.

	% D	----------------------------------------------------------------------------------------------
	\newpage
	\newsection{D}

	
	\subsection{Dashboard}
	\label{sec:dashboard}
	Una \emph{dashboard} (in italiano "cruscotto") è una schermata che permette di monitorare in tempo reale l’andamento dei report e delle metriche aziendali. Una \emph{dashboard} può presentare uno o più \underline{\hyperref[sec:panel]{panel}} al suo interno.

	
	\subsection{Design Pattern}
	\label{sec:designpatter}
	Un \emph{design pattern} è una soluzione generica ad un problema ricorrente. I \emph{design patter} non sono specifici di un linguaggio di programmazione e individuano piuttosto l'approccio ottimale in contesto ben definito per la risoluzione di un problema generico. Quando si intende usare un \emph{design pattern} è quindi necessario istanziarlo, ovvero è necessario implementarlo in modo appropriato rispetto ai requisiti del sistema da realizzare e al linguaggio di programmazione usato.

		
	\subsection{DevOps}
	\label{sec:devops}
	Termine derivante dalla contrazione delle parole inglesi \emph{development} ("sviluppo") e \emph{operations} ("operazioni" nel senso di "messa in produzione"). Indica un metodo di sviluppo \underline{\hyperref[sec:prodottosoftware]{software}} basato sulla comunicazione e collaborazione tra sviluppatori e addetti alle "\emph{IT operations}", per sviluppare e mantenere velocemente ed efficacemente prodotti e servizi software.


	\subsection{Disciplinato}
	\label{sec:disciplinato}
	Soggetto ad un insieme di regole pensate per garantire la massima \underline{\hyperref[sec:efficienza]{efficienza}} ed \underline{\hyperref[sec:efficacia]{efficacia}}.

	% E	----------------------------------------------------------------------------------------------
	\newpage
	\newsection{E}


	\subsection{Economicità}
	\label{sec:economicita}
	\underline{\hyperref[sec:efficacia]{Efficacia}} raggiungibile con \underline{\hyperref[sec:efficienza]{efficienza}}. Garantita dall'uso di standard.

	\subsection{Efficacia}
	\label{sec:efficacia}
	L'abilità di un'entità di portare a termine il compito assegnatole.


	\subsection{Efficienza}
	\label{sec:efficienza}
	Data la misura del consumo di risorse che avviene nel compimento di un obiettivo, minore il consumo di risorse, maggiore l'efficienza.
	
	
	%G ----------------------------------------------------------------------------------------------
	\newpage
	\newsection{G}
	
	\subsection{Grafana}
	\label{sec:grafana}
	Grafana è un compositore di grafici, che viene eseguito come applicazione web. Supporta \underline{\hyperref[sec:influxdb]{InfluxDB}}.


	\subsection{Grafana Plug-in Code Style Guide}
	\label{sec:grafana}
	Guida tecnica per la scrittura di codice per plug-in di Grafana.	
	

	% H ----------------------------------------------------------------------------------------------
	\newpage 
	\newsection{H}


	\subsection{HTML}
	\label{sec:html}
	In informatica l'HyperText Markup Language (HTML) è un linguaggio di markup  nato per la formattazione e impaginazione di documenti ipertestuali disponibili nel web. HTML è un linguaggio di pubblico dominio, la cui sintassi è stabilita dal \emph{World Wide Web Consortium} (W3C).
	

	% I	----------------------------------------------------------------------------------------------
	\newpage
	\newsection{I}


	\subsection{IEE 830-1998}
	\label{sec:iee830}
	Best practice raccomandate per la specifica dei \underline{\hyperref[sec:requisito]{requisiti}} di \underline{\hyperref[sec:prodottosoftware]{prodotti software}}.


	\subsection{Incremento}
	\label{sec:incremento}
	Procedura che prevede avanzamento per aggiunta ad un impianto base.


	\subsection{InfluxDB}
	\label{sec:influxdb}
	InfluxDB è un database di serie temporali open source. È ottimizzato per la memorizzazione e il recupero rapido e ad alta disponibilità di dati di serie temporali in campi come il monitoraggio delle operazioni, le metriche dell'applicazione, i dati di sensori e l'analisi in tempo reale.


	\subsection{Intelligenza artificiale}
	\label{sec:ia}
	Disciplina nel ramo dell'informatica, si occupa dello studio e della progettazione di sistemi informatici in grado di funzionare in maniera simile alla mente umana. Per raggiungere questo obiettivo, si occupa anche di elaborare modelli matematici che approssimino le modalità di pensiero ed apprendimento caratteristiche dei cervelli biologici.


	\subsection{In-Schedule}
	\label{sec:inschedule}
	Il termine in-schedule indica la conformità alle aspettative temporali o di costi determinate in fase di pianificazione.


	\subsection{ISO/IEC 12207}
	\label{sec:iso12207}
	Standard riferiti ai processi di \underline{\hyperref[sec:ciclodivita]{ciclo di vita}}, raggruppati in 3 categorie: primari, di supporto, organizzativi.


	\subsection{ISO/IEC 15504}
	\label{sec:iso15504}
	Standard riferiti alla \underline{\hyperref[sec:qualitaprocessi]{qualità dei processi}}, collettivamente noti anche come SPICE (\emph{Software Process Improvement and Capability Determination}). Definiscono una scala di maturità di processo a cinque livelli (più il livello base, detto "livello 0").


	\subsection{ISO/IEC 25010}
	\label{sec:iso25010d}
	Standard riferito alla \underline{\hyperref[sec:qualitaprodotto]{qualità di prodotto}}, noto anche come SQuaRE (\emph{Systems and software Quality Requirements and Evaluation}). Descrive un modello di qualità di prodotto che individua 8 caratteristiche principali da rispettare: \begin{itemize}
	\item Functional Suitability: idoneità funzionale, ossia quanto il prodotto rispetta i requisiti;
	\item Performance Efficiency: \underline{\hyperref[sec:efficienza]{efficienza}} della \underline{\hyperref[sec:performance]{performance}}, ossia quanto il prodotto è in grado di sfruttare efficientemente le risorse disponibili;
	\item Compatibility: compatibilità, ossia quanto il prodotto riesce ad interfacciarsi con altri prodotti o sistemi;
	\item Usability: usabilità, ossia quanto il prodotto risulta di semplice utilizzo;
	\item Reliability: affidabilità, ossia quanto il prodotto risulta stabile nell'esecuzione e nei risultati;
	\item Security: sicurezza, ossia quanto il prodotto protegge i dati trattati;
	\item Maintainability: manutenibilità, ossia quanto il prodotto si presta alla risoluzione di bug e problemi;
	\item Portability: portabilità, ossia quanto il prodotto risulta utilizzabile su ambienti diversi tra loro.
	\end{itemize}


	\subsection{ISO/IEC/IEEE 42010:2011}
	\label{sec:iso12207}
	Standard di \underline{\hyperref[sec:bestpractice]{best practice}} concernenti la \underline{\hyperref[sec:progettazionesoftware]{progettazione software }} e la definizione dell'architettura del \underline{\hyperref[sec:prodottosoftware]{software}}. Punti essenziali sono:
	\begin{itemize}
	\item La decomposizione del sistema in componenti (utile ad aumentare il parallelismo);
	\item Definizione delle interfacce dei componenti;
	\item Definizione dell'organizzazione delle interfacce che permettono l'interazione dei componenti;
	\item Paradigmi vari per la composizione dei componenti.
	\end{itemize}


	\subsection{ISO 90003:2004}
	\label{sec:iso90003}
	Standard di \underline{\hyperref[sec:bestpractice]{best practice}} per la valutazione della qualità di processi dei fornitori. I principi fondamentali sono:
	\begin{itemize}
	\item L'orientamento al cliente;
	\item L'obiettivo di leadership sul mercato;
	\item Il coinvolgimento del personale;
	\item L'approccio per processi;
	\item L'obiettivo del miglioramento continuo;
	\item La presa di decisioni basate su evidenze;
	\item La gestione delle relazioni.
	\end{itemize}
	A garantire l'adesione a questi principi dev'essere la documentazione verticale (specifica di progetto) ed orizzontale (specifica dell'azienda).


	\subsection{Issue Tracking System}
	\label{sec:issuetrack}
	\underline{\hyperref[sec:prodottosoftware]{Prodotto software}} che gestisce e mantiene liste di problematiche. Utile in situazioni di sviluppo collaborativo da parte di un team.


	\subsection{Iterazione}
	\label{sec:iterazione}
	Procedura che prevede avanzamento per raffinamento e rivisitazioni.


	% J	----------------------------------------------------------------------------------------------
	\newpage
	\newsection{J}


	\subsection{JPG}
	\label{sec:jpg}
	Formato lossy per memorizzare immagini.


	\subsection{JavaScript}
	\label{sec:javascript}
	Linguaggio di scripting orientato agli oggetti e agli eventi, comunemente utilizzato nella programmazione web, lato client, per la creazione di effetti dinamici interattivi tramite funzioni di script, invocate da eventi innescati a loro volta in vari modi dall'utente sulla pagina web in uso.


	\subsection{JSON}
	\label{sec:json}
	\emph{JSON(JavaScript Object Notation)} è un formato adatto all'interscambio di dati fra applicazioni client/server basato sul linguaggio \underline{\hyperref[sec:javascript]{JavaScript}}.
	


	% L	----------------------------------------------------------------------------------------------
	\newpage
	\newsection{L}

	
	\subsection{LaTeX}
	\label{sec:latex}
	Linguaggio di markup per la scrittura di testi, particolarmente usato in ambito accademico.


	
	\subsection{Likelihood}
	\label{sec:likelihood}
	Parola inglese, significa "verosimiglianza" in italiano. In ambito probabilistico si usa per indicare la probabilità che un evento si verifichi.

	
	\subsection{Liveliness}
	\label{sec:liveliness}
	Parola inglese, significa "vitalità" in italiano. In ambito informatico si riferisce ad un insieme di proprietà dei sistemi concorrenti, per cui un sistema deve funzionare nonostante i suoi componenti concorrenti debbano "fare a turni" in sezioni critiche.


	% M	----------------------------------------------------------------------------------------------
	\newpage
	\newsection{M}

	
	\subsection{Management}
	\label{sec:management}
	L'insieme delle funzioni amministrative, direttive e gestionali di un'impresa o di un'azienda; il compito del manager.

	
	\subsection{Manuale della qualità}
	\label{sec:manualequalita}
	Documento che specifica le strategie che un organizzazione adotta per operare processi di qualità.


	\subsection{Manutenzione}
	\label{sec:manutenzione}
	Processo correttivo e di sviluppo che avviene dopo il rilascio della versione finale di un prodotto software. Se ne distinguono diversi tipi, come di seguito.

		
		\subsubsection{M. correttiva}
		Ha come scopo la correzione di errori, bug, inesattezze, inefficienze, etc..

		
		\subsubsection{M. adattiva}
		Ha come scopo l'adattamento a diverse tecnologie, ambiti, contesti.

		
		\subsubsection{M. evolutiva}
		Ha come scopo l'adattamento a nuove tecnologie, ambiti, contesti, l'aggiunta di funzionalità, etc..


	
	\subsection{Miglioramento continuo}
	\label{sec:miglioramentocontinuo}
	Principio attorno al quale organizzare i processi per ottenere un miglioramento continuo, prevede 4 macro-fasi:
	\emph{Plan (keep track of what you're going to do)}, \emph{Do (as planned)}, \emph{Check}, \emph{Act (keep what works, throw what doesn't)}.

	
	\subsection{Milestone}
	\label{sec:milestone}	
	Letteralmente pietra miliare, rappresenta un traguardo (anche intermedio) di particolare rilevanza per l'andamento del progetto.
	
	
	\subsection{Minimizzazione}
	\label{sec:minimizzazione}
	Riduzione ad icona.

	
	\subsection{Modello di ciclo di vita}
	\label{sec:modelliciclodivita}
	Il \underline{\hyperref[sec:ciclodivita]{ciclo di vita}} di un \underline{\hyperref[sec:prodottosoftware]{software}} non è univocamente determinato, ma ne vengono individuati diversi modelli possibili. La scelta del modello dipende da 3 macro-fattori: cosa vuole il committente, dipendenza da terze parti, livello di coinvolgimento del committente nell'accertamento dello stato di avanzamento.

		
		\subsubsection{Sequenziale (A cascata)}
			Ha per principio cardine la ripetibilità dei processi.
			Il ciclo di vita sequenziale è lineare, le fasi si susseguono, e la direzione
			ammessa è una sola.
			Il modello fa forte uso di documentazione il che rende il sistema organizzato e tracciabile, prevede pre e post per ogni fase e associa ad ogni fase date di inizio e fine.
ISO 12207 definisce cosi' le fasi del ciclo di vita sequenziale: analisi, progettazione (intesa come \textit{design}), realizzazione, manutenzione.

		
		\subsubsection{Incrementale}
			Prevede un approccio che fa uso di incrementi, ha come vantaggi:
			\begin{itemize}
			\item Il valore aggiunto di ogni incremento
			\item La riduzione del rischio di fallimento portata da ogni incremento
			\item L'uso di abilitatori che facilitano il lavoro
			\end{itemize}

		
		\subsubsection{Evolutivo}
			Fa uso massiccio della fase di manutenzione viene fatta ad ogni versione rilevante del prodotto.

		
		\subsubsection{A componenti}
			Si sviluppa sull'idea di riutilizzare componenti . Prevede quindi tecniche di adattamento delle componenti e dei requisiti finalizzate al riuso del \underline{\hyperref[sec:prodottosoftware]{software}}.

		
		\subsubsection{A spirale}
				Modello di ciclo di vita utilizzato quando il progetto è innovativo e non esistono best practice applicabili per lo sviluppo del progetto.

		
		\subsubsection{Agili}
		Metodi che si rifanno a 4 principi fondamentali:
			\begin{itemize}
			\item Focus sugli \emph{stakeholder} e le loro interazioni, piuttosto che su processi e strumenti;
			\item Focus sul \underline{\hyperref[sec:prodottosoftware]{software}} piuttosto che sulla documentazione;
			\item Rapporto collaborativo piuttosto che contrattuale con il cliente;
			\item Risposta veloce anche se non pianificata al cambiamento.
			\end{itemize}
		Tali metodi offrono degli svantaggi poiché la documentazione è cruciale nella fase di manutenzione del \underline{\hyperref[sec:prodottosoftware]{software}} e l'assenza di pianificazione in faccia ai cambiamenti comporta dei rischi.

			
			\subsubsection{Scrum}
				Scrum ha come focus fondamentale le  \emph{user-stories}: una visione di ciò che vuole il cliente, da esse si ricava un \emph{product backlog}, insieme di attività/feature da svolgere/realizzare. L'attività di \emph{Sprint Backlog} é un passo iterativo che impone controlli frequenti (\emph{Daily Scrum}) sulla correttezza delle azioni intraprese, tramite \emph{Sprint Review} e \emph{Retrospective}.
				
	
	\subsection{Modulo}
	\label{sec:modulo}
	Si tratta di una componente o parte del software, un modulo:
	\begin{itemize}
	\item incapsula dati e comportamenti in maniera coerente e coesa;
	\item ha un interfaccia che ne permette l'uso;
	\item è pensato in modo da favorire la pratica della modularità.
	\end{itemize}
	
	
	\subsection{Monitoraggio dei rischi}
	\label{sec:monitoraggiorischi}
	Attività che ha lo scopo di anticipare l'insorgere di eventi di rischio per mitigarne gli effetti.	
	

	% O	----------------------------------------------------------------------------------------------
	\newpage
	\newsection{O}

	\subsection{Obiettivo SMART}
	\label{sec:smart}
	\begin{itemize}
			\item Specific: formalmente definiti;
			\item Measurable: il cui stato di ottenimento è misurabile;
			\item Achievable: raggiungibili;
			\item Realistic: pratici (raggiungibili);
			\item Time-Bound: costretti da un vincolo temporale.
		\end{itemize}


		
	\subsection{Open source}
	\label{sec:opensource}
	Tipo di prodotto \underline{\hyperref[sec:prodottosoftware]{software}} il cui codice sorgente viene rilascia	al pubblico mediante una licenza che ne permette a chiunque lo studio, la modifica e la distribuzione p	qualsiasi scopo.
		

	% P	----------------------------------------------------------------------------------------------
	\newpage
	\newsection{P}


	\subsection{Panel}
	\label{sec:panel}
	Un panel è una rappresentazione visiva di dati elaborati da un sistema mostrati sotto forma di grafici.


	\subsection{PDCA}
	\label{sec:pdca}
	Acronimo di \emph{Plan, Do, Check, Act}, anche noto come ciclo di Deming. Si tratta di un ciclo di miglioramento continuo dei processi e dei prodotti, suddiviso in quattro fasi, che sono appunto \emph{Plan, Do, Check, Act}.


	\subsection{Pianificazione delle attività}
	\label{sec:pianificazioneattivita}
	Pianificare le attività è uno dei compiti del \underline{\hyperref[sec:projectmanager]{project manager}}.
	La pianificazione include l'allineamento delle attività su un asse temporale, l'assegnazione delle attività a delle persone. \\Il project manager si avvale di diversi strumenti per pianificare le attività:
	\begin{itemize}
	\item Diagrammi di Gantt (Rappresentano durata prevista vs durata effettiva delle attività);
	\item Diagrammi PERT (\emph{Programme Evaluation and Review Technique}, tiene conto delle dipendenze tra attività e mostra lo \hyperref[sec:slack]{slack});
	\item WBS (\emph{Work Breakdown Structure}, divisione delle attività fino a raggiungere il compito minore assegnabile ad una persona. Diagramma che rappresenta l'assegnazione di tali attività alle persone nel tempo, rispettando e rappresentando dipendenze e vincoli temporali.).
	\end{itemize}


	\subsection{Piano di progetto}
	\label{sec:pianoprogetto}
	Include:
	\begin{itemize}
	\item Introduzione (scopo, struttura);
	\item Organizzazione del progetto;
	\item Analisi dei rischi;
	\item Risorse disponibili (tempo, persone);
	\item Suddivisione del lavoro;
	\item Calendario delle attività;
	\item Meccanismi di controllo e rendicontazione.
	\end{itemize}


	\subsection{Plug-in}
	\label{sec:plug-in}
	Un plug-in in campo informatico è un programma non autonomo che interagisce con un altro programma per ampliarne o estenderne le funzionalità originarie. Ad esempio, un plug-in per un \underline{\hyperref[sec:prodottosoftware]{software}} di grafica permette l'utilizzo di nuove funzioni non presenti nel software principale.


	\subsection{PNG}
	\label{sec:png}	
	Formato lossless per memorizzare immagini.

	
	\subsection{Processo}
	\label{sec:processi}
	Insieme di attività correlate (includono tutto ciò che è attinente) e coese (sono tutte necessarie) che trasformano ingressi (bisogni) in uscite (prodotti) secondo regole date dal controllo processo a seguito di decisioni prese sulla base di misurazioni delle risorse consumate dal processo.
	I processi si differenziano in base alla specificità dell'ambito di applicazione, possono essere standard (riferimento base generico), definito (standard adattato alle esigenze aziendali), di progetto (definito adattato al progetto).

	
	\subsection{Processo software}
	\label{sec:processisoftware}
	Insieme di attività che devono essere svolte per far avanzare un prodotto \underline{\hyperref[sec:prodottosoftware]{software}} nel suo ciclo di vita (nell'ambito SWE, vedi "Processo" per una voce più generale).

	
	\subsection{Processo specializzato per progetto}
	\label{sec:processispecializzati}
	Prevedono una fase di pianificazione, una di definizione, attenzione nella conduzione e analisi critica del funzionamento del processo.

	
	\subsection{Prodotto software}
	\label{sec:prodottosoftware}
	Una "entità" software progettata per essere rilasciata ad un cliente.
	Prende forme diverse in base al soggetto richiedente:

		
		\subsubsection{Commessa}
		Si tratta di un prodotto software le cui specifiche ed obiettivi sono indicati da un committente.

		
		\subsubsection{Pacchetto}
		Si tratta di un prodotto software le cui specifiche ed obiettivi sono indicati per la replicazione in o per altri software.

		
		\subsubsection{Componente}
		Si tratta di un prodotto software le cui specifiche ed obiettivi sono indicati per la composizione con altri software.

		
		\subsubsection{Servizio}
		Si tratta di un prodotto software le cui specifiche ed obiettivi sono determinati per la risoluzione di un problema.


	\subsection{Prodotti documentali}
	\label{sec:prodottidocumentali}
	\begin{itemize}
	\item \underline{\hyperref[sec:capitolato]{Capitolato d'appalto};}
	\item \underline{\hyperref[sec:studiofattibilita]{Studio di fattibilità}};
	\item \underline{\hyperref[sec:analisirequisiti]{Analisi dei requisiti}};
	\end{itemize}


	\subsection{Produttività}
	\label{sec:produttivita}
	Una misurazione di \underline{\hyperref[sec:efficienza]{efficienza}}, rapporto tra quantità di prodotto realizzato e risorse consumate.


	\subsection{Progettazione Software}
	\label{sec:progettazionesoftware}
	Insieme di attività che precede la realizzazione il cui scopo è quello definire l'architettura del \underline{\hyperref[sec:prodottosoftware]{software}} con l'obiettivo di sviluppare un prodotto corretto per costruzione piuttosto che per correzione.


	\subsection{Progetto}
	\label{sec:progetto}
	Insieme di attività e compiti, atti a raggiungere \underline{\hyperref[sec:smart]{obiettivi SMART}}, che hanno una data di inizio ed una di fine, e che consumano un pool di risorse limitate.

	
	\subsection{Project Manager}
	\label{sec:projectmanager}
	Individuo che ha il compito di gestione del progetto. Ciò prevede:
	\begin{itemize}
	\item Istanziare i processi nel progetto (quelli standard aziendali e quelli istanziati dai processi aziendali);
	\item Stimare costi e risorse necessarie;
	\item Pianificare attività (organizzarle nel tempo e a chi assegnarle);
	\item Controllare le attività e verificare i risultati.
	\end{itemize}


	\subsection{Proponente}
	\label{sec:proponente}
	Parte terza, un azienda, che propone un \underline{\hyperref[sec:capitolato]{capitolato}}. 

	
	\subsection{Prototipo}
	\label{sec:prototipo}
	Esemplare di prova del prodotto, serve a provare e scegliere soluzioni; può avere carattere usa e getta (in metodi iterativi) oppure essere la base per incrementi successivi (in metodi incrementali).
	Un prototipo di tipo usa e getta comporta dei costi che producono poco valore aggiunto.


	% Q	----------------------------------------------------------------------------------------------
	\newpage
	\newsection{Q}

	
	\subsection{Quantificabile}
	\label{sec:quantificabile}
	Misurabile.

	
	\subsection{Qualità}
	\label{sec:qualita}
	 Misura in cui un prodotto \underline{\hyperref[sec:prodottosoftware]{software}} soddisfa un certo numero di aspettative rispetto sia al suo funzionamento sia alla sua struttura interna.
	\subsubsection{Qualità del prodotto}
	\label{sec:qualitaprodotto}
	Caratteristiche di un prodotto di qualità:
	\begin{itemize}
	\item Sufficienza: Capacità di soddisfare tutti i requisiti;
	\item Comprensibilità: Facilità di utilizzo, comprensione del funzionamento da parte degli  \emph{stakeholder};
	\item Modularità: Suddivisione in parti chiaramente distinte e sconnesse, principio fondamentale dell' \emph{information hiding}.;
	\item Robustezza: Agli input di utilizzo;
	\item Sicurezza: Affidabilità in caso di malfunzionamento (hardware o software);
	\item Sicurezza bis: Resistenza alla intrusioni;
	\item Flessibilità: Misura della facilità con la quale il sistema può essere modificato o adattato per soddisfare nuovi requisiti;
	\item Riusabilità: Misura della facilità di riutilizzo in altri contesti dei moduli che formano il prodotto;
	\item Disponibilità: Più inconveniente il tempo di \emph{downtime} in caso di guasti o aggiornamenti, meno disponibile risulta il prodotto;
	\item \underline{\hyperref[sec:efficienza]{Efficienza}};
	\end{itemize}
	\subsubsection{Qualità dei processi}
	\label{sec:qualitaprocesso}
	Una buona qualità dei processi induce una buona qualità del prodotto di questi. Per assicurarsi che un processo sia di qualità, è necessario un meccanismo di controllo che effettui modifiche sul \underline{\hyperref[sec:wow]{\underline{\hyperref[sec:wow]{way of working}}}} ove ritenuto necessario sulla base di misurazioni sulla qualità dei processi. Per definire gli strumenti di valutazione della qualità dei processi dei fornitori in genere, non solo in ambito di sviluppo software, è stato messo a punto lo standard ISO 9000, in ambito specifico di ingegneria software si ha invece ISO 90003:2004.

		
	% R	----------------------------------------------------------------------------------------------
	\newpage
	\newsection{R}


	\subsection{Raintank}
	\label{sec:raintank}
	Raintank è una piattaforma di monitoraggio open source per raccogliere, archiviare e analizzare enormi quantità di dati diagnostici, da un'ampia varietà di fonti, in molti formati diversi.


	\subsection{Repository}
	\label{sec:repo}
	Termine inglese che significa "deposito", "ripostiglio". In ambito informatico viene usato per indicare una struttura dati in cui vengono gestiti i metadati per un insieme di file o una strutture di cartelle. Esempi di metadati gestiti da una repository sono lo storico dei cambiamenti avvenuti, o l'insieme dei \underline{\hyperref[sec:commit]{commit}} (modifiche).


	\subsection{Requisito}
	\label{sec:requisito}
	I requisiti sono le capacità che il sistema dovrà avere per svolgere le funzioni volute dal committente. È \underline{\hyperref[sec:bestpractice]{best practice}} definire requisiti:
	\begin{itemize}
	\item Non ambigui: Interpretati univocamente da tutti .
	\item Corretti.
	\item Completi.
	\item Variabili.
	\item Consistenti: che non si contraddicono a vicenda o da sè.
	\item Modificabili.
	\item Tracciabili
	\item Ordinati: tra obbligatori, desiderabili ed opzionali.
	\end{itemize}


	\subsection{Rete Bayesiana}
	\label{sec:retebayes}
	Una rete Bayesiana è un modello grafico probabilistico che rappresenta un insieme di variabili stocastiche con le loro dipendenze condizionali attraverso l'uso di un grafo. Per esempio, una rete Bayesiana potrebbe rappresentare la relazione probabilistica esistente tra dei sintomi e delle malattie. Dati i sintomi, la rete può essere usata per calcolare la probabilità della presenza di diverse malattie.

	
	\subsection{Rischio}
	\label{sec:rischio}
	I rischi di progetto sono: sforamento costi, tempi, o risultati insoddisfacenti. Fonti di rischio principali sono:
	tecnologie usate, rapporti interpersonali, organizzazione del lavoro, requisiti e rapporti con  \emph{stakeholder}, tempi e costi.
	La gestione del rischio si fa tramite:
	Identificazione (nel progetto, prodotto, mercato), Analisi (probabilità e conseguenze), Pianificazione (come evitare e o mitigare gli effetti), Controllo(attenzione agli indicatori di rischio, raffinamento strategie).

	
	\subsection{Riuso}
	\label{sec:riuso}
	Del codice, può essere di due tipi: opportunistico (copia e incolla), che ha un basso costo e poco impatto, o sistematico, che ha un maggior costo e un maggior impatto.

	
	\subsection{Ruolo}
	\label{sec:ruolo}
	Indica l'area di specializzazione.
	In un progetto informatico i ruoli principali sono:
	\begin{itemize}
	\item Analista: Si occupa dell'analisi dei requisiti, in generale: individua e definisce il "problema" in termini formali (in modo che sia possibile verificare se la soluzione è tale);
	\item Progettista: Si occupa del design del prodotto. In generale: si occupa della soluzione del "problema";
	\item Programmatore: Realizza il prodotto, non ha libertà di scelta, segue le direttive del progettista;
	\item Verificatore: Testano la qualità del prodotto, danno feedback;
	\item Amministratore: Garantisce il funzionamento dell'apparato informatico aziendale.
	\end{itemize}
	Tutte queste figure professionali sono coordinate nell'ambito di un progetto da un \underline{\hyperref[sec:projectmanager]{project manager}}.
	
	
	% S	----------------------------------------------------------------------------------------------
	\newpage
	\newsection{S}

	
	\subsection{Schedule Variance}
	\label{sec:schedulevariance}
	Schedule variance (SV) è una metrica che indica se le attività di progetto sono in linea rispetto alla loro schedulazione.
	Può assumere valori:
	\begin{itemize}
	\item Positivi: indicano che le attività procedono più velocemente rispetto a quanto pianificato;
	\item Negativi: indicano che le attività procedono più lentamente rispetto a quanto pianificato;
	\item Zero: indica che le attività rispettano la loro schedulazione.
	\end{itemize}
	Generalmente viene calcolata: $SV = EV - PV$
	Dove SV indica la schedule variance, EV indica l'earned value ovvero il valore del lavoro effettuato, PV indica il planned value ovvero il costo del lavoro pianificato.


	\subsection{Script}
	\label{sec:script}
	Il termine script, in informatica, designa un tipo particolare di programma, scritto in una particolare classe di linguaggi di programmazione, detti linguaggi di scripting.


	\subsection{Sistematico}
	\label{sec:sistematico}
	Metodico, rigoroso.


	\subsection{Slack (tempo)}
	\label{sec:slackt}
	Margine di tempo tra la fine di un attività e la scadenza di fine attività. \\
	Se è positivo, l'attività è prevista terminare prima della scadenza quindi vi è un margine per eventuali imprevisti. Se è vicino o 0 tale margine è piccolo o assente, si introducono quindi dei rischi. Se è negativo, la scadenza è passata e l'attività non è stata portata al termine.


	\subsection{Software Engineering}
	\label{sec:swe}
	Disciplina il cui scopo è la realizzazione di  \underline{\hyperref[sec:prodottosoftware]{prodotti software}}; nota anche con l'acronimo SWE.
	SWE si occupa dell'organizzazione e della gestione delle attività, compiti e interazioni di un team di sviluppatori,
	le quali hanno come obiettivo per l'attività di sviluppo e il software risultate \underline{\hyperref[sec:efficacia]{Efficacia}} ed \underline{\hyperref[sec:efficienza]{Efficienza}}.
	SWE si occupa inoltre delle metodologie di cura del progetto per l'intero ciclo di vita del software.
	SWE prevede un approccio sistematico, disciplinato e quantificabile all'attività di sviluppo.


	\subsection{Sorgente Dati}
	\label{sec:sorgentedati}
	Fonte di dati interpretabili per ottenere informazioni.
	
	
	\subsection{Stima dei costi}
	\label{sec:stimacosti}
	Influenzata dalle dimensioni del progetto, dalle esperienze pregresse, dalla familiarità con le tecnologie adottate, dalla produttività dell'ambiente di lavoro, dalla qualità attesa. \\Un modello algoritmico che può aiutare in ciò è il \underline{\hyperref[sec:cocomo]{CoCoMo}}.

	
	\subsection{Studio di fattibilità}
	\label{sec:studiofattibilita}
	Studio economico realizzato per determinare se è vantaggioso partecipare ad una gara d'appalto.
	
	
	% T ----------------------------------------------------------------------------------------------
	\newpage
	\newsection{T}
	
	
	\subsection{Task}
	\label{sec:task}
	Attività associata ad un segmento temporale.
		

	\subsection{Ticket}
	\label{sec:telegram}
	Un ticket è l'associazione tra una task ed un assegnatario. Un sistema di ticketing é una piattaforma per la gestione di task, ticket e assegnatari. 


	\subsection{Tracciamento dei requisiti}
	\label{sec:tracciamentorequisiti}
	Attività di monitoraggio dell'evoluzione e della scoperta dei requisiti che possono cambiare durante lo svolgersi dello sviluppo del progetto.Tale attività viene svolta tramite apparati informatici appositi, le informazioni importanti sui requisiti di cui viene tenuta traccia sono: \emph{status}, \emph{origin}, assegnatari.


	\subsection{TypeScript}
	\label{sec:typescript}
	Estensione di javascript increntrata nell'aumentarne la scalabilità. Sviluppata e manutenuta da Microsoft. Tra le feature di interesse vi è il typing statico e la possibilità di sviluppare applicazioni sia client-side che server-side.

	% V	----------------------------------------------------------------------------------------------
	\newpage
	\newsection{V}
	
	
	\subsection{Validare}
	\label{sec:validare}
	Attività di confronto tra i risultati ottenuti e quelli aspettati. Quest'attività si svolge al termine del progetto.

	
	\subsection{Verificare}
	\label{sec:verificare}
	Attività di verifica di presenza di errori e di rispetto del \underline{\hyperref[sec:wow]{way of working}}. La verifica si manifesta tangibilmente sotto forma di test, effettuati sui moduli che compongono il prodotto \underline{\hyperref[sec:prodottosoftware]{software}}. Le forme di verifica sono 2: \hyperref[sec:analisidinamica]{\underline{analisi dinamica}} e \hyperref[sec:analisistatica]{\underline{analisi statica}}. L'analisi dinamica prevede l'esecuzione del software mentre l'analisi statica prevede un analisi del codice sorgente senza eseguirlo.


	\subsection{Verificatore di requisiti}
	\label{sec:verificatorerequisiti}
	Individuo che ha il compito di verificare i requisiti, utilizza le tecniche descritte in \underline{\hyperref[sec:analisistatica]{analisi statica}}.

	
	\subsection{Versionamento}
	\label{sec:versionamento}
	Organizzazione e gestione delle versioni del \underline{\hyperref[sec:prodottosoftware]{software}} prodotte nel corso dello sviluppo.

	
	\subsection{Versione}
	\label{sec:versione}
	Vedi \underline{\hyperref[sec:baseline]{baseline}}.
	
	
	% W	----------------------------------------------------------------------------------------------
	\newpage
	\newsection{W}
	
	
	\subsection{Way of working}
	\label{sec:wow}
	Insieme di regole, pratiche, che caratterizzano l'attività di sviluppo. Il way of working è pensato per agevolare lo sviluppo ed evitare sprechi di risorse ed errori.

	
	\subsection{Webhook}
	\label{sec:webhook}
	Tecnologia nell'ambito del web development che permette di modificare il comportamento di una pagina web tramite callbacks personalizzate.


	\subsection{Workflow}
	\label{sec:workflow}
	Termine inglese che significa "flusso di lavoro". Indica la gestione dei processi lavorativi.

	 
	% X ----------------------------------------------------------------------------------------------
	\newpage
	\newsection{X}


	\subsection{XML}
	\label{sec:xml}
	Si tratta di un meta linguaggio che permette di definire \underline{\hyperref[sec:linguaggiomarkup]{linguaggio di markup}}.
	
	
	% Z	----------------------------------------------------------------------------------------------
	\newpage
	\newsection{Z}

	
	\subsection{Zero-latency}
	\label{sec:zerolatency}
	Termine inglese che significa "senza ritardo". Riferito al lavoro, indica un approccio prevede l'inizio della attività di sviluppo non appena possibile. Permette di avere un margine di tempo tra termine del progetto e consegna.

	
	\subsection{Zero-laxity}
	\label{sec:zerolaxity}
	Termine inglese che significa "senza margine". Riferito al lavoro, indica un approccio prevede l'inizio delle attività di sviluppo nel momento in cui la data di consegna del progetto meno il tempo richiesto dallo sviluppo equivale zero o meno.


\end{document}
