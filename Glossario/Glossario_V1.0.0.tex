\documentclass[12pt]{letter} % Copiata dal nostro stile

\usepackage{setspace}
\usepackage{geometry} % Per modificare margini e dimensioni varie
\usepackage{graphicx} % Per inserire immagini

\geometry{
	paper=a4paper, % Change to letterpaper for US letter
	top=3cm, % Top margin
	bottom=1.5cm, % Bottom margin
	left=2.2cm, % Left margin
	right=2.2cm, % Right margin
	%showframe, % Uncomment to show how the type block is set on the page
}

\usepackage[T1]{fontenc} % Output font encoding for international characters
\usepackage[utf8]{inputenc} % Required for inputting international characters
\usepackage{charter} % stix di default, copiato dagli answer credo
\usepackage{scrextend} % indentazione per il corpo della lettera
\usepackage{microtype} % Improve justification
\usepackage{eurosym} % wanna see the muney

\date{11/01/2019} % comment to show today date

\signature{
	Project manager 7DOS \\
	\includegraphics[width=6cm]{../PianoDiProgetto/Res/Firme/andrea.png}
} % firma a piè di pagina

%\address{123 Broadway \\ City, State 12345 \\ (000) 111-1111} % Your address and phone number











%\usepackage{calc} %introduce la notazione infissa per le op. aritmetiche interne a LaTeX

\usepackage[utf8]{inputenc}
\usepackage[T1]{fontenc}
\usepackage[italian]{babel} %il documento è in italiano
%\usepackage{textcomp} %The pack­age sup­ports the Text Com­pan­ion fonts, which pro­vide many text sym­bols
%(such as baht, bul­let, copy­right, mu­si­cal­note, onequar­ter, sec­tion, and yen), in the TS1 en­cod­ing.

\usepackage{graphicx}       %permette di inserire delle immagini
%\usepackage{caption}        %numerazione figure e loro descrizione testuale
\usepackage[labelformat=empty]{caption}
\usepackage{subcaption}     %sottofigure numerabili
\usepackage{float}  %permette di inserire un # qualsiasi di figure fluttuanti
\usepackage{xcolor}
\usepackage{rotating} %permette di ruotare le immagini
%\usepackage{changepage} %utile se c'è bisogno di aggiustare margini per centrare figure

%package utili per la math mode ( $ ... $ o \[ ... \] )
\usepackage{amsmath}
\usepackage{amssymb}
\usepackage{amsfonts}
%\usepackage{euler}    %font 'ams euler', lo stesso di 'Concrete Mathematics' di Knuth
\usepackage{amsthm}
\usepackage{mathtools}

% package utili per tabelle(\thead in particolare)
\usepackage{array, booktabs, caption}
\usepackage{makecell}
\renewcommand\theadfont{\bfseries}
\usepackage{boldline}

\usepackage{listings} %permette di inserire degli spezzoni di codice

\usepackage{tikz} %disegno di immagini vettoriali a schermo. Utile per grafi
\usetikzlibrary{arrows.meta}
\usetikzlibrary{graphs}
\usetikzlibrary{arrows}
%\usepackage{tikz-uml} %serve per disgnare l'UML, fantastica guida:
%https://perso.ensta-paristech.fr/~kielbasi/tikzuml/var/files/doc/tikzumlmanual.pdf
%download package: http://perso.ensta-paristech.fr/~kielbasi/tikzuml/

%package per le tabelle
\usepackage{booktabs} %permette di poter usare delle liste nelle tabelle
\usepackage{tabularx} 
\usepackage{longtable} %una tabella può continuare su più pagine
\usepackage{multirow} %utile per visualizzare una cella su più righe
%\usepackage{multicolumn} %cella su più colonne
%\usepackage[table]{xcolor} %rende disponibile l'utilizzo di un colore per lo sfondo
                        %delle celle di una tabella

%crea una cella per le tabelle in grado di andare a capo con \newline
%https://tex.stackexchange.com/questions/12703/how-to-create-fixed-width-table-columns-with-text-raggedright-centered-raggedlef
\usepackage{array}
\newcolumntype{L}[1]{>{\raggedright\let\newline\\\arraybackslash\hspace{0pt}}m{#1}}
\newcolumntype{C}[1]{>{\centering\let\newline\\\arraybackslash\hspace{0pt}}m{#1}}
\newcolumntype{R}[1]{>{\raggedleft\let\newline\\\arraybackslash\hspace{0pt}}m{#1}}


%indice con i puntini
\usepackage{tocloft}

%http://ctan.mirror.garr.it/mirrors/CTAN/macros/latex/contrib/appendix/appendix.pdf
\usepackage{appendix} %aggiunge dei comandi per l'appendice
\usepackage{parskip} %aiuta LaTeX a trovare il miglior stile per i page break
\setcounter{secnumdepth}{5} % numera i sottoparagrafi
\setcounter{tocdepth}{5} %aggiunge all'indice i sottoparagrafi
%\usepackage{titlesec} %\begin{paragraph} si può usare come subsubsubsection!


\usepackage{breakurl}%\url{...} può continare alla linea successiva. (si può andare a capo)

%Pacchetto per il simbolo degli euro
\usepackage{eurosym}

%Pacchetto per i colori delle tabelle
\usepackage{color, colortbl}

\definecolor{Maroon}{cmyk}{0, 0.87, 0.68, 0.32}
\usepackage[colorlinks=true]{hyperref}
\hypersetup{
    colorlinks=true,
    citecolor=black,
    filecolor=black,
    linkcolor=black, % colore dei link interni
    urlcolor=blue  % colore dei link interniesterni
}


% Creazione della copertina
\newcommand{\copertina}{
  \newgeometry{top=4cm}
  
  \begin{titlepage}
  \begin{center}

  \begin{center}
    %% qui metteteci l'immagine di copertina. Io ho messo quella dell'uni,
    %voi mettete quella del vostro grupo
    \centerline{\includegraphics[scale=0.1]{../logo}}
  \end{center}
  
  \vspace{1cm}

  \begin{Huge}
    \textbf{\Titolo{}} \\
  \end{Huge}

  \vspace{9pt}  
  
  \begin{large}
  \Gruppo{}\ - \Data{}
  \end{large}	  
  
  \vspace{15pt}

  \bgroup
  \def\arraystretch{1.3}
   \centering
   \begin{tabular}{c|L{5cm}}
      \multicolumn{2}{c}{\textbf{Informazioni sul documento} } \\ \hline
      \textbf{Versione} &  \Versione{}\\
      \textbf{Responsabile} & \Responsabile{}\\
      \textbf{Verifica} & \Verifica{}\\
      \textbf{Redazione} & \Redazione{} \\
      \textbf{Stato} & \Stato{} \\
      \textbf{Uso} & \Uso{} \\
      \textbf{Destinato a} & \Destinatoa{} \\
      \textbf{Email} & \Email{} \\
    \end{tabular}
  \egroup
  
  \vspace{15pt}

  \begin{center}
    \textbf{Descrizione\\}
    \DescrizioneDoc{}
  \end{center}

  \end{center}
  \end{titlepage}
  
  \restoregeometry
}

\newcommand{\code}[1]{\flextt{\texttt{#1}}}

\newcommand{\gl}[1]{\textit{#1}\ped{g}}

% VARIABILI

\newcommand{\sectiontitle}{}
\newcommand{\newsection}[1]{\renewcommand{\sectiontitle}{#1}\section{#1}}
%\newcommand{\titleAppendice}[1]{\renewcommand{\sectiontitle}{#1}\section*{#1}}

\newcommand{\Titolo}{Norme di Progetto}

\newcommand{\Gruppo}{7DOS}

\newcommand{\Versione}{3.0.0}

\newcommand{\Responsabile}{Giacomo Barzon}

\newcommand{\Verifica}{Giovanni Sorice \newline Nicolò Tartaggia}

\newcommand{\Redazione}{Lorenzo Busin \newline Andrea Trevisin \newline Michele Roverato}

\newcommand{\Destinatoa}{Prof. Tullio Vardanega \newline Prof. Riccardo Cardin \newline 7DOS}

\newcommand{\Uso}{Interno}

\newcommand{\Stato}{Approvato}

\newcommand{\Data}{22 Marzo 2019}

\newcommand{\Email}{\url{7dos.swe@gmail.com}}

\newcommand{\DescrizioneDoc}{Questo documento descrive le regole, gli strumenti e le convenzioni adottate durante la realizzazione del progetto \emph{G\&B}.}



\begin{document}
\copertina
\newpage 
\section*{\centering{Diario delle modifiche}}
\begin{table}[H]
	\centering
	\begin{tabular}{p{4cm}|C{3cm}|C{3cm}|R{2.5cm}|R{2cm}}
		\hlineB{3}
		
		\thead{Modifica} &\thead{Autore} &\thead{Ruolo} &\thead{Data} &\thead{Versione} \\
		
		\hlineB{3}
		
		\emph{Approvazione del documento} & Giacomo Barzon & Responsabile & 2019-01-06 & 1.0.0 \\
		\hline
		
		\emph{Verifica del documento} & Lorenzo Busin & Verificatore & 2019-01-05 & 0.8.0 \\
		\hline
		
		\emph{Verifica del documento} & Giovanni Sorice & Verificatore & 2019-01-02 & 0.7.0 \\
		\hline
		
		\emph{Verifica del documento} & Andrea Trevisin & Verificatore & 2018-12-28 & 0.6.0 \\
		\hline
		
		\emph{Stesura Meccanismi di controllo e e Consuntivi} & Nicolò Tartaggia & Analista & 2018-12-24 & 0.5.0 \\
		\hline
		
		\emph{Completamento stesura Pianificazione} & Nicolò Tartaggia & Analista & 2018-12-22 & 0.4.3 \\
		\hline
		
		\emph{Completamento stesura Suddivisione risorse} & Michele Roverato & Analista & 2018-12-17 & 0.4.2 \\
		\hline
		
		\emph{Completamento stesura Analisi dei Rischi} & Marco Costantino & Analista & 2018-12-14 & 0.4.1 \\
		\hline
		
		\emph{Inizio stesura Suddivisione risorse} & Michele Roverato & Analista & 2018-12-07 & 0.4.0 \\
		\hline
		
		\emph{Stesura Modello di Sviluppo} & Michele Roverato & Analista & 2018-12-05 & 0.3.0 \\
		\hline
		
		\emph{Inizio stesura Pianificazione} & Nicolò Tartaggia & Analista & 2018-12-04 & 0.2.0 \\
		\hline
		
		\emph{\textit{Inizio stesura Analisi dei Rischi}} & Marco Costantino & Analista & 2018-12-02 & 0.1.0 \\
		\hline
		
		\emph{Stesura della sezione Introduzione} & Marco Costantino & Analista & 2018-11-30 & 0.0.2 \\
		\hline
		
		\emph{Stesura dello scheletro del documento} & Nicolò Tartaggia & Analista & 2018-11-28 & 0.0.1 \\
		
	\end{tabular}
	
\end{table}


\clearpage

\tableofcontents

\newpage
	% A	----------------------------------------------------------------------------------------------
	{\Huge{\textbf{A}}} \\
	\line(1,0){450}
	
	% Attività ------------------------------------------------------
	\section{Attività}
	\label{sec:attivita}
	L'attività è una componente essenziale di un \hyperref[sec:progetto]{progetto}. Un attività prevede dell'intenzionalità: le specifiche dell'attività sono determinate da chi la svolge. 	
	
	% Analisi dei requisiti ------------------------------------------------------
	\section{Analisi dei requisiti}
	\label{sec:analisirequisiti}
	L'obiettivo dell'analisi dei requisiti è quello di individuare e definire i \hyperref[sec:requisito]{requisiti} di progetto. Output di tale processo è un documento contrattuale che andrà fornito all'azienda appaltatrice nell'ambito di una gara d'appalto. Sulla base di tale documento l'azienda deciderà a chi affidare l'appalto. \newpage

	% B	----------------------------------------------------------------------------------------------
	{\Huge{\textbf{B}}} \\
	\line(1,0){450}
	
	% Baseline ------------------------------------------------------
	\section{Baseline}
	\label{sec:baseline}
	La baseline è in generale un punto di partenza, il piano di progetto originale. La baseline ha diverse declinazioni che indirizzano un obbiettivo strategico e il loro scopo è di aiutare a misurare l'avanzamento del processo nella direzione degli obbiettivi. Gli obbiettivi vengono concordati con il committente in modo di dimostrare l'avanzamento del progetto. La baseline è suddivisa in parti, definite nel modo migliore per aiutare a raggiungere gli obbiettivi. Ogni parte evolve nel tempo ed ha quindi un numero di versioni, a cura dell'owner (responsabile) della parte. Il mantenimento della baseline avviene tramite la \hyperref[sec:controlloconfigurazione]{gestione della configurazione}. 
	
	% Best Practice -------------------------------------------------
	\section{Best Practice}
	\label{sec:bestpractice}
	Una best practice è il migliore modo di approciare un problema. Le best practice prevedono l'applicazione di principi noti ed autorevoli. In ingegneria è richiesta la conoscenza e l'uso di best practices, non di crearne di nuove.

	% Best Practice -------------------------------------------------
	\section{Brainstorming}
	\label{sec:brainstorming}
	Discussioni collaborative, creative, viene data voce ad ogni persona a turno, si tiene traccia della discussione. \newpage


	% C	----------------------------------------------------------------------------------------------
	{\Huge{\textbf{C}}} \\
	\line(1,0){450}

	% Ciclo di vita del software ------------------------------------
	\section{Capitolato d'appalto}
	\label{sec:capitolato}
	Documento prodotto da un azienda che descrive il prodotto di cui l'azienda sente il bisogno. Il capitolato è una chiamata ai fornitori, una notifica di bisogno che andrà esaminata per verificare se è il caso di partecipare al bando d'appalto per la realizzazione del prodotto voluto dall'azienda. Il capitolato d'appalto è la prima fonte di informazioni da analizzare per l'analisi dei requisiti. Il dominio dell'azienda appaltatrice influenzerà le informazioni contenute nel capitolato che quindi dovrà essere attentamente analizzato tenendo conto del contesto professionale da cui arriva, il dialogo con l'azienda appaltatrice sarà necessario per la corretta comprensione del capitolato. 
	
	% Ciclo di vita del software ------------------------------------
	\section{Ciclo di vita del software}
	\label{sec:ciclodivita}
	Insieme di fasi, lo sviluppo del software è una di esse, le fasi sono diverse in base al modello di ciclo di vita utilizzato.

	% CoCoMo -------------------------------------------------------
	\section{CoCoMo}
	\label{sec:cocomo}
	Modello algoritmico per la stima di costi e risorse. Il modello stima le risorse necessarie e le esprime in mesi/persona MP. Ha come input:
	\begin{itemize}  
	\item La complessità del progetto.
	\item Le dimensioni del SW da sviluppare.
	\item Il peso della complessità sullo sviluppo.
	\item Un coefficente moltiplicativo (parte da 1).
	\item Un fattore di espansione del tempo (parte da 2.5).
	\item Un coefficiente di complessità. 
	\end{itemize}		

	% Code-'n-Fix -------------------------------------------------------
	\section{Code-'n-Fix}
	\label{sec:codenfix}
	Vecchia pratica nell'ambito di produzione del software, autodescrittiva: l'intera attività di produzione del software consisteva nel codificare e riparare software.
	
	% Compiti -------------------------------------------------------
	\section{Compito}
	\label{sec:compiti}
	Un compito è una componente essenziale di un \hyperref[sec:progetto]{progetto}. I compiti vengono assegnati e non lasciano spazio alla decisione di chi li riceve.
	
	% Configurazione ------------------------------------------------
	\section{Configurazione}
	\label{sec:configurazione}
	Insieme di regole che determina come assemblare le sezioni di un software, quali \hyperref[sec:versione]{versioni} usare per ogni sezione, come le sezioni interagiscono, quali sezioni con quali versioni producono una baseline eccetera.
	
	% Controllo di configurazione ----------------------------------
	\section{Controllo di configurazione}
	\label{sec:controlloconfigurazione}
	Gestione e controllo della \hyperref[sec:configurazione]{configurazione} che permette di assemblare le varie componenti di un software.\newpage

	% D	----------------------------------------------------------------------------------------------
	{\Huge{\textbf{D}}} \\
	\line(1,0){450}

	% Disciplinato -------------------------------------------------	
	\section{Disciplinato}
	\label{sec:disciplinato}
	Soggetto ad un insieme di regole pensate per garantire la massima \hyperref[sec:efficienza]{efficienza} ed \hyperref[sec:efficacia]{efficacia}.\newpage

	% E	----------------------------------------------------------------------------------------------
	{\Huge{\textbf{E}}} \\
	\line(1,0){450}

	% Economicità --------------------------------------------------
	\section{Economicità}
	\label{sec:economicita}
	Efficacia raggiungibile con efficienza. Garantita dall'uso di standard.
	
	% Efficacia ---------------------------------------------------
	\section{Efficacia}
	\label{sec:efficacia}
	L'abilità di un entità di portare a il compito assegnatole.
	
	% Efficienza ---------------------------------------------------
	\section{Efficienza}
	\label{sec:efficienza}
	Data la misura del consumo di risorse che avviene nel compimento di un obiettivo, minore il consumo di risorse, maggiore l'efficienza.\newpage

	% I	----------------------------------------------------------------------------------------------
	{\Huge{\textbf{I}}} \\
	\line(1,0){450}

	% Incremento ---------------------------------------------------
	\section{Incremento}
	\label{sec:incremento}
	Procedura che prevede avanzamento per aggiunta ad un impianto base.

	%ISO 12207 -----------------------------------------------------
	\section{IEE 830-1998}
	\label{sec:iee830}
	Best practice raccomandate per la specifica dei requisiti di prodotti software. Vedi la voce \hyperref[sec:requisito]{requisito}.
	\linebreak

	%ISO 12207 -----------------------------------------------------
	\section{ISO/IEC 12207}
	\label{sec:iso12207}
	Standard riferiti ai processi di \hyperref[sec:ciclodivita]{ciclo di vita}, raggruppati in 3 categorie: Primari, di supporto, organizzativi.
	\linebreak

	%ISO 42010:2011 -----------------------------------------------------
	\section{ISO/IEC/IEEE 42010:2011}
	\label{sec:iso12207}
	Standard di best practice riferiti alla progettazione software, alla definizione dell'architettura del software. Punti essenziali sono:
	\begin{itemize}  
	\item La decomposizione del sistema in componenti (utile ad aumentare il parallelismo)
	\item Definizione delle interfaccie dei componenti 
	\item Definizione dell'organizzazione delle interfaccie che permettono l'interazione dei componenti
	\item Paradigmi vari per la composizione dei componenti 
	\end{itemize}		

	%ISO 90003:2004 -----------------------------------------------------
	\section{ISO 90003:2004}
	\label{sec:iso90003}
	Standard di best practice per la valutazione della qualità di processi dei fornitori. I principi fondamentali sono:
	\begin{itemize}  
	\item L'orientamento al cliente.
	\item L'obiettivo di leadership sul mercato.
	\item Il coinvolgimento del personale.
	\item L'approccio per processi.
	\item L'obiettivo del miglioramento continuo.
	\item La presa di decisioni basate su evidenze.
	\item La gestione delle relazioni.
	\end{itemize}	
	A garantire l'adesione a questi principi dev'essere la documentazione verticale (specifica di progetto) ed orizzontale (specifica dell'azienda).	
	
	% Iterazione ---------------------------------------------------
	\section{Iterazione}
	\label{sec:iterazione}
	Procedura che prevede avanzamento per raffinamento e rivisitazioni.\newpage

	% M	----------------------------------------------------------------------------------------------
	{\Huge{\textbf{M}}} \\
	\line(1,0){450}
	
	% Manutenzione ------------------------------------------------	
	\section{Manuale della qualità}
	\label{sec:manualequalita}
	Documento che specifica le strategie che un organizzazione adotta per operare processi di qualità.
	
	
	\section{Manutenzione}
	\label{sec:manutenzione}
	La manutenzione di un prodotto software è di diversi tipi:
		
		% Correttiva ------------------------------------------------	
		\subsection{Correttiva}
		Ha come scopo la correzione di errori, bug, inesattezze, inefficienze, etc..
		
		% Adattiva ------------------------------------------------
		\subsection{Adattiva}
		Ha come scopo l'adattamento a diverse tecnologie, ambiti, contesti.
		
		% Evolutiva ------------------------------------------------
		\subsection{Evolutiva}
		Ha come scopo l'adattamento a nuove tecnologie, ambiti, contesti, l'aggiunta di funzionalità, etc..

	% Modelli di ciclo di vita -------------------------------------
	\section{Miglioramento continuo}
	\label{sec:miglioramentocontinuo}
	Principio attorno al quale organizzare i processi per ottenere un miglioramento continuo, prevede 4 macrofasi:
	Plan (keep track of what you're going to do), Do (as planned), Check, Act (keep what works, throw what doesn't)
		
	% Modelli di ciclo di vita -------------------------------------
	\section{Modello di ciclo di vita}
	\label{sec:modelliciclodivita}
	Il \hyperref[sec:ciclodivita]{ciclo di vita} di un software non è univocamente determinato. Diversi modelli descrivono il ciclo di vita di un software.
	Il ciclo di vita determina quali processi attivare. La scelta del modello dipende da 3 macrofattori: cosa vuole il committente, dipendenza da terze parti, livello di coinvolgimento del committente nell'accertamento dello stato di avanzamento.
	
		%Sequenziale ----------------------------------------
		\subsection{Sequenziale (A cascata)}
			Ha per principio cardine la ripetibilità dei processi.
			Il ciclo di vita sequenziale è lineare, le fasi si susseguono, e la direzione
			ammessa è una sola.
			Il modello fa forte uso di documentazione il che rende il sistema organizzato e tracciabile, prevede pre e post per ogni fase e associa ad ogni fase date di inizio e fine.  
ISO 12207 definisce cosi' le fasi del ciclo di vita sequenziale: analisi, progettazione (intesa come \textit{design}), realizzazione, manutenzione. 		
		
		%Incrementale ----------------------------------------
		\subsection{Incrementale}
			Prevede un approccio che fa uso di incrementi, ha come vantaggi:
			\begin{itemize}  
			\item Il valore aggiunto di ogni incremento 
			\item La riduzione del rischio di fallimento portata da ogni incremento
			\item L'uso di abilitatori che facilitano il lavoro 
			\end{itemize}			
			
		%Evolutivo ----------------------------------------
		\subsection{Evolutivo}
			Fa uso massiccio della fase di manutenzione viene fatta ad ogni versione rilevante del prodotto.
		
		%A componenti ----------------------------------------
		\subsection{A componenti}
			Si sviluppa sull'idea di riutilizzare componenti software. Prevede quindi tecniche di adattamento delle componenti e dei requisiti finalizzate al riuso del software.
		
		%Spirale ----------------------------------------
		\subsection{A spirale}	
				Modello di ciclo di vita utilizzato quando il progetto è innovativo e non esistono best practice applicabili per lo sviluppo del progetto.
		
		%Agili ----------------------------------------
		\subsection{Agili}
		Metodi che si rifanno a 4 principi fondamentali:
			\begin{itemize}  
			\item Focus sugli stakeholder e le loro interazioni, piuttosto che su processi e strumenti
			\item Focus sul software piuttosto che sulla documentazione
			\item Rapporto collaborativo piuttosto che contrattuale con il cliente
			\item Risposta veloce anche se non pianificata al cambiamento. 
			\end{itemize}				
		Tali metodi offrono degli svantaggi poichè la documentazione è cruciale nella fase di manutenzione del software e l'assenza di pianificazione in faccia ai cambiamenti comporta dei rischi.
		
			%Scrum ----------------------------------------
			\subsubsection{Scrum}
				Scrum ha come focus fondamentale le  user-stories: una visione di ciò che vuole il cliente, da esse si ricava un product backlog, insieme di attivita'/feature da svolgere/realizzare. L'attivita' di Sprint Backlog e' un passo iterativo che impone controlli frequenti (Daily Scrum) sulla correttezza delle azioni intraprese tramite Sprint Review, e Retrospective.\newpage

	% P	----------------------------------------------------------------------------------------------
	{\Huge{\textbf{P}}} \\
	\line(1,0){450}
	
	% Pianificazione delle attività --------------------------------------------

	\section{Pianificazione delle attività}	
	\label{sec:pianificazioneattivita}
	Pianificare le attività è uno dei compiti del \hyperref[sec:projectmanager]{project manager}.
	La pianificazione include l'allineamento delle attività su un asse temporale, l'assegnazione delle attività a delle persone. \\Il project manager si avvale di diversi strumenti per pianificare le attività:  	
	\begin{itemize}  
	\item Diagrammi di Gantt (Rappresentano durata prevista vs durata effettiva delle attività)
	\item Diagrammi PERT (Programme Evaluation and Review Technique tiene conto delle dipendenze tra attività e mostra lo \hyperref[sec:slack]{slack})
	\item WBS (Work Breakdown Structure divisione delle attività fino a raggiungere il compito minore assegnabile ad una persona. Diagramma che rappresenta l'assegnazione di tali attività alle persone nel tempo, rispettando e rappresentando dipendenze e vincoli temporali.) \ldots 
	\end{itemize}	
	
	\section{Piano di progetto}	
	\label{sec:pianoprogetto}
	Include:
	\begin{itemize}  
	\item Introduzione (scopo, struttura)
	\item Organizzazione del progetto
	\item Analisi dei rischi
	\item Risorse disponibili (tempo, persone)
	\item Suddivisione del lavoro
	\item Calendario delle attività
	\item Meccanismi di controllo e di rendicontazione	
	\end{itemize}	
		
	% Processo -----------------------------------------------------------------
	\section{Processo}	
	\label{sec:processi}
	Insieme di attività correlate (includono tutto ciò che è attinente) e coese (sono tutte necessarie) che trasformano ingressi (bisogni) in uscite (prodotti) secondo regole date dal controllo processo a seguito di decisioni prese sulla base di misurazioni delle risorse consumate dal processo.
	I processi si differenziano in base alla specificità dell'ambito di applicazione, possono essere standard (riferimento base generico), definito (standard adattato alle esigenze aziendali), di progetto (definito adattato al progetto).
	
	% Processi Software --------------------------------------------------------
	\section{Processo Software}
	\label{sec:processisoftware}
Insieme di attività che devono essere svolte per far avanzare un prodotto software nel suo ciclo di vita (nell'ambito SWE, vedi "Processi" per una voce più generale).

	% Processi specializzati per progetto --------------------------------------------
	\section{Processo specializzato per progetto}
	\label{sec:processispecializzati}
	Prevodono una fase di pianificazione, una di definizione, attenzione nella conduzione e analisi critica del funzionamento del processo.
	
	% Prodotto Software --------------------------------------------	
	\section{Prodotto Software}
	\label{sec:prodottosoftware}
	Un prodotto software prende forme diverse in base al soggetto richiedente:
		
		% Commessa -------------------------------------------------
		\subsection{Commessa}
		Si tratta di un prodotto software le cui specifiche ed obiettivi sono indicati da un committente.
	
		% Pacchetto ------------------------------------------------		
		\subsection{Pacchetto}
		Si tratta di un prodotto software le cui specifiche ed obiettivi sono indicati per la replicazione in o per altri software.
		
		% Componente ------------------------------------------------		
		\subsection{Componente}
		Si tratta di un prodotto software le cui specifiche ed obiettivi sono indicati per la composizione con altri software.
		
		% Servizio ------------------------------------------------		
		\subsection{Servizio}
		Si tratta di un prodotto software le cui specifiche ed obiettivi sono determinati per la risoluzione di un problema.
	
	% Prodotti documentali --------------------------------------------		
	\section{Prodotti documentali}	
	\label{sec:prodottidocumentali}
	\begin{itemize}  
	\item \hyperref[sec:capitolato]{Capitolato d'appalto}
	\item \hyperref[sec:studiofattibilita]{Studio di fattibilità}
	\item \hyperref[sec:analisirequisiti]{Analisi dei requisiti} 
	\end{itemize}	
		
	% Produttività --------------------------------------------	
	\section{Produttività}	
	\label{sec:produttivita}
	Una misurazione di efficienza, rapporto tra quantità di prodotto realizzato e risorse consumate.

	% Progettazione Software ---------------------------------------------------
	\section{Progettazione Software}
	\label{sec:progettazionesoftware}
	Insieme di attività che precede la realizzazione il cui scopo è quello definire l'architettura del software con l'obiettivo di sviluppare un prodotto corretto per costruzione piuttosto che per correzione.
	
	% Progetto ---------------------------------------------------
	\section{Progetto}
	\label{sec:progetto}
	Insieme di attività e compiti, atti a raggiungere obbiettivi SMART, che hanno una data di inizio ed una di fine, e che consumano un pool di risorse limitate.	
	
	% Project Manager ---------------------------------------------------	
	\section{Project Manager}
	\label{sec:projectmanager}
	Individuo che ha il compito di gestione del progetto. Ciò prevede:
	\begin{itemize}  
	\item Istanziare i processi nel progetto (quelli standard aziendali e quelli istanziati dai processi aziendali)
	\item Stimare costi e risorse necessarie
	\item Pianificare attività (organizzarle nel tempo e a chi assegnarle)
	\item Controllare le attività e verificare i risultati
	\end{itemize}	
		
	% Prototipo ---------------------------------------------------
	\section{Prototipo}
	\label{sec:prototipo}
	Esemplare di prova del prodotto, serve a provare e scegliere soluzioni, puo' avere carattere usa e getta (in metodi iterativi) 
	o puo' essere la base per incrementi successivi (in metodi incrementali).
	Un prototipo di tipo usa e getta comporta dei costi che producono poco valore aggiunto.\newpage	

	% O ----------------------------------------------------------------------------------------------
	{\Huge{\textbf{O}}} \\
	\line(1,0){450}
		
	\section{Obiettivo SMART}
	\label{sec:smart}
	\begin{itemize}  
	\item Specific: formalmente definiti
	\item Measurable: il cui stato di ottenimento è misurabile
	\item Achievable: raggiungibili
	\item Realistic: pratici (raggiungibili)
	\item Time-Bound: costretti da un vincolo temporale
	\end{itemize}\newpage

	% Q	----------------------------------------------------------------------------------------------
	{\Huge{\textbf{Q}}} \\
	\line(1,0){450}
	
	% Quantificabile -----------------------------------------------
	\section{Quantificabile}
	\label{sec:quantificabile}
	Misurabile.
	
	% Qualità -----------------------------------------------------
	\section{Qualità} 
	\label{sec:qualita}
	Riferita alla qualità del prodotto o dei processi:
	\subsection{Qualità del prodotto}
	Caratteristiche di un prodotto di qualità:
	\begin{itemize}  
	\item Sufficienza: Capacità di soddisfare tutti i requisiti
	\item Comprensibilità: Facilità di utilizzo, comprensione del funzionamento da parte degli stakeholder.
	\item Modularità: Suddivisione in parti chiaramente distinte e sconnesse, principio fondamentale dell'information hiding. 
	\item Robustezza: Agli input di utilizzo.
	\item Sicurezza: Affidabilità in caso di malfunzionamento (hardware o software).
	\item Sicurezza bis: Resistenza alla intrusioni. 
	\item Flessibilità: Misura della facilità con la quale il sistema può essere modificato o adattato per soddisfare nuovi requisiti.
	\item Riusabilità: Misura della facilità di riutilizzo in altri contesti dei moduli che formano il prodotto.
	\item Disponibilità: Più inconveniente il tempo di downtime in caso di guasti o aggiornamenti, meno disponibile risulta il prodotto.	
	\item Efficienza.
	\end{itemize}
	\subsection{Qualità dei processi}
	Una buona qualità dei processi induce una buona qualità nel prodotto di tali processi. Per assicurarsi che un processo sia di qualità, è necessario un meccanismo di controllo che effettui modifiche sul way of working ove ritenuto necessario sulla base di misurazioni sulla qualità dei processi. Per definire gli strumenti di valutazione della qualità dei processi dei fornitori in genere, non solo in ambito di sviluppo software, è stato messo a punto lo standard ISO 9000, in ambito specifico di ingegneria software si ha invece ISO 90003:2004.
	
		\newpage

	% R	----------------------------------------------------------------------------------------------
	{\Huge{\textbf{R}}} \\
	\line(1,0){450}

	% Requisito -----------------------------------------------------
	\section{Requisito}
	\label{sec:requisito}
	I requisiti sono le capacità che il sistema dovrà avere per svolgere le funzioni volute dal committente. È best practice definire requisiti:
	\begin{itemize}  
	\item Non ambigui: Interpretati univocamente da tutti .
	\item Corretti.
	\item Completi.
	\item Variabili.
	\item Consistenti: che non si contraddicono a vicenda o da sè.
	\item Modificabili.
	\item Tracciabili
	\item Ordinati: tra obbligatori, desiderabili ed opzionali. 
	\end{itemize}

	% Rischio -----------------------------------------------------
	\section{Rischio}
	\label{sec:rischio}
	I rischi di progetto sono: sforamento costi, tempi, o risultati insoddisfacenti. Fonti di rischio principali sono:
	tecnologie usate, rapporti interpersonali, organizzazione del lavoro, requisiti e rapporti con stakeholder, tempi e costi.
	La gestione del rischio si fa tramite:
	Identificazione (nel progetto, prodotto, mercato), Analisi (probabilità e conseguenze), Pianificazione (come evitare e o mitigare gli effetti), Controllo(attenzione agli indicatori di rischio, raffinamento strategie).
	
	% Riuso -----------------------------------------------------
	\section{Riuso}
	\label{sec:riuso}
	Del codice, puo' essere di due tipi: opportunistico (copia e incolla) ha un basso costo e poco impatto, sistematico, ha un maggior costo e un maggior impatto.		

	% Ruolo -----------------------------------------------------
	\section{Ruolo}
	\label{sec:ruolo}
	Indica l'area di specializzazione (importante per garantire skill eccezionali).
	In un progetto informatico i ruoli principali sono:
	\begin{itemize}  
	\item Analista: Si occupa dell'analisi dei requisiti, in generale: individua e definisce il "problema" in termini formali (in modo che sia possibile verificare se la soluzione è tale). 
	\item Progettista: Si occupa del design del prodotto. In generale: si occupa della soluzione del "problema".
	\item Programmatore: Realizza il prodotto, non ha libertà di scelta, segue le direttive del progettista.
	\item Verificatore: Testano la qualità del prodotto, danno feedback.
	\item Amministratore: Garantisce il funzionamento dell'apparato informatico aziendale. 
	\end{itemize}
	Tutte queste figure professionali sono coordinate nell'ambito di un progetto da un project manager.
	\newpage
	
	% S	----------------------------------------------------------------------------------------------
	{\Huge{\textbf{S}}} \\
	\line(1,0){450}
	
	% Sistematico -------------------------------------------------
	\section{Sistematico}
	\label{sec:sistematico}
	Metodico, rigoroso.

	% Slack -------------------------------------------------
	\section{Slack}
	\label{sec:slack}
	Margine di tempo tra la fine di un attività e la scadenza di fine attività. \\
	Se è positivo, l'attività è prevista terminare prima della scadenza quindi vi è un margine per eventuali imprevisti. Se è vicino o 0 tale margine è piccolo o assente, si introducono quindi dei rischi. Se è negativo, la scadenza è passata e l'attività non è stata portata al termine.
	
	% Software Engineering ----------------------------------------
	\section{Software Engineering}
	\label{sec:swe}
	Disciplina il cui scopo è la realizzazione di prodotti software.
	SWE si occupa dell'organizzazione e della gestione delle attività e compiti e dell'interazioni di un team di sviluppatori
	che hanno per obiettivo per l'attività di sviluppo e per il software risultate efficacia ed efficienza.	
	SWE si occupa inoltre delle metodologie di cura del progetto per l'intero ciclo di vita del software.
	SWE prevede un approccio sistematico, disciplinato e quantificabile all'attività di sviluppo.

	% Stima dei costi -------------------------------------------------
	\section{Stima dei costi}
	\label{sec:stimacosti}
	Influenzata dalle dimensioni del progetto, dalle esperienze pregresse, dalla familiarità con le tecnologie adottate, dalla produttività dell'ambiente di lavoro, dalla qualità attesa. \\Un modello algoritmico che puo' aiutare in ciò è il \hyperref[sec:cocomo]{CoCoMo}.

	% Stima dei costi -------------------------------------------------
	\section{Studio di fattibilità}
	\label{sec:studiofattibilita}
	Studio economico realizzato per determinare se è vantaggioso partecipare ad una gara d'appalto.\newpage

	% Tracciamento dei requisiti --------------------------------------
	\section{Tracciamento dei requisiti}
	\label{sec:tracciamentorequisiti}
	Attività di monitoraggio dell'evoluzione e della scoperta dei requisiti che possono cambiare durante lo svolgersi dello sviluppo del progetto.Tale attività viene svolta tramite apparati informatici appositi, le informazioni importanti sui requisiti di cui viene tenuta traccia sono: status, origin, assegnatari.\newpage

	% V	----------------------------------------------------------------------------------------------
	{\Huge{\textbf{V}}} \\
	\line(1,0){450}
	
	% Validare ---------------------------------------------------
	\section{Validare}
	\label{sec:validare}
	Attività di confronto tra i risultati ottenuti e quelli aspettati. Quest'attività si svolge al termine del progetto.
	
	% Verificare -------------------------------------------------
	\section{Verificare}
	\label{sec:verificare}
	Attività di verifica di presenza di errori e di rispetto del way of working. La verifica si manifesta tangibilmente sotto forma di test, effettuati sui moduli che compongono il prodotto software. Le forme di verifica sono 2: analisi dinamica e analisi statica. L'analisi dinamica prevede l'esecuzione del software mentre l'analisi statica prevede un analisi del codice sorgente senza eseguirlo. 
	\subsection{Analisi Dinamica}
	Coincide con i test: 
	\begin{itemize}  
	\item Di unità: verificano il corretto funzionamento dell'unità.
	\item Di integrazione: verifica la corretta interazione tra più unità che interagiscono dando luogo ad un build.
	\item Di sistema: verificano la corretta interazione tra più build che danno luogo ad un sistema complesso.
	\end{itemize}	
	Il collaudo, test finale eseguito con il committente, è solo un attività formale se i test precedenti sono stati eseguiti rigorosamente. I test devono essere ripetibili e deterministici, perciò dev'essere documentata la loro esecuzione, le condizioni iniziali di esecuzione del test, il risultato atteso e quello ottenuto. I test inoltre devono essere automatizzati per ridurre al minimo la possibilità di errore umano in fase di test.
	\subsection{Analisi Statica}
	Walkthrough e Inspection sono le 2 tecniche di analisi statica prevalenti. Walkthrough prevede un analisi a tappeto del codice sorgente alla ricerca di possibili criticità, mentre la inspection somiglia di più all'analisi a campione e prevede l'analisi specifica di parti considerate di importanza critica.
	
	% Verificare -------------------------------------------------
	\section{Verificatore di requisiti}
	\label{sec:verificatorerequisiti}
	Individuo che ha il compito di verificare i requisiti, tecniche usate sono: \\
	Walkthrough: verifica a tappeto dei requisiti: ogni requisito viene analizzato alla ricerca di non conformità alla best practice stabilita. \\
	Inspection: verifica a campione, campione selezionato sulla base di conoscenze pregresse sulle caratteristiche di requisiti problematici.
	
	% Versionamento ----------------------------------------------
	\section{Versionamento}
	\label{sec:versionamento}
	Organizzazione e gestione delle versioni del software prodotte nel corso dello sviluppo.

	% Versione ----------------------------------------------
	\section{Versione}
	\label{sec:versione}
	Vedi \hyperref[sec:baseline]{baseline}. \newpage

	% W	----------------------------------------------------------------------------------------------
	{\Huge{\textbf{W}}} \\
	\line(1,0){450}
	
	% Way of working ------------------------------------------------
	\section{Way of working}
	\label{sec:wow}
	Insieme di regole, pratiche, che caratterizzano l'attività di sviluppo. Il way of working è pensato per agevolare lo sviluppo ed evitare sprechi di risore ed errori. \newpage

	
	% Z	----------------------------------------------------------------------------------------------
	{\Huge{\textbf{Z}}} \\
	\line(1,0){450}
	
	% Zero-latency ------------------------------------------------
	\section{Zero-latency}
	\label{sec:zerolatency}
	Riferito al lavoro: senza ritardo. L'approccio prevede l'inizio della attività di sviluppo non appena possibile. Permette di avere un margine di tempo tra termine del progetto e consegna.
	
	% Zero-laxity ------------------------------------------------
	\section{Zero-laxity}
	\label{sec:zerolaxity}
	Riferito al lavoro: senza margine. L'approccio prevede l'inizio delle attività di sviluppo nel momento in cui la data di consegna del progetto meno il tempo richiesto dallo sviluppo equivale zero o meno. 
\end{document}